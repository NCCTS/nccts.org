% ------------------------------ sample ----------------------------------------

% WARNING!  Do not type any of the following 10 characters except as directed:
%                &   $   #   %   _   {   }   ^   ~   \

% \section{Simple Text}          % This command makes a section title.

% Words are separated by one or more spaces.  Paragraphs are separated by
% one or more blank lines.  The output is not affected by adding extra
% spaces or extra blank lines to the input file.

% Double quotes are typed like this: ``quoted text''.
% Single quotes are typed like this: `single-quoted text'.

% Long dashes are typed as three dash characters---like this.

% Emphasized text is typed like this: \emph{this is emphasized}.
% Bold       text is typed like this: \textbf{this is bold}.

% \subsection{A Warning or Two}  % This command makes a subsection title.

% If you get too much space after a mid-sentence period---abbreviations
% like etc.\ are the common culprits)---then type a backslash followed by
% a space after the period, as in this sentence.

% Remember, don't type the 10 special characters (such as dollar sign and
% backslash) except as directed!  The following seven are printed by
% typing a backslash in front of them:  \$  \&  \#  \%  \_  \{  and  \}.
% The manual tells how to make other symbols.

% ------------------------------------------------------------------------------

\documentclass[oneside]{book}
\usepackage{graphicx}
\usepackage{hyperref}
\usepackage[utf8]{inputenc}
\usepackage{lxRDFa}
\usepackage[explicit]{titlesec}
\usepackage{tocloft}

\title{\textbf{CELEBRATING LIFE} \\ AS A \\ \textbf{CATHOLIC CHRISTIAN}}
\author{Fr. Francis A. Novak, C.Ss.R.}  \date{June 2013}

\newcommand*\Hide{%
\titleformat{\part}
  {}{}{0pt}{}
}

\begin{document}
\pagestyle{plain}

% ------------------------------------------------------------------------------

\frontmatter

\setcounter{secnumdepth}{-1}
\section*{} \lxRDFa{property=manual-top-title,resource={manual}}

CELEBRATING LIFE AS A CATHOLIC CHRISTIAN

% ------------------------------------------------------------------------------

Nihil Obstat: + Kelvin E. Felix, D.D.  Archbishop of Castries Castries,
St. Lucia, West Indies

Imprimatur: + John F. Donoghue, D.D.  Archbishop of Atlanta October 15, 2004

Celebrating Life as a Catholic Christian Copyright 2010 by Fr. Francis A. Novak,
C.Ss.R.  St. Clement, Liguori, Missouri, 63057, USA www.clcc.info

ISBN: 978-0-615-31588-1

The contents of this book are distributed under the terms of the Creative
Commons license, Attribution-NonCommercial 3.0 Unported. The full text of the
license may be found at the back of the book, beginning on page 309.

A summary of the license terms You are free: to share to copy, distribute and
transmit the work; to remix to adapt the work. Under the following conditions:
attribution you must attribute the work in the manner specified by the author or
licensor (but not in any way that suggests that they endorse you or your use of
the work); noncommercial you may not use this work for commercial purposes. With
the understanding that: any of the above conditions can be waived if you get
permission from the copyright holder; public domain where the work or any of its
elements is in the public domain under applicable law, that status is in no way
affected by the license; other rights in no way are any of the following rights
affected by the license: your fair dealing or fair use rights, or other
applicable copyright exceptions and limitations; the author's moral rights;
rights other persons may have either in the work itself or in how the work is
used, such as publicity or privacy rights; notice for any reuse or distribution,
you must make clear to others the license terms of this work.

Scripture selections, unless otherwise noted, are taken from the New American
Bible, Copyright 1991, 1986, 1970 by the Confraternity of Christian Doctrine,
Incenterfold found on pages 144-145 by Chris Schwartz Studios, Saint Louis,
Missouri.

For more information and additional electronic resources, please visit the CLCC
website.  http://www.clcc.info e-mail: questions@clcc.info

\section*{} \lxRDFa{property=cc-lic-section,resource={cc-lic}}

\begin{center}

This work is distributed under the terms of the Creative Commons license,
\emph{\href{https://creativecommons.org/licenses/by-nc/3.0/us/legalcode}{
Attribution-NonCommercial 3.0 Unported}}.

\href{https://creativecommons.org/licenses/by-nc/3.0/us/}{
  \includegraphics[scale=0.4]{by-nc-nd}
\lxRDFa{property=cc-lic-graphic,resource={cc-lic}}}

\end{center}

% ------------------------------------------------------------------------------

To teach in order to lead others to the faith is the task of every preacher and
of every believer.

--- St. Thomas Aquinas

% ------------------------------------------------------------------------------

The world aflame with Pentecost's fire, Evangelization, the Good News spreading,
God's people true conversion seeking; Enlightened by the Spirit's fire,
Enlivened by the Savior's Body and Blood, Protected under Virgin Mother's care,
Entrusted with the Church's teaching faithfully to live and share.

% ------------------------------------------------------------------------------

To my Father and Mother, and siblings, Sister Elizabeth, Marie, Fr. Henry,
Philip and Bishop Alfred and to my Redemptorist family and friends

% ------------------------------------------------------------------------------

\maketitle

% ------------------------------------------------------------------------------

Celebrating Life as a Catholic Christian is a proven and efficacious tool for
the New Evangelization, that is, the teaching and celebrating and living of our
Catholic faith with the enthusiasm and engagement of the first Christians and
the first missionaries to every part of the world.  In small communities formed
by the parish, Catholics are helped to deepen their own knowledge and love of
Christ, so that they can give an account of their faith to others, especially
those who do not know Christ or have drifted from His friendship.  The small
communities are particularly effective in carrying out the catechesis of those
who are new members of the Church.  The deepening of faith is secured by the
study of the authoritative sources of the Church's Magisterium, Her official
teachings, which are presented in a most accessible manner.  Through the
discussion of the texts of recent Papal Magisterium, members of the small
communities, under the pastoral guidance of the Vicar of Christ on earth, are
prepared in the best manner possible to be effective heralds and agents of the
New Evangelization.

Raymond Leo Cardinal Burke
Archbishop Emeritus of Saint Louis
Prefect, Supreme Tribunal of the Apostolic Signatura

% ------------------------------------------------------------------------------

\chapter{Foreword}

In 1980 Father Francis A. Novak, C.Ss.Rdioceses and parishes, it became clear
that some changes in the process would be useful. Also gracious and positive
feedback from clergy and laity prompted some changes. A few users of the CCL
felt that the process ought to be made more contemporary to counter the
aggressive secularization of culture. In 1998 changes were made, the most
obvious change was in the title, Celebrating Life as a Catholic Christian
(CLCC). The new title was chosen to suggest the tensions Catholics have and will
increasingly have in their struggle to live authentic Christian lives in a
culture hell-bent on its de-Christianization.

Unfortunately in 1995, Fr. Novak contracted a staph infection during a knee
replacement operation. This event greatly curtailed and slowed down the
promotion of the newly named and improved CLCC process. Six years and nineteen
surgeries later, the continuing infection necessitated removal of his right leg
above the knee on Tuesday of Holy Week, April 10, 2001.
Functioning on a prosthetic leg, Fr. Novak is back in action, though with some
limitations. This book, Celebrating Life as a Catholic Christian, is written for
priests and lay leaders in parishes strongly drawn to bring God's people to a
deeper knowledge of the Catholic faith, growth in holiness and commitment to
evangelize one's family, the parish community and world, with special focus on
helping lapsed Catholics return to the Church.

G. William Sefton
Chairman of NCCTS Board of Directors

% ------------------------------------------------------------------------------

\chapter{Acknowledgments}

Some thirty years ago a seminal idea about a faith formation process was planted
in my psyche. I prayed it ``fell on good soil.'' Half way through the stretch of
nurturing, a title for the process came to me: Celebrating Life as a Catholic
Christian (CLCC). By no means was the laborious move from idea to reality a
one-man project. The Lord provided a partnership of talented people to help me
advance the project, namely Mr. Bill Sefton, first chairman of the National
Catholic Conference for Total Stewardship (NCCTS). He was a man of exceptional
commitment to the CLCC apostolate, and a friend of nearly five decades. Our two
episcopal moderators, Archbishop John F. Donoghue, now emeritus, of Atlanta and
Archbishop Kelvin E. Felix, formerly of St. Lucia in the Caribbean and now in
Grand Bay, Roseau, Dominica, provided their friendly presence faithfully  and
their input has always been positive. I thank also all other original donors of
their time and gifts to help define the nucleus of a much-needed ``renewal''
apostolate in the Church, the CLCC process.

The CLCC faith formation process has incalculable potential and requires strong
participant commitment. Inspiring and exemplary commitment has been given by a
small group eager to grow personally in the faith, and eager to spread it by
reaching out to non-practicing, lapsed Catholics. According to the latest
survey, the number of non-church-going Catholics in the U.S. stands at a
staggering 45 million. Helping these millions return to Christ and his Church is
the CLCC's apostolic outreach. Groups formed in parishes that carry out the CLCC
process are known as Small Ecclesial Faith Communities (SEFC). Members of one
such group have met faithfully on Tuesday evenings without interruption for 4
years in over 200 consecutive ninety-minute Faith Formation Sessions. Their
commitment to the CLCC process is not only transparent, but their growth in
knowledge of the faith and in their spiritual life has been a climb of the
mountain. Lay leaders in this group, noted for their extraordinary dedication
and perseverance, are Rob Winkler, Adelaide Herrell and Kimberly Dowd.

The CLCC process, we feel confident, is the work of the Holy Spirit. How else
explain the following occurrence? Out of the masses emerged mysteriously a
Michael Bradley, the right man at the right time, a computer professional. He
came to help us put the CLCC faith formation Manual into final form for
publication. To Michael Bradley very special and sincere thanks for the amount
of selfless time he has generously donated to the CLCC process, even taking six
weeks off from work. Thanks also to Fr. Brian Van Hove, S.J, Board member of the
NCCTS, for introducing Michael Bradley to us to become part of this
mission. Finally, warm thanks to family and many dear friends, who by their
daily prayers, were in large and small ways instrumental in making the CLCC
faith formation process a successful journey from idea to reality.

% ------------------------------------------------------------------------------

\pagebreak
\tableofcontents

% ------------------------------------------------------------------------------

\chapter{Primary Sources}

Citations in this manual are selected from the following list of official Church
documents: Vatican II documents, encyclicals, apostolic exhortations and papal
pronouncements. They are listed here in the order in which they appear in this
manual, not alphabetically. Also listed is nomenclature proper to the CLCC
process.

CL    Christifideles Laici, The Lay Faithful of Christ
    Apostolic Letter, John Paul II, 1988

CCC    Catechism of the Catholic Church
    Libreria Editrice Vaticana, Second Edition, 1997

PDV    Pastores Dabo Vobis, I Will Give You Shepherds
    Apostolic Exhortation, John Paul II, 1991

NMI    Novo Millennio Ineunte, Beginning of the New Millennium
    Apostolic Letter, John Paul II, 2001

LG    Lumen Gentium, Light of the Nations,
    Dogmatic Constitution of the Church
    Second Vatican Council, 1964

EN    Evangelii Nuntiandi, On Evangelization in the Modern World
    Apostolic Exhortation, Paul VI, 1975

RMiss    Redemptoris Missio, Mission of the Redeemer
    Encyclical Letter, John Paul II, 1990

PO    Presbyterorum Ordinis, Decree on the Ministry and Life of Priests
    Second Vatican Council, 1965

CD    Christus Dominus, Decree on the Bishops' Pastoral Office
      in the Church
    Second Vatican Council, 1965

AA    Apostolicam Auctuositatem, Decree on the Apostolate of the Laity
    Second Vatican Council, 1965


DV    Dei Verbum, Dogmatic Constitution on Divine Revelation
    Second Vatican Council, 1965

CT    Catechesi Tradendae, Catechesis in Our Time
    Apostolic Exhortation, John Paul II, 1979

GS    Gaudium et Spes, Pastoral Constitution of the Church
      in the Modern World
    Second Vatican Council, 1965

EE    Ecclesia de Eucharistia, Church of the Eucharist
    Encyclical Letter, John Paul II, 2003

SC    Sacrosanctum Concilium, Constitution on the Sacred Liturgy
    Second Vatican Council, 1963

DI    Dominus Iesus, On the Unicity and Unity of Christ's Church
    Declaration, Congregation for the Doctrine of the Faith, 2000

EV    Evangelium vitae, The Gospel of Life
    Encyclical Letter, John Paul II, 1995

DD    Dies Domini, Day of the Lord
    Apostolic Letter, John Paul II, 1998

DeV    Dominum et vivificantem, The Holy Spirit in the Life of
      the Church and the World
    Encyclical Letter, John Paul II, 1986

Did    Didascalia, Teaching of the 12 Apostles and Disciples of
      Our Savior
    Catechetical text of the 3rd century

DC    Deus Caritas Est, God is Love
    Encyclical Letter, Benedict XVI, 2005

SC    Sacramentum Caritatis, The sacrament of charity
    Apostolic Exhortation, Benedict XVI, 2007

SD    Salvifici Doloris, The Christian Meaning of Human Suffering
    Apostolic Letter, John Paul II, 1984

CV    Caritas in veritate, Charity in Truth
    Encyclical Letter, Benedict XVI, 2009

CLCC        Celebrating Life as a Catholic Christian

SEFC        Small Ecclesial Faith Communities

TS        Total Stewardship

% ------------------------------------------------------------------------------

\chapter{Introduction}

1. Celebrating Life as a Catholic Christian (CLCC) is a process by which lay
Catholics are formed in spirituality within Small Ecclesial Faith Communities
(SEFC) which are established in and by the parish. In addition to becoming
better formed in the faith, the CLCC process prepares SEFC members to carry out
the critically needed apostolate of reaching out to lapsed, non-practicing
Catholics with the aim of helping them return to the Church. For those who are
willing to join a SEFC for re-introduction to the faith, the faith community
then acts as the staging ground for their re-entry into the Church. Fundamental
to this process is Saint James' astounding and unequivocal teaching ``faith of
itself, if it does not have works, is dead ... [F]aith without works is
useless.'' (cf. Jas 2:17,20; cf. CCC 1815) Obviously the goal of this process is
to develop faith more deeply in the People of God, and its apostolic outreach is
conversion of fallen-away Catholics for building up the kingdom of God.
Recently a saddened father confided to me that five of his six adult children do
not go to Mass on Sunday. All were educated in the best of Catholic schools from
the elementary level to college. All were trained in the faith. ``They don't
deny the faith,'' he said,`` they just ignore it, or try to justify their
troubled conscience by saying they don't 'buy' it all.''

2. This book will show why Pope John Paul II has stressed the need of formation
in holiness and the need of New Evangelization that he describes in several
apostolic letters and exhortations issued during his pontificate. For example in
the apostolic exhortation, Christifideles Laici, ``The Lay Faithful of Christ,''
he states that between God who offers his gifts, and the person who receives
them, that person is ``called to exercise responsibility, indeed the necessity,
of a total and ongoing formation.'' Quoting the Synod Fathers (1988), the Holy
Father writes, ``Christian formation (is) a continual process [emphasis added]
of maturation in faith and a likening to Christ in the individual, according to
the will of the Father, under the guidance of the Holy Spirit.'' The Holy Father
emphasizes that ``formation of the lay faithful must be placed among the
priorities of a diocese. It ought to be so placed within the plan of pastoral
action that the efforts of the whole community (clergy, lay faithful and
religious) converge on this goal.'' (cf. CL 57)

3. The ``plan of action'' which the Pope has in mind is embodied in this
process. Because Catholics for two or more generations have been minimally
catechized and therefore minimally formed in the faith, the need of sound,
orthodox catechesis is not only imperative but may not be delayed. The RCIA
boasts of newcomers and returnees to the faith in the tens of thousands each
year. But in not a few cases because they feel unanchored to the Church many
drift way.

4. Unique to Small Ecclesial Faith Communities in the CLCC process is that not
only do fallen-away Catholics have a welcome place to come to, but also new
Catholics, graduates of the RCIA who have the glow of their Easter Vigil
sacramental experience, have a haven to come to for bonding, support, prayer
life, ongoing formation in doctrine and ``maturation in the faith.''

5. Many conscientious parents express concern when they see in their adolescent
and college age children a faith crisis. Their concern is selection, to which
Catholic school should we send them, since some Catholic schools are ``Catholic
in name only.'' They see the faith crisis in their own parish: low Mass
attendance, often irreverent liturgies, and certain sex education programs in
their schools which instead of promoting and preserving chastity destroy the
children's innocence.

6. The growing priest shortage in both diocesan and religious orders is finally
hitting home. Where will they go for Mass? Hard to believe but true, the priest
shortage began some forty years ago when an admission policy to seminaries was
adopted which has been tabbed as ``artificial and contrived.'' This means that
screening of applicants favored those who were liberal, ``open'' and modern. Men
found to be traditional, obedient and prayerful were believed to be ``rigid''
and therefore not accepted. The fruits of this policy are being reaped today. We
ought never to feel that God is not calling men to the priesthood. He is. For
reasons too numerous to mention here, young men are not responding, but in time
they will. Moreover, the new priests ordained in a given year are far too few to
replace old and ill priests who retire from ministry or die, increasing the
number of ``priestless parishes.''

7. Hope should be our trademark. The SEFC is an ideal environment for a young
man to experience ``Church'' and discern his vocation to serve it. Members will
help him follow his vocation quest, pray for him, support him and see him to
ordination. ``The duty of fostering vocations falls on the whole Christian
community,'' stated Pope John Paul II, ``and they should discharge it
principally by living full Christian lives.'' (cf. PD 41)

8. ``Do something'' is the outcry today. A Spirit-filled laity is emerging to do
something about the crisis of faith they see around them. No longer do they feel
comfortable shifting problems to Father or the Bishop. The Holy Spirit is
working in their lives and in families awakening them to the powers that
Baptism, Confirmation and Eucharist impart, especially awareness of their
``right and duty'' to participate in the saving mission of the Church. They
learn that to share in the mission of the Church does not give them license to
act independently of the local pastor or of the bishop or pope. Rather they
strive to share responsibility in the Church's apostolates in friendly
``communion'' with the pastor, the designated head and shepherd of the flock.
Part I of this book presents the CLCC process in its 5 stages. Part II, titled
Authoritative Sources consists of passages taken from 20 key teachings of the
Church. Both Part I and II present almost inexhaustible material for faith
formation, preaching and the apostolate to the lapsed.

% ------------------------------------------------------------------------------

\chapter{Putting the CLCC Process to Work}

PUTTING THE CLCC PROCESS TO WORK

The Process
1. Celebrating Life as a Catholic Christian (CLCC) is a process which
systematically evangelizes God's people in spirituality. At the same time it
stimulates them to carry out their ``missionary duties,'' the major one, helping
lapsed, non-church going Catholics return to the Church. In addition to citing
many official Church teachings, this process incorporates a number of the new
initiatives that Pope John Paul II calls for in his apostolic letter, Novo
Millennio Ineunte, ``On Entering the New Millennium,'' for spreading the Gospel
in the new century.
This process confronts the faith and cultural crisis that currently plagues
millions of Catholics, many of whom seem blissfully unaware of the gravity of
their situation.
By means of New Evangelization and a catechesis for holiness, the CLCC process
aims to give God's people a bold Catholic identity, a way to live and a way to
help other people live genuinely Christian lives not succumbing to the pressures
of secular society nor to the corrosive influences of anti-Christian and
atheistic forces that are corrupting the culture. In opposition to these forces
is this thoroughly Christ-centered CLCC faith formation process. It presents
Jesus' public ministry in Galilee in five stages, while at the parish level the
CLCC process enacts Jesus' Galilean ministry in five parallel stages.

Key People
2. Key people who endorse the process are the priest, pastoral council and heads
of parish organizations. They make a united decision to undertake the CLCC
process in the parish and commit to support it.
Their first job is to select 12 men and women, after the 12 apostles, called the
Matrix. These 12 become the core group of lay persons who lead the process. They
agree to lead the process through its 5 Stages. Their first responsibility is to
learn the process in a general way. As the process unfolds they learn the
specific responsibilities relative to each stage.

Small Ecclesial Faith Communities
3. The Matrix 12, the core lay leadership group, is the first of many Small
Ecclesial Faith Communities (SEFC) formed in the parish. Many more Small
Ecclesial Faith Communities will follow. Indeed small communities of different
kinds already exist in parishes. However, it is advised and highly useful for
parish community building (``communion'') that existing groups be willing to be
part of the CLCC process. The following and all others are invited and urged to
join: existing bible study groups, prayer groups, religious education
instructors, parish school teachers, the liturgy committee, choir members, the
youth group, young adults, music ministers, the married, single, widowed, the
pro-life group, St. Vincent de Paul Society, ushers, greeters, collection
counters, the finance committee, sports directors, maintenance people,
all. There is no one in the parish not in need of further spiritual growth and
updating in his faith. After Jesus ``called'' the 12 Apostles (cf. Mt 4:18,21;
10:1-4), and ``appointed'' 72 disciples (cf. Lk 10:1; Mk 3:16), they followed
Jesus to learn his way of life. They grew in the ``blessedness'' Jesus
proclaimed and exemplified. Jesus formed the first small faith
community. Through the centuries thousands of small faith communities have grown
into ``communions,'' called dioceses and parishes, which constitute the
infrastructure of the Church, the Body of Christ.

CLCC is not a Quick-Fix
4. The CLCC is decidedly a process, and not a program! A program has a beginning
and an end, whereas a process is developmental. Through a series of small steps
taken within several stages it advances toward certain desired goals. Unlike
programs, the CLCC process is ongoing. It's open-ended. Being a process of
formation in holiness and in knowledge of the faith, usually a radical change in
lifestyle begins. The CLCC process is, therefore, not a quick-fix. Saints are
not made overnight. Time is needed for lay leaders and parishioners to get the
process underway. People need time to understand the process and act on it,
especially the astounding prospect of becoming holy. True ``spiritual warriors''
who do battle with Satan to save lapsed Catholics need time for solid spiritual
formation. Thus, the CLCC process may not be hurried or short-circuited. Every
temptation to get it over must be banished. Time is the soul of the CLCC
process. Time will show that the process achieves hoped for and measurable
results. Personal holiness and zeal for recovering ``lost'' and lapsed Catholics
cannot be bought online by credit card.

How to Start the CLCC Process
5. Parishes have a choice of three ways to start the CLCC process.

A five day mission, Sunday to Thursday, with preaching centered on the meaning
and need of the process and an explanation of the 5 stages, one stage explained
each evening.
A three, four or five day traditional mission on the eternal truths: need of
salvation, the reality of death, judgment, God's loving mercy, the need of
making a good, integral confession, heaven or hell, and the Blessed Mother.
The parish priest announces the process over several weekends, explaining what
it is, and the absolute usefulness of everyone entering in and belonging to a
Small Ecclesial Faith Community. He emphasizes its twofold focus: catechetical
formation in the faith and in holiness, and outreach to recover lapsed Catholics
who have strayed from the faith. He points out the glory that will be given to
the Lord when all Catholics in the parish, active and alienated, will begin
Celebrating Life as Catholic Christians.

Holding a mission prior to initiating the process has its advantages, but a
mission is optional. A mission can bring the people to a spiritual ``high.'' But
to implant a faith formation process in the parish like the CLCC as a follow-up
to a mission is a perfect way to build on the mission's ``high'' and its
momentum. In other words, ``strike while the iron is hot!'' In any case the
parish priest's preaching on the process has value that ought not be
underestimated. Hearing him parishioners will know immediately that he and the
lay leaders alike are making the CLCC a top parish priority.

Like advertising on TV, parish leaders have a sell job that is non-stop. Those
slow to buy into the process will eventually come around. However parishioners
who for reasons of health, age, handicap or other good reasons cannot join a
SEFC are encouraged to take up the spiritual formation process privately. They
are also urged to pray for the success of the CLCC process, for example, praying
the rosary daily or doing some good work, invoking God's blessing on all those
entering the process and for a great spiritual awakening in the parish.

Importance and Necessity of Prayer
6. Jesus said solemnly and emphatically, ``Whoever remains in me and I in him
will bear much fruit.'' (cf. Jn 15:5) Prayer unites us to the Lord. Prayer
creates a bond between Christ and the person who prays. ``For where two or three
are gathered together in my name, there am I in the midst of them.'' (Mt 18:20)
In prayer we come to him, and he comes to us. Through prayer good things happen,
and good things are given us as Jesus promised. Thus, to begin the CLCC process
is one thing. To have the CLCC process ``bear much fruit'' is another. The
``fruit'' is a new level of holiness in all parish members who participate in
the formation process. Holiness is brought about through the experience of
planting, pruning, cultivation and reaping of New Evangelization, orthodox
catechetical formation, holiness, conversion of the lapsed and the return of
many to Christ.
Why the emphasis on remaining united to the Lord through prayer? Jesus gives
this straight reply, ``because without me you can do nothing.''Jesus' analogy of
the vine and branches applies to the importance and necessity of prayer. He
said, ``I am the vine, you are the branches.'' He exhorts, ``Remain in me, as I
remain in you.'' He warns, ``Just as a branch cannot bear fruit on its own
unless it remains on the vine, so neither can you unless you remain in me.'' (Jn
15:4-5) In sum, the CLCC process cannot bear fruit in the parish unless all
persons involved are united to the Lord through the grace-filled power of
prayer.

Desired Results of the Process
7. What are some of them? When parishioners show greater faith and fervor in
their worship at Mass; when growing numbers of parishioners, by God's grace,
show real earnestness about living their lives as Catholic Christians; when
after a period of time a detectable air of humility and holiness wafts through
the parish; when a new and fresh spirit of involvement and cooperation become
evident among the People of God.
Evidence of progress in holiness will also be seen when Catholics who have been
away from the sacraments for many years seek reconciliation with the Church
through the sacrament of penance; when new faces begin to appear at Sunday Mass
alongside the regulars, thus increasing Mass attendance; when liturgical
celebration becomes more transcendent and less banal; when people ask for public
adoration of the Blessed Sacrament and come faithfully to make a personal hour
of adoration; when marriage becomes esteemed as a sacred life-long covenant;
when spouses become more forgiving and compatible; when family life becomes more
peaceful and stable and children are parent-directed, rather than parents being
directed by children.
As holiness grows among the ``actives'' in SEFCs, so interest in ``missionary
activities'' will grow. Awakened will be a sense of responsibility to reach out
to Catholics who are religiously comatose and critically in need of New
Evangelization. The agent of this responsibility is God's grace. The Holy Spirit
impels them to seek out and help lapsed brothers and sisters to join a SEFC for
return to the Church. The grace-filled actives recover the ``lost.'' They are
builders of ``communion'' in the Church. They help bring the Body of Christ to
full stature.

Lay Leaders and Small Ecclesial Faith Communities
Organization
8. Lay leaders of the CLCC process are chosen for having shown some
organizational and management skills. More importantly they possess a strong
desire to grow in personal holiness and in knowledge of the faith and want to
help fellow parishioners do likewise. Recognizing their own hunger for truth and
holiness, they feel certain others experience the same hungers but have not
diagnosed their condition as primarily spiritual rather than psychological,
social, marital or material. Thus leaders are willing to learn and lead a
spiritual conversion process, the CLCC, which is of utmost importance for soul,
mind, heart and body.
Lay leaders first organize themselves. They form the first Small Ecclesial Faith
Community (SEFC). Their main function is to hold weekly spiritual formation
sessions for themselves. In them they learn how to conduct a spiritual formation
session, how to use the material in the 5 Stages starting with Stage 1, and
later going into Part II, titled Authoritative Sources. Formation consists of
internalizing the content of each paragraph in each stage following the order
for proper conduct of a session.

9. Lay leaders in the first SEFC, also called the Matrix 12 after the 12
Apostles, choose from among themselves a Moderator and an Associate
Moderator. They are advised not to ``hire'' an outside facilitator. Educated,
qualified and motivated people that they are, they can and will become expert
facilitators themselves. Should questions arise, you are welcome to e-mail the
NCCTS at anytime: questions@clcc.info.
Lay leaders are also responsible for initiating in the parish other Small
Ecclesial Faith Communities (SEFC). Like the Matrix 12, SEFCs should ideally
have 12 members. They may have 10 but ought not to go below 7. There is no limit
to the number of SEFCs that can be formed in a parish, the more the better. They
should dot every area within the parish boundaries. Like the Matrix 12, members
of SEFCs ``moderate'' themselves, choosing persons within the group whom they
feel are best qualified to lead.

Functions
10. Moderators call and conduct the sessions. They follow strictly the Order for
Conducting Spiritual Formation Sessions (see the next chapter). All members
should be willing and given a chance to take turns moderating a
session. Sessions may be held in people's homes or in parish
facilities. Experience shows that preference tilts to people's homes. A private
home is less institutional and less threatening to returnees to the faith than a
church meeting hall.
Most important is that the moderator keeps the session on track without
deviation, that is, he stays strictly to the content of the paragraph under
consideration. Allowing participants to wander off track, letting some members
dominate the discussion, to air personal gripes and distract the group from its
spiritual focus -- these are sure ways to ``kill'' the group and destroy the
process.
Moderators are to begin the session at the agreed upon time, giving each member
a chance to briefly share (without coercion) their spiritual insight or
experience with the group. The session must conclude exactly at the agreed upon
time. To go overtime, no matter how lively the discussion, is also a sure way to
``kill'' the formation process and the group's commitment to it.
If some members choose to continue the session's discussion, they are free to do
so after it concludes. The moderator must conclude the formation session in 90
minutes in consideration of those who have other obligations.

The weekly session is 90 minutes, 45 minutes for the first half, Interior
Formation, then 45 minutes for the second half, Exterior Action and Conclusion.

% ------------------------------------------------------------------------------

\chapter{Order for Conducting Formation Sessions}

in Small Ecclesial Faith Communities

Hymn
Select an appropriate religious song familiar to all  missalette, hymn book or
CD.

Time Frame
90 minutes
Start on time; end on time; do not go overtime.

Opening Prayer
Suggestion, use a liturgical prayer in accord with Church's season.

Source
Part I, material as given in 5 stages.
Part II, material in Authoritative Sources.

Subject Matter
Each stage contains many short and long paragraphs. For example, Stage 1 has
about 45 paragraphs, enough material for 45 formation sessions. Read aloud only
one paragraph per session, and from it choose a sentence, phrase or short
excerpt that strikes you. That choice is your subject matter; every person in
the SEFC makes a similar choice from the same paragraph.

Interior Spiritual Formation
45 minutes
All spend a short time in quiet, prayerful reflection, each member opening
himself to the Holy Spirit to receive insights for spiritual development. After
a short time members of the SEFC share their insights (gifts) with the full
group. Stay with the subject matter of the paragraph; do not wander off to other
paragraphs, as this will distract you and confuse the group. Strive for
togetherness, bonding, and the all-important ``communion'' in spirit and truth
that links the People of God in SEFCs.

Exterior Action
45 minutes, includes Action Resolution and Conclusion
In personal, quiet time all reflect briefly on an action the Holy Spirit
suggests. The action centers always on how you or the group can advance the CLCC
process. Invoke the Holy Spirit for light and Mary, Mother of the Church, for
heavenly assistance. Share with your group the action you choose to do either
alone or with the group; try to reach consensus on the best action to take.

Action Resolution
Deciding on a personal or group action has unlimited possibilities. Example:
help make the formation sessions run smoothly so that all members learn and get
used to the Order given here; make interior formation prayerful and holy;
respect everyone's right to speak; start thinking of parishioners to invite into
your SEFC to bring the number up to ten or twelve; begin making a list of
lapsed, non-church going Catholics who need to be visited and brought into your
SEFC. Exterior action includes, if possible, going to weekday Mass, making a
visit to the adoration chapel and praying for the conversion of lapsed
Catholics.

Conclusion
Prayers of intercession come next led by the moderator or a group member. Such
prayers may be printed, pre-prepared or spontaneous. Always include a petition
that ALL parishioners enter into the CLCC formation process.

Closing Prayer
Selected from the Church's liturgical prayers for the previous Sunday, the
current feast or memorial, etc.

Hymn
From the missalette, hymn book or CD.
Total session time should not exceed 90 minutes.

% ------------------------------------------------------------------------------

\mainmatter
\addtocontents{toc}{\cftpagenumbersoff{part}}
\part{The Process}
\addtocontents{toc}{\cftpagenumberson{part}}

% ------------------------------------------------------------------------------

\chapter{Stage 1.\ Call of Apostles and Disciples}

\section*{} \lxRDFa{property=stage-main-content,resource={manual}}

Important note:  Before beginning Stage One, you are asked, please, not to
quick-read this book as you may read a novel or the newspaper. This is a study
manual of deep content that aims to help Catholics rise to higher levels of
formation in the faith for growth in personal holiness. The Small Ecclesial
Faith Community (SEFC) is the best suited environment for implanting in
participants' minds, hearts and souls some of the great truths taught by Christ
and his Church. So that everyone is on the same page, please go to the front
pages of this manual and reread Putting the CLCC Process to Work. Like a three
dimensional image you will ``see'' clearly the two specifics of the CLCC
process: formation of God's people in personal holiness, and motivating them for
mission:  to reach out to and help lapsed and alienated Catholics return to
Christ and his Church.

The first action in a CLCC formation session is the leader's reading aloud a
segment, a paragraph or group of paragraphs, from the manual to the SEFC. At the
bottom of the segments stand the letters RDS. R stands for Reflection: quiet
time given to understand the content of the section. D stands for Discussion:
personal prayer time directed to the Holy Spirit asking him for light,
dialoguing with him within oneself and ordering the received insights. S stands
for Sharing: each participant contributes to the group the insights he or she
has received and wants to share. Alongside RDS is given a page number to turn to
for aiding the group's Reflection, Discussion and Sharing.


Introduction
1. Jesus' three-year public ministry in Galilee was a divinely planned process
of salvation for the whole human race. It is divided into five stages, each
stage is distinguishable from the other, yet each stage is connected to the
other. Celebrating Life as a Catholic Christian (CLCC) is a five-stage faith
formation process modeled on Jesus' public ministry in Galilee. It is designed
for implementation in parishes of the local Church.
RDS, page \#

2. Jesus' ministry has two goals: giving his disciples a new way of life in
faith, the other is missionary outreach. The two goals of the CLCC go together:
interior spiritual formation conjoined to an exterior action.

The first goal focuses on personal and communal spiritual growth that gives the
believer a deeper knowledge of Christ and his Church, and on improving his or
her prayer life. To do this believers enter a small faith group to experience
bonding with like-minded people for community building. These goals help God's
people answer Christ's call, ``Come, follow me,'' (Lk 18:22) and his command,
``Love one another.'' (Jn 13:34)

The second goal focuses on a particular exterior missionary action, reaching out
to lapsed, non-church going Catholics, estimated at 45 million in the US, with a
view to helping them return to Christ for reconciliation and salvation, thus
building up the kingdom of God on earth, the Church.
RDS, page \#

For better understanding and results-producing pastoral use of the CLCC process,
each of the five stages has essentially two parts: A. What Jesus did in Galilee
and B. Enacting in the Parish What Jesus did in Galilee.

A. What Jesus Did in Galilee
1. The first thing Jesus did in his public ministry in Galilee was to gather
followers and from among them he called and selected 12 apostles. They were
select witnesses of his teachings, his miracles, his way of life and his
communion with the Father in prayer. Through his Passion, Death, Resurrection
and Pentecost, Christ founded his Church, the kingdom of God's people on earth
as designed by the Father, a kingdom that will last till the end of time through
the sustaining presence and power of the Holy Spirit. In founding his Church on
earth Jesus guaranteed its continuity and triumph despite the fierce powers of
hell raging against it. (cf. Mt 16:18)
RDS, page \#

2. The terms ``Christian'' and ``Catholic'' were unknown in Jesus'
day. (cf. Acts 11:26) Yet what Jesus was teaching and sharing with his Apostles
and disciples was Celebrating Life as Catholic Christians (CLCC). With their own
eyes and ears they saw and heard the primacy Jesus gave to Commandments of love
of God and love of neighbor. Daily he showed this love by his self-giving to
everyone to the point of exhaustion. They saw the compassion he showed toward
repentant sinners and his rebuke to those who exalted themselves; how he
detested hypocrisy, religiosity, pseudo-piety and false worship. In the many
different ways that Jesus dealt with people his followers were to imitate him.
RDS, page \#

3. Catholics and other people are immersed in the present-day culture of
secularism and neo-paganism. They need a defense against the barrage of
anti-Christian influences that surround them. They must be armed to deal with
the growing, overt hostility against Christianity, the Catholic Church in
particular, at themselves and the values they live by. Thus, an empowerment
process like Celebrating Life as Catholic Christians becomes imperative, and
should be classed among the leading parish apostolates. God's people under siege
need guidance, group support and grace to live authentic Catholic lives, gifts
God gives through the CLCC formation process.
RDS, page \#

4. Call of the 12 Apostles and 72 Disciples
Biblical texts

Mt 4:18-22
``As Jesus was walking by the Sea of Galilee, he saw two brothers, Simon whom he
called Peter, and his brother Andrew casting a net into the sea; they were
fishermen. He said to them, 'Come after me, and I will make you fishers of men.'
At once they left their nets and followed him. Walking a little farther he saw
other brothers, James and John, Zebedee's sons, and he called them and
immediately they left their boat and their father and followed him.''

Mt 9:9-10
``He also called Levi, named Matthew, who was sitting in his customs post
collecting taxes. He said to him, 'Follow me.' And he got up and followed
him. While he was at table in his house, many tax collectors and sinners came
and sat with Jesus and his disciples.''

Lk 6:12-13
``In those days he departed to the mountain to pray, and he spent the night in
prayer to God. When day came, he called his disciples to himself, and from them
he chose Twelve, whom he named apostles.''

Mt 10:1-4
``The names of the twelve apostles are: first, Simon called Peter, and his
brother Andrew, James, son of Zebedee and his brother John; Philip and
Bartholomew, Thomas and Matthew, the tax collector; James, son of Alpheus, and
Thaddeus, Simon the Cananean and Judas Iscariot who betrayed him.''
RDS, page \#

5. Mission of the Apostles and Disciples
Biblical Texts

Lk 9:1-2,6
``He summoned the Twelve and gave them power and authority over all demons and
to cure diseases, and he sent them to proclaim the kingdom of God and to heal
[the sick]. Then they set out and went from village to village proclaiming the
good news and curing diseases everywhere.''

Lk 10:1-3
``After this the Lord appointed seventy [two] others whom he sent ahead of him
in pairs to every town and village he intended to visit. He said to them, 'The
harvest is abundant but the laborers are few; so ask the master of the harvest
to send laborers for his harvest. Go on your way; behold, I am sending you like
lambs among wolves.'''
RDS, page \#

6. Catechism of the Catholic Church
The following quotations give a concise summary of the Church's teaching on the
Mission of the Apostles.

CCC 75
``Christ the Lord, in whom the entire Revelation of the most high God is summed
up, commanded the apostles to preach the Gospel, which had been promised
beforehand by the prophets, and which he fulfilled in his own person and
promulgated with his own lips. In preaching the Gospel, they were to communicate
the gifts of God to all men. This Gospel was to be the source of all saving
truth and moral discipline.''

CCC 1575
``Christ himself chose the apostles and gave them a share in his mission and
authority. Raised to the Father's right hand, he has not forsaken his flock but
keeps it under his constant protection through the apostles, and guides it still
through these same pastors who continue his work today. Thus, it is Christ whose
gift it is that some be apostles and others pastors. He continues to act through
the bishops.''
RDS, page \#

B. Enacting What Jesus Did in Galilee in the Parish
1. Leadership of the Priest
Jesus chose 12 apostles to assist him in his mission and provided for their
continuing his mission. Likewise the parish priest must find lay persons to
assist him in his mission, an important one, of launching the CLCC process. Pope
John Paul II states that the parish priest is ``configured to Christ, the good
shepherd.'' (cf. PD 22) It is the pastor who goes out among the sheep and lambs
of his parish. He scans the flock, selects and calls 12 men and women of all age
groups, young, middle aged and older, who can help him carry out the CLCC
process in the parish. They become the lay leaders of the CLCC process. They are
called the Matrix 12.

The priest informs the people in the pews about the CLCC process, explains what
it is, and asks for their participation in it. The mission of the lay leadership
group, the Matrix 12, is to learn the process and guide it through its 5 stages,
receiving throughout support and encouragement from the priest and cooperation
from fellow parishioners. For its success the priest's initiative and active
involvement in the process cannot be emphasized enough. Strongly recommended is
that this process becomes a parish priority, taking precedence over less urgent
ministries and parish activities.

In order to select the right people for carrying out the process, the priest
should pray, as Christ did, and greatly depend on the light of the Holy Spirit
for the gift of discernment to ensure he chooses the right parishioners to
become the Matrix 12. He should also pray to be open to the Holy Spirit's
bidding, that he chooses other gifted parishioners to form SEFCs, and resist the
temptation to select ``never say no'' workers, favorites or super-involved
people often already over-burdened with parish projects.

The priest must always keep in mind the fundamental purposes of the CLCC
process.The first purpose of the CLCC process is to help people attain new
levels of personal and communal holiness in preparation for doing effective
evangelization. The second purpose is missionary, to find and help save the
wayward sheep, lapsed Catholics. Recall the gospel parable of the lost
sheep. ``What man among you having a hundred sheep and losing one of them would
not leave the ninety nine in the desert and go after the lost one until he finds
it? When he does find it  he calls together his friends and neighbors  Rejoice
with me because I have found my lost sheep  in just the same way there will be
more joy in heaven over one sinner who repents than over ninety-nine righteous
people who have no need of repentence.'' (cf. Lk 15:4-7)
RDS, page \#

2. Teachings of Vatican Council II
``Just as the Son was sent by the Father, so too he sent apostles.'' (cf. Jn
20:21) The Church has received from the apostles the task to be Christ's
``witnesses'' and proclaim his saving truth ``even to the ends of the earth.''
(cf. Acts 1:8) ``Hence she makes the words of the Apostle her own: 'Woe to me,
if I do not preach the gospel' (cf. 1 Cor 9:16), and unceasingly sends heralds
of the gospel [to]  carry on the work of evangelizing. The Church is compelled
by the Holy Spirit to do her part towards the full realization of the will of
God, who has established Christ as the source of salvation for the whole
world. By proclamation of the gospel, she prepares her hearers to receive and
profess the faith, disposes them for baptism, snatches them from the slavery of
error, and incorporates them into Christ so that through charity they may grow
up into full maturity in Christ  The obligation of spreading the faith is
imposed on every disciple of Christ according to his ability.'' [emphasis added]
(cf. LG 17)
RDS, page \#

3. Apostles and Disciples, the First Small Faith Community!
During his three-year public ministry Jesus not only chose 12 apostles and 72
disciples, but also formed ``the 12'' into a small, close-knit community, the
first of its kind. They were not men who merely followed Jesus passively. Jesus
``called'' them, formed them and ``appointed'' them by divine design to become
active co-workers with him. They were the first to come to know the purpose of
his mission: ``to be evangelized [in order to] evangelize,'' words of Blessed
John Paul II.

``He appointed 72 others whom he sent ahead of him in pairs to every town and
place he intended to visit.'' (Lk 10:1). They were witnesses to Christ. He
empowered them to ``cure the sick'' in whatever town they entered and welcomed
them, and to say, ``The kingdom of God is at hand for you.'' (cf. Lk 1:5,8-9)

The community of apostles and disciples in Galilee has become a model for the
establishment of Small Ecclesial Faith Communities (SEFCs) in parishes and
dioceses. The importance of SEFCs cannot be measured, for they implant in God's
people a good that is at once spiritual, personal and communal. They strive to
achieve an in depth sense of community, which is most needed in today's secular
and individualized culture. Fostering community among God's people is imperative
for families, for parishes and dioceses, and above all in the secular work
places of the laity.
RDS, page \#

4. Models of Leadership in the Old Testament
Words like ``call,'' ``chosen,'' ``mandated,'' ``sent,'' and ``mission'' were
central in Christ's ministry. Consider some of the leading figures in the Old
Testament who were also specially called by God and formed to carry out a
mission. They were models for Christ's disciples and are models for our
formation and mission.

Examples
Samuel, when only a boy, was called by God (1 Sam 3:3b-19); became a great seer
(prophet) and judge (ruler) of the 12 tribes of Israel (1 Sam 7:12,15-17).

Jonah, a prophet, was sent on mission to the city of Nineveh to predict
destruction for its evil ways. In sackcloth the king and people turned away from
their sins and God preserved them and their city (cf. Jon 3:1-5,10).

What about Abraham, his ``call'' (cf. Gen 12:1-3), the spiritual seed (Gen 15),
and the test of his faith? (Gen 22)

Joseph, son of Jacob, born of Rachel (Gen 30:24), a shepherd boy (Gen 37:2-3),
sold to slavery in Egypt only to rise to be steward of Pharaoh's vast country
(Gen 41:41).

Moses who was ``called'' (Ex 3:2-10), led Israel out of Egypt across the Red Sea
(Ex 13-14), and to whom God gave the 10 Commandments on Mt. Sinai (Ex 19-20).

Who can forget the courageous Esther, a Jewess, who by divine plotting became
queen to King Ahasuerus of Persia where Jews were targeted for extermination by
order of the king, but who by her brave intervention with him and admission she
was Jewish saved her people from massacre? (Esth 1-9; cf. also Dictionary of the
Bible, John L. McKenzie, S.J)

To appreciate how God in his grand plan of salvation called, chose and
dispatched certain people to carry out his will, it is useful to look up the
above texts in your bible. Many more models could be added.
RDS, page \#

5. Small Ecclesial Faith Communities, Why Are They Needed?
Most parishes, large, medium-sized and small have a sizable number of persons
who call themselves Catholic, but are in fact lapsed, alienated, non-practicing
Catholics. Many are weak in their knowledge of the faith, and in numerous cases
almost totally ignorant of even the fundamentals of the Catholic Faith.

Some who dissent from the faith nurse a bias against certain truths of the faith
they don't like, for example, artificial contraception, cohabitation, same sex
marriage, etc. Some reject outright various teachings of the Church, and justify
their rejection by appealing to the principle of ``religious freedom.'' They
claim their conscience is at peace. Not a few ``in name only Catholics'' are
graduates of Catholic colleges. They enrolled in the school with shaky faith and
not infrequently left the school with ``freedom'' not to practice the faith.

Catholics such as these need to be reached, befriended and won back to the
Church. Some may be looking for a way back into the fold. Others may be
closed-minded and resist any invitation to return to the Church. The power of
God's grace can never be under-estimated. For some, the RCIA program fills their
faith gap. Others, really the majority, need to be brought to a more lively
practice of the Faith through the slower-moving, more in-depth dynamics of Small
Ecclesial Faith Communities where welcome is warm and camaraderie
long-term. Within them results are more predictably measurable.Two outstanding
Pontiffs have explicitly endorsed and promoted small faith communities in their
teachings (cf. Evangelii Nuntiandi, EN, Paul VI, 1975, 58, 60; Redemptoris
Missio, RM, John Paul II, 1990, 51). In faith communities, Church-going
Catholics advance admirably to higher levels of knowledge of their Church's
teachings and in holiness. Alienated, lapsed and ``Christmas and Easter''
Catholics who accept invitations to come into such a community will first
experience bonding with the members; then, by association with them in study,
prayer and sharing, and by the power of God's grace, they will become disposed
to return to the Church.
RDS, page \#

6. Blessed Trinity, Archetype of Small Ecclesial Faith Communities
Under the heading THE PERSON AND SOCIETY, the Catechism of the Catholic Church
includes this sub-heading, The Communal Character of the Human Vocation, and
makes this statement, ``All men are called to the same end: God himself. There
is a certain resemblance between the unity of the three divine persons and the
fraternity that men are to establish among themselves in truth and love. Love of
neighbor is inseparable from love of God.'' (CCC 1878) It goes on to say that
each community is defined not only ``by its purpose  and consequently obeys
specific rules; but the human person  is and ought to be the principle, the
subject and the end of all social institutions.'' (CCC 1881)

Singled out as first is the communal family, followed by ``the state,''
``voluntary associations and institutions  on both national and international
levels  various professions, economic, cultural and recreational, including
sports.'' (cf. Mater et Magistra, John XXIII, 1961, 60; cf. also CCC 1882)

In his encyclical, Peace on Earth, Blessed John XXIII wrote: ``Human society
must primarily be considered something pertaining to the spiritual. Through it,
in the bright light of truth, men should share their knowledge, be able to
exercise their rights and fulfill their obligations, be inspired to seek
spiritual values  and eagerly strive to make their own the spiritual
achievements of others.'' (Pacem in Terris, PT, 1963, 36; cf. also CCC 1886)

Addressing the issue of authority, Blessed John XXIII stated: ``Human society
can be neither well-ordered nor prosperous unless it has some people invested
with legitimate authority to preserve its institutions and to devote themselves
as far as is necessary to the work and care for the good of all.'' (PT 46;
cf. also CCC 1897) ``Authority is exercised legitimately only when it seeks the
common good of the group concerned, and if it employs morally licit means to
attain it, [otherwise it is] not binding in conscience.'' (PT 51; cf. also CCC
1903)
RDS, page \#

7. SEFCs, Front Doors for Lapsed Catholics to Come Home to the Church
Small Ecclesial Faith Communities (SEFCs) are front doors with welcome mats out
where Catholics away from the Church can enter the house of the Lord and begin
to feel accepted and and be enkindled with love of the Catholic faith.

Every Catholic, including the lapsed, on entering a Small Ecclesial Faith
Community, begins a process of change. SEFCs are established for the specific
purpose of change. Within them members become notably better formed in the
faith, strive to develop a deep love for Christ from the truths they learn about
him and his Church, and grow spiritually. Invigorated by the power of truth,
many then are inspired to tell others about their conversion experience
(evangelization). They want to evangelize others.

The second purpose of an SEFC is for its members to reach out to and invite
lapsed Catholics to join their community. Having entered the group, the lapsed,
by association with practicing Catholics and exposed to the community's
formation and prayer life, are re-introduced to God's saving grace. They set
foot on the road that leads to a great change of heart. They leave their former
life of separation from the Church. They begin Celebrating Life as Catholic
Christians.

As the group bonds together, practicing Catholics must be on their toes to give
good example in Christian living. They ``celebrate'' the Christian life not for
show, but from conviction that they are Christ's disciples, genuinely committed
to Jesus' way of life in spite of the influences of a secular, anti-Christian
world. On the other hand, newcomers to the group become witnesses of this
counter-cultural way of life. Through gradual formation in the faith and by the
outpouring of grace received from the Holy Spirit, they move toward re-entry
into the larger community of Christ  the One, Holy, Catholic and Apostolic
Church.
RDS, page \#

8. Priest's Mission is ``To Beget''
Vatican Council II's Decree on the Ministry and Life of Priests, states
``Acknowledging Christ's desire and inspired by the Holy Spirit, the apostles
considered it their duty to select ministers 'who shall be competent in turn to
teach others.' (2 Tim 2:2) This duty then 'to beget' is part of the priestly
mission by which every priest is made a partaker in the care of the whole
Church, so that workers may never be lacking for the People of God on earth.''
(PO 11). This statement applies to bishops, successors of the apostles, and to
priests who aid their bishops in the offices of teaching, sanctifying and
governing. Stated here are the basic principles of shared responsibility.

In their mission priests also share responsibility with the laity, particularly
in the role of teaching: priests teach truth from the pulpit, teachers of
religion form children in the faith in Catholic schools, in the religious
education programs, in the Rite of Christian Initiation of Adults, in
preparation for the Sacraments of Baptism and Confirmation, and in the
consultative body called the parish pastoral council. Vatican Council II's
statement about church leaders, bishops, having the ``duty to select ministers
'who shall be competent to teach others,''' points first to priests. Part of the
priests' mission is to select competent laypersons to help teach eternal truths
in pastoral work such as in the CLCC process. Theirs is an educational,
formational and spiritual responsibility shared with the bishop.The Decree sums
up shared responsibility in these words, ``This duty then is a part of the
priestly mission by which every priest is made a partaker in the care of the
whole Church, so that workers may never be lacking for the People of God on
earth.''(cf. also LG 34-37; CCC 783; RH 19)

Thus, the meaning ``to beget'': Jesus, the God-Man came to earth to save all
mankind. He is the one and only Savior. There is no other. (cf. DI 13) To Peter
and the 12 Apostles he entrusted the great responsibility to carry out his plan
of salvation in the whole world. (cf. Mt 28:18-20; also 16:18) Through the
apostles' successors, bishops, their helpers, priests, and lay persons, Christ
continues his work, by their preaching the good news, administering the
sacraments to his people, and sharing responsibility with the laity who can
contribute much to the development and spread of the faith of Christ's Church
throughout the world.
RDS, page \#

9. Formation in and Functions of Small Ecclesial Faith Communities
The primary goal of every person who belongs to an SEFC is to strive to advance
in personal holiness through the CLCC faith formation process as given in this
study manual. Formation in holiness of life is based on Jesus' teachings during
his public ministry, as recorded in the Bible. Members of SEFCs are also formed
in official Church teachings, many of which may relatively unknown to them
(cf. Part II, Authoritative Sources). Jesus' teachings recorded in the gospels
and the Church's official teachings interlink. Each participant in an SEFC
should be open ``to be evangelized [in order to] evangelize'' in his or her
parish; that is, willing, if possible, to do mission outreach to lapsed,
alienated, non-regular church goers, befriend them and invite them to join a
Small Ecclesial Faith Community for conversion to Christ and his Church.
RDS, page \#

10. Catechetical Instruction Holds Primacy of Place
Jesus assures us, ``Where two or three are gathered together in my name, there
am I in the midst of them.'' (Mt 18:20) Moreover he dispenses his graces
generously to those who ask. It is in the close-knit Small Ecclesial Faith
Community where good will and fraternity exist, and where training in the faith
and motivation to reach out to non-regular Church-going Catholics takes place.

In the lapsed who join one of the SEFCs a change of heart begins. Enkindled by
Christ's love they begin to have a positive ``feeling'' for return to the
Church. Through God's grace Small Ecclesial Faith Communities are the ``rich
soil'' that ``produce(s) a hundred or sixty or thirtyfold.'' (cf. Mt 13:8)
Lapsed Catholics are like ``the harvest [that] is abundant but the laborers are
few; so [as Jesus urged] ask the master of the harvest to send out laborers for
the harvest.'' (cf. Mt 9:37-38) To gather in the ``abundant harvest,'' laborers
for the harvest are needed. Thus, SEFCs are a ``must'' where laborers are
schooled both in personal holiness and for gathering the harvest.
RDS, page \#

Summary
Stage One of the CLCC process is summed up by this statement from Vatican II's
Decree on the Bishops' Pastoral Office in the Church, ``Catechetical training is
intended to make men's faith living, conscious and active through the light of
instruction. This instruction is based on Sacred Scripture, tradition, the
liturgy, the teaching authority and life of the Church.'' (CD 14) This statement
is directed to the bishops. However, various forms of ``catechetical training,''
as stated above, are the primary ministry of most priests with assistance from
laity.

Good preaching by clergy and formation of laity through catechesis given in
Small Ecclesial Faith Communities by, through and with competent laypersons
holds primacy of place in the process called Celebrating Life as a Catholic
Christian.

The first thing Christ did was call 12 Apostles and 72 disciples. After being
with him awhile he declared them ``fishers of men.'' He then sent them into the
villages and towns around Lake Genesareth  to announce the Good News about his
coming as Messiah and to cure the sick. People who accepted their words become
followers of Jesus.

Enacting in a parish what Jesus did in Galilee consists of establishing Small
Ecclesial Faith Communities (SEFCs) modeled on Jesus' small faith community of
the apostles and disciples. SEFCs are composed of regular church-going
Catholics. The focus of SEFCs is to provide, through the CLCC process, formation
of the members in Catholic doctrine and Church teachings for growth in personal
holiness. The CLCC faith formation process can be summed up in the famous words
of Blessed John Paul II, ``be evangelized'' in order ``to evangelize.''

``To evangelize'' includes seeking out lapsed and alienated Catholics,
befriending them and winning them over to join one of the SEFCs in the
parish. Moved by the Holy Spirit and by association with exemplary practicing
Catholics, they begin their journey back to full union with Christ and his
Church.
RDS, page \#

% ------------------------------------------------------------------------------

\section{Reflection and Discussion}
\lxRDFa{property=stage-rd-content,resource={manual}}

Note:  for obtaining the most value from the CLCC faith formation process,
please read again Putting the CLCC Process to Work found in the introductory
pages of this manual; also look again at Order for Conducting Formation
Sessions.  The aids and insights given in this Reflection and Discussion section
correspond to paragraphs in the main text of Stage One.


Introduction
1. The first paragraphs contain two issues for reflection and formation. Choose
one issue, one that appeals to you most. Reflect on it prayerfully under the
guidance of the Holy Spirit. Be open to his insights. He is helping you make
your first tiny step in the CLCC faith formation process.

Also open your manual to the centerfold and study the graphic. Note the five
stages of Jesus' ministry in Galilee (left side) and the parallel five-stage
structure of the CLCC process for SEFCs (right side). Reflect how you and your
fellow parishioners can become more Christlike by carrying out this process.

2. In his public ministry Jesus had two goals: to give his people a new way of
life, and do missionary outreach to the lapsed and ``lost.'' For us, his new way
of life means first answering the ``call to holiness.'' In silence, reflect on
his ``two greatest commandments'' and his ``call'' to us to become holy. Share
your insights with your SEFC group.

Missionary outreach can be done in different ways. In the CLCC process, outreach
means contacting and helping an estimated 45 million lapsed and fallen-away
Catholics return to Christ and his Church. On a scale of 1 to 10 how important
and urgent do you rate this outreach in your family and parish. Reflect: how do
we help the lapsed return to Christ and his Church?

A. What Jesus Did in Galilee
1. Choosing apostles and disciples was a major step in Jesus' great divine plan
of founding his Church on earth. Reflect prayerfully on the two reasons Jesus
chose these men as stated in the main text. Reflect how you fit into Jesus'
divine plan and how you can benefit spiritually from it. Be open to the insights
the Holy Spirit gives you.

2. To Celebrate Life as a Catholic Christian means to imitate Christ in the ways
he has shown us. Grade yourself. Choose one of the ways given here and reflect
how you live it: well, sort of, need improvement, news to me.

3. What you know of the CLCC process to this point, do you feel it can
``empower'' you and other Catholics to live more authentic Christian lives in
our current anti-Christian culture? Reflect on the area of empowerment you feel
will bring about the most change in your life. Reread the paragraph. Share.

4. Call of the 12 Apostles and 72 Disciples
The two texts taken from St. Matthew's gospel have action words such as
``walking,'' ``casting,'' ``followed'' and so forth. Find a few more. Reflect
how these actions can apply to your relationship with Jesus. St. Luke's gospel
states, Jesus ``prayed'' a whole night before choosing the 12. This action shows
us the importance and necessity of praying prior to making major choices or
decisions.

Before Jesus' time and even in his day people were filled with great expectation
of the promised Messiah. When Jesus came and called certain men, they were drawn
to his extraordinary persona. They may also have been dissatisfied with the work
they were doing, e.g. fishing, tax collecting. God respects the free will with
which he endowed every human being. What teaching does Jesus' ``call'' suggest
to you about vocations that people respond to for different states of life:
single, married, religious, priestly? Each individual is free to accept or
reject God's invitation. He forces no one. Choose up to three ``action words.''
Prayerfully reflect on them. Call on the Holy Spirit for light. What meaning do
they have for you? Share.

5. Mission of the Apostles and Disciples
The two texts from St. Luke's gospel are Jesus' mission statements for the 12
apostles and 72 disciples. Choose an action that Jesus gave them to do and
reflect on it prayerfully. Ask yourself, how can it apply to me? Listen to the
Holy Spirit. Try to think of some comparable action you can do. Share.

6. Catechism of the Catholic Church
The first of the two paragraphs taken from the Catechism of the Catholic Church
(CCC 75), centers on Jesus' command to his apostles to preach the gospel, the
source of truth and moral discipline. The second paragraph (CCC 1575) states how
Jesus protects his flock through the apostles and their successors, bishops, who
are pastors. Reflect on how these statements complement each other. Tell Jesus
how safe you feel believing it is he who cares for us, his Church, through these
responsible shepherds. Share your insights.

B. Enacting in the Parish What Jesus Did in Galilee

Note: Under ``Leadership of the Priest'' are four paragraphs. They describe the
priest's all-essential role in getting the CLCC process underway in the
parish. The priest, whether resident and full-time, part-time, itinerant or a
once a month priest to his people, should take on the process with great
enthusiasm especially since the laity in SEFCs share the greater responsibility
for carrying it out. Like Christ, the priest should take the lead in choosing
and calling ``apostles and disciples,'' lay leaders from among the
parishioners. He helps them launch the process. To assure its progress and
growth he oversees it as best he can and preaches about it as often as he deems
it necessary to make the ``fire burn brightly.''

1. Leadership of the Priest
The priest's initiative for successful launching of the CLCC process is
imperative. What about parishes with only one priest who is overwhelmed with
work? Do you feel lay leaders should enact it counting on only limited oversight
from the priest? What about getting help from a deacon? Recall the two purposes
of the CLCC process! Neither are ``take it or leave it'' options.  The call to
holiness is mandated by Jesus: ``Be perfect as your heavenly Father is perfect''
(cf. Mt 5:48; also CCC 2013) and the Dogmatic Constitution of the Church, Light
of the Nations (Lumen gentium, LG, 39-42). The second purpose is missionary, to
reach out to and help save the wayward sheep (Lk 15:4-6). The two purposes are
complimentary to each other. Choose a question above. Call on the Holy Spirit
for light. Reflect in quiet. Share your inmost thoughts.

2. Teachings of Vatican Council II
The two teachings are power-packed mandates deriving from Christ and the
apostles. Ask yourself: how do these mandates pertain to me? How obligated am I
to do what they say?  Humbly be open to the Holy Spirit. Be inspired. Share your
insights with your SEFC.

3. Small Ecclesial Faith Communities (SEFCs)
This paragraph contains phrases and words that show how Jesus formed his first
community. They are ``community,'' ``faith,'' ``small,'' ``by design,''
``co-workers,'' and ``model.'' Asking the Holy Spirit for light, what meaning do
these words have for you, for your growth in holiness and mission outreach to
lapsed fellow Catholics? Reflect quietly in prayer. Share.

4. Models of Leadership in the Old Testament
The models given here are ordinary people whom God called to do extraordinary
things in his great plan of salvation. Select one model. Open your Bible to the
texts that describe the model you chose. In prayer to the Holy Spirit reflect on
this person specially chosen to work with God on behalf of his people. Reflect
on how you, also special to God, are to carry out his plan of salvation within
your family, parish and work place through the CLCC process.

5. Small Ecclesial Faith Communities, Why Are They Needed?
In the three paragraphs given here, identify the reason for establishing SEFCs,
the faith problems that many Catholics have, and by what means they can be
brought back to the faith. Reflect.

If the cited documents are accessible, you may want to read what Popes Paul VI
and Blessed John Paul II have said about small faith communities. Both Pontiffs
gave them strong endorsement. Marginal Catholics has reached astronomic
numbers. An estimated 45 million Catholics do not go to Church regularly or not
at all. They need to be reached and won back to the Church. Try to recall some
of the ways you know or have heard to win wayward Catholics back into the
Church. What you know of the CLCC and SEFCs, do you feel this non-intimidating,
sensitive formation process will in the long run be the most effective
conversion method? Prayerfully reflect. Share.

6. Blessed Trinity, Archetype of Small Ecclesial Faith Communities
The following sentence taken from the Catechism (CCC 1878) identifies the
source, character and the links that connect Small Ecclesial Faith Communities:
``There is a certain resemblance between the unity of the three divine persons
[source, God] and the fraternity (character of SEFCs) that men are to establish
among themselves in truth and love [connecting links].''

What should this sentence mean to you in the CLCC formation process? (a) The
Blessed Trinity, the source, is certainly a ``community'' of Father, Son and
Holy Spirit, the source of life, redemption and agent of salvation; (b) the
character of SEFCs consists of Christian brotherhood, neighborly fellowship and
solidarity among its members; (c) the connecting links between the Trinity and
SEFCs are truth and love. Christ said, ``I am the truth'' (Jn 14:6) and God is
defined as ``love'' (1 Jn 4:8). Clearly SEFCs have their foundation in and flow
from the Triune God.

Central to this reflection is the human person, also the close-knit society of
persons, the communal family, each having a spiritual life. Their rights and
obligations, their spiritual values and spiritual achievements must be
respected, as taught by Blessed John XXIII (cf. MM 60; PT 36) Having supreme
authority over all creation and man, God also set up civil authority which must
be ``morally licit'' to further the common good and maintain good order.

Prayerfully reflect on these fundamental truths. Choose one, preferably the
first one. Make the connection between the Blessed Trinity and the CLCC
process. See how the CLCC process flows radically from the Blessed
Trinity. Invoke the Holy Spirit. Share your insights.

7. SEFCs, Front Doors for Lapsed Catholics to Come Home to the Church
The four paragraphs here give the reasons for the existence (raison d'etre) of
SEFCs, what they do, and why they are valuable in pastoral ministry. Lapsed
Catholics have reached astronomic numbers. An estimated 45 million Catholics do
not go to Church regularly in the United States. They need to be reached and won
back to the Church. Try to recall some of the ways you know or have heard to win
wayward Catholics back into the Church. What you know of the CLCC to this point,
do you feel this non-intimidating, formative process is an effective apostolate
for helping people grow in holiness and reaching out to help save the lapsed?
In quiet time, pray and reflect on at least one or two of these key words and
phrases: ``welcome mat,'' ``re-introduce,'' ``change of heart,''
``association,'' ``good example,'' ``witness,'' ``grace of the Holy Spirit.''
What impact do these key words or phrases have on you? Pray for light. Share
your God-given thoughts.

8. Priest's Mission in the CLCC Process is ``To Beget''
``To beget'' means the ``duty'' to propagate the kingdom of God for one's own
salvation and the salvation of others. Jesus' mission was ``to beget.'' He
called apostles first, who chose successors (bishops), who then selected priests
to share their care of the Church, and priests in turn seek out lay persons
(workers), so that the people of God are adequately cared for. The pastoral term
for ``duties'' is shared responsibility. The laity's duty is to share
responsibility, helping the priest by exercising primarily their prophetic role
of teaching and formation of God's people in the faith. The CLCC is in fact a
process of ``begetting.'' Participants train in personal holiness, a quality
that helps them deliver lapsed Catholics into the arms of Mother Church.  Pray
to the Holy Spirit to get the impact of these truths. Reflect in quiet. Share
with your SEFC how you see the CLCC as a ``begetter'' of higher levels of
holiness and of conversion in your parish.

9. Formation in SEFCs Consists of Two Functions
They are:  (a) growth in personal holiness of the group's members, and (b)
reaching out to help lapsed Catholics to return to the Church.  The two
functions are inter-linked to ensure good results. Reflect why and how formation
in personal holiness is a vital prerequisite for doing successful outreach
evangelization to those in need of conversion. Pray to the Holy Spirit for his
gifts. For productive evangelization, all seven gifts of the Holy Spirit need to
be active within you: wisdom, council, knowledge, understanding, fortitude,
piety and fear of the Lord (CCC 1831). Invoke the Lord to energize the Holy
Spirit's ``gifts'' in you. Share the light you receive.

10. Catechetical Instruction Holds Primacy of Place in SEFCs
Small Ecclesial Faith Communites are invaluable for parishes. Parishioners in
great part are in need of a booster shot of truth. Too often, they are strangers
to each other, having time only for a quick ``Hi'' as they rush to the parking
lot after Sunday Mass. SEFCs have two purposes. First to invite, draw and
convince ``Hi Catholics'' that SEFCs are also for them. Systematic formation in
the faith is necessary for them, and for their own growth in personal holiness
perhaps a stunning surprise. The second purpose is that members of SEFCs do not
stop with ``interior formation.'' They are expected to follow up with an
``exterior action,'' to reach out to and help save lapsed Catholics return to
the Church. This was Jesus' mode of operation in Galilee. He called apostles,
formed them in his truths by his preaching (interior formation), then sent them
out two by two (exterior action) to towns and villages to teach, preach, convert
and drive out demons. The CLCC process is modeled on Jesus' method of
evangelizing. See the centerfold, page \#. Invoke the Holy Spirit for
light. Reflect in quiet on the remarkable parallel between Jesus' form of
evangelizing in Galilee and the CLCC's corresponding form of evangelizing in the
parish. Share the insights the Holy Spirit gives you.

Summary
In Stage One, the main text of the CLCC process is summed up by a single
authoritative statement from Vatican Council II's Decree on the Bishops'
Pastoral Office in the Church (Christus Dominus, CD, 14). Please reread it. In
this one sentence do you see the CLCC in outline? Do you feel this sentence
gives a glimpse of the CLCC process, highlighting its main points? Prayerfully
reflect. Share.

% ------------------------------------------------------------------------------

\chapter{Stage 2.\ Formation of Disciples and Apostles}

\section*{} \lxRDFa{property=stage-main-content,resource={manual}}

Introduction
1. Jesus' public ministry began when he called 12 men whom he named
Apostles. From a growing number of followers he ``appointed'' 72 disciples. He
brought them together. They formed community, heard his teachings and were
``insiders'' to his new way of life. They were eyewitnesses of his miracles and
his power to expel demons and the way he went about accomplishing his messianic
mission. He had two purposes in mind: to form them to live in community with
him, and after a period of formation in his way of life he sent them to share
the good news first in local Galilee and after Pentecost into the whole world.

2. Stage Two centers on formation of small faith communities in parishes, the
way Christ formed community with his followers. Members are equipped with the
good news of the Catholic faith, and at the same time motivated to reach out to
lapsed Catholics, non-regular church-goers, and those who say they are
``Catholic'' but do not in fact live either the faith or by the Church's
disciplines.

3. Jesus' community of apostles was called ``the Twelve.'' Celebrating Life as a
Catholic Christian (CLCC) begins with a group of 12 lay person known for their
commitment to the Church and gifted with some leadership qualities. They are
called the Matrix 12. They are also called the Lay Leadership group. For the
most part they are ``called'' by the pastor. Their responsibilities are to learn
the CLCC process and implement it exactly as given in this manual, all the while
receiving whole-hearted support from the pastor.

4. Similar to Christ's ``appointed'' 72 disciples are groups of people who form
Small Ecclesial Faith Communities (SEFCs). Like the Matrix 12, they are
parishioners in search of a deeper spiritual life, better knowledge of the
Catholic faith and who feel drawn to do some kind of meaningful apostolic work,
such as the CLCC faith formation process. Simultaneously, they receive
preparation for and motivation to do this outreach ministry, the aim being to
help non-practicing Catholics return fully to Christ and his Church.

There is no limit to the number of SEFCs that can be established in a parish. In
fact, the more the better! Take note that the Matrix 12 and parishioners in
SEFCs should work together and not stand isolated from one another. Rather, they
should work together collaboratively to achieve their goal: to grow holy and
build community in the parish.
RDS, page \#

A. Community, Communio and Communion
1. Meaning of these terms
Community is a group or class of people who bond together because they have a
common interest, a cause or special purpose, e.g. neighborhood watch, the
family, home-schooling parents, a college community, a parish, a diocese, a
civic organization.

Communio (Latin) is the action of the Holy Spirit, the divine, invisible agent
of heaven and earth bonding God's people into the Body of Christ, making it the
Church through the sacraments of Baptism, Eucharist, Confirmation and the gift
and virtue of faith.
Communion is the visible reality, people spiritually bonded who pursue the
reason for their existence, namely to become holy and reach out to others in
missionary activity.
RDS, page \#

2. Joseph Cardinal Ratzinger, later Pope Benedict XVI, while Prefect of the
Vatican's Congregation of the Doctrine of the Faith has stated that the concept
of communion in the Church (ecclesiology) has a ``double dimension: the vertical
(communion with God) and the horizontal (communion among men).''

The ``vertical'' is the up and down dimension, God intervening in peoples' lives
through an abundance of graces which empower them to live their lives in the
pursuit of holiness. This dimension includes God's constant providence,
protection and guidance.

The ``horizontal'' dimension is oneness, ``communio,'' which people have with
each other because they believe the same revealed truths. They belong to the
same Church, the one, true Church of Jesus Christ, worship in the same approved
liturgy and share the same goals and values. This kind of ``communio'' among
church-going Catholics is what makes Small Ecclesial Faith Communities, and
makes them work.
RDS, page \#

3. St. Paul gave us the analogy that the Church is the Body of Christ (1 Cor
12:12), an expression that is well known to us. It means that it is the
Eucharist that transforms us into one body. Thus, the Body of Christ is at once
a visible communion (e.g. the assembly of worshipers at Sunday Mass, SEFCs and
other groups) and an invisible communion when believers share the one, true
faith, which resides in their minds, souls and hearts. Invisible communion binds
together the pilgrim church on earth, the heavenly church already united with
Christ in heaven, and the church in purgatory waiting to be fully purified. The
Body of Christ is the union of the three churches. In the creeds we profess them
as the communion of saints. (cf. CCC 954)
In addition to the infinite sacrifice of Christ on the Cross and his
intercession for us before the Father, the saints and in particular the Virgin
Mary, Mother of God, also pray for us. Thus the communion of saints is a
communion of love and caring between the Church in heaven and the pilgrim church
on earth. (cf. CCC 956)
RDS, page \#

4. On October 22, 2003, Pope John Paul II celebrated the 25th year of his
pontificate. On that day he bestowed on each of the thirty newly designated
Cardinals (one was held in pectore, in the Holy Father's heart) the cardinal's
ring, a sign of dignity, solicitude and solid communion with the See of
Peter. Communion! In his homily he described, ``the Church as one People of God
rooted in the multiplicity of nations.'' He was speaking of the wondrous
phenomenon of communio in the Catholic Church.
RDS, page \#

B. The Matrix 12
1. The 12 lay leaders of the CLCC process, the Matrix, like the Apostles are
intimately linked with Mary, Mother of the Church. (cf. CCC 963) Mary's
relationship with the Apostles and all humanity was forever confirmed at the
foot of the Cross (``Woman, behold your son''), and then sealed on Pentecost
when, with the Apostles, Mary received an outpouring of the Holy Spirit's
abundant graces, spiritual gifts and charisms. From that day till the end of
time Mary plays a key role in the apostles' ministry and in all apostolates of
the Church. (cf. Acts 2)
RDS, page \#

2. Matrix comes from the Latin word, mater, meaning mother. Matrix is a
generative term. It means that something originates in or develops from
something that already is. Mary was a virgin living in Nazareth. The angel
announced to her that she would ``bear a son'' by the overshadowing of the Holy
Spirit, and he would be ``called Son of the Most High.'' Mary became
``incarnate'' with the Son of God (cf. Lk 1). Mary conceived this Child by the
Holy Spirit. Having developed in her womb, after nine months she delivered the
Christ-Child in Bethlehem. The Second Person of the Blessed Trinity, Jesus, who
existed from all eternity at a given time took on humanity in the womb of Mary
as God-Man. Thus the term matrix fittingly applies to Mary.
RDS, page \#

3. How does the term matrix apply to the Matrix 12? Through weekly, systematic
formation in official Church teachings, they become ``incarnate'' with
Christ. By receiving the sacraments, praying and sharing, they develop further
in Christ. Formation takes place in their Small Ecclesial Faith
Community. ``Delivery'' means they begin setting up other Small Ecclesial Faith
Communities in the parish, starting with one, then a second, a third and adding
more till all the parishioners belong to a SEFC. Parishioners who enter a SEFC
and participate in the CLCC process can be likened to conception in the
womb. There they are ``formed'' in holiness and prepared for mission. After a
time they ``deliver.'' They reach out to lapsed Catholics in the parish.
RDS, page \#

4. Mary, Mother of the Church
The Church bestowed on Mary the title, Mother of the Church, because she is
eminently deserving of it through her deep faith (Annunciation, cf. CCC 965,
969), her martyr-like suffering (Calvary, CCC 964), her unbounded prayer and
charity as Mother of the Risen Christ (Pentecost, CCC 965) and her singular
glory (Assumption, CCC 966).
RDS, page \#

5. We Catholics believe that Mary is the Mother of the Church and is therefore
our Mother, the Mother of every human being. How did this phenomenon, so
fortunate for us, come about? According to the Catechism of the Catholic
Church``... she is our Mother in the order of grace.'' The following are the
Catechism's explanations for this phenomenon.

CCC 967
By her complete adherence to the Father's will, to her Son's redemptive work,
and to every prompting of the Holy Spirit, the Virgin Mary is the Church's model
of faith and charity. Thus she is a ``pre-eminent and wholly unique member of
the Church''; in fact, she is the ``exemplary realization'' (typus = type) of
the Church.

CCC 968
Her role in relation to the Church and to all humanity goes still further. ``In
a wholly singular way she cooperated by her obedience, faith, hope and burning
charity in the Savior's work of restoring supernatural life to souls. For this
she is a mother to us in the order of grace.''

RDS, page \#

6. The phenomenon is summarized in this way. ``This motherhood of Mary in the
order of grace continues uninterruptedly from the consent which she loyally gave
at the Annunciation and which she sustained without wavering beneath the cross
until the eternal fulfillment of all the elect. Taken up to heaven she did not
lay aside this saving office, but by her manifold intercession continues to
bring us the gifts of eternal salvation. Therefore the Blessed Virgin Mary is
invoked by the Church under the titles of Advocate, Helper, Benefactress and
Mediatrix.'' (cf. CCC 969)
RDS, page \#

7. Not only is Mary Mother of the Church, but the Church is a mother like
Mary. ``The Church in her apostolic work, (for example, carrying out the CLCC
process in a parish or diocese), also rightly looks to her who brought forth
Christ ... so that through the Church Christ may be born and grow in the hearts
of the faithful also. The Virgin Mary in her own life lived an example of that
maternal love by which all should be fittingly animated who cooperate in the
apostolic mission of the Church on behalf of the rebirth of men.'' (LG 65)

The paragraph above taken from the Vatican Council II's Dogmatic Constitution on
the Church is charged with words such as ``brought forth,'' ``born,''
``rebirth.'' Other life-giving words such as revitalization, recycling, renewal
and renaissance are contained in the arch-symbolic term aggiornamento, which
means, ``beginning anew.'' This is the term Blessed Pope John XXIII used in his
address at the opening of the Second Vatican Council on Saturday, October 20,
1962 in Rome. Thus, when a pastor and his Matrix 12 initiate the CLCC process in
the parish they are ``beginning'' its ``rebirth,'' aggiornamento.
RDS, page \#

C. Small Ecclesial Faith Communities (SEFC)
1. As the Matrix 12 provide leadership and initiative to the CLCC process
through their Small Ecclesial Faith Community, so too parishioners who
establish, join and are trained in other Small Ecclesial Faith Communities give
tremendous muscle to the faith-quality in themselves and to the People of
God. They take their clue from the way Jesus sized up the religious condition of
the people in his day. They were starving for the good news of the Messiah. Thus
he exclaimed, ``The harvest is abundant but the laborers are few; so ask the
master of the harvest to send out laborers for the harvest.'' (Lk 10:2)
RDS, page \#

2. What then is the main purpose of both the Matrix 12 and members of SEFCs?
They become laborers of the abundant harvest. Using the words of John Paul II,
Matrix leaders and members of SEFCs need to be evangelized [in order] to
evangelize. (emphasis added) (cf. CL 51)
``To be evangelized'' means that the Matrix 12 and members of SEFCs are formed
in the faith more deeply, and become better acquainted with the authoritative
teachings of the Church as given in official Church documents (cf. Part II of
this book) and the Catechism of the Catholic Church. The first aim is to grow in
holiness, a fundamental prerequisite for doing effective evangelization.
RDS, page \#

3. ``To evangelize'' means that SEFC members and the Matrix 12 carry out the
second purpose of the CLCC process. They reach out to lapsed, non-church going
Catholics, also former Catholics in an effort to win them over and join a
SEFC. Through association with the CLCC members in a SEFC, the lapsed are
introduced again to the faith through magisterial and catechetical
formation. All-important for them is to be open to the gentle prodding of the
Holy Spirit. He will give the person a good feeling toward the Church and a
desire to ``come home.''
RDS, page \#

4. The following statement ought to be not just read, but pondered, perhaps
memorized and indelibly marked in one's store of CLCC convictions. Pope John
Paul II uttered the statement on Sunday, October 19, 2003 when he beatified in
Rome the world-renowned missionary-evangelizing nun, Mother Teresa of Calcutta,
before a massive crowd of 300,000 people. Speaking of her extraordinary
holiness, he said categorically and concisely, ``Sanctity is the secret of
evangelization and of all genuine spiritual renewal.''
RDS, page \#

5. The reason for the CLCC's dual purposes is that church leaders, clergy and
laity, especially in large, long established parishes tend to feed the sheep in
the fold but fail, due to being understaffed, to go after the strays and
lost. Who can deny that sometimes not a few nominal Catholics live within the
parish boundaries, some only a few blocks away, who do not know Christ and know
little about the Church.

If these lapsed Catholics had been catechized in early adolescence, not a few
have forgotten what they were taught including basic prayers such as the Our
Father and Hail Mary. Often enough they may be the sorry products of shabby,
meager and superficial religious education. Now as adults they ignore the
Church, have lost interest in it, feel it has nothing to offer them, and some
may harbor anger toward it. These kinds of Catholics number in the
millions. They need ``disciples,'' ``laborers'' from SEFCs to find and visit
them, speak to them kindly, provide a setting for the Holy Spirit to touch their
hearts, and begin the slow process of gathering them in an abundant harvest.
RDS, page \#

6. Why ``Ecclesial'' Communities?
Small communities called ``base communities'' (communidades de base) had their
origin largely in South America in the past century. In a sense they are the
predecessors of faith communities in the United States. The nature and focus of
faith communities have gone through many ups and downs: some because their
purpose was not clearly defined, others for being top heavy with unworkable
internal structures, some lacked the missionary spirit, and there were other reasons.

Pope Paul VI explains clearly why small communities, if they are to be regarded
as true faith communities, must have an ``ecclesial'' connection, that is, must
be identifiable as spiritual and religious. They must be distinguished for their
``worship, deepening of faith, fraternal charity, prayer [and] contact with
pastors.'' (emphasis added) He said that communities that claim to ``remain
within the unity of the Church,'' but in fact are ``hostile to the hierarchy''
(is) a misuse of terms. Then definitively the Holy Father states, ``groups
which come together within the Church in order to unite themselves to the the
Church and to cause the Church to grow'' will be ``a hope for the Universal
Church.''  (cf. EN 58)

Small faith communities of various kinds, however, do exist in many parishes in
the United States. For participants they have considerable social, educational,
religious and pragmatic value. Here in the United States due to the severe
breakdown of social interchange and decline of neighborliness many people are
``too busy'' to talk to each other, such as married couples, parents and
children, relatives and friends. But other factors enter in. The materialistic
and secular culture in which we live makes small faith communities a virtual
necessity for survival and preservation of personal sanity. In them people
experience humanness, their dignity as persons is respected, and their religious
faith is not downgraded but strengthened.

These are priceless qualities of soul that kindred spirits can and do share with
each other. Small faith communities are refreshing havens where people enjoy
sincere, not false fellowship. Being with loving, trustful persons they can
disclose and discuss personal, family, marital, social, and religious concerns,
including political issues. Most importantly members of authentic faith
communities will experience the warm congeniality of communal prayer (people
praying together) and not feel embarrassed.

Members give good example to each other, and to fellow Christians who are
newcomers to the group. They also remind each other about receiving the
sacrament of reconciliation, that peace-giving, healing, uplifting encounter
with the merciful Lord.
RDS, page \#

7. The Popes' Teachings on ``Ecclesial'' Communities
In his encyclical letter, Mission of the Redeemer (Redemptoris Missio, RMiss),
Pope John Paul II states forcefully that small faith communities must be
``ecclesial.'' ``Ecclesial base communities,'' he writes, must be ``good centers
for Christian formation and missionary outreach.'' A papal one-liner, it could
be said, that describes the CLCC process! These ``groups of Christians,'' he
continues, ``... come together for prayer, Scripture reading, catechesis and
discussion of human and ecclesial problems with a view to common commitment.''
He adds reassuringly that such communities ``are a sign of vitality within the
Church, an instrument of formation and evangelization and a solid starting point
for a new society based on a 'civilization of love.''' (RMiss 51) What a
blessing that the CLCC process has been able to appropriate these Church
teachings!
RDS, page \#

8. Why the emphasis on ``ecclesial?'' Not only are parishes well served by
having many functioning small faith communities, but they are best served when
the communities are strongly knitted to the parish and committed to advancing
the goals of the parish mission statement, especially through the CLCC faith
formation process. Union with the parish heightens ``communion'' within the
parish. Members of SEFCs help raise the level of holiness among other parish
members.

By their formation SEFC members commit to do the most challenging apostolate in
the parish, reaching out to lapsed, non-church going Catholics and helping them
return to Christ and his Church. What the Holy Father is calling for is this: he
wants Small Ecclesial Faith Communities to ``organize the parish community, to
which they always remain united ... become a leaven of Christian life, of care
for the poor and neglected, and of commitment to the transformation of
society.'' (cf. RMiss 51)
RDS, page \#

9. Members of SEFCs, he goes on to say, ``play an active role in the task of
evangelization and initial proclamation of the Gospel and become a source of new
ministries.'' ``Every community, if it is to be Christian, must be founded on
Christ and live in him as it listens to the words of God, focuses its prayer on
the Eucharist, lives in a communion marked by oneness of heart and soul, and
shares according to the needs of its members.'' (cf. RMiss 51) (cf. Acts
2:42-47)
RDS, page \#

10. Following are the words of Pope Paul VI on ecclesial communities taken from
his apostolic exhortation Evangelii Nuntiandi, ``Evangelization in the Modern
World,'' EN, 58. ``Every community (example, a CLCC SEFC) must live in union
with the particular and universal Church, in heartfelt communion with the
church's pastors and the magisterium, with a commitment to missionary outreach
and without yielding to isolationism or ideological exploitation.''

Pope John Paul II said, ``if they (Small Ecclesial Faith Communities) truly live
in unity with the Church, a true expression of communion and a great means for
the construction of a more profound communion, they are thus cause for great
hope for the life of the Church.'' (emphasis added) (cf. RMiss 51)
RDS, page \#

D. The Laity's Role in the Mission of the Church
1. ``The layperson's apostolate in the Church derives from his Christian
vocation (call), and his activity has been fruitful from the beginning of the
Church.'' These are centuries-old truths clearly affirmed in Vatican Council
II's Apostolicam Auctuositatem, Decree on the Apostolate of the Laity. (AA 1)

Persons who join Small Ecclesial Faith Communities should not only rejoice that
they are called to a Christian vocation but also that they are called to a
Christian apostolate. ``In the Church,'' states the Decree above, ``there is a
diversity of service but a unity of purpose.'' ``The laity too share in the
priestly, prophetic and kingly office of Christ and therefore have their role to
play in the mission of the People of God in the Church and in the world.''
``They exercise a genuine apostolate by bringing the gospel and holiness to men
... perfecting the temporal sphere of things through the spirit of the gospel.''
(AA 2)
RDS, page \#

2. Right and Duty
``The laity derive their right and duty with respect to the apostolate from
their union with Christ the Head.'' The sacraments of baptism, confirmation and
``especially the most holy Eucharist communicate and nourish that charity which
is the soul of the apostolate.''

``The apostolate is carried on through faith, hope and charity,'' gifts received
from the Holy Spirit. ``For the exercise of this apostolate, the Holy Spirit
gives to the faithful special gifts as well, and to the individual.'' St. Peter
exhorts, ``according to the gift that each has received, administer it to one
another, and become good stewards of the manifold grace of God.'' (1 Pet 4:10)
(AA 3)

Having received ``these various charisms and gifts ... each believer [has] the
right and duty to use them in the Church and in the world for the good of
mankind and for the up-building of the Church.'' (AA 3)
RDS, page \#

3. Success of the lay apostolate
``The success of the lay apostolate depends upon the laity living in union with
Christ.'' Intimate union with Christ in the Church ``is nourished by spiritual
aids which are common to all the faithful, especially active participation in
the sacred liturgy.'' The most perfect example of this type of spiritual and
apostolic life is the most Blessed Virgin Mary, Queen of Apostles. (AA 4)
RDS, page \#

4. Goals
``The mission of the Church is not only to bring to men the message and the
grace of Christ, but also to penetrate and perfect the temporal sphere with the
spirit of the gospel.'' ``The laity, therefore, exercise their apostolate both
in the Church and in the world, in both spiritual and temporal orders
... although distinct, are so connected in the one plan of God that he himself
intends Christ to appropriate the whole universe into a new creation.''
(emphasis added) (AA 5; also cf. 6,7,8)
RDS, page \#

5. Methods of the Apostolate
This part of Stage Two centers on the importance and value of ``associations,''
a term applicable to Small Ecclesial Faith Communities. The Decree states, ``The
laity can engage in their apostolic activity either as an individual or as
members of various groups or associations.'' (AA 15) From this statement one can
conclude that SEFCs are ``associations.''

The Matrix 12 and SEFCs are indeed associations. ``For this reason the faithful
should exercise their apostolate by way of a united effort ... in their family
communities, and in their parishes and dioceses.'' ``The group apostolate is
highly important because the apostolate is best implemented through joint
action.''

Moreover, ``associations established to carry on the apostolate in common
sustain their members, form them for the apostolate, and rightly organize and
regulate their apostolic work so that much better results can be expected than
if each member were to act on his own.'' (AA 18)
The general aims of associations are ``to evangelize and sanctify ... to infuse
a Christian spirit in the temporal order.'' and to ``bear witness to Christ in a
particular way through works of mercy and charity.'' (AA 19)

The particular aims of laypersons who join associations such as the Matrix 12
and SEFCs are to strive ``to make the gospel known and men holy,'' and to
``contribute the benefit of their experience to the running of these
organizations.'' The contribution is really willingness to do the pastoral work
of carrying out a specific action plan. Carrying out a specific pastoral action,
such as Celebrating Life as a Catholic Christian, is vintage ``Catholic
Action,'' one of its finest manifestations. (cf. AA 20)
RDS, page \#

6. Pastoral Activity of Lay Persons
Carrying out the CLCC process by the Matrix 12 and members of SEFCs is a
singular lay activity that provides ``a variety of opportunities for apostolic
activity,'' chief of which is the need of outreach to lapsed Catholics.

Today, women ``have an ever more active share in the whole life of society
... (and) participate more widely in various fields of the Church's
apostolate.'' (AA 9) ``As sharers in the role of Christ the Priest, the Prophet
and King,'' the laity's ``activity is so necessary within church communities
that without it the apostolate of pastors is generally unable to achieve its
full effectiveness.'' (AA 10) (cf. Acts 18:18,26; also cf. Rom 16:1-16, names of
lay men and women who helped and were closely associated with St. Paul.)

``The laity with the right apostolic attitude supply what is lacking to their
brethren, and refresh the spirit of pastors and of the rest of the faithful.''
(cf. 1 Cor 16:17-18) Note well the next statement regarding the CLCC's second
goal, lay Catholics reaching out to lapsed Catholics: ``They lead to the Church
people who perhaps are far removed from it.'' (emphasis added) (AA 10) What more
explicit words are needed to declare that the second purpose of the Matrix 12
and SEFCs is the high priority apostolic activity of outreach to millions of
lapsed, non-church going Catholics with the view of helping them return to the
faith and living it again?
RDS, page \#

7. ``The laity should accustom themselves to working in the parish in close
union with their priests, bringing to the church community their own and the
world's problems ... concerning human salvation. As far as possible, the laity
ought to collaborate energetically in every apostolic and missionary undertaking
sponsored by their local parish.'' ``They should above all make missionary
activity their own by giving material and even personal assistance, for it is a
duty and honor for Christians to return a part of the good things they receive
from him.'' (AA 10)
RDS, page \#

8. Preservation of Good Order
``In the Church there are many apostolic undertakings which are established by
the free choice of the laity.'' However, ``no project may claim the
name. 'Catholic' unless it has obtained the consent of lawful Church
authority.'' (AA 24) ``Bishops, pastors of parishes, and other priests
... should keep in mind that the right and duty to exercise the apostolate is
common to all the faithful, both clergy and laity, and that the laity also have
their proper roles in building up the Church.''

For this reason priests ``should work fraternally with the laity in and for the
Church and take special care of the lay persons engaged in apostolic works.''
Priests ``should devote themselves to nourishing the spiritual life and
apostolic mentality in Catholic societies entrusted to them.'' ``Through
continuous dialogue with the laity, priests should carefully search for forms
which make apostolic activity fruitful.'' (AA 25) The CLCC process is certainly
one form of apostolic activity, which can be and is ``fruitful'' when carried
out correctly and faithfully.
RDS, page \#

9. Formation for the Apostolate
The Decree on the Apostolate of the Laity speaks of ``maximum effectiveness''
which can be attained ``only through diversified and thorough formation.'' (AA
28) The apostolic process presented in this manual, Celebrating Life as a
Catholic Christian, is indeed a ``diversified and thorough formation'' of the
Matrix 12 and members of SEFCs. The formation is at once spiritual, doctrinal
and missionary. All laypersons in the parish are cordially invited, in fact,
urged to participate in the CLCC process by becoming members of a SEFC.
RDS, page \#

10. ``Apostolic formation'' of the laity has also a ``distinctively secular
quality and its own form of spirituality,'' that is, it differs from the
formation of religious and clergy. Based on his ``natural abilities ... the lay
person should learn to advance the mission of Christ and the Church by basing
his life on belief in the divine mystery of creation and redemption, and being
sensitive to the movement of the Holy Spirit who gives life to the People of God
and who impels all men to love the Father as well as the world and mankind in
him. This formation should be deemed the basis and condition for every
successful apostolate.'' (AA 29)
RDS, page \#

11. ``Spiritual formation'' should include ``solid doctrinal instruction in
theology, ethics and philosophy.'' (AA 29) But ``formation for the apostolate
cannot consist in merely theoretical instruction; from the very beginning of
their formation laity should gradually and prudently learn how to view, judge
and improve themselves and others through action, thereby entering into
energetic service of the Church.'' (emphasis added) (AA 29)

The ``laity'' are non-ordained persons. Their ``service'' to the Church is
service rendered to the People of God through different apostolates and
ministries. One such unique service and apostolate is this CLCC process.

For the reasons given above, great attention must be given to the study of and
reflection on Vatican II documents, the magisterial teachings of the Church and
pronouncements of her Supreme Pontiffs, many of which are presented in each of
the five stages of this process.

Part I of this manual, as is obvious, is laden with informative, explanatory and
affirming quotations from official church teachings. These quotations are solid
assurance that the CLCC process is eminently orthodox. Part II consists of
select paragraphs, excerpts and key passages for spiritual and apostolic
formation taken also from conciliar, magisterial and papal teachings. They have
been chosen for their high relevancy for Catholics striving to live
counter-cultural lives within a culture of hedonism, materialism, irreligion and
growing atheism.
RDS, page \#

% ------------------------------------------------------------------------------

\section{Reflection and Discussion}
\lxRDFa{property=stage-rd-content,resource={manual}}

Note:  newcomers should read Putting the CLCC Process to Work and then Order for
Conducting Formation Sessions, found in the front of this manual. Following the
Order is most essential for good and proper formation in the CLCC process. Like
all the stages, Stage Two has corresponding paragraphs for Reflection,
Discussion and Sharing.

Introduction
The introduction of Stage Two is a summary of Stage One and a review of the many
implications which Christ's ``call'' to the 12 apostles and 72 disciples has for
us in our day. The work of Stage One is to enact in the parish what Jesus did in
Galilee: he gave formation in doctrine and designation of mission. Turn to the
centerfold of this book, page \#. Study the parallel between the five stages of
Jesus' public ministry and the five stages of the CLCC process.

Enactment of the process in a parish begins with the priest (the Christ
figure). He selects a core lay leadership group called the Matrix (a figure of
the 12 apostles); then follows establishment of Small Ecclesial Faith
Communities (figures of the 72 disciples).  As a member of your SEFC, what truth
stood out in Stage One that helped you?

A. Community, Communio and Communion
Stage 2 has to do with Formation. When the apostles through their own free will
chose to answer Jesus' call to follow him, they had no idea what they were
getting into, except that they were attracted to Jesus, liked what they heard
him say and apparently were dissatisfied with their occupations.  The ``12'' and
``72'' became a group of men who had something in common: their desire to follow
Christ.

1. The lay leadership group, the Matrix 12, and parishioners in Small Ecclesial
Faith Communities are people who bond together. With light from the Holy Spirit
they reflect on the three ``Cs'' and what they mean to each person. Ask
yourself, in which ``C'' am I? Share your insights.

2. Cardinal Ratzinger tells us definitively that communion has a ``double
dimension,'' the vertical and horizontal. Read the explanation of the two
dimensions and reflect on them prayerfully. Ask yourself, am I part of both
dimensions as I should be? Share your thoughts.

3. St. Paul gave us his famous analogy of the Universal Church as the Body of
Christ. Understand and reflect on its ``visible communion'' and its ``invisible
communion.'' Praise God that you are part of this privileged communion, now in
this life and later in eternity. Share.

4. On October 22, 2003 John Paul II celebrated his 25th year as Pope and Pastor
of the Universal Church. On that day he bestowed upon each of the 30 new
Cardinals (one was held in pectore, in the Holy Father's heart) the cardinal's
ring, a sign of dignity, solicitude and solid communion with the ``See of
Peter''. In his homily he described, ``the Church as one People of God rooted in
the multiplicity of nations.'' This is a profound and beautiful description of
the Church. Where is ``communio'' in it? Reflect how you fit into this
communion. Share.

B. The Matrix 12
1. The assertion made here, that the Blessed Virgin Mary is intimately linked
with the Apostles and all the apostolates of the Church, requires making an act
of faith. Reread the paragraph. Guided by the Holy Spirit, how strongly do you
believe this great truth?

2. The divine action of the Incarnation is without doubt the world's most
stupendous mystery: God coming to earth to dwell in human form and who was born
of a Virgin Mother. Reflect how Mary is a matrix. Share your insights.

3. ``Incarnate'' is the word the Church uses for biological term ``gestation,''
the carrying of offspring developing in the uterus to term. In this case the Son
of God was carried and developed in the womb of the Virgin Mary. The Matrix 12,
the lay leadership group, begin their formation by holding weekly sessions in
the CLCC process in their Small Ecclesial Faith Community. After a short time
they begin to ``deliver'', that is, bring forth other SEFCs in the rest of the
parish. Reflect prayerfully on the action of ``incarnating'' the parish with
many SEFCs, doing ``communio'' for community building. This is ``communio,''
community building par excellence.

4. Mary, Mother of the Church
Mary is indeed eminently deserving of the title, Mother of the Church. Though
believed by the Church for centuries, it was Pope Paul VI who officially
conferred this title, Mother of the Church on the Blessed Virgin Mary in his
allocution at the end of the third session of Vatican Council II on November 21,
1964. Reflect quietly on the four major events in Mary's life stated in the main
text. Choose one event, reflect on it and share the insights the Holy Spirit
gives you with your SEFC.

5. Mary is not only Mother of the Church but is also our Mother, each individual
believer's Mother in the order of grace. Find in the next two short paragraphs
at least two major theological reasons that make Mary our Mother. Reflect on
them and share your insights.

6. That Mary is our Mother ``in the order of grace'' is indeed an extraordinary
truth. She didn't ``lay aside this saving office'' after she was assumed into
heaven. Rather from her exalted place in heaven she continues uninterruptedly
``to bring us the gifts of eternal salvation.'' This flood of graces flows from
her ``fiat'' in Nazareth to be the Mother of God. Reflect on the flood of graces
you receive from Christ through Mary, on the love you have for her in prayer and
how much you love her under the sweet title: Mother. Share a personal
experience.

7. Not only is Mary mother of the Church, but the Church is a mother like
Mary. Read the explanatory paragraphs. As Mary brought forth Jesus to the world
in Bethlehem, so the Church through her many birth-giving apostolic works brings
forth Christ to its members, to families, parishes, dioceses and the Universal
Church. Reflect on Celebrating Life as a Catholic Christian, a premier apostolic
process that brings a ``beginning'' of ``rebirth'' of faith and mission to the
People of God. Do we not speak of Holy ``Mother'' Church?  Mother Church is
God's ``birthing'' organism in the world. Share your insights.

C. Small Ecclesial Faith Communities (SEFCs)
1. The first statement under C.1. presents a huge challenge to the Matrix 12 and
to SEFCs. God's people were starving for the good news of the Messiah. Jesus
said, ``The harvest is abundant but the laborers are few.'' Reflect on how the
CLCC process does the ``harvest'' and how people in SEFCs become the ``laborers
for the harvest.'' Be open to the Holy Spirit. Share.

2. Laborers need ``to be evangelized,'' equipped with better and deeper
knowledge of the faith and increase in personal holiness. Receiving a doctrinal
upgrade in the faith should help each member in a SEFC to grow in
holiness. Reflect how well you feel you are ``equipped'' to evangelize. Be
honest. Share your insights, your joy, hope and anticipation.

3. The second goal of the CLCC process is ``to evangelize.'' The specific target
is lapsed, non-church going Catholics who are in critical need of evangelization
for conversion. Reflect on your responsibility to help reach out to these kinds
of Catholics. Pray to the Holy Spirit for light and grace, courage and
strength. Share.

4. On October 18, 2003, Pope John Paul II concluded the congress of prelates in
Rome with 149 cardinals, seven Eastern Catholic patriarchs, and 109 presidents
of episcopal conferences in attendance. Two days before on October 16th, the
College of Cardinals honored the Holy Father on his 25th year as Pope. In his
talk to them he made a strong plea for holiness of life, and love of the poor
and weak. In his talk on Sunday, October 19, before the massive crowd squeezed
into St. Peter's square for the beatification of Mother Teresa of Calcutta, he
recalled her concise statement and proclaimed it: ``Sanctity is the secret of
evangelization and of all genuine spiritual renewal.''  Let this statement ring
long in our minds and hearts. It demands our most serious reflection on the
depth of wisdom it contains, what we are called to become and do. Share the
inspirations the Holy Spirit gives you.

5. Would you agree that many parishes are busy feeding the sheep and lambs, that
is, regular church-going Catholics, but unintentionally neglect going out
specifically in search of perhaps hundreds of stray sheep and wounded lambs? Do
you agree that the mission of SEFCs, the ``disciples'' and ``laborers'', should
be to go out and find the lost and lapsed Catholics in the parish? In quiet
reflection ask the Holy Spirit to help you do this ``harvest.''

6. Why ``Ecclesial'' Communities?
The first paragraph in the main text is background material stating that small
communities originated mostly in South America. Today, small communities of
every imaginable kind are formed but are neither of ``faith'' nor ``ecclesial.''
Although only a few now the future of SEFCs is promising. The parish is a large
ecclesial faith community. Celebrating Life as a Catholic Christian is a process
which promotes ``small'' ecclesial faith communities within the parish.  Members
share the same values and strive to attain the same goal, Heaven, but with
strong focus on growing in personal love of Christ and reaching out to and
helping lapsed, non-practicing Catholics return to the faith.

Many Catholics in parishes, though persons of faith, feel alone and unconnected
with other Catholics, like blood relatives feeling like strangers at a family
reunion. Many Catholics are shy and hesitate to be joiners, yet long for
Christian camaraderie. Like the lapsed, they too need to be contacted,
befriended, given recognition and invited into an SEFC where they will
experience social welcome and begin their journey with others toward higher
levels of holiness and reaching out to the lapsed. For understanding of this
purpose of the CLCC process, pray to the Holy Spirit. Reflect on this
teaching. Ask yourself, is a small ecclesial faith community a ``must'' for me?

7. The Popes' Teachings on ``Ecclesial'' Communities
The Popes are strong on the establishment of small faith communities in the
Church. They insist that they be solidly religious and culturally productive to
the joiners and to those whom they aim to serve. Members must be strongly
focused on growing in authentic Catholic truth. They must attached to the
parish, thereby contributing to its overall apostolic mission. Small Faith
Communities are ``ecclesial'' when they are connected with and attached to the
parish. Reflect on your SEFC. Do you feel it is ``a sign of vitality'' in your
parish? Is it advancing a ``civilization of love''? Share.

8. Blessed John Paul II states what he expects small faith communities to do in
a parish. Five things! Identify as many as you can. Reflect on the immense good
you do for yourself and others by belonging to an SEFC. Share your vision.

9. In this quotation taken from his encyclical, Mission of the Redeemer, Pope
John Paul II states explicitly what elements make a faith community visibly
Christian and ecclesial. He gives at least six elements. Chose one that is most
meaningful to you. Reflect on how you and your SEFC friends can, by doing them,
``play an active role in the task of evangelization.'' Share.

10. Reflect prayerfully on the two paragraphs in the main text. Pope Paul VI
sees SEFCs not as parochial only, but inserts them into the global picture of
the Universal Church. Remember your study on ``communio'' (A.1.) and Communion
of Saints (A.3.)? Invoke the Holy Spirit and connect the two thoughts. Now read
and reflect on John Paul II's vision of SEFCs as a ``great hope for the life of
the Church.'' Share.

D. The Laity's Role in the Mission of the Church
1. The Christian Vocation
The Decree on the Apostolate of the Laity speaks of the Christian vocation as a
``call'' not only to be a disciple of Christ but also to play a role in the
Church's apostolate, namely ``bringing the gospel and holiness to men'' and
``perfecting the temporal sphere of things through the spirit of the gospel.''
In quiet prayer to the Holy Spirit, ask yourself what should all this mean to
me? Share your inmost thoughts.

2. Right and Duty
That the laity have a ``right and duty with respect to the apostolate'' sounds
like upbeat language. Note the sources from which the ``right and duty'' come:
Christ, the three sacraments of Baptism, Confirmation and the Holy Eucharist,
and special gifts to the faithful and to individuals  all through the Holy
Spirit. St. Peter calls them ``manifold.'' Check the gifts you've received. Give
humble thanks to the Lord. Praise him! Share.

3. Success of the lay apostolate
Charisms and gifts not only give the laity the ``right and duty'' to use them
for the good of mankind and the up-building of the Church, but also assures
success in their apostolates, such as the CLCC process. Note the conditions for
success: union with Christ, spiritual aids, the sacred liturgy and Mary, who is
the highest exemplar of holiness and apostolic life. Reflect on your formation
in these areas thus far. Share.


4. Goals, Methods and Pastoral Activity of Lay Persons
Consider the dual mission of the Church. It functions in both the spiritual and
temporal spheres. It makes men holy in the Church and at the same time strives
to implant Christ in the secular city. In quiet time, examine yourself, ask, do
I consciously try to bring Christ into the Public Square: where I work, at the
mall, while driving, grocery-shopping, eating out, at the gas station, bank,
hairdresser, golf course, etc.?

5. Methods of the Apostolate
Vatican II's Decree on the Apostolate of the Laity gives considerable attention
to the importance, need, advantages and activity of what it calls associations,
which in fact are Small Ecclesial Faith Communities.

Read the four paragraphs from the main text. In official Church language they
spell out what SEFCs (associations) are and do: ``best implemented through joint
action,'' (``united effort''), ``sustain members,'' ``form them,'' ``organize
and regulate apostolic work,'' ``evangelize and sanctify,'' ``infuse a Christian
spirit in the temporal order,'' ``make the gospel known and men holy,'' and
``contribute their experience to running the organization.'' Reflect on these
defining phrases drawn from official Church teaching, which, in fact, define the
CLCC process. Try to see how all these elements combine into Celebrating Life as
a Catholic Christian? Praise God! Share your insights.

6. Pastoral Activity of Lay Persons
Vatican Council II's Decree on the Apostolate of the Laity goes further. Women,
it says, ``have an ever more active share in the whole of society [and] in
various fields of the Church's apostolate.'' It says that pastors need the
laity, women and men, to make their apostolate more effective. Take special note
of this statement: ``They [the laity] lead to the Church people who perhaps are
far removed from it.'' Clearly, church authority is saying that an imperative
pastoral activity is for laypersons to reach out to and help lapsed Catholics
return to the Church. In fact, this is the prime exterior action of lay persons
in the CLCC process.

7. This paragraph, it can be said respectfully, is really a ``pep talk'' the
Decree gives to the laity. They should work closely with their priests,
collaborate with real energy and commitment in parish undertakings (such as the
CLCC process), and consider it ``a duty and honor'' to give back a part of
themselves to the Lord. Let the Spirit enlighten you on the immense good you do
and will do by participating in the CLCC process. Share with your group some
blessing you have already received through the process.

8. Preservation of Good Order
Examine the wisdom contained in this paragraph. The CLCC process is indeed
Catholic. It invites, in fact pleads, with you, the laity, to take up your
``proper roles in building up the Church.'' Remember, however, that your
formation becomes authentic and your activity fruitful in the CLCC process only
when it has ``the consent of lawful Church authority,'' bishops, pastors and
other priests. In this way you grow in holiness and help souls to
salvation. Reflect on the CLCC process. It has the Imprimatur (approval) of
lawful Church authority.

9. Formation for the Apostolate
Does what you know of the CLCC process to this point provide ``diversified and
thorough formation'' to you and your SEFC for achieving ``maximum
effectiveness'' to attain its goals: greater holiness in each person and
outreach to lapsed Catholics? Call on the Holy Spirit to help you express your
feelings. Share.

10. Apostolic formation of laypersons differs from formation of religious and
clergy. It has a ``distinctively secular quality,'' meaning, it is based on
``the divine mystery of creation and redemption and is sensitive to the movement
of the Holy Spirit who gives life to the People of God.'' In your quiet
reflection time, ask, do I really open myself to the Holy Spirit, to his wisdom,
his counsel, knowledge and understanding, to fortitude, piety and fear of the
Lord, all the gifts which the Spirit is eager to give you? Which gift do you
feel you need most? Which gift do you feel you have received at least in some
measure? Reflect prayerfully and share.

11. Spiritual formation consists of two parts. Laypersons in SEFCs are raised to
a new level of holiness through theological, scriptural and magisterial
teachings, as presented in this manual. The first aim of SEFC members is to grow
in personal sanctity. Simultaneously, they are formed to do apostolic works,
which is service to the Church. The service laypersons are expected to do
through the CLCC process is winning lapsed Catholics and those who have strayed
from the Church back into the Church.

In a nutshell this is what the two-part CLCC process is about: formation in the
faith and outreach to non-practicing Catholics. Reflect on the abundance of
graces and gifts God has already given you and has in store for you for greater
spiritual formation and service of his kingdom. Praise him and thank him from
your heart! Share.

% ------------------------------------------------------------------------------

\chapter{Stage 3.\ Stewardship of Our Gifts}

\section*{} \lxRDFa{property=stage-main-content,resource={manual}}

Note: Stage 3 has 52 numbered segments, they form the main text. In its
Reflection and Discussion section, beginning on page 67, there are corresponding
numbered segments designed to stimulate prayerful reflection, discussion and
sharing. When you have chosen a paragraph from the main text as the subject
matter for your formation session, please refer to the like number segment in
Reflection and Discussion as they are meant to be used together.

Introduction
1. Stage 3 of this process, Celebrating Life as a Catholic Christian (CLCC),
focuses on Stewardship of our Gifts. Fundamental to stewardship is that
recognition be given the Lord who pours out his gifts on his creatures in
inestimable abundance. Added to the gifts given mankind is yet another gift:
``dominion,'' understood as sovereignty over our gifts, a sovereignty that is
not absolute but has limits. (cf. CCC 373,2415)
When gifts are freely given certain obligations fall on the one who receives
them: the duty to acknowledge them as coming from God and not ourselves, manage
them wisely and use them in a manner pleasing to God. Thus, for the gifts we
receive, certain religious and moral obligations are imposed: that we
acknowledge God as the Supreme Giver of all gifts and in prayer give him humble
and daily thanks.

Stewardship of our Gifts implies more
2. Because God's gifts are so many and varied and come in a continuous flow, we
tend to take many for granted, for example the everyday gifts of the air we
breathe, the water we use, the sunshine that lights our day. There are also
personal gifts that are unique to us and no one else.
These we often accept without even fleeting awareness of where they came from or
who gave them. These are the gifts of life, health, and capacity to think,
understand, read, study and retain, to find a job, provide food and have a house
to call home. Ultimately the chief agent of all gifts, those given in abundance
or in lesser supply, is the Lord, who is the just distributor of all gifts. He
gives them to each person not equally but according to his infinite wisdom for
each person's ultimate well being. (cf. Mt 25:14-30, also Mt 20:9-16)

3. Pre-eminent of all the categories of God's gifts are the supernatural gifts
of faith, hope and charity, faith being the foundational gift without which the
practice of religion is not possible. (cf. Heb 11:6) The gift of faith enables
us to ``give free assent to the whole truth that God has revealed.'' (cf. CCC
150). As a gift received from God, faith is a supernatural virtue that he
infuses into a person's soul, mind and heart. (cf. CCC 153) It is cultivated
through prayer, and prayer obtains more grace from God who helps the person
strive to love Him, serve his neighbor and advance in holiness.

God is good and generous beyond words
4. The abundance of his gifts and graces should prompt us not only to give him
prayerful thanks but also to realize that God gives his gifts freely without our
deserving or meriting them. God's gratuitous giving should lead to a humble
realization urging us to give him more praise and thanksgiving.

5. Celebrating Life as a Catholic Christian is therefore an aggregate of many
elements. They belong to and come under the encompassing concept of Total
Stewardship that is the bedrock of the CLCC process. This phrase is in fact an
arch-symbol. It expresses the three pervasive components of the bible. It is not
limited to buzzwords and even reaches beyond the popular concepts of ``time,
talent and treasure,'' used for gaining certain pastoral ends.
Rather the phrase, Total Stewardship, expresses the three inseparably linked
components of the written word of God: evangelization, discipleship and
service. As responsible stewards of our gifts, we are to use these gifts for the
building of the kingdom of God on earth. This use of our gifts entails spreading
the good news (evangelization), becoming a committed follower of Christ
(discipleship), and aiding our neighbor as far as we can in his every need
(service). Together the three components add up to not partial but Total
Stewardship. Harmonizing the three components in one's life is to Celebrate Life
as a Catholic Christian.
6. Lastly, all of God's gifts, though freely given, have a string attached to
them: accountability. An accounting must be given to the Lord how the gifts
received have been used, how they were increased, what good works they produced,
and how the receiver advanced toward the Giver in love and holiness or regressed
from him through irresponsible behavior. (cf. Mt 25.31-46)

A. God's Word, First of His Greatest Gifts
1. God's gifts are situated in several strata of importance, lesser gifts rising
to the highest of gifts, the word of God, where God the Father reveals himself
through his Son, Jesus, the God-Man, who is the WORD. (Jn 1:1) The apostles and
disciples were indeed systematically formed in the great truths, which Jesus
proclaimed about his Father and himself.
But not infrequently the truths were so radically new the disciples failed to
understand what Jesus was talking about. Many times Jesus had to take them aside
and in private give them a fuller explanation of his teaching. Despite the
clarification, often the disciples still failed to get the full impact of his
message.

2. The episode when Jesus spoke ``privately'' to four of his apostles took place
on the Mount of Olives. (cf. Mk 13:3) He told them of the destruction of the
temple in Jerusalem, its meaning for the ``end time,'' the coming persecution,
the great tribulation, the coming of the Son of man, the lesson of the fig tree,
and the need for watchfulness. (cf. Mk 13:1-37) Jesus spoke of all these future
happenings publicly but in parables. For other significant passages where Jesus
explains difficult teachings to his disciples privately, see Mt 17:19; Mk 9:28;
Lk 10:23.

3. Among Jesus' many parables one stands out. In it he shows that the most
important and greatest of God's gifts is the word of his Father, which he came
to earth to proclaim and to die for on the cross. He describes the gift in
dramatic images. It is the parable of the sower and the seed. (Mk 4:1-20)
Perhaps nowhere in the New Testament does the Lord God announce his word more
explicitly and interpret it most precisely to the crowds than the parable of the
sower and the seed.


4. At the Sea of Galilee Jesus got into a boat, seated himself in it and staying
near the shore, began to teach a very large crowd gathered around him on
land. Mark's gospel states, ``he taught them at length in parables.'' Then he
startled them. He struck upon a parable that froze them with attention. ``Hear
this!'' he said. ``A sower went out to sow. And as he sowed, some seed fell on
the path, and the birds came and ate it up. Other seed fell on rocky ground
where it had little soil. It sprang up at once because the soil was not
deep. And when the sun rose, it was scorched and it withered for lack of
roots. Some seed fell among thorns, and the thorns grew up and choked it and it
produced no grain. And some seed fell on rich soil and produced fruit. It came
up and grew and yielded thirty, sixty and a hundredfold.'' He added, ``Whoever
has ears to hear ought to hear.'' (Mk 4:3-9)

5. Puzzled about the meaning of the parable, the Twelve along with others
questioned Jesus ``alone'' about the meaning of the parable. Jesus is the
sower. The seed is God's word, and God's word is Christ. ``In the beginning was
the Word, and the Word was with God, and the Word was God.'' (cf. Jn 1:1) Then
Jesus, the Word, describes in four descriptive analogies how his word would and
would not be received by those who hear it. Jesus explained:

Mark 4:15
Some fell ``on the path where the word was sown.'' ``As soon as they hear the
word, Satan comes at once and takes away the word sown in them.''
vv. 16-17
Some was ``sown on rocky ground.'' These are the ones ``when they hear the word,
receive it at once with joy. They have no root; they last only for a time. Then
when tribulation or persecution comes because of the word, they quickly fall
away.''
vv. 18-9
Seeds ``sown among thorns are another sort. They are the people who hear the
word, but worldly anxiety, the lure of riches, and the craving for other things
intrude and choke the word, and it bears no fruit.''
v. 20
But seeds ``sown on rich soil are the ones who hear the word and accept it and
bear fruit thirty and sixty and a hundredfold.''
6. The centrality of the word in God's great plan of salvation ought neither be
denied nor underestimated. This is why the Dogmatic Constitution on Divine
Revelation, Dei Verbum (henceforth cited as DV) states clearly that ``the word
of God, which is the power of God for salvation to everyone who has faith, is
set forth and displays its power in a most wonderful way in the writing of the
New Testament.'' (DV 17)
Moreover, the Catechism of the Catholic Church, CCC 124 teaches that these
writings ``hand on the ultimate truth of God's Revelation. Their central object
is Jesus Christ, God's incarnate Son: his acts, teachings, Passion and
glorification, and his Church's beginning under the Spirit's guidance.''
Therefore, the Christian faith is the religion of the ``Word'' of God, a word
that is as St. Bernard preached ``not a written and mute word, but the Word
which is incarnate and living.'' (CCC 108)

7. The Word is Necessary for Formation of Conscience
a. What is conscience?
Conscience is like an ADT alarm system built into a person's heart to sound
caution to avoid evil and encourage it to choose what is good.
b. Conscience can be likened to a restaurant menu
You have choices to select which food you feel is good to eat and turn down
foods that you don't want to eat.
c. Conscience is like a magnet
It draws you to truth, to God the Supreme Truth, to accept his Commandments as
the ultimate etiquette of proper conduct toward God and one's neighbor.
d. Conscience is also like a ``voice''
That is, a conduit through which God speaks to persons with gentle
authority. The prudent person keeps his conduit open to ``hear'' all the Lords
messages, which guide him to do good and avoid evil. (cf. CCC 1777)
e. Explaining Conscience
Because conscience is a voice in the inmost being of a person, its function is
to prod a person to make judgments, choices, to do good and avoid evil. These
are called ``moral'' judgments because conscience prods the person to choose
what is good and reject what is evil. Thus, the Church speaks of a moral
conscience.
f. Forming a Right Conscience
Conscience is formed primarily through education from infancy and childhood to
the end of adulthood. In fact a good, well-formed conscience seeks not only to
do good and avoid evil, but is constantly in search of truth. This means that it
``formulates its judgments according to reason, in conformity with true good
willed by the wisdom of the Creator.'' (cf. CCC 1783)
g. The education of conscience
``This is indispensable for human beings who are subjected to negative
influences and tempted by sin to prefer their own judgment and to reject
authoritative teaching.'' (CCC 1783) Some samples of negative influences:
tolerance, diversity, being non-judgmental, political correctness, ``I'm against
abortion but pro-choice,'' relativism (there are no absolutes), subjectivism (I
can do what I want if it feels good).
Education of conscience is more than listening to a half hour talk on conscience
in the parish hall during Lent. Educating our conscience consist of a continuous
search for what is true and right, and obeying it, ``a life-long task.'' (CCC
1784)
h. Erroneous Conscience
The opposite of a well-formed conscience is an erroneous conscience. This kind
of conscience makes bad judgments due to ignorance. Ignorance can be culpable,
that is blameworthy: the person is responsible for the wrong he commits despite
his ignorance. He is responsible because he refuses to learn the truth about
what is right or wrong, or rationalizes about it, or rejects the truth
outright. Thus the person is guilty of the sin he commits. On the other hand
ignorance can be invincible. This means the person, not knowing the nature of
his evil act, is not responsible for the guilt of his act even though the act
committed is sinful. Such a person has a serious obligation to learn the truth,
correct his moral error and cultivate a good and right moral
conscience. (cf. CCC 1790-1794)

8. Examination of Conscience
a. To prepare for the Sacrament of Reconciliation, it is necessary and useful to
examine one's conscience in the light of God's word. The best biblical passages
to use for an examination of conscience are the Ten Commandments, the moral
catechesis of the Gospels, especially the beatitudes, and St. Paul's
letters. (cf. Gal 5:19-21; also cf. CCC 1852) The Catechism of the Catholic
Church gives three lists of the 10 Commandments, the list in the Book of Exodus
20.2-17, the list given in the Book of Deuteronomy 5.6-21, and the traditional
list given in catechisms. It is highly recommended that you open your CCC to
these pages, 496-497.

b. To refresh your memory the Traditional Catechetical List of the Ten
Commandments is as follows:

I am the Lord your God, you shall not have strange gods before me.
You shall not take the name of the Lord your God in vain.
Remember to keep holy the Lord's Day.
Honor your father and your mother.
You shall not kill.
You shall not commit adultery.
You shall not steal.
You shall not bear false witness against your neighbor.
You shall not covet your neighbor's wife.
You shall not covet your neighbor's goods.

c. In the Sermon on the Mount Jesus taught his disciples and us the beatitudes
(cf. Mt 5:1-12). They are: Blessed are the poor in spirit, for theirs is the
kingdom of heaven. (3) Blessed are they who mourn, for they will be
comforted. (4) Blessed are the meek for they shall inherit the land. (5) Blessed
are they who hunger and thirst for righteousness, for they will be
satisfied. (6) Blessed are the merciful, for they shall be shown mercy. (7)
Blessed are the clean of heart, for they shall see God. (8) Blessed are the
peacemakers, for they shall be called children of God. (9) Blessed are they who
are persecuted for the sake of righteousness, for theirs is the kingdom of
heaven. (10) Blessed are they when they insult you and persecute you and utter
every kind of evil against you [falsely] because of me. (11) Rejoice and be
glad, for your reward will be great in heaven. Thus they persecuted the prophets
who were before you. (12) See also Lk 6:20-26.

d. For an outstanding layout of Duties of Christians, see St. Paul's letter to
the Romans chapters 12-15. For St. Paul's short treatment of Spiritual Gifts,
see 1 Corinthians chapters 12 and 13. In his letter to the Galatians chapter 5,
St. Paul presents his Exhortation to Christian Living, which is excellent for an
examination of conscience and private reflection for growth in holiness (see
especially verses 13-26). Lastly, in his letter to the Ephesians, chapters 4-6,
St. Paul speaks of the Church's Unity amid Diversity and what its conduct should
be to foster unity.

B. God's Word in the Liturgy
1. Sacramental celebration in the Church consists of various sacred signs and
symbols drawn from the revealed word of God, which are used to encounter Jesus
Christ in acts of worship.
Signs for the most part are physical
2. Examples are the Sign of the Cross and the sign of peace at Mass. Making the
Sign of the Cross is external language, which says we believe in, and have a
loving relationship with the three Divine Persons of the Blessed Trinity. The
sign of peace shows the fraternal communication we have with one another in
Christ. When the priest holds his hands outstretched over the bread and wine
immediately prior to the consecration this is a visible and physical sign of
calling the Holy Spirit to effect transubstantiation. (cf. CCC 1145-1146)

3. Symbols
Man is God's most exalted creature of all visible and animate creation. He is
exalted because he is made in the ``image'' of God. Thus, God speaks to him
through the material cosmos.
Light and darkness
Christ is the Light. (Jn 1:4-9; 8:12; 9:5; 12:46) Satan is Christ's archenemy
who is the agent of darkness. (Mk 4:15; Mt 4:10; Lk 10:18)
Wind and Fire
In these symbols the Holy Spirit came to the Apostles and Mary on Pentecost
Sunday. (Acts 2:2-4)

Water
Jesus' baptism in the Jordan River was a symbol of our baptism in Christ giving
us birth in the divine life of the Spirit. (CCC 694; Mt 3:16; Jn 3:5; Col 3:27)
Earth, our Exile, is the place of pilgrimage where the ``people of the poor''
await and prepare for Christ's coming through purification according to the Holy
Spirit. (cf. CCC 710, 716)
Fruits
Fruits are a symbol of the innumerable gifts of God in material creation and
spiritual re-creation. (cf. Num 18:12) Eating the forbidden fruit on tree of
knowledge of good and evil is the symbol of sin. (cf. Gen 2:17)
``First Fruits,'' the best of the harvest, were offered to the Creator in humble
thanksgiving for his bounty: sheaves of wheat, grapes from the vineyards, wine,
olives, figs, flocks and herds. (cf. CCC 1334-1335) The quantity of the produce
offered to the Creator was the tithe, a tenth part of the product. (cf. Deut 14:22-27)

``First Fruits'' and first-born have like connotations. In regard to the Lord's
incarnation, passion, death and Resurrection Paul writes, ``Now Christ has been
raised from the dead, the first fruits of those who have fallen asleep.'' (1 Cor
15:20) Again Paul states, Christ is ``the firstborn of all creation'' (Col
1:15), ``he is before all things ... he is the head of the body, the church. He
is the beginning, the firstborn from the dead, that in all things he himself
might be preeminent.'' (Col 1:15,17-18) Christ is ``the firstborn of the dead
and ruler of the kings of the earth.'' (Rev 1:5)

4. More Symbols
Washing and ``purifications''
Examples: Jesus washed the feet of the apostles. (Jn 13:8,14) The priest washes
his hands (``lavabo'') after he receives the prepared gifts brought to the
altar.
Anointing
This symbol is usually associated with the sacraments of Baptism, Confirmation,
Holy Orders and the healing Sacrament of the Sick. It signifies the presence and
power of the Holy Spirit coming to the person receiving the sacrament. (cf. CCC
695)
Bread and Wine
At the Last Supper Jesus took bread and wine, gave thanks, blessed them and
pronounced the words of consecration changing the bread into his Body and the
wine into his Blood. Jesus multiplied the loaves enough to feed five
thousand. This miracle pre-figures the great mystery of the Eucharist (Mt
14:13-21; 15:32-39; cf. CCC 1335)
Breaking of Bread
This action was part of the Jewish meal. During the Last Supper Jesus also
``broke the bread'' before distributing it to the apostles. (cf. CCC 1329) Jesus
must have done something significant in the breaking of the bread. On the Sunday
evening of Jesus' Resurrection the two disciples walking with Jesus to Emmaus
recognized him in the ``breaking of bread.'' (Lk 24:30) When celebrating the
memorial of Jesus' Passion and Resurrection, the new Christians referred to it
as ``the breaking of the bread.'' (Acts 2:42)

The Cup
Jesus' whole life from the Incarnation to his death on the Cross was an offering
of himself to the Father for men's sins. At the beginning of his passion in the
Garden of Gethsemani, Jesus had a vision of the mountain of sin that he would
atone for by his death. Peter thought he might protect Jesus from arrest, but
the Lord said to him, ``Put your sword into its scabbard. Shall I not drink the
cup which the Father has given me?'' (Jn' 18:11; CCC 607) In the words of
institution at Mass the priest says over the chalice filled with wine, ``This
cup which is poured out for you is the New Covenant in my blood.'' (Lk 22:19-20;
CCC 1365)
The Altar
The altar is a symbol of Christ himself. It is at once the ``altar of
sacrifice'' where Christ is offered to the Father, and the ``table of the
Lord,'' where the assemblies of the faithful come to be reconciled to the Lord
and to receive him as ``food'' from heaven. The Eucharistic liturgy expresses
the inseparable connection between the sacrifice of Christ and the communion of
the faithful with Christ. (cf. CCC 1383)

The Word
Finally, God's word is celebrated in the Liturgy of the Word at Mass. The
readings are taken from ``the book of the Word,'' the lectionary (cf. CCC 1154),
first from Old Testament prophets, followed by the Letters of the Apostles and
finally the Gospels of the four evangelists. These readings are generally
grouped according to cycles of three years. The homily comes next. It should be
an inspiring application of the readings to the spiritual and moral needs of the
particular assembly. On Sundays and other great feast days, the assembly then
makes a public profession of faith by reciting the Nicene Creed. The Prayer of
the Faithful, General Intercessions, conclude the Liturgy of the Word. These are
``supplications, prayers, intercessions and thanksgivings ... made for all ...''
(cf. 1 Tim 2:1-2; CCC 1349)

C. Stewardship of Our Three Baptismal Gifts
1. Among the 19,000 Catholic parishes in the United States listed in the
Official Catholic Directory, all parishes have at least some members involved in
their activities. Common roles for the laity include: the parish council,
greeters and ushers at the Sunday Masses, lectors and extraordinary ministers of
Holy Communion, teachers in the parish school, religious education instructors,
RCIA, food pantry, St. Vincent de Paul Society, liturgy team, youth group, bible
study groups, prayer groups, and CLCC Small Eccclesial Faith
Communities. Working with the priest, dedicated lay men and women and young
adults carry out these activities. They are persons of faith, prayer, blessed
with good will and the desire to serve God, his Church and the community.

2. However active the laity may be in parish ministries, often through no fault
of their own, they hunger for deeper spirituality and catechesis. This means
they need better knowledge of the faith, the meaning of the Creed, the need of
the Sacraments and the necessity of Prayer. Often they lack understanding of the
mystery of the Church, the Body of Christ and its structure. Many concede they
lack basic familiarity with Sacred Scripture, thus the growing interest in bible
study. The Catechism of the Catholic Church, though known to them, seems
overwhelming (a book of 904 pages), yet it is the official and orthodox source
of truth about Jesus, his teachings and his Church.
3. Learning and being able to talk to others confidently about God and his Son,
Jesus, is not just nice or useful but essential for living life as a Catholic
Christian. In the Catechism we read: ``The Old Testament attests that God is the
source of all truth. His Word [Jesus] is truth. His law is truth. Since God is
'true,' the members of his people are called to live in the truth.'' (CCC 2465)

4. ``In Jesus Christ, the whole of God's truth has been made manifest. 'Full of
grace and truth,' he came as the 'light of the world,' he is the truth.''
``Whoever believes in me may not remain in darkness.'' (Jn 12:46) To follow
Jesus is to live in ``the Spirit of truth.'' Jesus who was sent by the Father
leads ``into all the truth.'' (CCC 2466)

5. Moreover, those who live by the Truth who is Christ become ``by their baptism
the community of believers [who] are made sharers in the priestly, prophetic and
kingly functions of Christ.'' And they have ``their own part in the mission of
the whole Christian people with respect to the Church and the world.''
(cf. Dogmatic Constitution of the Church, Lumen Gentium, LG, 31)

6. Neither in holy orders nor being vowed religious, the vocation of the laity
``is to seek the kingdom of God by engaging in temporal affairs and by ordering
them according to the plan of God.'' The laity ``are called by God ... led by
the spirit of the Gospel ... to work for the sanctification of the world from
within, in the manner of leaven.'' (LG 31) It is the laity, the People of God,
who make up the Body of Christ under the one Head, the Lord Jesus. ``As living
members'' of this one Body, they are to ``expend all their energy for the growth
of the Church and its continuous sanctification.'' (cf. LG 33)

D. We are a Priestly People
What are the gifts?
What then are the three baptismal gifts, and how do we practice good stewardship
of them?


1. At the time of our baptism, we, the People of God, receive in faith a unique
gift. We share in the priesthood of Christ. We are given a ``priestly
vocation,'' a call to share in the priestly functions of Christ. ``Christ the
Lord, high priest taken from among men, has made this new people 'a kingdom of
priests to God, his Father.''' Thus, the baptized in Christ, the new ``high
priest,'' make up a kingdom of priests to God, our Father. (cf. CCC 784)

2. By baptism the People of God are ``regenerated,'' that is, original sin is
washed away. They are anointed with the Holy Spirit and become part of the Body
of Christ. In brief, they ``are consecrated to be a spiritual house and a holy
priesthood.'' By this consecration they become the ``living stones ... built
into a spiritual house of God to be a holy priesthood to offer spiritual
sacrifices acceptable to God through Jesus Christ.'' (1 Pet 2:5; cf. CCC 784)

3. ``Hence the laity, dedicated as they are to Christ and anointed by the Holy
Spirit, are marvelously called and prepared'' to produce ``even richer fruits of
the Spirit.'' ``For all their works, prayers and apostolic undertakings, family
and married life, daily work, relaxation of mind and body, if they are
accomplished in the Holy Spirit-indeed even the hardships of life if patiently
borne-all these become spiritual sacrifices acceptable to God through Jesus
Christ. In the celebration of the Eucharist these may most fittingly be offered
to the Father along with the body of the Lord. And so, worshiping everywhere by
their holy actions, the laity consecrates the world itself to God, everywhere
offering worship by the holiness of their lives.'' (CCC 901)

E. We are a Prophetic People
1. This term, prophetic, may suggest that baptized Catholics are like weather
forecasters who predict climate changes, but who preach doctrinal and
disciplinary truths of the Church. The term, however, has a very specific
meaning. At their baptism Christians receive the gift and role of being
spokespersons of God in ways proper to them. They are called by God to make his
Son known to others by sharing the good news about Jesus and his works, making
this sharing a part of the mission of their lives.

2. The term ``prophetic'' comes from the original notion of prophets in the Old
Testament. These men were special messengers chosen and gifted by God. He
inspired them with his word, and by the power of the Holy Spirit impelled them
to proclaim the Lord's word, sometimes under extreme suffering, to the people in
various periods of time prior to Jesus' coming on earth.

3. John the Baptizer, a relative of Jesus, was the last of the great Old
Testament prophets. (cf. Mk 11:32; Lk 20:6; CCC 523,719) He preached a baptism
of repentance (Mt 3:2) and announced the coming of the Messiah. (CCC 522, 555,
702) Of Jesus, John said, ``the one who is coming after me is mightier than I.''
(Mt 3:11).

4. Jesus was the last and the ``greatest of prophets.'' (cf. Mt 12:41ff; Lk
11:31ff) As such he was recognized by the Samaritan woman at the well, ``Sir, I
see you are a prophet.'' (Jn 4:19) On entering Jerusalem triumphantly the crowd
exclaimed, ``This is Jesus the prophet, from Nazareth in Galilee.'' (Mt 21:11)
When the blind man began to see, the Pharisees questioned him who gave him his
sight. Referring to Jesus, he said, ``He is a prophet.'' (Jn 9:17; also cf. Lk
1:67-76)

5. Inseparable from the role of prophet is the office of teacher. The four
gospel writers record numerous episodes where the disciples and people of Jesus'
day addressed Jesus as teacher or master. And rightly so! They saw and heard
Jesus teach the crowds the truths of God with authority on the mountain, in the
desert, in villages, in the synagogue, at the seashore, in people's private
homes, and in his 7 last words on the Cross. (cf. Mt 23:10; Mk 10:17; Lk 20:21;
cf. CCC 2605)

6. The ``Sacred Deposit'' of the faith consists of Sacred Scripture and
Tradition. It is the collection of teachings of the Apostles entrusted to the
Church's pastors for transmission to the faithful. (cf. CCC 84) Authentic
interpretation of the Word of God is entrusted to the living, teaching office of
the Church alone, the magisterium. (cf. CCC 86) ``This means that the task of
interpretation has been entrusted to the bishops in communion with the successor
of Peter, the Bishop of Rome.'' (CCC 85)

7. Keeping in mind the above material for background, we come now to see how the
laity participates in the prophetic office of Christ. ``Christ ... fulfills this
prophetic office, not only by the hierarchy ... but also by the laity. He
accordingly both establishes them as witnesses and provides them with the sense
of faith (sensus fidei) and the grace of the word.'' The great St. Thomas
Aquinas wrote, ``To teach in order to lead others to faith is the task of every
preacher and of each believer.'' (cf. CCC 904)

8. Thus, lay people fulfill their prophetic mission by evangelization, ``that
is, the proclamation of Christ by word and testimony of life.'' For lay people,
``this evangelization ... acquires a specific property and peculiar efficacy
because it is accomplished in the ordinary circumstances of the world.'' Witness
of life, however, is not the sole element in the apostolate; the true apostle is
on the lookout for occasions of announcing Christ by word, either to unbelievers
... or to the faithful. (CCC 905) Lay people who are capable and trained may
also collaborate in catechetical formation, in teaching the sacred sciences, and
in the use of the communications media. (CCC 906)

9. All the holy People of God, therefore, have a share in Christ's prophetic
office, ``above all in the super-natural sense of faith that belongs to the
whole People, lay and clergy, when it 'unfailingly adheres to this faith
... once for all delivered to the saints,' and when it deepens its understanding
and becomes Christ's witness in the midst of the world.'' (CCC 785)
In view of the several authoritative Church teachings quoted here, one could
imagine that the Council Fathers of Vatican II and the authors of the Catechism
of the Catholic Church envisioned a process like Celebrating Life as a Catholic
Christian, whose aim and purpose is to form baptized Catholics more deeply in
the faith, and with the help of God's grace influence lapsed Catholics to return
to the Church. This process was produced and is promoted by the National
Catholic Conference for Total Stewardship.

F. We are a Royal People
1. The last office, which Christ shares with us, is his royal (kingly)
office. How did Jesus come to royalty? Was he not a humble carpenter in
Nazareth? When Nathaniel who later was named Bartholomew was invited by Andrew
to meet Jesus, he commented jokingly ``can anything good come out of Nazareth?''
(Jn 1:46) Later when Jesus met Nathaniel knowing of his comment, he complemented
him his good-natured humor, ``Here is a true Israelite. There is no duplicity in
him.'' (Jn 1:47)

2. Slowly Jesus' ``royalty,'' really his divinity, began to surface. When he
taught in the synagogue in his hometown, Nazareth, his own people were
astonished at his wisdom and the ``mighty deeds he wrought by his hands.'' They
said, ``where did he get all this?'' ``Is he not the carpenter, the son of
Mary?'' ``And they took offense at him.'' (cf. Mk 6:2-3)

3. Authority is almost always part and parcel of royalty. The following are
sample texts affirming Jesus' royal authority. Speaking on the importance of
building one's house on rock rather than sand, the crowds were astonished, ``for
he taught as one having authority, and not as their scribes.'' (Mt 7:29) When
Jesus healed the paralytic in Capernaum and forgave his sins, the scribes were
discussing by what authority he was doing these things? Jesus knowing their
thoughts said, ``But that you may know that the Son of Man has authority to
forgives sins on earth-he said to the paralytic, I say to you, rise, pick up
your mat, and go home.'' (Mk 2:6-11; also cf. 11:28)

In his prayer at the Last Supper, Jesus attests to his universal
authority. ``Father, the hour has come. Give glory to your son, so that your son
may glorify you, just as you gave him authority over all people, so that he may
give eternal life to all you gave him.'' (Jn 17:1-2) These sample texts are
clearly royal in tone and content.

4. To share in Christ's kingly office is an amazing gift, which he gave us at
baptism. How do we know we share in Jesus' role as king? By his death and
resurrection Jesus draws all men to himself. In baptism we die with Christ and
rise with Christ; we die to sin and death and rise to holiness and eternal
life. Doing this Jesus ``made himself the servant of all, for he came 'not to be
served but to serve, and to give his life as a ransom for many.''' (Mt 20:28) In
this way Christ exercises his royal prerogative. He chose to be a servant
king. By this fact He is King and Lord of the universe. (cf. CCC 786)

5. We Christians share in Christ's kingship when we ``serve him'' especially
when we recognize him ``in the poor and suffering'' and see in them the poor and
suffering Redeemer King. ``To reign is to serve him.'' (LG 8) In this way the
People of God fulfills its royal dignity by living their vocation, which is ``to
serve Christ.'' (CCC 786)

6. Exercising a kingly role
How can the laity exercise their kingly role in the CLCC process? Start with
fundamentals.
a. Overcome ``the reign of sin'' in yourself
Do this by controlling your passions and allurement to evil and strive to live a
holy life. (cf. CCC 908)
b. Join forces with other like-minded lay persons
Christians should ``remedy institutions and conditions of the world [which] are
an inducement to sin,'' examples: media, books, education, malls, lewd and
pornographic TV shows and immoral Internet videos. Parents, grandparents,
baby-sitters, daycare attendants have a serious responsibility to monitor
children's consumption of TV, Internet, video games, electronic messaging,
smartphones and other popular media.
c. Impregnate culture and human works with a moral value
Combat the social suicide of artificial contraception, abortion, partial birth
abortion (consider our aging population), Internet pornography, rampant egoism,
individualism, divorce, re-marriage, cohabitation, abused and neglected
children, the active homosexual lifestyle, dishonesty, selfishness, perversion
of the judicial system, justice, murder, suicide, drugs. These undeniable
realities not only are hindering the practice of virtue but are destructive of
the fabric of a healthy society and its culture. As a kingly people Christians
must penetrate the culture and inculcate moral values. (cf. CCC 909)
d. Cooperate with pastors
``The laity can feel called, or in fact be called (example, to become a Matrix
member or join a Small Ecclesial Faith Community) to cooperate with their
pastors in the service of the ecclesial community, for the sake of its growth
and life'' (the CLCC process) ``This can be done through the exercise of
different kinds of ministries according to the grace and charisms which the Lord
has been pleased to bestow on them.'' (cf. CCC 910)
e. Distinguish one's Christian duties carefully
Christ's kingly people should ``distinguish carefully between the rights and
duties which they have as belonging to the Church, and those which fall to them
as members of the human society.'' For example Catholic politicians who say they
are pro-life but vote pro-choice explaining they cannot ``impose'' their
morality on others. This is a position which is contradictory and therefore
untenable. Although they try they cannot reconcile the contradiction. ``They are
to be guided by a Christian conscience, since no human activity, even of the
temporal order, (e.g. abortion) can be withdrawn from God's dominion.'' (CCC
912)

f. Serve as witnesses and instruments
``Thus, every person, through these gifts given to him (priestly, prophetic and
kingly offices), is at once the witness and the living instrument of the mission
of the Church itself `according to the measure of Christ's bestowal.''' (CCC 913)

G. Emphasize the laity's ``rights and duties'' in
  the Church's mission
1. All that has been said thus far in Stage 3 centers on the Lord's outpouring
of his gifts to his people. Therefore, because gifts have been received
obligations follow. They are: to be acknowledged with gratitude, managed and
used responsibly. And as receivers of the gifts we should be acutely aware that
as a priestly, prophetic and kingly people we have a ``right and duty'' to
practice good stewardship in our families and in the mission of the Church.

2. The many quotations cited above from the Dogmatic Constitution of the Church
and the Catechism of the Catholic Church attest mightily that the laity have
``rights and duties'' to participate in the life of Christ and in its vast
mission of evangelization. One area of the Church's apostolic mission, which may
not be swept under the rug because of commitment to other ministries, is the
shocking estimated 47 million U.S. Catholics who do not go to Mass
regularly. This reality suffers no delay to correct.
3. The National Catholic Conference for Total Stewardship feels its work is cut
out for it: to initiate in all parishes this formation and conversion process,
Celebrating Life as a Catholic Christian (CLCC), for the up building and
re-building of the kingdom of God on earth. It forms God's people more deeply in
the faith and equips them to reach out to lapsed and wayward Catholics with the
hope of bringing their brothers and sisters back into the arms of Mother Church
through Jesus, the Redeemer.

Stage 3 follows the same composition-format as given in Stages 1 and 2 and will
continue in Stages 4 and 5.

The format is simple:

What Jesus did in Galilee
He went to villages to proclaim the Good News first to his own people, the Jews.
Enacting in the parish what Jesus did in Galilee
The pastor preaches the good news of the CLCC process to his people at the
Sunday Masses. He recommends Lay Leaders, the Matrix 12, helps them organize
into a Small Ecclesial Faith Community, and they begin holding their weekly
spiritual formation sessions. Their job is also to begin establishing other
Small Ecclesial Faith Communities, showing parishioners who join them how to
hold their weekly spiritual formation sessions.

H. The Indispensability of Preaching
1. Spiritual formation is really catechesis, and in the CLCC process it's done
in two ways. The first is by ``preached catechesis.'' This means hearing and
learning about the CLCC process through the proclaimed word in Sunday
homilies. Preached catechesis also means ``getting a feeling'' of the CLCC
process from not one but many explanations given by the preachers, thus
impacting the people with the ``feeling.'' The second way is by ``shared
catechesis.'' This means giving and receiving spiritual formation within the
formation sessions conducted in the Small Ecclesial Faith Communities (SEFC).
2. Preaching the CLCC is vital in all 5 Stages of the process for achieving the
intended results. Fatal to the process is if the priest feels that one talk on
the process will do the job. It won't. It can't. The process is big and
ongoing. The priest needs to explain aspects of the process to his people, to
show his full support of the process, his encouragement to those involved and
his urging to those not yet involved. By showing his own involvement in the
process, though limited, he shows he wants and is pushing this spiritual
formation and conversion process one hundred percent. This assures his people he
wants the intended results to be attained, and knowing this the people will
respond.

3. Priests and deacons are kindly asked to preach the material as given in each
of the 5 Stages. Most recommended is that a single paragraph or passage,
quotation or text be selected and expanded upon, keeping in mind always to
instill clearer and deeper understanding of a particular aspect of the formation
process. One cannot say it all in one talk. Nor can we expect the people to
understand the meaning and purpose of the whole process in one talk, nor to
digest a plate full of explanation when a morsel will satisfy.

4. Ongoing preaching does not mean preaching on the CLCC every week but
frequently with prudent spacing. The rationale? The people need repeatedly to be
informed, motivated, supported and encouraged: the Matrix 12, all parishioners
in SEFCs and those not yet participating. After the five stages in Part I have
been preached, and all five stages carried out in weekly spiritual formation
sessions, then material in Part II is begun. Again it is shared within
SEFCs. Priests and deacons preach, as prudence dictates, the topics as given in
Part II. Some preachers may prefer to go directly to the primary sources to
select passages other than those given in Part II in support of their message;
this is commendable. But please stay on track of the CLCC process.

I. Preview of Part II, More Gifts
1. The documents listed in Part II, like those in Part I, are invaluable gifts
by which Paul VI, John Paul II and Benedict XVI, pontiffs of towering stature,
have enriched the Church. These ``gifts'' are encyclicals, apostolic
exhortations and apostolic letters that have immediate relevance for the
faithful in the present age. These documents have the weight of official
teachings of the Church's magisterium. The list also includes an Instruction on
``Authentic Liturgy'' and a Declaration ``On the Unicity and Salvific
Universality of Jesus Christ and the Church.'' These are pronouncements authored
and signed by high-level Vatican officials and approved by the Pope.
Sad to say most Catholics know nothing about these high-charged teachings. Many
have not even heard of them. This situation is not entirely their fault. They
have not been informed about them, nor has anyone told them of their usefulness
for updating their knowledge of Catholic doctrine, their need for formation of a
right conscience, for strengthening marriage and family life, for preparation
for various apostolates such as reshaping culture with Christian values, and for
personal and communal spiritual growth for Celebrating Life as Catholic
Christians.

Thus, the documents in Part II present a welcome and urgent catechesis for
weekly formation sessions in the parish, for CLCC leaders, for the people who
already belong to one of the Small Ecclesial Faith Communities, and for people
in the parish still undecided whether to upgrade their knowledge of the faith
and spiritual life due to various personal circumstances.

2. To get a preview of the titles of the documents in Part II, please turn to
Authoritative Sources. Added to this list are the following suggested Church
documents for study, reflection and further interior formation: Vatican Council
II documents (1962-1965), in particular Lumen Gentium, LG, the Dogmatic
Constitution of the Church, Apostolicam Actuositatem, AA, the Decree on the
Apostolate of the Laity,, and Sacrosanctum Concilium, SC, the Constitution on
the Sacred Liturgy SC, all quoted in Part I.
Other papal and magisterial teachings for formation are: Fides et Ratio, FR,
Faith and Reason, John Paul II, 1998, Ecclesia in America, EA, the Church in
America, John Paul II, 1999, Novo Millennio Ineunte, NMI, On Entering the New
Millennium, John Paul II, 2000, Ordinatio Sacerdotalis, OS, Apostolic Letter on
Ordination of Women, John Paul II, 1994,Dives Misericordia, DM, Rich in Mercy,
John Paul II, 1980, Rediscovering the Sacrament of Reconciliation, Holy Thursday
Letter to Priests, John Paul II, 2001, and other teachings for our time.

3. Clergy are again asked to preach faithfully the CLCC process as it unfolds in
the 5 Stages of Part I and continues in Part II. The laity will find the heart
of the 20 documents in the paragraphs and excerpts selected. The documents in
Authoritative Sources and those listed above are catechesis at its highest
level. Reading and studying these documents will give ever-greater incentive to
the people in SEFCs to grow in knowledge of the faith and in holiness, and want
to do the urgent apostolate of helping wayward Catholics return to the
Church. Also hearing the teachings preached re-enforces the lay leaders of the
CLCC process and parishioners already members of SEFCs to dig into their
formation sessions more deeply with passionate zeal and love of God and
neighbor.

Our fervent prayer is that as the process progresses, all parishioners, as far
as they are able, will become members of Small Ecclesial Faith Communities
intent upon Celebrating Life as Catholic Christians.

% ------------------------------------------------------------------------------

\section{Reflection and Discussion}
\lxRDFa{property=stage-rd-content,resource={manual}}

Newcomers should read: Putting the CLCC Process to Work found in the front of
this manual, and then Order for Conducting Formation Sessions. These
explanations and instructions will help you to gain the greatest good from the
manual's contents.

Note: Stage 3 has 52 numbered segments, they form the main text. In this
Reflection and Discussion section there are corresponding numbered segments
designed to stimulate prayerful reflection, discussion and sharing. When you
have chosen a paragraph from the main text as the subject matter for your
formation session, please refer to the like numbered segment in this section as
they should be used together.

Introduction
1. The issues stated in this paragraph center on the elements proper to the
concept of stewardship. Though generally known they deserve repeating for
reflection because they are recorded in various ways throughout God's revealed
word. The elements are: ``recognition'' of gifts received, ``dominion'' (loaned
not owned), ``manage them wisely,'' ``use them in a manner pleasing to God,''
``acknowledge God who is the Supreme Giver,'' ``give him humble, daily thanks.''
Ask yourself, do I do these things consciously and prayerfully?

2. Stewardship of our gifts! Reflect on the ``every day gifts'' you enjoy, and
the gifts that are personal to you. Spiritual formation based on the gift of
faith means that we acknowledge God as the ``chief agent of all gifts,'' that
out of his love and wisdom, God distributes to each person gifts special to him
or her for the present life and that lead to one's ultimate life in
eternity. Think how great and good God is to you!

3. Pre-eminent! Of God's many and varied gifts, pre-eminent are the supernatural
gifts of faith, hope and charity. Faith is the foundational gift, which makes
the practice of religion possible. Reflect: what a miserable person I would be
without the strength of my faith especially in times of stress and
trial. ``Lord, help my unbelief.'' (cf. Mk 9:24)

4. God is good and generous. Here we come to the great mystery of God's
gratuitous giving. What does this mean? It means that God gives us his gifts
freely, and we receive them without deserving them, meriting or earning
them. Example: We may say, ``I got this gratis,'' meaning you received a favor
without the giver expecting a return, payment or recompense. In no way can we
recompense God except by giving him humble and prayerful thanks and
praise. Giving God prayerful thanks daily is a key component of
stewardship. Reflect and share.

5. Celebrating Life as a Catholic Christian means practicing responsible
stewardship of the innumerable gifts we receive from God daily and through our
lifetime. These gifts include no less than the divinely revealed word of God,
the Holy Bible. It is composed of three inter-connected elements: evangelization
(the good news), discipleship (becoming a faithful follower of Christ) and
serving one's neighbor by sharing one's various gifts and resources with him
(service). The sum of these linked elements is called Total Stewardship. To
become a good steward of all of one's gifts is to Celebrate Life as a Catholic
Christian.

6. Lastly, to the gifts given and received a string is attached,
accountability. Ponder God's expectations of you. Your reward in heaven or
retribution (punishment) depends on the kind of account you give God for your
actions, for the way you have used or misused, perhaps even abused, your
gifts. (cf. Mt 25:31-46)

A. God's Word, First of His Greatest Gifts
1. Jesus taught many great truths. Many were new, radical and even for the
apostles and disciples difficult to understand. Open your bible to chapter 1 of
St. John's gospel, the Prologue, verses 1-18. It says it all. Prayerfully
reflect on it. Which sentence means the most to you? Share.

2. An episode! Four apostles, Peter, James, John and Andrew came to Jesus
privately for an explanation of the future happenings he preached about in
parables. He gave them an ear full, seven happenings in all. Please be open to
the Spirit. Yes, we are in the ``end time.'' Share, but avoid speculating on
these predictions as some contemporary fundamentalists do. Stay on track and
within the time frame.

3. Among Jesus' many parables, please read this paragraph and go to par. 4.

4. At the Sea of Galilee! Carefully read this paragraph and go to par. 5.

5. Puzzled! Reflect on Jesus' four-part explanation of the seed. Ask yourself,
in which category am I? In your faith journey from childhood to adulthood, would
you say, ``I've been in all four at one time or another?'' Now as a member of a
SEFC, am I in the ``rich soil'' category? Pray! Share.

6. The centrality of the word! Vatican II's teaching (DV) and the CCC are
explicit and emphatic that the word of God is ``the power of God for salvation''
and ``the ultimate truth of God's Revelation.'' Reflect on these statements,
what they mean for your spiritual formation and their impact on lapsed Catholics
called to return to the faith. Ask the Holy Spirit for light.

7. The Word is Necessary for Formation of Conscience
a. b. c. and d. The four analogies given here describe through commonplace
things what conscience is like. Look closely at D. Conscience is like a
voice. Reflect, do I listen to the voice of my conscience? Am I grateful when it
approves my action(s), shut it out when I disagree with it, or talk myself into
justifying it? Be honest.

e. and f. When a person makes a ``judgment,'' a decision between what is good
and what is evil, that person is consulting his conscience and making a choice
in a moral matter. Sometimes a person makes a wrong choice and does what is evil
and feels comfortable with it as he claims. He needs to be informed what is
right and wrong and helped to form a ``right conscience.'' This obliges him to
seek out the truth. When he or she is informed of the truth, which is situated
in the wisdom of God, that person is obliged to accept the truth, adhere to it
and live by a right moral conscience. Prayerfully reflect on this vastly
misunderstood issue. Ask: where am I on it?

g. To educate your moral conscience is no easy task due to the ``negative
influences'' swirling around us. Some examples: ``Political correctness''
demands that we be ``tolerant'' even of grave sin, e.g. legalized abortion; that
we accept the most extreme forms of ``diversity,'' e.g. so-called same-sex
marriage; that we be ``non-judgmental,'' e.g. not to report to the FBI a plot by
terrorists who have infiltrated our country to poison a U.S. city's water supply
for fear of being accused as a ``whistle blower,'' or of a ``hate crime.'' Do we
or do we not need to educate our moral conscience? Education of conscience is a
life-long task because cultural and societal situations are constantly
changing. This task requires information reflection, conviction and prayer.

h. Erroneous Conscience! Please reflect on the two distinctions, which
follow. An erroneous conscience is one that is in error. A person has an
erroneous conscience, when he commits sin:
In willful ignorance, that is, when he deliberately ignores, neglects or rejects
information about the rightness or wrongness of his action. This is called
culpable or blameworthy ignorance. His conscience is in error and he incurs the
guilt of sin.
The other ``error'' of conscience occurs when a person commits sin in ignorance
which is not deliberate, non-intentional. This is called invincible or no-fault
ignorance. Though not guilty of sin, he is responsible to learn the truth about
his action.
Keep in mind these distinctions when examining your conscience in preparation
for confession especially if you've been away from the sacrament for a long
time. Reflect in quiet. Share.

8. Examination of Conscience
a. and b. Review the two lists of Commandments in the Old Testament, and refresh
your memory on the traditional list of Commandments, which you learned and
perhaps memorized as a child. Knowing the Ten Commandments by heart is certainly
part of your spiritual formation.
A case history in the ``temporal order'': recall that Supreme Court Chief Judge,
Roy Moore, of the State of Alabama was ordered by a decision of a federal court
to the remove a small marble monument of the Ten Commandments from the rotunda
of the Judicial Building following fierce agitation by the American Civil
Liberties Union (ACLU), claiming the monument violated separation of church and
state. Was this decision a violation of the right of ``free exercise'' of
religion as stated in the First Amendment? Are believers being persecuted for
their faith? Reflect on the ``free exercise'' clause, not the politics. Share
but stay on track.

c. and d. Examination of conscience is invaluable not only in preparation for
receiving the Sacrament of Reconciliation, but also for becoming aware of sins
which we commit but hardly ever advert to, and perhaps never confess, but which
stand in the way of our advancing on the road to holiness.
The fundamental place to begin examination of conscience is the Ten
Commandments. The Eight Beatitudes included here under c. is another guide for
examining one's conscience. Under d. in several of his letters St. Paul presents
lists of vices and virtues. Checking out these lists is also useful for making
improved confessions. They will aid you in your progress toward sanctity. Check
your Bible for one of these lists. Quietly reflect on it and share the insights
you receive.

B. God's Word in Liturgy
1. and 2. Sacramental celebration is done through visible signs that express
invisible and transcendent realities. The reality is encountering the living
Christ through faith and through the sacred sign. In par. 2, reflect on familiar
sacred signs and their meaning for you.

3. Symbols!
There is a difference between signs and symbols. Signs are visible objects that
express invisible realities. Example: bread and wine at the consecration are
visible to the eye but they express the invisible real presence of Christ on the
altar. Symbols are different. They are an object, a word or phrase that has its
own immediate, limited meaning but at the same time signify other unexpressed
meanings. Example: kitchen, a single object which has many unexpressed meanings:
your family's kitchen, your grandma's kitchen, a restaurant kitchen, a military
field kitchen. ``Kitchen'' is a symbol with many meanings. Pick a symbol from
the list in para. 3. Open to the Holy Spirit, what meaning does it have for you?
Share.
4. More Symbols
Religion is the celebration of a set of beliefs which are made visual by symbols
that signify transcendent realities, namely the presence and power of God who
forgives sin, dispenses grace and effects change for good in people's
lives. Please review the symbols listed, choose one, reflect on it and how it
effects change in you. Share.

C. Stewardship of our Three Baptismal Gifts
1. Just read.

2. Please read and reflect. Many practicing Catholics of good will who are
active in the Church realize they lack adequate formation in catechetics. They
know, love and live by the basic truths of the faith but many express a desire
to become better formed in the truths of the Catholic faith. Providing formation
in the faith in parishes is what the CLCC process is about. Formation takes
place in SEFCs. This manual presents a process of orthodox and systematic
formation in the faith. It follows Jesus' format of formation taken directly
from his public ministry in Galilee. Praise God for the gift of Jesus'
format. Share your insights.

3. and 4. Center on truth. Reflect on these statements: ``God is the source of
all truth.'' His Son Jesus, the Word, is truth.``His people are called to live
in the truth.'' Jesus, the light of the world, is the truth. Jesus leads us
``into all the truth.'' What do these uncompromising, explicit statements say to
you? Share.

5. and 6. Center on functions. Because Christ is Truth, his community of
believers, the majority of whom are lay persons, have a ``vocation.'' They are
``called'' to sanctify the world (as priestly people), to give growth to the
Church (as prophetic people) and to engage in temporal affairs by ordering them
according to the plan of God (as kingly people). A tall order! Invoke the Holy
Spirit and reflect how these three functions pertain to you and how you carry
them out.

D. We are a Priestly People.
1. 2. and 3. Each paragraph presents a specific teaching, but the three
baptismal gifts are intimately linked. They empower us to practice Total
Stewardship.

1. We are given a ``priestly vocation.'' This means we are called to share in
the priestly functions of Christ. By our baptism in Christ we are consecrated to
be a spiritual house and a holy priesthood. Quietly reflect. This authoritative
statement leaves little doubt that we are a priestly people. Ask yourself: am I
aware what this ``vocation'' should mean to me as a Catholic Christian?

2. By virtue of the sacrament of baptism, we are ``regenerated,'' washed clean
of original sin, anointed with the Holy Spirit, and consecrated to offer
spiritual sacrifices acceptable to God through Jesus Christ. As a practicing
Catholic, ask yourself, what kind of ``spiritual sacrifices'' do I offer to God
through Jesus Christ? Name some. (Don't look at par. 3. yet where the answers
are given.) Share.

3. The laity are baptized, anointed and dedicated to God. They are expected to
produce. What? Spiritual sacrifices acceptable to God! Study the list. Do they
sound familiar? When you offer to God your everyday works they are ``holy
actions,'' worshiping actions by which you are consecrating yourself, your life
and the world to Christ, and at the same time they are aiding you to grow in
holiness. Putting the three teachings together in par. 1, 2 \& 3, do you see how
you are a priestly people? Thank Him!

E. We are a Prophetic People
1. and 2. Understand the term ``prophetic'' correctly. Prophetic does not mean
being a forecaster (e.g. of weather) or a predictor of the outcome of an event
(e.g. winner of a Super Bowl game). To be prophetic means to be a spokesperson
of God, making his Son Jesus known and accepting this role as one's mission. The
Old Testament had a long line of prophets. They were called, chosen, and made
inspired messengers of God. Some didn't want the job, even resisted it, felt
unqualified, but in the end did their job very successfully, often at the
expense of terrible suffering and even death. Reflect. Am I called and chosen?
Do I accept my mission to be a spokesperson of God, really?
3. and 4. The top biblical prophets were St. John the Baptist and Jesus. We know
much about Jesus as prophet and will learn more in this CLCC process. What do we
know about St. John? Go to the texts cited, either from Mark, Luke or Matthew or
the CCC. Get the full story on John. Reflect and apply as much of it as you can
to yourself. Share.

5. Inseparable from the role of prophet is the office of teacher. Me, a teacher?
Never! If you are a parent you are a teacher. You teach your children right from
wrong, how to be polite, who is God, the power of prayer, how to be gentle,
loving, honest and truthful. Right? The Catechism of the Catholic Church states
``Parents are the principal and first educators of their children.'' (CCC 1635)
Vatican Council II states, parents ``by word and example [are] the first heralds
of the faith with regard to their children. They should encourage them in the
vocation which is proper to each child, fostering with special care any
religious vocation.'' (cf. LG 11; also cf. CCC 2225-2226)
Whether you are a grandparent, parent, single or young adult, or teenager in
your SEFC you are a teacher. Reflect: each time you share the truth, you are
God's messenger. When you share in your SEFC, you are a ``teacher.'' Let your
sharing spill over into your family, to your friends and at work. By sharing
truth, you are ``teaching'' others some aspect of God's goodness, mercy and love.

6. The ``Sacred Deposit'' of faith. Reflect how blessed you are that all truths
about God, Jesus our Savior and his Church are revealed to you in the Scriptures
and in Tradition. In addition you have the Magisterium, the Church's official
teaching office, to rely on for correct interpretation and explanation in order
to walk safely in truth on your journey to salvation. Reflect how the Sacred
Deposit of faith steers you past the pitfalls of personal opinion, dissent,
erroneous conscience and the insecurity of ``personal experience'' which can
lead an individual astray that is contrary to truth and Church teaching. Living
by the Sacred Deposit of faith is the safety net of the Catholic Church and its
faithful members. Praise God for the gift of truth!



7. Keeping in mind! The prophetic office of Christ is shared by the hierarchy,
clergy and laity. Christ makes bishops, priests and laypersons witnesses to his
life and teaching. He gives them the ``sense of faith,'' a tremendous gift, and
endows them with grace for the office. Reflect: am I trying to be a prophetic
person? Am I striving to use this gift generously within my SEFC? Am I using my
prophetic gift in my mission of outreach to non-church going Catholics? Be open
to the Holy Spirit's light and grace to aid you.

8. and 9. Here evangelization is defined again, how it is done and to whom the
good news is announced, even to unbelievers who sometimes are among the
lapsed. Note that ``catechetical formation'' is the purpose of SEFCs both for
practicing Catholics and returnees to the faith. The truly Catholic prophetic
person is one who ``unfailingly adheres to this faith,'' the faith Christ gives,
and with God's help and the help of Christ's Mother Mary does not deviate from
it.

F. We are a Royal People
1. 2. and 3. Great was the people's confusion about Jesus in his day who he
really was. They knew he lived in lowly Nazareth and was a carpenter. In
Paragraphs 2. \& 3. Jesus projects his quality of royalty by the authority of
his words, deeds and demeanor. These were attributes of his divinity that were
surfacing. Reflect in particular on the episode of the spiritual and physical
healing of the paralytic. Solid evidence of his divine power and kingly
authority! That same Jesus heals us. Recall a healing you received, or someone
who was healed physically or spiritually, or both. Share.

4. and 5. As a believer who shares in Jesus' role as King, reflect on the
following great truths. The sacrament of Baptism unites us with Christ in order
to die with him, that is, die to sin. At the same time like Christ we rise with
him. After rising out of the baptismal water as Jesus rose from the tomb, three
things are given to us: our sins are forgiven, grace is given for growing in
holiness, and we receive the status of adopted children of God. We become heirs
of his heavenly kingdom. Only Christ could by his divine and kingly power united
with his Father and the Holy Spirit rise from the dead. Joined as we are to
Christ he will raise us from the dead on the last day. In this way Christ makes
himself a servant to us. He is a servant king, ``a ransom for many.'' By dying
and rising with Christ in baptism we are given a share in his royal
servanthood. Your servanthood is expressed when you render service to your
neighbor, to your friends in your SEFC where you receive and share formation in
the faith, and in your outreach to win lapsed Catholics back to the Church. We
are a kingly people because we are a servant people. ``To reign is to serve
him.''

6. How can the laity exercise their kingly role?
a. b. c. d. e. and f.  Read carefully. To complement your commitment to and
participation in the CLCC process, keep in mind that its focus is on formation
and outreach to lapsed Catholics. Reflect on the five ways listed that will help
you exercise your kingly role in the CLCC process. Select one. Reflect on what
you need to pray for, to learn more about it, and internalize it. Then share it
with your SEFC group.

G. Emphasis on the Laity's ``Rights and Duties'' in
  the Church's Mission
1. This paragraph reawakens awareness that as God's People we are privileged
receivers of many undeserved gifts. We are, therefore, obliged to recognize that
gifts received demand giving thanks to the Father who gave them. Attached to
one's gifts are ``rights and duties.'' ``Rights'' give us freedom to use them
according to God's plan. ``Duties'' impose the obligation to use them
productively for personal good and the good of others. ``Duties'' include
carrying out the mission of the Church. The sum of these truths and actions
constitute Total Stewardship. Reflect on the basic concept of Total
Stewardship. It is responsible and productive management of the multiplicity of
gifts received for one's own well-being, our neighbor, community and Church.

2. and 3.  Just read.

H. The Indispensability of Preaching
1. Catechesis is not catechism. It is systematic formation in the teachings of
the Church accompanied by training how to put the Church's teachings into
practice in one's life. Catechesis gives God's people an experience in faith not
just knowledge of the faith. Preached catechesis on the CLCC in weekend homilies
is essential for successful initiation of the process in the parish. Equally
essential to the process is that laypersons in SEFCs experience growth in the
faith through shared catechesis. This is the catechesis that members of SEFCs,
who open themselves to the light of the Holy Spirit, share their insights with
each other following quiet reflection.

2. This paragraph is meant to give encouragement to the priest to commit to the
process personally as much as he can. To attempt shortcuts, or to give the
impression he is aloof from the process is a sure way to kill it. Parishioners
will become enthusiastic about their spiritual formation and outreach for
converting lapsed Catholics only to the degree of the priest's commitment to the
process. Members of SEFCs will find it helpful to the priest and to themselves
if they encourage and support the priest. Reflect in prayer how to support your
priest in the CLCC process.

3. and 4. How should the priest present the CLCC process to the parishioners?
Through his weekend homilies! The initial announcement of the process should be
an enthusiastic, interesting overview. Homilies should say what the process
gives parishioners, namely ongoing systematic formation in doctrine and
spirituality in weekly 90 minute sessions. Later homilies should explain how
spiritual formation is conjoined to planned, methodical outreach to lapsed
Catholics to help them return to the faith.
This is followed with particulars: that the process is lay-led by a core group
called the Matrix 12, that participation of all parishioners is the goal, and
that participation takes place in SEFCs. Homilies should focus strongly on the
establishment of Small Ecclesial Faith Communities throughout the parish, what
they are and why are they ``ecclesial.'' Each stage has an abundance of material
for preached catechesis that is nearly inexhaustible.
Reflect: how can I help, encourage, pray and respectfully prod the priest to
preach the process to make it ``bear much fruit'' for all parishioners? Pray and
share the insights you receive.

% ------------------------------------------------------------------------------

\chapter{Stage 4.\ Home Visitation and Evangelization}

\section*{} \lxRDFa{property=stage-main-content,resource={manual}}

for Community Building

Note: Stage 4 has 37 numbered segments, they form the main text. In its
Reflection and Discussion section, beginning on page 95, there are corresponding
numbered segments designed to stimulate prayerful reflection, discussion and
sharing. When you have chosen a paragraph from the main text as the subject
matter for your formation session, please refer to the like number segment in
Reflection and Discussion as they are meant to be used together.

Introduction
1. Stage 4 is about ``saints'' reaching out to those ``called to be saints.''
(cf. 1 Cor 1:2) In his letters St. Paul addressed his followers at times as
``holy ones,'' saints. (cf. Rom 12:13) He said boldly, ``This is the will of
God, your sanctification.'' (1 Thes 4:3; cf. Eph 1:4) Holiness, sanctification,
is not for a few but for all. The Apostles Creed, the prayer at the beginning of
the rosary, expresses this well and succinctly. ``I believe in the holy catholic
Church, the communion of Saints.''

2. Church teaching on the Communion of Saints on earth is the body of redeemed
Christians linked together, who avoid serious sin, live in the state of grace
and who strive to become holy. They are one in unity and charity with Christ
especially in the Eucharist and with each other. (cf. CCC 948)

3. The Communion of Saints is the great ``community'' of blessed enjoying their
reward in heaven, the saved though suffering temporarily in purgatory for final
purification, and us still on earth on journey to join the saints in
heaven. (cf. CCC 957) In short the Communion of Saints is the ``unity of
believers who form the body of Christ.'' (CCC 960)

``Saved'' and ``Non-saved''
4. A tendency in Church ministry, it seems, though there are noteworthy
exceptions, is that the ``saved'' are pre-occupied with ``saving the saved.''
The ``saved'' are Catholics who attend Mass regularly and drop in their
envelope, but beyond weekly Sunday Mass are uninvolved in any parish life. It
seems they choose rather to stand on the sidelines and not be on the playing
field of parish activities. It can be said on the other hand that the
``non-saved'' are lapsed, non-church going Catholics, many of whom are not even
faintly aware of the precarious spiritual condition they are in. Are the
``saved'' in many parishes not giving the ``non-saved'' the urgent attention
they need for their salvation?

5. Non-church going Catholics in the United States are estimated at about 47
million. These are baptized Catholics, most of whom have been very likely
educated in Catholic elementary and secondary schools, and not a few have
attended Catholic colleges, who come to Mass only at Christmas, Easter or attend
a funeral or wedding of a relative or friend during the year.

What is meant by the expression ``the saved saving the saved?''
6. It means that church-going men, women, youth and children are given greater
religious and spiritual services by the parish than are given to lapsed,
non-church going Catholics. This does not imply that the lapsed are wholly
ignored. Not at all! It does mean that woefully insufficient attention is given
to helping the lapsed return to the Church. Celebrating Life as a Catholic
Christian (CLCC) is a results-producing process of systematic spiritual
formation which equips and motivates the ``saved'' to reach out to the loosely
called ``non-saved,'' lapsed, non-church going Catholics, including alienated,
indifferent and marginal Catholics. It reaches out to Catholics ``on the edge''
and to those who are outside the Church for whatever reason, and helps them
return to it.
Pope John Paul II's New Evangelization means precisely this: to awaken a
missionary spirit among the People of God to go out to those who need to hear
the truth of faith and be formed in it and by it.

7. In the United States 19,000 Catholic parishes are spread over 196 Catholic
dioceses. Surveys show that about 1 out of 4 baptized Catholics go to Mass
regularly, or about 25\%. What about the 75\% who do not go to Mass regularly or
not at all? The high percentage of non-church-goers would be lowered
significantly if, a) non-church going Catholics living within the boundaries of
a parish were sought out, and b) if they were brought into Small Ecclesial Faith
Communities, re-introduced gradually into the faith--the aim being to help them
make a decision to return to the faith--returnees over time would be so
overwhelming that Catholic churches would not be able to hold them all.
The CLCC process is not keeping a scorecard or looking at numbers. The
missionary aim of the CLCC process is winning former church members back to the
faith. The heart of the Church's mission and the heart of the CLCC mission are
one and the same, to have a passionate desire to bring Christ's salvation to
all.

8. Many Catholic parishes throughout this country already have outreach programs
in place such as ``Come Home,'' ``Come Home for Christmas,'' and others. For
years the Legion of Mary has been in the forefront of this apostolate. The RCIA
is doing a formidable job in giving growth to the Church each year. The CLCC,
however, is a process not a program that unfolds over five stages. It has a
beginning but no end. It is ongoing due to the kind of apostolic ministry it
is. Great results take place over time, not in a couple of semesters or seasons
or a year.
The CLCC process forms people to conform to Christ in order to reform their
lives and come to full communion with Christ in holiness in his Church, the
Mystical Body. The crowning goal of the CLCC process is to help all the Lay
Faithful of Christ become saints. This requires pastoral planning, perseverance
and God's grace working uniquely in each person over time.

Work Cut Out
1. Recall that the CLCC process in all of its 5 Stages runs in perfect parallel
with Jesus' public ministry. What Jesus did in Galilee in five stages, the CLCC
process does in five stages in the parish. In Stage 4 Jesus reached out to
non-Jews, to Samaritans, Gentiles, Romans and others to bring them the good
news. (cf. Centerfold, page 144) In Stage 4 the CLCC process reaches out to
non-practicing Catholics in the parish to bring them the good news and an
invitation to conversion.

2. Following are three biblical episodes, really classic models of the CLCC
process. Each of the three models demonstrates a particular element of the
process.
The first model is the Parable of the Lost Sheep (Lk 15:4-7), searching and
finding the lost. Jesus asks, ``What man among you having a hundred sheep and
losing one of them would not leave the ninety-nine and go after the lost one
until he finds it?'' (4)
The second model is the Parable of the Prodigal Son (Lk 15:11-32), the lost has
been found. The father rejoices and calls a celebration. ``This son of mine was
dead, and has come to life again; he was lost, and has been found.'' (24)
The third model is the Conversion of the Samaritan Women at the Well (Jn
4:4-42), her formation and evangelization. After her person-to-person
conversion in the one-Person Small Faith Community of Jesus and herself, she
went with heart burning with zeal to tell the good news about Jesus to her town
people of Shechem. (verses 29-30,39)

3. At the parish level spiritually formed parishioners in Small Ecclesial Faith
Communities begin to evangelize. They go out to non-practicing, non-church going
Catholics scattered throughout the parish. They visit them in their homes,
invite them to join a Small Ecclesial Faith Community for formation in the
faith, and in prayer help them to begin considering returning to the Church. In
this way koinonia, building up ``communion'' within the Church takes
place. Reclaiming lapsed and negligent Catholics is a prime way of bringing
Christ's Church to ``full stature.'' (Eph 4:13)

A. Home Visitors
1. Home visitors are faithful Catholics who strive to live the truths of faith
intelligently and conscientiously, and out of love for Christ want to spread the
same truth and love to others. They are Catholics who reach a point in their
spiritual formation when they want to do ministry of substance. They are ready
for a challenge and want to produce results. Finding lapsed Catholics within a
parish, visiting them in their homes, selling them on entering a non-threatening
Small Ecclesial Faith Community for formation, and getting them to open up to
the graces of the Holy Spirit for reentry into the Church is indeed a ministry
of substance. This was Jesus' ministry. He visited people in their towns,
villages and homes, and taught groups on mountainsides and lakeshores. He told
them about his new way of life. His purpose was to win them for his
kingdom. Home visitors have an identical ministry.

2. To be successful home visitors, leaders and members of SEFCs need on-going
spiritual formation to keep growing in the faith and keep their commitment at
high level to evangelize lapsed Catholics. Never should they yield to the
temptation to think they have arrived, know all that is needed to know, and that
their own formation is complete. This is to deceive oneself. The agent of
deception never rests. Because the Lord has blessed us with ``many and varied
gifts'' (cf. Stage 3), we have an obligation to use them responsibly for
building up his kingdom on earth and for giving him the glory he is due. For
best results, formation in SEFCs is non-stop. Newcomers to a SEFC depend on the
seasoned members for guidance, example and inspiration for their
conversion. Rookie returnees should never feel they were let down by the
veterans in their search for Christ.

3. Contacting, visiting and befriending lapsed, non-church going Catholics is
one thing. Another thing is that a lapsed Catholic may feel the visit is an
intrusion on his or her privacy and resist the idea of coming into a SEFC in the
parish. Pray and persevere. Do not stop the visits. Practice patience; have
confidence! Your zeal and charity are your leverage. A change of heart will come
even in the ``hardest cases.'' One day, perhaps after the third or fourth
friendly visit, the person will come around and tell his or her story. Listening
to the person's story whether told in the home or later in a SEFC, but always
with attention and compassion, is the first step in the conversion process. The
newcomer to a SEFC must feel welcome, loved and made comfortable in the
group. Present at his first group session the newcomer will immediately see how
a faith formation session is conducted in a SEFC. He or she must be encouraged
to participate.
More important than anything else, the newcomer will observe the joyful
seriousness of the members. He will observe the dignity of the group, its
prayerfulness and un-pretended commitment to its apostolate. Seeing these
inspiring qualities will impact the lapsed person. Spiritual formation after his
long abandoned faith will seem less alien and less difficult to
regain. Gradually the person will feel he belongs to the group. In this
environment of Christian love, the power of the Holy Spirit will ``touch'' his
heart nudging him to return to the faith for re-entry into the Church, and
reunion with Christ.
4. To make return to the Church permanent and solid, returnees need on-going
formation in the faith. Where and how? By belonging to a Small Ecclesial Faith
Community! Remaining in it the returnee experiences how the faith is practiced
on a deeper level. He will receive correct answers to continuing questions he
may have about the faith. He experiences the wondrous power of group prayer,
bonding with like-minded friends for trust, support, strength and help. In the
SEFC he begins to get a ``sense'' of the mystery of ``communion,'' koinonia,
within the Church, and how this divine glue makes the Church Universal ``One and
Holy.'' ``Communion'' within the Church also means the loving interaction of the
People of God in grace who are concerned about the salvation of one another and
who strive to help each other on the uphill journey to eternal life.

5. Essential also for the returnee is the good example and lifestyle of
practicing Catholics. These are Catholics who attend Sunday Mass regularly,
participate devoutly in the liturgies, receive the sacrament of penance and
reconciliation frequently, and receive Holy Communion in a manner that shows
they know what they are doing. They are also Catholics who live by the 10
Commandments, obey the moral law of God, have children according to the plan of
God, and who raise and educate their children in the Catholic faith, some at
great sacrificial cost. They train them to be responsible stewards of their
gifts, to be honest, just and respectful of all people, to be patriotic and
productive American citizens and abide by the Constitution of the United
States. Lastly they pray and inculcate the habit of family and private prayer in
their children.
These are some of the great treasures that the Church has, practices and enjoys
for increasing love of God and neighbor in people's hearts. Home Visitors yearn
to share them with Catholics in need of returning to the Church. They sustain
conversion to Christ in everyone. They are fundamental requirements for
Celebrating Life as a Catholic Christian.

B. Home Visitors are Evangelizers
1. Jesus' strongest and most explicit mandate to his Apostles immediately prior
to his Ascension into heaven was to evangelize. ``Go, therefore, and make
disciples of all nations, baptizing them in the name of the Father, and of the
Son, and of the Holy Spirit, teaching them to observe all that I have commanded
you. And behold, I am with you always, until the end of the age.'' (Mt 28:19-20)
These solemn words of Jesus make evangelization the priority of the Church, the
diocese and the parish to ``end of the age.''

2. The above text deserves long and deep reflection. Why? Because it pertains
not only to the apostles, but also to home visitors in our day and to all
followers of Christ of every age! Therefore, clergy and laity have a solemn duty
to evangelize, to proclaim the good news to those present to them and willing to
hear it. This duty also extends without exception to those far removed from
them, unknown to them, indifferent to hearing the word, and perhaps deliberately
resisting it. The latter group for the most part are lapsed, non-church going,
alienated Catholics.

3. Lay Catholics, called home visitors, have an obligation to go out to them,
find them and visit them. Because this obligation is serious, it is not a choice
to discuss: ``should we go,'' ``can we go,'' ``will we go?'' Lapsed Catholics
are an ailing part of the People of God. They are in critical need of
salvation. They need to be contacted, and with utmost discretion and
sensitivity, urged to return to communion with the Church. They need to be
brought into an environment which will help them return to Christ. In Stage 4 of
the CLCC process, home visitation is a ``duty'' incumbent on Catholic men and
women in Small Ecclesial Faith Communities to help estranged Catholics re-unite
with Christ in his Church.

4. Regarding conversion of sinners, St. James offers this appealing incentive to
persons who are diffident or fearful about making home visits: ``My brothers, if
anyone among you should stray from the truth and someone bring him back, he
should know that whoever brings back a sinner from the error of his way will
save his soul from death and will cover a multitude of sins.'' (cf. Jas 5:19-20)

5. Home visitors reach out to non-church going Catholics and help them with the
power of God's grace return to the faith. This is no small challenge but worth
every effort. Pope John Paul II addresses this challenge from the cultural
perspective. He said in sum, that Catholics like other persons are not immune to
powerful, cultural, neo-pagan influences that undermine not only Christian
values but also the values of humanity itself.

6. Quoting Vatican Council II's document, Gaudium et Spes, GS, the Pastoral
Constitution of the Church in the Modern World, the Holy Father deplores ``the
development of a culture [which] becomes dissociated not only from Christian
faith, but even from human values.'' (GS 58) John Paul II also quotes Pope Paul
VI's apostolic exhortation Evangelii Nuntiandi, Evangelization in the Modern
World, EN, ``What matters is to evangelize humanity's culture and the cultures
of human families ... The split between the Gospel and culture is without a
doubt the drama of our time, just as it was of other times. Therefore, every
effort must be made to ensure full evangelization of culture, or more correctly
cultures.'' (EN 18-19) It seems clear that a major factor contributing to the
great falling away from the Church is due to the cultures in which we live.

7. Parishioners in Small Ecclesial Faith Communities do two things: they join a
Small Ecclesial Faith Community, akoinonia, to grow personally and with others
in their spiritual life, and to become home visitors, representatives of their
parish, to lapsed, non-church going Catholics living within the parish. As a
member of a SEFC they assume the duty to evangelize, a duty of highest
importance. Pope Paul VI in Evangelii Nuntiandi stated emphatically that
``person-to-person'' contact is ``indispensable'' for the transmission of the
gospel. (cf. EN 46) Thus, what better way to evangelize a lapsed,
non-church-going Catholic than by making friendly contact with the person in his
home?

8. Paul VI goes on to say, that ``side by side with the collective proclamation
[that is, the Liturgy of the Word with homily on Sunday], the other form of
transmission [is] person-to person.'' He asks, ``is there any other way of
handing on the gospel than by transmitting to another person one's personal
experience of faith?'' He answers his own question. ``It must not happen that
the pressing need to proclaim the Good News to the multitudes should cause us to
forget this form of proclamation whereby an individual's conscience is reached
and touched [emphasis added] by receiv[ing] [it] from someone else.'' (EN 46)

9. Formation of home visitors for evangelization is necessary and
imperative. Reaching perfection in holiness that God has in his plan for them is
not fully attained on earth, but in heaven. While on earth formation in the
spiritual life is an on-going growth process. We may never say without danger of
blasphemy, ``I've arrived, I'm there,'' ``I am holy.'' Lay leaders of the Matrix
12 and the groups of 8 to 12 persons who make up SEFCs in the parish must
recognize this fundamental reality of spiritual formation. Advancing in
knowledge and love of Christ and his Church cannot be measured on an imaginary
barometer that has a top level marked, ``HOLY.'' What characterizes Stage 4 is,
it centers on and provides systematic spiritual formation to SEFC
participants. It fills them with the gift of grace to want to reach out and
evangelize lapsed Catholics by visiting them in their homes as Jesus did.

10. The ``catechesis'' given in Parts I and II of this manual is in fact a
catechesis specifically for formation of parishioners in SEFCs. It is a
wholesome experience for them to hear, accept and discuss the words of Christ
and his Church, planting them deeply in their minds and hearts and turning these
great truths they into convictions to live by. The first source of these truths
is the Bible, the divinely inspired writings handed down by the apostles to us
through the Church. The second source is the rich and profound body of
magisterial and papal pronouncements. These are Vatican Council II
Constitutions, Decrees and Declarations, papal encyclicals, apostolic
exhortations, pastoral letters, instructions and various clarifications issued
by the Vatican.

11. We are living in an age of cultural decline. The essentially religious
culture by which our nation thrived for two hundred years is being
systematically de-Christianized by the powerful forces of atheism, secularism,
materialism and relativism. Greed, power seeking and self-interest are diseases
that undermine the culture. Influenced by leftist higher education, the media,
TV, a greatly corrupted legal system, cruel disregard for the sacredness of
human life and living in a radical humanistic environment, it becomes
understandable how Catholics can fall from the faith and communion with the
Church, and from each other.
The culture in which we live currently is a culture of self-destruction. Pope
John Paul II's phrase for this phenomenon is, ``culture of death.'' He has
stated many times that the ``culture of death'' must be replaced by a
``civilization of love.'' The CLCC process is rooted in and promotes a
``civilization of love.'' For the sake of building koinonia, community, in
parishes and the Church, an army of committed parishioners, properly formed
spiritually, must be deployed to fan out over the parish to home- visit all
lapsed Catholics in need of evangelization for return to the Church. To use
St. Paul's battle cry, they must ``put on the armor of God. Stand fast
... girded in truth, clothed with righteousness as a breastplate, feet shod in
readiness for the gospel of peace.'' (Eph 6:13-15)

C. Home Visitors are Builders of Community
1. The second motivation of members of Small Ecclesial Faith Communities, the
first being growing spiritually, is to prepare to go out into the parish to
``retrieve'' the ``living stones'' which have fallen out of the house of God
(non-practicing Catholics) and ``re-insert'' them into the structure of the
Church, the ``spiritual house ... through Jesus Christ.'' (cf. 1 Pet 2:5) Doing
this makes home visitors builders of koinonia, community. They seek out the
lapsed, inactive, alienated, wayward and fallen-away Catholics. With utmost
sensitivity toward the visited, home visitors are reminded that all baptized
persons belong irrevocably to the intimate family of God.
In graphic language St. Peter speaks of those who belong to Christ, the baptized
(cf. CCC 1141), as living stones, a royal priesthood, a chosen race, a holy
nation, a people of his own, a spiritual house. (cf. 1 Pet 2:9) Meaningful as
these designations are, they are generally unknown to Catholics, especially
those away from the Church a long time. They need to come to know the meaning of
these phrases. A good place is in a SEFC. Learning that the terms apply to them
can with God's grace enkindle in their hearts and minds a desire to return to
the Church.

2. Why are the baptized called living stones? Because God the Father made his
Son ``a cornerstone'' for those who believe in him! Believers who assemble in
one place for worship of the Son are living stones gathered together, being
built and building up ``a spiritual house.'' (cf. CCC 1179, also 1268) Why are
those baptized called a royal priesthood? Because they participate in a ``common
priesthood'' with Christ who is Priest and King (cf. CCC 1141), and offer
spiritual sacrifices acceptable to God through Jesus Christ. Why are those
baptized called a chosen race? Because Abraham's faith was so great, God chose
his race and made a covenant with it that continued till after David's time into
Christ's end time. Through the Holy Spirit the ``promised restoration'' began.
The chosen who make up the kingdom ``would belong to the poor according to the
Spirit.'' (cf. CCC 709,710; also cf. Mt 5:3) Why are baptized persons called a
holy nation?`` Because once they were ''no people' but now ... are God's
people.`` Why are the baptized called ''a people of his own?`` Because they had
once received no mercy, now (they) have received mercy.'' (cf. 1 Pet 2:4-10)
Finally why are the baptized called ``a spiritual house?'' Because baptized
persons present themselves to Christ to be built into a spiritual house, an
``edifice of spirit,'' the place where God dwells! They become a Holy City, the
New Jerusalem, a community of believers, (cf. CCC 756) whose God is Christ the Lord.

3. Home visitors are moved to seek out the lapsed in their parish by these words
which God the Father spoke in prophecy about his God-Man Son, Jesus:
1 Pet 2:6
Behold, I [God] am laying a stone [Jesus, my Son] in Zion, a cornerstone, chosen
and precious, and whoever believes in it [in Jesus] shall not be put to shame.
This prophecy, states St. Peter, has ``value'' for those ``who have faith.'' But
he adds, ``those without faith,'' the ``stone ... will make people stumble ... a
rock that will make them fall.'' ``They stumble by disobeying the word ...'' (1
Pet 2:7,8) Many, indeed millions, have ``stumbled'' because they are ``without
faith,'' even though they are baptized. Those who have stumbled, are, therefore,
in great need of evangelization.
Home visitors seek them out, visit them in their homes, hear their story
sympathetically, sometimes a sad, angry story, and warmly invite them to join a
SEFC in the parish in order to begin their trek back into the ``spiritual
house'' of God.

4. Each baptized Catholic, even if he is non-practicing, must come to realize
that he or she is a ``living stone'' of God's house. Those having stumbled for
lack of faith, and who find themselves fallen away from the Church, should not
avoid those who seek them out. They should not resist the effort of home
visitors to retrieve them. Prayer and penance, as Jesus advised his disciples,
are the weapons by which to expel this kind of demon. (cf. Mk 9:29)

Unfortunately the number of ``living stones'' that have fallen out of the
structure of the house of God, the Church, has reached a shocking estimated
75\%, or about 47 million out of a total 63 million Catholics in the United
States. This falling out has severely weakened the Church. Thus home visitors
have a tremendously important and urgent apostolate: the job of finding those
who have stumbled, retrieving them in order to help them, strengthening their
weakened faith in a SEFC through the power of the Holy Spirit, and
``re-inserting'' them into the ``communion of believers,'' the Church. Home
visitors are builders of community, koinonia.

5. Models of home visitation abound in the New Testament. Jesus and Mary are of
course the principal models. Below are a few, though more could be added,
well-known gospel episodes showing the importance and effectiveness of making
visits to homes. We suggest that you look up these texts and be inspired by the
results that came from each visit:
Lk 1:39-45
Mary visits her cousin Elizabeth.
Mt 8:14-15
Jesus visits Peter's mother-in-law.
Mt 26:6
Jesus visited Simon the Leper in his house in Bethany.
Lk 10:38-42
Jesus visits Mary and Martha.
Lk 19:1-10
Jesus visits Zaccheus, the tax collector.
Jn 2:1-2
Jesus and Mary visit a couple at their wedding in Cana.
Jn 4:4-42
Jesus visits with the Samaritan woman at the well
Jn 20:19-22
Jesus visited his apostles on the evening of the first day.
6. Front-line home visitors in a parish are the Matrix 12, the lay leaders of
the CLCC process and the priest. He is the leader, motivator and supporter of
this apostolate. Next are the laypersons that make up Small Ecclesial Faith
Communities (SEFC). Together they form a veritable brigade of missionaries who
visit the lapsed in their homes. Over many weeks of formation within their SEFC,
they have grown spiritually and gained a broader knowledge of the faith. Weekly
formation equips them spiritually to begin making home visits for
evangelization. Continuing to participate faithfully in ongoing weekly formation
in their SEFC is imperative. The better they are formed, the better home
visitors they make. Continuing their weekly formation sessions builds in them
confidence and zeal.
In the first two or three visits, they are unsure of themselves, as were the 72
disciples whom Jesus sent to evangelize in the villages. Home visitors reach out
to lapsed Catholics with the same feeling. Continuing spiritual formation in
SEFCs leads members to become confident missionaries unafraid to do everything
they prudently can to bring those who have ``stumbled'' into darkness ``into his
[Christ's] wonderful light.'' (cf. 1 Pet 2:9)
What qualities should home visitors have for building up community in a parish?

7. The first thing the person visited will look for is how genuine and friendly
is the visitor? What they see, what they hear and how it is said will have a
profound ``make or break'' impact on the person. The visitor must project
sincere humility. The visited, being somewhat on the defensive, will see through
any facade of faked friendliness. With a kindly voice the visitor states clearly
and simply the reason for the visit.

Sample
``I am N. N., a member of St. N. N. Parish. We are making house visits. We want to visit and
get acquainted with the people in every neighborhood in our parish ... May I
come in?''
If personal interior peace, born of holiness, shines forth from the visitor, a
promising start has been made. Showing these kinds of qualities in voice, manner
and appearance will immediately ignite confidence in the visited toward the
visitor.

8. If the person discloses that he or she is Catholic but a non-practicing
Catholic, disclosure is the core of the visit. Usually the visited will want to
talk. The visitor must project total interest in the person's story and
encourage him to go, while showing understanding and no impatience. After the
story is told, whether it is happy and grateful or sad, angry, accusatorial of
the church or priest or others, without comment and remaining completely
neutral, the visitor ventures to say kindly:
``Perhaps we can help you ... [pause] ... May we help you?''
The story out, the visited may be crying. If the visitor detects that the person
is open, the visitor shares what the Church can offer, for example a parish
mission statement, a weekly bulletin with Mass schedule, and other
information. The visitor briefly explains the CLCC process (offering a
brochure), explains what a Small Ecclesial Faith Community is, how it works,
what its goals are, and would the person be willing to come to a formation
session? Add pleasantly, ``I'll be happy to pick you up.'' Tell the invited
visited person you would like to have him or her experience a spiritual
formation session and introduce the person to all the friendly members in the
SEFC. Assure the person he will enjoy meeting the group, and that the group is
anxious to meet the person.
The offer to pick him or her up shows the visited person that you are earnest
and that you are inviting him to something he will find engaging and
enlightening. Whatever personal views or problems the visited shares with the
visitor is held in strictest confidentiality. If the person's problem centers on
an invalid marriage or is of another kind of religious or moral nature, the
visitor should encourage the person to speak with a priest, and offer, not urge,
to make an appointment for the person with the priest. All of this can and will
succeed when carried out in the framework of profound Christian charity,
gentleness, prudence and prayer. Jesus' powerful words must underlie the whole
visit, ``without me you can do nothing.'' (Jn 15:5)

9. In addition to the Matrix 12, all members of the parish are invited,
encouraged and urged to belong to a Small Ecclesial Faith Community for personal
spiritual formation,for growth in knowledge of the Catholic faith, and go in
search of ``lost sheep'' in the parish. Everyone without exception is included
except perhaps legitimately unable, handicapped persons. However, they will
profit greatly by using this CLCC Manual for formation privately at home.

10. Who are the people in a parish in need of formation in a SEFC? The
following: all regular church-going Catholics, all lay leaders of parish
organizations and commissions, the parish pastoral council, teachers in the
parish school, all catechists in the public school religion program, all
lectors, ushers, hospitality persons, cantors, musicians, choir members, all
young people from junior high school to college, the youth club, young adults
group, graduates of RCIA as part of their mystagogy, returnees to the Church,
engaged and married couples, single and widowed persons, persons considering
becoming Catholic, in short ALL people who live within the parish boundaries.
Through the catechesis given in the five stages of the CLCC process in Part I
they will receive fundamental formation in the faith and feel impelled to do the
apostolate of outreach to lapsed Catholics. In Part II, they will be introduced
to and formed in several of the Church's most authoritative teachings, so
necessary for Celebrating Life as Catholic Christians in today's world and
culture. Their collective goal of the CLCC is koinonia, building ``communion''
among the baptized for building up the kingdom of God on earth, Church of Jesus
Christ, the ``House of God.''

% ------------------------------------------------------------------------------

\section{Reflection and Discussion}
\lxRDFa{property=stage-rd-content,resource={manual}}

Introduction
1.St. Paul's statements about ``saints'' and ``holy ones,'' and the Church's
belief in the ``communion of saints'' are explicit. Do these statements suggest
an option, to be or not to be a saint, or are they a mandate coming from Jesus
himself, to ``be holy as my Father is holy?'' Reflect: do I really want to
become a saint?

2. and 3.The ``Communion of Saints!'' What elements do you find in these
paragraphs that make up this beautiful mystery? Select the one that appeals and
applies to you most. In quiet, reflect. Share.

``Saved'' and ``Non-saved''
4. and 5.Would you say that a relatively small core of parishioners are active
in parish apostolates, and that the majority settle for being church-goers? Does
your parish have a dynamic outreach apostolate to restore lapsed, non-church
going Catholics to the faith? Rate this apostolate:important, very important; we
have other priorities, it's too difficult, who will do it?In quiet time listen
to the Holy Spirit. Share.
What is meant by the expression ``the saved saving the saved?''
6.Do you buy Pope John Paul II's idea of New Evangelization capsulized here:
that parishioners in SEFCs should have a ``missionary spirit,'' motivating them
to go out to those in need to hear the truth of faith and be formed in it and by
it? Do you feel the Holy Spirit is touching you with this kind of spirit?
Prayerfully reflect, how can I develop and do it?

7.Bishops, pastors and lay leaders are greatly concerned about low Mass
attendance and the indifference which many adult Catholics exhibit toward the
faith and the Church of their youth. Many reasons exist for this tragedy. How
should parish leaders react: be complacent, feel it's a hopeless situation, do
something with vigor? Fallen-away Catholics are like the lost coin and lost
sheep in the Gospel. Should we search for it till it's found or put it off for
``later?'' Reflect, what should be the feeling of SEFC: be committed to do it,
advise the pastor to do it? Share.

8.Four words give the CLCC process uniqueness and distinction in the
Church. They are ``form,'' ``conform,'' ``reform'' and ``transform.'' They are
words often used by John Paul II in his various teachings to the Church. They
connotechange. Change is first for people in SEFCs from being indifferent to
enthusiastic, from mediocrity to excellence, from ``let Joe do it,'' to ``Let's
do it with the pastor's blessing!'' Secondlychangeis for Catholics outside of
the Church to come back to the Church, to move from dormant faith to vibrant
faith, to rise from ``penny catechism'' to adult catechesis as given, for
example, in the CLCC process. Reflect: Am Ichanging? Am I instrumental in
bringingchange? Am I zealous for the conversion of lapsed Catholics? Praise God.

Work Cut Out
1. 2. and 3.Consider Stage 4 and the work cut out for us in our
SEFC. It'soutreach. Reflect on the three great outreach models given in the
gospels. Christ was speaking of himself. It was he who initiated the outreach in
each case. We cannot expect the lapsed in general to return to the Church
without some outside initiative. I, we, must be the initiators. Inexpressible
joy even in heaven will mark their return to Christ. Think how blessed you are
to do the work of outreach! Thank the Lord. Share.
A. Home Visitors
1.As people grow older, they mature in wisdom. So too Catholics who grow in the
faith are impelled to do greater work in their parish. Running bingos, doing
bake sales and collecting pledges for fund drives have a value for sure. But
people who do these things begin to have a hunger for the ``red meat'' of
ministry. Spiritual and catechetical formation in SEFCs leads to seeing the need
of more fundamental pastoral action and the desire to do it. Home Visitation is
one such ``red meat'' ministry. Reflect on doing it; the good you will do for
others; the encouragement and example you will give to the timid and hesitant,
and the joy heaven will have over one sinner who repents. (cf.Lk15:7)

2.God has generously endowed us with many gifts to use for ourselves and for
building up his kingdom on earth, the Church. We may not ``bury'' our gifts. We
must use them. What better way than to reach out to non-practicing Catholics and
help them returnto the Church? Be alert. The Evil One never rests and will lay
obstacles and excuses in your path. Overcome them! Pray, share!

3.Kindness, gentleness and perseverance in making repeat visits will pay
off. Nothing influences people more than another person showing courage and
commitment to his conviction. Cold, defensive and resistant hearts are moved by
compassion, love and dedication. These are the breakthrough virtues of home
visitors. Prayer, personal and within your SEFC, does the rest.

4.So that returnees to the Church become steadfast in the faith they are urged
to remain in the security of their SEFC where prayer, the strength of bonding,
mutual trust and support are assured. This iskoinonia, ``communion,'' the common
union of God's People with each other in faith and love of God. This is ministry
of substance. Thank and praise God you are a part of it. Share how you feel you
are part of it.

5.Living a truly authentic Catholic lifestyle by members of SEFCs is a powerful
and exemplary teaching tool for returnees to the faith. They need the good
example of faithful Catholics who attend Sunday Mass, receive the Sacraments,
some daily, who live the 10 Commandments faithfully, who have, raise and educate
their children according to the plan of God and Church teaching, and who train
them to be faith-filled, productive, patriotic Catholic citizens. For many
returnees to the faith, conversion may be a radical lifestyle change. Reflect:
am I a good example in these areas of Catholic conduct?

B. Home Visitors Are Evangelizers
1. and 2.You are familiar with Jesus' last mandate to his Apostles before he
ascended to heaven:go and evangelize. The mandate applies not only to the
Apostles, bishops and priests but equally to lay men and women, and in
particular to CLCC members who become home visitors. In quiet time reflect on
Jesus' mandate and its implications for you, how it applies to you and gives you
``heart'' to act. Share.

3. and 4.To act or not to act, to home visit or not to home visit is not a
choice but an obligation of the parish and a duty of responsible parishioners,
especially those in SEFCs. Ask yourself: what gives rise to this obligation and
duty? Charity! ``Love your neighbor as yourself.'' If charity is shallow, look
at incentive. Reread St. James' statement, an apostle known for saying it as it
is: ``whoever brings back a sinner from the error of his way will save his soul
from death.'' (cf.Jas5:20) Be open to the Spirit. Share.

5. and 6.Outreach to non-church going Catholics, Pope John Paul II observed, is
a ``challenge.'' It's not easy. Pope Paul VI put evangelization into a cultural
context. He said, cultures need to be evangelized because they are being
woefully de-Christianized in the name of freedom, personal rights, tolerance and
avoidance of discrimination. Prayerfully reflect on some cultural abominations
being legalized in this country that are totally contrary to the Christian way
of life. Share.

7. and 8.Personal and group spiritual formation through catechesis and outreach
to lapsed, non-church going Catholics are the two legs on which the CLCC process
stands. How else reach the lapsed if not by the ``person-to-person'' contact
that Pope Paul VI described as ``indispensable?'' Evangelization through the
Sunday homily is what he calls ``collective proclamation.'' He implies this
starts the job but does not complete it. An individual's conscience must be
reached and touched by another individual, such as a home visitor. Invoke the
Holy Spirit for courage! Share.
9.Formation in holiness is a requirement without which one cannot do ministry
effectively. One may never say, ``I've reached the top in holiness, now I'm
ready to evangelize.'' Growth in holiness is a never-ending process that moves a
person forward and upward toward God in love. To stop the growth movement is to
go backward. Holiness is the dynamic that excites home visitors to retrieve
lapsed Catholics. In quiet time reflect how the Holy Spirit is inspiring you to
do home visitation.

10. and 11.We Catholics are not short on official Church teachings. They are
issued in several kinds of documents from the Vatican, and some come from the
United States Conference of Catholic Bishops. Unfortunately the majority of
American Catholics know far too little about these documents.
Fortunately this book,Celebrating Life as a Catholic Christian, contains the
central teaching of many of them in both Part I and II. Because millions of
Catholics live in the world like wheat in the midst of weeds (cf.Mt13:24-30),
they are exposed to and influenced by the world's cultures, chief of which is
the ``culture of death,'' as stated by Pope John Paul II. Outreach to lapsed
Catholics, helping them return to Christ, is to practice the ``civilization of
love.'' It is in this way that koinonia, community, the Church of Christ's love,
is built up. ``Love one another as I love you,'' said our Lord. (Jn15:12)
Reflect deeply on paragraph 11. Share.

C. Home Visitors Are Builders of Community
1.By virtue of their baptism non-practicing Catholics continue to be ``living
stones,'' but they are stones which have fallen out of the structure of the
Church. They need to be found and re-inserted into the structure, the
``spiritual house'' which is Christ. Can you imagine anything more pleasing to
Christ than home visitors going out to find and gather up the living stones,
like pieces of a precious mosaic which have wiggled out of God's art-piece, and
then restoring them to his Body, the Church? Reflect on the wonderful apostolate
you are called to do. Praise and thank the Lord. Share your insight.

2.Good question, why are baptized Catholics, even non-church going Catholics,
calledliving stones? In his First Letter, chapter two, St. Peter gives the
answer in just five key phrases. Chose one, reflect on it. Invoke the Holy
Spirit for insights. Internalize your choice. Make it your own. Share.
3.The perfect descriptive word for non-practicing Catholics is ``stumble.'' They
have stumbled because ``without faith'' (not always their fault), they did not
have the light of God's word or the strength of God's grace. Thus, living in the
darkness of ignorance, anxiety, anger, sin and absence of grace they stumbled,
that is, fell away from the Church. Many may now be secretly looking for a way
to return ``home.'' God's spiritual house is still their house. This is the cue
for home visitors in SEFCs to act. Reflect: this ismyjob,ourjob. Share.

4.When Jesus sent out his disciples two by two to visit homes, proclaim
salvation and expel demons in Jewish villages, they were stumped when in some
places the demons didn't go away. They asked Jesus about this. His reply? Prayer
and penance are needed before the exorcism. Lapsed Catholics today are in the
millions. To succeed in retrieving these ``living stones'' take Jesus'
advice. Disarm the Tempter with prayer and penance. Then the ``found'' can
rejoin the ``communion of saints.'' Be community builders. Ask yourself: do I
pray enough?

5.Here, I suggest, that each member of your SEFC look up the text in the gospel
wherein Mary visited her aged cousin Elizabeth. Or consult the text wherein
Jesus visited Simon, the leper, in his house in Bethany before his Passion. In
these and all his visits, Jesus made good and great things happen. Believe that
good and great things will happen to the people in the homes you visit. Ask the
Holy Spirit to give you his light when you make home visits to lapsed Catholics.

6.Note, making home visits should not replace holding the weekly formation
session in spirituality and catechesis in your SEFC. Members of the Matrix and
SEFCs should do both works in the week: the formation session and home
visitation. If a choice must be made, make it in favor of the formation
session. But do not get slack on making home visits. Postponement is like a
virus. If not checked it will weaken you. The weekly formation session gets
priority. It is these sessions that keep your commitment to spiritual growth
aflame. Remember, Christ is the light. We ought not walk in darkness. (cf.1
Jn1:6; also cf.Jn1:5)


What qualities should home visitors have for building up
   community in a parish?
7.Ask yourself, am I really a loving person? Am I committed to building
community in the parish by searching out Catholics who have strayed from the
faith? Examine yourself: am I sincere, genuinely friendly, humble, willing to
make a sacrifice? Am I aware that I am an agent of the Holy Spirit to the person
I am visiting? Am I gentle, firm, confident like Jesus was with Zaccheus?
(cf.Lk19:1-10)

8.Out of this long paragraph pick a sentence or thought that has special meaning
for you as a home visitor. Reflect on the privilege the Lord is giving you to be
a ``good shepherd'' in search of lost sheep! Reflect at the same time on the
immense grace given you to be a stand-in for Christ to the person visited. Try
to project Christ, the caring Servant, in your demeanor, speech, conduct and
appearance Never patronize. Remember you are a ``disciple'' (cf. Stage 1)
``called and gifted'' with special grace who is exercising his ``prophetic
role'' (cf. Stage 3) to bring the good news of Jesus, the Redeemer, to one
gravely in need of him.

9.Try to imagine the kind of parish you would have if every able-bodied man and
woman, every young person from high school and upward belonged to a Small
Ecclesial Faith Community, all being nourished in weekly spiritual and
catechetical formation, and all fanning out to make home visits to every
household in every section of the parish. What New Evangelization this would be!
This is not a pipe dream. This is the goal of the CLCC process.
As school educators like to say. ``No child will be left behind,'' so too
committed people in SEFCs say, and mean it, no fallen away Catholic will be left
behind in this parish. We will reach them, form them in SEFCs, and by means of
God's infinite mercy count them to be among the saved in heaven. Reflect, pray
and share how strongly you feel about reaching out to ``the lapsed'' and
bringing them into the fold of Christ!

10.In this extensive, though not comprehensive, list of people, add any other
groups of Catholics who may have been missed.Celebrating Life as a Catholic
Christianis foreveryone. SEFCs are not just for super-religious or super active
persons in the parish. Nor are they for Catholics who may be unfairly considered
elites, or who, God forbid, consider themselves elites.
On the contrary, the reality of Stage 4,Home Visitation and Evangelization for
Community Building, is done by repentant sinners in search of other sinners;
sinners in need of truth and community, sinners in need of God's mercy and
forgiveness and return to the faith, sinners feeling alone and isolated in need
of welcome and warmth. Christ died on the Cross for all human beings. Reflect on
the magnitude of this truth. We are ALL fallen creatures in the hands of a
loving God who seeks to lift us up by his truth, forgiveness, mercy and love
that we may be one with him forever. Share.

% ------------------------------------------------------------------------------

\chapter{Stage 5.\ Stewards of the Eucharist}

\section*{} \lxRDFa{property=stage-main-content,resource={manual}}

Note: Stage 5 has 39 numbered segments, they form the main text. In its
Reflection and Discussion section, beginning on page 129, there are
corresponding numbered segments designed to stimulate prayerful reflection,
discussion and sharing. When you have chosen a paragraph from the main text as
the subject matter for your formation session, please refer to the like number
segment in Reflection and Discussion as they are meant to be used together.

Introduction
``Let everyone see us as servants of Christ and stewards of the
  mysteries of God.'' (1 Cor 4:1)
1. One of the great laments of the Catholic Church in the past several decades
is the millions of her members who have drifted away, no longer go to Mass, and
yet still call themselves ``catholic.'' Exodus from the Church began in the
troubled and rebellious 1960s. Whether the exodus has slowed down, or whether in
this new century significant numbers of lapsed Catholics will return to the
practice of the faith remains to be seen. Hopeful signs, however, of Catholics
in search of the true faith are appearing.
Millions of young people have gone on pilgrimage to be with the Holy Father for
World Youth Day celebrations in Denver, Rome, Toronto, Cologne, Sydney and
elsewhere. Here in the United States tens of thousands of young people, from
grade school to college, go to Washington, D.C. each year to join the March for
Life, bravely demonstrating they reject ``choice'' for abortion and defend
publicly the ``right to life.'' Thousands of college students are living
chastity in the midst of a culture of pre-marital sex. These are promising
signs. They are like tiny spring plants peering out of the still cold soil of
winter, showing the world around them their promising beauty when they bloom in
the spring and summer.

2. Another hopeful sign is the growing number of faith-filled Catholics who are
requesting pastors to have public exposition of the Blessed Sacrament in their
parishes. They want exposition in order to adore Jesus Christ during the day or
through the night, to deepen their faith in the mystery of Christ's love. Their
desire is to grow in holiness, pray for peace in the world, invoke the Lord of
mercy for the conversion of sinners, especially lapsed family members, and to
make reparation for the multitude of sins, sacrileges and outrages committed
against the Lord in this sacrament. More and more devout Catholics in all parts
of the country are known to make a long drive, if necessary, to a church that
has public exposition all day or day and night when their own parish churches
have locked doors.

3. When Jesus performed the two astounding miracles of the multiplication of
loaves feeding five thousand on one occasion (Mt 14:13-21) and four thousand on
another occasion (Mt 15:32-39), Jesus did not do this only out of compassion to
feed their bodily hunger. He performed the miracles to prefigure his institution
of the Holy Eucharist at the Last Supper, by which he would feed countless
masses of believing disciples for all ages with the food of his Body and
Blood. This is the mystery of God, the mystery of the REAL PRESENCE, Jesus, true
God and true Man, present in his body and blood, soul and divinity on our altars
in the Mass and in church tabernacles. Puzzling and inexplicable, this is the
``mystery'' of which the Lord made us ``stewards.'' It draws millions of
believers to adore it on the one hand, and on the other hand is the excuse
millions of Catholics give for leaving the Church because they no longer believe
it.

4. Recall Jesus' discourse to the crowd. He said seriously that his body was
real food and his blood real drink, and that whoever eats and drinks of it will
have ``eternal life'', that is, the person, the believer, would be raised up
``on the last day.'' (cf. Jn 6:54,55) On hearing this stupendous truth coming
from the mouth of Jesus, many believed. But many did not. They dissented. ``This
saying is hard,'' they said. Refusing to ``accept it'' and ``murmuring about
this ... many ... returned to their former way of life.'' (cf. Jn 6:60-66) Those
of faith, however, feel privileged and unworthy to be stewards of the Eucharist.

5. Consistent throughout the first four stages of the CLCC process, Stage 5 also
has a parallel. What Jesus did in Galilee, the laity re-enact in the parish. In
Stage 5, Jesus came up from Galilee to Jerusalem to carry out his final and
enduring acts of redemption. He held his Last Supper with his Apostles followed
by his Passion, Crucifixion, Death and Resurrection, all of which were climaxed
with the power of the Lord on the day of Pentecost. At the parish level, Stage 5
is carried out when all members of Small Ecclesial Faith Communities, and all
parishioners assemble to celebrate the Eucharist. Through the power of the Holy
Spirit, they take ``part in the Eucharistic Sacrifice, which is the source and
summit of the whole Christian life ... offer the divine Victim to God, and
... themselves along with it.'' (LG 11; also Ecclesia de Eucharistia, Church of
the Eucharist, EE, 1,21)

A. The Eucharist, God's Greatest Gift to Man
1. It is no coincidence that the 3,000 bishops who participated in Vatican
Council II (Years 1962-1965) made, in the words of Pope Paul VI, ``the liturgy
... the first subject to be examined,'' showing thereby its ``intrinsic worth
and importance for the life of the Church.'' Its worth and importance for the
Church's worship was carefully crafted by the Council Fathers and issued in 1963
in a document titled Sacrosanctum Concilium, SC, better known as the
Constitution on the Sacred Liturgy.

2. Part Two, Section Two, Chapter One, Article 3 (paragraphs 1322-1419) of the
Catechism of the Catholic Church treats the Sacrament of the Eucharist. It
quotes the Council's Constitution on the Sacred Liturgy and other documents of
Vatican II, especially Lumen Gentium, the Dogmatic Constitution of the Church,
where references are made to the Holy Eucharist. In his encyclical Ecclesia de
Eucharistia, ``the Church of the Eucharist,'' issued on Holy Thursday, April 17,
2003, Pope John Paul II quotes the Catechism, several documents of Vatican
Council II, many Greek and Latin Fathers of the Church, and saints who have
distinguished themselves for their writings and preaching on the Holy
Eucharist. Thus, we, the Church, are gifted with an enormously rich body of
authentic and beautiful teachings on the Holy Eucharist. They are to be read,
studied and applied to ourselves first for advancing in personal holiness and
for building community,koinonia, in the Church and the world through the dynamic
of ``communio.''
The Eucharist is God the Father's supreme gift of his beloved Son to us
3. How do we know this? We have been raised to the dignity of royal priesthood
by Baptism, have we not? Have we not been configured to Christ in the Sacrament
of Confirmation? In the Eucharist we participate in the Lord's own sacrifice
(cf. CCC 1322), the most central action of this mystery. Active participation in
it began to take shape on Holy Thursday evening ``at the Last Supper, on the
night he was betrayed,'' when the Father's Son, our ``Savior instituted the
Eucharistic sacrifice of his Body and Blood.'' ``This he did in order to
perpetuate the sacrifice of the cross throughout the ages, until he should come
again, and so to entrust to his beloved Spouse, the Church, a memorial of his
death and resurrection: a sacrament of love, a sign of unity, a bond of charity,
a Paschal banquet, in which Christ is consumed, the mind is filled with grace,
and a pledge of future glory is given to us.'' (CCC 1323; emphasis added)
Participation by baptized believers in this memorial is then not only a gift of
unfathomable value but the monumental truth by which the Church exists, acts and
worships.

4. In his encyclical, Ecclesia de Eucharistia, the fourteenth of his
pontificate, Pope John Paul II titled the first chapter ``The Mystery of
Faith.'' Indeed, the Holy Eucharist is a ``mystery,'' a reality that the human
mind cannot fathom, but by the gift of faith accepts and believes. The Holy
Father emphasizes that the Eucharist is a sacrifice, a sacrifice ``marked
indelibly by the events of Jesus' passion and death.'' The Eucharist is not just
a ``reminder'' of these events but the ``sacramental re-presentation'' of his
sacrifice on the Cross. That this ``sacramental re-presentation'' is a mystery
is proclaimed at every Mass. After the consecration does not the presiding
priest call out to the assembly, ``Let us proclaim the mystery of faith,'' to
which the assembly responds enthusiastically with one of the approved
acclamations such as: ``Dying you destroyed our death, rising you restored our
life, Lord Jesus, come in glory.''
``When the Church celebrates the Eucharist, the memorial of her Lord's death and
resurrection, this central event of salvation becomes really present and 'the
work of our redemption is carried out.''' (EE 11; also cf. LG 3) This sacrifice
is so focal to the whole plan of man's salvation, that Christ would not think of
returning to the Father ``only after he left us a means of sharing in it
[emphasis added] as if we had been present'' beneath the Cross on Calvary. The
``means of sharing in it'' is the Eucharist. It is ``the gift par excellence.''
It is the person of Christ in his sacred humanity. It is the gift of himself as
God-Man. It is ``the gift of his saving work'' that ``transcends all times.''
(cf. CCC 1085) It is the ``inestimable gift,'' the Mystery of Faith. It is also
the mystery of mercy, God showing his love for man ``to the end'' (cf. Jn 13:1),
a love which knows no measure. (cf. EE 11)

Absolutely central to the Holy Eucharist is its sacrificial meaning
5. Sacrifice and its meaning must be fully understood and appreciated. The
Eucharist is at once a real sacrifice and a sacrament. The sacrifice on Calvary
happened once. The sacrifice on our Catholic altars takes place repeatedly. It
is a re-presentation of Christ's one sacrifice on the Cross. It is through the
Eucharistic re-presentation, the Mass, that the sacrifice of Christ on the Cross
is visibly enacted on the altar. By this Eucharistic sacrifice Christ reconciles
men and women to himself in every age. Thus, ``the sacrifice of Christ and the
sacrifice of the Eucharist are one single sacrifice.'' (CCC 1367)
The one sacrifice is also the sacrament, the sacrifice on the Cross
repeated. What appears to be ``bread'' is`` living bread'' (cf. Jn 6:51; also
cf. 6:35,48), the real Body of Christ. What appears to be ``wine'' is Jesus'
Sacred Blood. In appearance, taste and smell, the sacramental Body and Blood of
Jesus look like bread and wine. They are the sacrament, the visible sign. In
this sign Jesus makes his real Body and Blood eatable and drinkable. Only
through the eyes of faith do we know that the sacrificial meaning, the
Eucharist, is really and truly the Body and Blood of Christ.

What about the ``real presence?''
6. Practicing Catholics of faith deeply revere the real presence of Christ in
the Eucharist. It is an awesome reality. They firmly believe that Jesus Christ,
true God and true Man, resides in this mysterious form. Some, however, harbor
doubts about it, others tend to disbelieve it and others deny it outright as
myth. What is the truth? Is Christ really and truly present in the Eucharist or
is he not?
The Catechism of the Catholic Church identifies at least ten ways by which
Christ is present in his Church. The most outstanding way is that ``he is
present most especially in the Eucharistic species.'' (cf. CCC 1373) His
presence is ``unique.'' In the Sacrament of the Eucharist his presence is
superior to all other sacraments because all other sacraments tend to it. In the
most holy Eucharist the Lord Jesus, ``body and blood, together with soul and
divinity,'' the whole Christ is truly, really and substantially
contained. ``Real'' means that he is present ``in the fullest sense: that is to
say, it is a substantial presence by which Christ, God and man, makes himself
wholly and entirely present.'' (CCC 1374; also Paul VI, Mysterium Fidei 39)

Transubstantiation! What is it?
7. The word is heard in homilies, used in catechetical works, and taught in RCIA
and bible studies. What exactly is it? Transubstantiation is the complete and
total ``conversion'' of bread and wine into the body and blood of Christ. When
Jesus spoke the words at the Last Supper, ``This is my Body'' over the bread,
and ``This is my blood'' over the cup of wine, he was speaking as God and
Man. His human words had the power of God. A complete change, a ``conversion,''
of the material bread and wine took place. (cf. CCC 1375) The power of Jesus'
words was witnessed by the crowds when for example he said to the paralytic,
``stand and walk.'' He got up and walked.
When Jesus speaks the words of institution (``consecration'') over the bread and
wine through the voice of the priest, the bread and wine are changed, that is,
substantially ``transformed.'' (cf. CCC 1375) The bread becomes Christ's Body;
the wine becomes Christ's blood. What we see are only ``visible signs,'' the
sacrament. What we do not see with our eyes but believe with all our hearts and
minds is that Jesus makes himself present on the altar, his body and blood, his
soul and divinity. This total and complete change, ``conversion,'' is called
transubstantiation. (CCC 1376)
This conversion is the ``ultimate Mystery of Faith.'' It ``surpasses our
understanding and can only be received in faith.'' St. Cyril of Jerusalem
exhorts his congregation, ``Do not see in the bread and wine merely natural
elements, because the Lord has expressly said that they are his body and blood:
faith assures you of this, though your senses suggest otherwise.'' (cf. EE 15)

The next Mystery of Faith, Communion
8. The Eucharist is often referred to as a banquet. Is this metaphorical
language? Is Holy Communion a banquet? We know that the Eucharist is God's
greatest gift to man. It is God the Father's gift of his only-begotten Son,
Jesus, to us. Holding the bread, he said to the Apostles, ``take and eat...''
Holding the cup of wine, he said, ``take this and drink...'' Jesus' Body and
Blood on the altar create a ``real presence'' to us. But they were not meant to
sit there, but to be taken, eaten and drunk. And the eating and drinking were to
have an effect on those who ate and drank. The effect would be a ``saving
efficacy.'' (EE 16)
To receive the Body and Blood is called communion, to have union with Christ, so
intimate that there is no comparable union on earth. The saving efficacy is
expressed in these words of Jesus at the institution: ``It [my blood] will be
shed for you and for all so that sins may be forgiven.'' Familiar words?
In Holy Communion ``we receive the very One who offered himself for us; we
receive his body which he gave up for us on the Cross and his blood which he
'poured out for many for the forgiveness of sins.''' (cf. Mt 26:28; EE 16) This
means that Eucharistic Communion forgives venial sins, nourishes us to avoid all
sin and strengthens us to strive for greater holiness. (CCC 1394) How necessary
it is to recall these words of Jesus: ``Just as the living Father sent me and I
have life because of the Father, so also the one who feeds on me will have life
because of me.'' (Jn 6:57) The Father is ``life'' and the Eucharist is divine
food which Jesus effected by the Holy Spirit. To receive communion is therefore
to receive the eternal Godhead, the Blessed Trinity, God in his indivisible
Oneness. Thus, the Eucharist is a true banquet. ``Amen, amen, I say to you,
unless you eat the flesh of the Son of Man and drink his blood, you do not have
life within you.'' (Jn 6:53) Clearly this is no metaphorical food. Jesus said,
``For my flesh is true food, and my blood is true drink.'' (Jn 6:55; cf. EE 16)

B. Stewardship of the Eucharist
Responsibility to oneself
1. This Stage of the CLCC process is titled Stewards of the Eucharist. Intrinsic
to the concept of stewardship is the all-important element of responsibility,
for stewardship and responsibility are twin realities that are tightly
linked. What responsibilities do Stewards of the Eucharist have? The first is to
themselves. They must be aware of what they are doing when they receive the Body
and Blood of Christ in Holy Communion. Specifically they must be without serious
sin, alive in the state of grace. They should have other dispositions too:
repentance, love of God and full awareness of what they are doing. They must
believe firmly in the gift of the Eucharist, have sorrow for even venial sins,
fast at least one hour before receiving Holy Communion, be mindful of the ONE
they are receiving and do so with reverence. They should be respectfully
attired, and after receiving the Lord make the sign of the cross devoutly. Upon
return to their places, each communicant should make a loving thanksgiving.

Responsibility to others
2. The second responsibility is to others. We must be aware that we belong to
the earthly branch of the tri-level Communion of Saints. We belong to the
worldwide body of over one billion baptized Catholics on journey to heaven. Our
second responsibility therefore is to the vast mass of baptized Catholics within
that body who have left the Church and no longer go to Mass. Stewardship of the
Eucharist requires us to reach out to non-church-goers. They must be made aware
of the unsurpassable gift they are missing, and who either out of ignorance,
doubt or even denial do not realize that their salvation is in grave jeopardy.
Stewardship of the Eucharist demands that on our pilgrim way to eternity, we try
to plant the seeds of conversion to Christ in those in need of conversion,
reaching out to them, bringing them into a Small Ecclesial Faith Community for
re-introduction to the Mystery of Faith and the Holy Eucharist. (cf. CCC 1396)
Each time we receive the Eucharist we should strengthen our sense of
responsibility for helping lapsed Catholics return to Christ, and be especially
eager to help those outside of the Church who are sincerely seeking salvation to
come into the Church of the Eucharist. (cf. EE 20)

3. What needs to be remembered is that Catholics who are lapsed are not always
personally at fault. They may be victims of injustice and discrimination. They
may be de-humanized through terrible poverty, made powerless and victims of
exploitation with almost no hope for a future. They need the gifts given at the
Last Supper. They need their faith and hope awakened in Christ in the
Eucharist. They need ``washing of the feet,'' the symbol of cleansing in order
to have a place in Christ's kingdom. ``Cleansing'' begins in Small Ecclesial
Faith Communities. Here the lapsed are wanted, welcomed and ``rehabilitated''
spiritually, even materially when possible, receiving catechesis from caring
fellow Catholics, love, and a way for re-entering the Church.

4. As responsible stewards of the Eucharist, attention must be given to low Mass
attendance and to action that will correct it. One effective action is
Celebrating Life as a Catholic Christian, the CLCC process. One American
Cardinal called this process ``the best'' in his opinion. The CLCC process is
not just subjectively centered on an individual's personal spiritual growth, but
is simultaneously focused outward. It reaches out to baptized Catholics who have
strayed from the Church. It seeks them out, and strives to help them celebrate
their lives as Catholic Christians. The reasons for low Mass attendance are
many. The first is wholly deficient training of young people in their formative
years in solid, authentic religious education; the woeful lack of doctrinal
substance in many catechetical materials; religion teachers giving fundamental
truths contained in the Ten Commandments, the Creed, the Sacraments, especially
Penance and Eucharist, hurried treatment, and failure to give students strong
training on the importance, need and power of prayer.
Also many seminaries have failed to give its students for priesthood in the late
60s up to the early 90s a vigorous training in orthodox theology and
ecclesiology. Many seminarians did not receive a firm foundation in philosophy,
in the discipline of liturgy stated explicitly in Sacrosanctum Concilium, SC,
Vatican II's Constitution on the Sacred Liturgy and clarified in numerous
documents issued since then by the Vatican for uniform worship and the
correction of irregularities which creep into sacramental
celebration. Fortunately many seminaries today give their students a rigorous
training in all the curricular disciplines, in total fidelity to the Church, the
Magisterium and inculcate respect for and obedience to the Holy Father, their
local bishop and major superiors.

5. Other reasons responsible for deterioration of the already weak faith of
millions resulting in departure from the Church are Entertainment Masses and
Self-Celebration Masses. These in the long run scandalize rather than edify the
People of God. Some priests and parish liturgical committees are ``sold'' at
times on bringing a comedy atmosphere to the holy Sacrifice of the Mass. It's a
way, they believe, to attract children, keep their attention and instill in them
a good feeling toward Church-going, and lessen the hassle parents have with kids
who do not want to go to Mass. The kids say they're bored, won't open their
mouths to sing or respond to prayers, say it's too long (60 minutes at most) and
that they don't get anything out of devoutly celebrated Masses. But three to
four hours of TV every day at home is too little. Entertainment Masses are those
where the celebrant is the center of attention, not the Lord present on the
altar or in the tabernacle. They are Masses where the presiding priest doesn't
miss a chance to say something cute or funny whether in the liturgy of the Word
or the liturgy of the Eucharist to get a laugh, even applause to make the people
``feel good.''
Another kind of self-celebration Mass is when the assembly celebrates itself. We
are Church! Jesus is in us and with us (``where two or three are gathered....'')
We are gathered to celebrate oneness and we top it off with a communal meal,
meaning Holy Communion. Might this idea of Eucharistic worship remind us of the
parable of the self-glorifying Pharisee and the self-effacing tax-collector, the
Publican, worshiping in the temple? (cf. Lk 18:9-14) Masses which are banal,
trivialized and exhibitionistic, are they building up faith or tearing it down?
In the measure that transparent holiness at Mass decreases, absence from Mass
increases.
In Ecclesia de Eucharistia, the Holy Father speaks sadly of ``shadows'' that are
blurring the ``lights'' which shine in Eucharistic celebrations. He speaks of
the need to banish effectively ``the dark clouds of unacceptable doctrine and
practice, so that the Eucharist will continue to shine forth in all its radiant
beauty.'' He states that it is his hope that his Encyclical Letter will
correct the ``abuses [that] have occurred in various parts of the Church.''
(cf. EE 10). The only way non-practicing Catholics can and will return to the
Church is when Mass becomes again an experience of transcending beauty, an
experience in faith and love, devotion to and reverence for the real presence of
Christ in the Eucharist.

Return to the Church will assuredly begin when the Eucharist is preached often
from personal conviction and sincere belief. Like the tax collector many lapsed
Catholics are looking for the Lord's loving mercy to cleanse their guilt-laden
hearts. They will return to the Lord with repentant hearts. They plead for
forgiveness and it is given them. They thank him for his goodness, and he
accepts it. They want the warmth of the Savior's love, and he returns it a
hundred fold. This is Mass and this is what the Lord does in the Mass if we let
him.

6. Not a few Catholics who were poorly educated in the faith have adopted some
trendy heresies that militate against belief in the Holy Eucharist, either
directly or indirectly. These heresies also contribute to departure from the
Church. One is the mistaken idea of ``freedom'', and in particular freedom of
conscience. Today the notion of freedom has gone wild. Great numbers of
Catholics think freedom means one can do as he pleases. They reject the true
meaning of freedom that one can and should do what is right. Out of control
freedom is largely the cause of crimes committed by adults and youth. It is also
the cause of abuse of the sacred institution of matrimony, the widespread
breakdown of family, and the practice of grave moral and social evils such as
artificial contraception, masturbation, abortion, active homosexuality and same
sex partnerships. The liberal notion of freedom, they feel, allows them to break
the laws of God with impunity. For example, couples feel free to live together
shamelessly in public sin before marrying, giving untold scandal to the young
and painful sorrow to older people especially God-fearing, moral parents.
Co-habitating couples who marry are off to a shaky start as studies show. Many
receive the sacrament of matrimony in sin because they do not go to confession
before their wedding. Without the aid of God's grace, these marriages have only
about a 50\% chance of being permanent and happy. They face a 50\% prospect of
divorce and all the societal ravages that flow from divorce. Children of such
marriages are often abused, neglected and made to feel unwanted. Many grow up
severely maladjusted and emotionally immature. Many launch into early sexual
activity, into drugs, become rebellious toward parental authority and defiant of
civil authority. Not all separations, however, are the result of a misguided
notion of freedom. Other factors enter that make marital separation, divorce or
annulments justifiable in order to preserve sanity and human dignity. (cf. CCC
2384)



7. The second heresy is denial of sin. Having been told or taught that there is
no serious sin, no punishment for sin, no hell only heaven, and no matter how
grossly sinful one may be, God's love and mercy prevail, and salvation is
certain. THIS IS NOT TRUE. Did not Jesus, the God-Man come to earth specifically
to forgive men's sins and save man from the eternal death of damnation? In the
Eucharistic Sacrifice of the Mass does not Jesus say most solemnly at the
consecration, ``this is the cup of my blood...It will be shed for you and for
all so that sins may be forgiven?'' Related to the denial of sin is the
destructive popular heresy of tolerance. No matter what people do or say
regardless how corrupting it is to the fabric of society, how offensive to God,
how evil and repugnant to decent persons, as long as it is not legally criminal,
look the other way. Be tolerant! Live with ``diversity!''

8. As informed and compassionate Catholics, we may not accept the gay and
lesbian lifestyles as normal. Yet powerful, gay political pressure is being
forced upon us. Homosexuality means that certain men and women ``experience an
exclusive or predominant sexual attraction toward persons of the same sex.''
Homosexual actions between two men or two women is according to Sacred Scripture
a ``grave depravity,'' and therefore ``intrinsically disordered.'' (CCC 2357;
cf. Gen 19:1-29; Rom 1:24-27; 1 Cor 6:10; 1 Tim 1:10; also cf. Lev 18:22; 20:13)
Such actions are ``disordered'' because they close off the possibility of
transmission of life that is achievable only through the ``sexual
complementarity'' of husband and wife. (CCC 2357) Thus, homosexual activity may
not be tolerated, and under no circumstances approved. Same-sex inclination is
present in many persons and is for them a real ``trial.'' ``Every sign of unjust
discrimination ... should be avoided, and respect, compassion and sensitivity
should be shown them, always careful not to give the least sign of
endorsement.'' (CCC 2358)
Like everyone else homosexual persons are called to chastity. No matter how
strong the inclination, it can be brought under control through honest
self-evaluation, freedom of will to want to choose between good and evil, and
through the powerful strength of God's grace obtained by daily prayer,
sacramental confession and Holy Communion. Homosexual persons can and must
travel the road of Christian perfection like every other Christian. (cf. CCC
2359)

9. As responsible Stewards of the Eucharist, we have an obligation to defend our
true faith and Church against the influences of a third heresy, religious
relativism, yet another cause of Catholics defecting from the Church. We are
irrevocably members of the one, holy, Catholic and apostolic Church which Christ
founded and preserves through his divine power and loving protection. Despite
countless assaults, plots and persecutions in fierce efforts to destroy her over
her two thousand year history, the Church stands firm and indestructible as
Jesus promised, ``The powers of hell shall not prevail against her.'' (Mt 16:18)
Religious relativism claims all religions are equal, that one religion is as
good as another. Skeptics claim also that ``salvation,'' if really necessary,
can be attained without Christ. This is gross heresy. To refute popular errors,
the Vatican's Congregation for the Doctrine of the Faith issued a Declaration,
titled Dominus Iesus, August 6, 2000, On the Unicity and Salvific Universality
of Jesus Christ and the Church (DI).
What does this mean? That the Church is One in Christ and that salvation for all
mankind is through Christ alone and no one else and in his Church. ``Christ
... Jesus of Nazareth ... is the Word of God made man for the salvation of
all.'' (DI 10) ``Since Christ died for all, and since all men are in fact called
to one and the same destiny, which is divine, we must hold that the Holy Spirit
offers to all the possibility of being made partners ... in the paschal
mystery.'' (DI 12) St. Peter standing before the Sanhedrin in defense of healing
the crippled man said boldly, ``There is salvation in no one else, for there is
no other name under heaven given men by which we must be saved.'' (Acts 4:12; DI
13) ``The Lord Jesus, the only Savior, did not only establish a simple community
of disciples, but constituted the Church as a salvific mystery: he himself is in
the Church and the Church is in him.'' (DI 16) ``Jesus Christ continues his
presence and his work of salvation in the Church and by means of the Church.''
(DI 16) ``The Church is the 'universal sacrament of salvation.''' (DI 20)
To some ears the above statements and biblical texts may sound discriminatory,
elitist and intolerant of other religions. The Declaration is a definitive
reiteration of basic truths that the Catholic Church has preached, taught and
lived by for over two thousand years and objectors will not intimidate
her. Jesus did say, ``I am the Way, the Truth and the Life. No one comes to the
Father except through me.'' (Jn 14:6)

10. A fourth destructive influence that causes loss of faith and departure from
the Church are the Culture Wars. These are conflicts between certain Catholic
truths and secular powers and civil legislation. The most deadly conflict is
between celebrating life versus killing unborn life. It is the lead issue in the
socio-political and socio-ecclesial culture wars. The reason is that the
majority of Catholics, Christians and persons of good will are passionately
pro-life and a minority of other people equally passionate but more vocal,
organized and well-funded promotes the so-called ``right'' of women to kill
their unborn child. The conflict becomes messy when high-profile Catholic
politicians defend and vote for abortion and partial birth abortion even though
Church teaching rejects their ``dual conscience'' stance as untenable:
``Personally I'm pro-life; publicly I'm pro-choice. I can't impose my morality
on others.''
Pope John Paul II and other pontiffs say that this dualistic position may not be
held by a Catholic and still be in good standing in the Church. In his 11th
encyclical, Evangelium Vitae, The Gospel of Life, EV, John Paul II raises the
issue of ``personal conscience ... in the public sphere'' as he does in other
teachings. In a democracy the ``majority rules.'' If abortion or any other
anti-life issue is legalized, the Catholic politician must separate himself or
herself from the public position. In conscience the Catholic politician must
acknowledge ``an objective moral law which is natural law written in the human
heart'' as superior and overriding civil law. Not to do so is to espouse
``ethical relativism which characterizes much of present-day culture.'' Catholic
politicians and ordinary Catholic citizens are in conscience bound to take a
strong, consistent pro-life stand. Not to do so is to remove oneself from
communion with the Church. (cf. EV 69,70; also 57,62)
Other cultural conflicts exist: atheism vs. Catholicism, error vs. truth,
dissent vs. obedience, abortion vs. sacredness of life, public education
vs. home schooling, the culture of death vs. the civilization of love, liberal
vs. conservative, progressive vs. traditionalist, war hawks vs. peace doves,
radical feminists vs. white males. More could be added. The reality is that
culture wars are pervasive and impact powerfully all persons, adults, young
people, seniors and children. Let's be bold and wise. Let's be Stewards of the
Eucharist. All the above cultural conflicts challenge us to be responsible
stewards of the Eucharist. We have Christ, the God Man, He is with us. He gave
himself to us. Therefore who can be against us? Let us reach out to help the
lapsed, religiously illiterate and excommunicated Catholics return to the Church
and the Eucharist. Celebrate Life as Catholic Christians!

C. Eucharist and Catholic Worship
1. In order to worship in faith and receive Holy Communion in the state of
grace, it is highly recommended that the Sacrament of Penance and Reconciliation
be received frequently. If a person is in grievous sin he is obliged to confess
his sins and be in the state of grace before receiving Holy Communion. Only when
in the state of grace does Holy Communion benefit the communicant
spiritually. To receive Holy Communion while in known mortal sin the communicant
commits sacrilege, ``the profaning or treating unworthily the sacraments.''
Sacrilege is most grave ``when committed against the Eucharist ... the sacrament
of the true Body of Christ made substantially present to us.'' (CCC 2120)
Prior to confession it is necessary to examine one's conscience. This means to
recall and know the nature of one's serious sins. Examination of conscience is
also useful in preparation for confession of venial sins. To grow in holiness
frequent confession of venial sins and faults increases grace in the soul. What
can be more remedial than to know that through the absolution that the priest
gives, it is Christ himself who forgives our sins? And he does so as often as we
receive the sacrament. The grace received in this sacrament helps us grow in
love of God, love of neighbor and in holiness of life. Parishioners who belong
to Small Ecclesial Faith Communities are urged to receive the Sacrament of
Penance often for their own spiritual benefit first of all. They also give good
example to lapsed Catholics who have come into a SEFC. By example returning
Catholics will gain some idea of the value of frequent confession, and the joy,
peace and blessings it gives. Purification and sanctification of the person
through this sacrament makes Catholic worship of the Eucharist more meaningful
and spiritually fruitful.

2. The basis of Catholic worship is the first of God's 10 Commandments given to
Moses for his people on Mt. Sinai. The traditional formula of the First
Commandment is: I am the Lord your God, you shall not have strange Gods before
me. Lengthier texts of the First Commandment are given in Exodus 20:2-17 and
Deuteronomy 5:6-21. (cf. CCC pp. 496-497) The formula above is the approved
synthesis of the two longer texts. The 10 Commandments are called the
``Decalogue,'' the ``ten words.'' The Pharisees and Sadducees, in animated but
inconclusive discussion among themselves about the Decalogue, decided to test
Jesus by asking, ``Teacher, which commandment in the law is the greatest?''
Jesus' reply to the ``scholar'' who asked the question, ``You shall love the
Lord, your God, with all your heart, with all your soul, and with all your
mind. This is the greatest and the first commandment.'' (Mt 22:37-38). St. Luke
in his gospel adds ``...and with all your strength.'' (Lk 10:27) Worship of God
therefore is worship of the God of love with total love.
When the Evil One tempted Jesus on the mountain showing him ``all the kingdoms
of the world in their magnificence'' and bribing him said, ``All these I shall
give to you, if you prostrate yourself and worship me.'' With scorching disdain,
Jesus said to him, ``Get away, Satan! It is written: 'The Lord, your God, shall
you worship and him alone shall you serve.''' (Mt 4:8-10, Jesus citing Deut 6:13)

Worship is an act of religion (cf. CCC 2096)
3. Broadly defined religion is a person's relationship with God. It consists of
belief in him (faith), placing confidence in his goodness, trusting he will care
for and protects us (hope), and loving him, uniting oneself with him through
prayer and service (charity). These fundamental virtues, the top virtue being
charity, require, in fact demand, that human creatures render to God what is
owed him as Creator in justice. We owe God first of all adoration. To adore God
means ``to acknowledge him as God and Savior, as Lord and Master of everything
that exists, and thank him for his infinite and merciful love.'' (CCC 2096)
Worship includes adoring God with the right dispositions, namely with ``respect
and absolute submission'' realizing the ``nothingness of the creature'' who
would not exist were it not for God. Worship means to adore God with praise,
exalting him and humbling oneself before him, expressing gratitude for all the
great things he has done and does for us each moment of every day. To worship
God turns us outward from ourselves setting us free to overcome temptations to
self-importance and reject the enticements and idolatry of the world. (cf. CCC
2097) Worship includes also key components such as prayer, vocal and mental,
meditation and contemplation, veneration of the Blessed Virgin Mary and the
saints, and above all adoration of the One God in Three Persons of the Trinity,
God the Father, God the Son and God the Holy Spirit.
4. The Catholic Church believes strongly in what takes place at the consecration
of the Mass. Christ gives himself wholly and truly to his Church in the
Eucharist. Millions of martyrs have died in defense of this great truth. The
truth is that Christ himself instituted the Eucharist and is himself, ``the
source and summit of the Christian life.'' (cf. LG 11) Why does the Church
believe so strongly in this mystery that takes place at the consecration?
Because on the authority and power of Jesus' words an extraordinary change takes
place, a change of the whole substance of the bread into the substance of the
body of Christ our Lord and of the whole substance of the wine into the
substance of his blood. This change is properly called
transubstantiation. (cf. A.7. above; also CCC 1376)

The Holy Eucharist, therefore, is the ``sublime cause'' which unites divine life
with human life. The Eucharist brings baptized believers into oneness with
Christ, the God Man, and with each other. It is the Eucharist that builds ``a
spiritual edifice,'' that unique and holy community called the Church. The
Eucharist ``is the culmination both of God's action of sanctifying the world in
Christ, and the worship men offer to Christ, and through him worship the Father
in the Holy Spirit.'' (cf. CCC 1325) Because the Eucharist is ``the source and
summit of the Christian life,'' it demands our most devout and universal worship.

5. Central to devout worship of the Eucharist, whether in the Mass and outside
of Mass, is that we express our belief that Christ is really present in the
signs of consecrated bread and wine. Reverence towards these sacred signs
demands that we genuflect or bow profoundly on entering and leaving a Catholic
church. When crossing before the tabernacle where the Sacrament is reserved a
genuflection or deep bow should be made. These are signs of adoration of the
Lord. When the Eucharist is exposed on the altar for solemn public veneration,
or carried in procession, utmost care must be given the Sacrament that it be
adored appropriately by all, clergy, altar servers, worshipers in the pews or
those standing on the street if the Eucharist is carried in public to another
place for adoration as perhaps on the solemn feast of Corpus Christi. (cf. CCC
1378)

6. Worship of the Eucharist is first and foremost a celebration of the community
of believers who acknowledge and adore the Lord Jesus in his real presence. The
Constitution on the Sacred Liturgy states it in this way. ``The Church has never
failed to come together to celebrate the paschal mystery ... the Eucharist in
which the victory and triumph of his death are again made present and at the
same time giving thanks to God for his unspeakable gift of Christ Jesus 'to the
praise of his glory,' through the power of the Holy Spirit.'' (SC 6) In fact,
the Constitution also states: ``every liturgical celebration, because it is an
action of Christ the priest and of his Body, the Church, is a sacred action
surpassing all others. No other action of the Church can match its claim to
efficacy, nor equal the degree of it.'' (SC 7) Without doubt, therefore, Christ
is the principal agent and center of Eucharistic worship. ``Christ always
associates the Church with himself in the truly great work of giving perfect
praise to God and making men holy. The Church is his dearly beloved Bride who
calls to her Lord, and through him offers worship to the Eternal Father.'' (SC 7)

D. Eucharist and the Celebration of Sunday
1. What is the primary day for worship of the Eucharist? Sunday, the Day of the
Lord! The third commandment of God is expressed by this traditional catechetical
formula: Remember to keep holy the LORD'S Day. (CCC page 496) On July 7, 1998
Pope John Paul II issued an apostolic letter to the Universal Church titled Dies
Domini, the Day of the Lord, DD. In it he speaks of the Day of the Lord, the Day
of the Church and the Day of Man.

Sunday, the Day of the Lord
2. After creating the world God celebrated his creation. It was the seventh day,
a day of rest, (Gen 2:2), not a day of idleness or divine ``inactivity,'' but a
``day of divine contemplation.'' He ``gazed'' upon the beauty of his
achievement. (DD 11) His day of rest was also a ``gaze'' into the
future. ``All-knowing'' he saw the ``glory of the risen Christ'' in ``the
festival of the 'new creation,''' when ``the Word was made flesh in the fullness
of time.'' (Gal 4:4; DD 8) The inseparable unity of Father, Son and Holy Spirit,
the One God in three divine Persons, was at work. His work was fully revealed in
the paschal mystery, in Jesus, his God-Man Son, in his passion, death and
resurrection. Thus the divine creating Father had much to celebrate ``at the
beginning'' and continues to celebrate till the end of time. (cf. DD 8)

``God blessed the Seventh Day and Made It Holy.'' (Gen 2:3) The ``seventh day''
is therefore special. The all-holy One made it ``holy'' in his creative plan. It
is holy because it defines and indelibly expresses God's relation to man and
man's relation with God. Furthermore the Day was ``announced and expounded by
biblical revelation.'' This means that God explicitly stated it was ``holy'' in
his revealed word. (cf. DD 13) Because the Seventh Day is holy, it is also a day
of ``rest,'' a day set apart from other days, (DD 14) and a day of
``remembering.'' ``Remembering'' centers on the grand work of God in his
creation, and on God's creatures nourishing their religious life on the ``day of
rest.'' (cf. DD 16) ``Remembering'' includes ``the work of liberation
accomplished by God in the Exodus, '' (cf. Deut 5:15; DD 17); and remembering
the ``new beginning'' which Jesus inaugurated by his death and resurrection, the
``festive day ... on which the Lord rose from the dead.'' (DD 18) As responsible
Stewards of the Eucharist, we remember to keep holy the Lord's Day.

Sunday, The Day of Christ, Dies Christi
3. Even though Sunday is the Lord's Day, a day of ``rest'' which is ``holy,''
Sunday is the venerable day for celebrating the Lord's resurrection. Christian
Sunday leads the faithful each week to ponder and live the event of Easter, the
true source of the world's salvation. (cf. DD 19) Sunday has the distinction of
being the first day of the week. After the resurrection that occurred on ``the
first day after the Sabbath,'' the apostles began to celebrate the first day of
the week as the day upon which the faithful gathered ``for the breaking of
bread.'' (Acts 20:7) It was also the day when Paul restored the young man
Eutychus to life who fell asleep while listening to one of Paul's hours long
sermon and tumbled to his death from a third story window where he was
sitting. (cf. Acts 20:8-12; DD 21) Besides being called the first day, Sunday is
also called ``the eighth day.'' The seventh day being the Sabbath, the next day,
Sunday, became the ``eighth day.'' It is an image of eternity when Christians
consider ``the age to come,'' eternal life. (cf. DD 26)
On Saturday night of Holy Week, the first part of the Easter Vigil is the
``Service of Light.'' Fire is blessed and the Easter Candle, a symbol of Christ,
is lighted. Carried up the aisle in a darkened church, the priest/deacon lifts
high the candle and sings three times, Christ our Light. The people sing, Thanks
be to God. The candle symbolizes the light of early dawn on Easter Sunday
morning when Jesus, who said, ``I am the Light of the world,'' (Jn 9:5) rose
from the tomb.
Pentecost is holy and solemn because it was on Sunday, the first day of the
week, when Christ sent his Spirit. It is ``remembered'' because the Holy Spirit
came upon the Apostles and Mary gathered in the ``upper room'' (Acts 1:13) in
the form of ``a mighty wind'' and ``fire.'' (Acts 2:2-3) ``Pentecost is the
founding day of the Church, the mystery which forever gives life to the
Church.'' (cf. DD 28 ; John Paul II, Dominium et Vivificantem, The Holy Spirit
in the Life of the Church and the World, DeV, 22-26). Sunday is ``the Day of
Faith.'' In fact, ``Sunday appears as the supreme day of faith.'' By the power
of the Holy Spirit, (cf. DD 29) the Church ``remembers'' all the major events of
Christ's life celebrating them through meaningful liturgies on Sundays. Christ
himself encourages us to believe and strengthens us against unbelief. Thus he
says to us what he said to the Apostle Thomas, ``Blessed are those who have not
seen and have believed.'' (Jn 20:29) In the Creed we publicly profess what we
believe, and with the help of God's indispensable grace live what we believe.

Sunday, Day of the Church, Dies Ecclesiae
4. a. Sunday is indeed the day when all the members of Christ's mystical body
come together to celebrate ``the living presence of the risen Lord in the midst
of his own people.'' (DD 31) Baptized persons are not saved alone as
individuals, but as members of Christ's Church. They come together to express
their identity with Christ and with each other. ``The scattered children of
God'' (Jn 11:52) reunite and become one in the Eucharist through the power of
the Holy Spirit. (cf. DD 31) The Church, therefore, is a ``Eucharistic
assembly.'' The assembly becomes ``church,'' ecclesia, because it is the
Eucharist, which forms and feeds the assembled Church. (cf. DD 32)
The Catechism of the Catholic Church states: ``the Sunday celebration of the
Lord's Day and his Eucharist is the heart of the Church's life.'' (CCC 2177)
Pastors, therefore, must stress that the Sunday Eucharist is a community
celebration. Vatican II's Constitution on the Sacred Liturgy stated, ``within
the parish a lively sense of community celebration of Sunday Mass'' must ``be
fostered in the thinking and practice of both laity and clergy.'' (cf. DD 35; SC
42) Community celebration requires carrying it out reverently and exactly
according to liturgical norms, sustaining it and building it up by bringing to
and forming lapsed Catholics in Small Ecclesial Faith Communities in preparation
for incorporation into the Church. Essential to the Church's Sunday assembly is
``the twofold table of the word and of the Bread of Life.''
The liturgical document of Vatican Council II noted, ``The Liturgy of the Word
and the Liturgy of the Eucharist are so closely joined together that they form a
single act of worship.'' (DD 39) By reading the sacred texts of God's word ``in
a spirit of prayer and docility'' individuals and families can gain from them
great spiritual formation, and from the interpretation the Church gives
them. (cf. DD 40)

b. The table of the Bread of Life is of course the Holy Eucharist. Sunday is the
day of thanksgiving for this gift. ``The whole community gathers to celebrate
the 'Lord's Day,' the Eucharist ... the great 'thanksgiving.''' Each of the
Eucharistic prayers are the Church's prayers of thanksgiving for the gift of
Christ in the Eucharist. The culmination of the Eucharistic prayer of
thanksgiving is expressed by the assembly by the ``Great Amen,'' meaning ``so be
it;'' It means giving affirmation, support, approval. Filled with the Spirit the
Church gives resounding, not weak, thanks to the Father through the Son in the
``Great Amen.'' (cf. DD 42)
Why is the table of the Eucharist so special, precious and without equal? It is
the sacrifice of Jesus on the Cross re-presented on the altar. Christ not only
re-enacts his own sacrifice but also unites the sacrifice of the Church with his
own. (DD 43) The Body of Christ, the Church, participates in the offering of her
Head, the Lord. ``In the Eucharist the sacrifice of Christ becomes also the
sacrifice of the members of his Body. The lives of the faithful, their praise,
sufferings, prayer and work are united with those of Christ and with his total
offering, and so acquire a new value.'' (CCC 1368) We bring our burdens to Jesus
and He makes them lighter, easier and holy. (cf. Mt 11:30)
In the very earliest times Sunday Eucharist was so deeply felt by Christians
that rather than miss Mass they were willing to suffer martyrdom. Many early
Christians were in fact martyred, under cruel emperors such as Diocletian,
Valerian, Hadrian and others. The catacombs and many churches in Rome and
elsewhere attest to it. (cf. DD 46) Only later due to weak faith or negligence
did the Church ``make explicit the duty to attend Sunday Mass,'' at first ``in
the form of exhortation,'' then by imposing ``penalties.'' Only in 1917 the Code
of Canon Law ``gathered this tradition into a universal law,'' saying that ``on
Sundays and other holy days of obligation the faithful are bound to attend
Mass.'' ``This legislation has normally been understood as entailing a grave
obligation.'' (cf. DD 47) ``Those who deliberately fail in this obligation
commit grave sin.'' (CCC 2181)

Sunday, the Day of Man, Dies Hominis
5. The reason that Pope John Paul II gave for issuing his apostolic letter, Dies
Domini, The Day of the Lord, on the Solemnity of Pentecost Sunday, May 31, 1998
was to highlight the importance of observing the Day of the Lord, and also to
respect it as the Day of Man. Observance of Sunday gives Christians a unique
identity as the Pope stated. Himself a very human person and fully in touch with
human beings in every part of the world, he has characterized the Day of Man,
Sunday, in very human terms. He said man needs a day of joy, rest and solidarity
with fellow human beings.
These characterizations have a basis in the creation story in the Book of
Genesis, but they are in part original to him. On the seventh day God ``rested
from all the work he had undertaken'' (Gen 2:2) Implicit in ``rest'' is the
feeling of joy and the joy of rest is enhanced when shared with family, friends
and loved ones in togetherness, thus solidarity. (cf. CCC 2184) Drawn perhaps
from his almost four years of forced labor as a rock breaker in a limestone
quarry in Poland in summer heat and sub-zero winter weather during the Nazi
occupation in 1940, Karol Wojtyla, the future pope, knew how welcome a day of
rest was. Workers today look forward to Friday and the following two days of
freedom from work and business.
Unfortunately growing commercialization of Sundays is eroding the sanctification
of the Day of the Lord and the Day of Man. In early times of Christianity Sunday
was not a day set aside on the civic calendar. However, to experience the value
and sacredness of Sunday ``Christians celebrated the weekly day of the risen
Lord primarily as a day of joy.'' (DD 55) The Didascalia, a source book of
liturgical practice in the young Catholic Church in Northern Syria in the third
century, advised Christians: ``On the first day of the week, you shall all
rejoice.'' (cf. DD 55)

The Day of Man is a Day of Rest and Joy
6. Before his passion Jesus forewarned his Apostles, ``You will be sorrowful,
but your sorrow will turn into joy.'' (Jn 16:20) Because Sunday Eucharist has a
festive character (example, it is appropriate to dress up to go to Mass) it
expresses joy. Christ communicates to his Church joy through the gift of the
Holy Spirit, and ``joy is one of the fruits of the Holy Spirit.'' (Rom 14:17;
also cf. DD 56) To complain that one gets bored at Mass is not to be open to the
Holy Spirit. Spiritual joy has distinctive qualities. It is not shallow; nor
does it just titillate the senses or excite the emotions. Rather Christian joy
is enduring. It consoles even when the soul is burdened or the body is
suffering. (cf. DD 57)
True Christian joy as distinguished from human joy comes from the heart of the
glorified Christ. Christian joy is the kind of joy the Apostles experienced on
seeing Christ on the evening of Easter. Thinking they were seeing a ghost, he
showed them his hands and feet. It was He! ``They were incredulous for joy.''
(cf. Lk 24:37-41; DD 58) A similar Christian joy is witnessed in parishes when,
following Sunday Eucharist, members gather in joy in front of church to chat or
go to the parish hall for coffee to visit, meet and socialize with fellow
Catholics. Sunday Eucharist does give joy if we want it.

7. The Creator rested with joy, and ``looked at everything he had made and found
it very good.'' (Gen 1:31) Only in about the fifth century were work laws
finally relaxed in the Roman Empire, and rest became part of Sunday along with
joy and leisure. But in some countries hostile to Christianity, Christians
worship on Sunday when they can and often secretly. Sunday not being a public
day of rest, they are not free to keep Sunday ``sacred'' by going to Mass
openly. (cf. DD 64)
Similar situations exist in some under-developed countries where poor people
must work long hours in miserable work places, are exploited and their personal
dignity debased. Often they too are unable to participate in Sunday Eucharist
and are denied a much needed day of rest and joy. In his famed social
encyclical, Rerum Novarum issued in 1891, Pope Leo XIII stated emphatically that
workers have a right to Sunday rest and the state must guarantee it. (cf. DD 66)

The Day of Man is also a Day of Solidarity
8. By this statement the Holy Father means that Sunday should also be a day of
fraternal, inter-personal sharing; a day when Christians have time and take time
to bond lovingly with others, show concern for them and bring solace to brothers
and sisters in bodily or spiritual need and care for them. Pope John Paul II put
it this way: ``Sunday should also give the faithful an opportunity to devote
themselves to the works of mercy, charity and the apostolate.'' (DD 69) In line
with Church tradition ``works'' can be ``corporal'' that is, giving material aid
to the poor. Example: ``[F]raternal sharing with the very poor...'' ``the
collection organized for the poor churches in Judea.'' (cf. DD 70) ``Works''
refer no less to spiritual works of mercy. Such works palpitate the heart of the
giver and the heart of the one who receives.

The Catechism of the Catholic Church speaks of Christians doing spiritual works
of mercy, and calls them ``missionary witness.'' The two words in fact spell out
the aim and goal of the CLCC process. This means reaching out to others to bring
them to faith in God and to his Church. The Catechism's full statement on
``missionary outreach'' is the following: ``The fidelity of the baptized is the
primordial condition for the proclamation of the Gospel and for the Church's
mission in the world. In order that the message of salvation can show the power
of its truth and radiance before men, it must be authenticated by the witness of
the life of Christians.'' ``The witness of a Christian life and the good works
done in a supernatural spirit have great power to draw men to the faith and to
God.'' (CCC 2044) More simply, the mission of Catholic faithful in the world is
to live the Christian life authentically and thereby draw men to faith and God
by means of good works.

The Day of Solidarity and the Commandment to Love One Another
9. At the Last Supper in one of his two discourses to the Apostles Jesus said
passionately, ``This is my commandment: love one another as I love you.'' (Jn
15:12) Being a day of rest and joy, Sunday is also a day of solidarity with
other Christians, in particular the hungry and thirsty, the stranger, the naked,
the ill and those in prison. (cf. Mt 25:35-46) All such works are rooted in the
law of love, charity. But they are not limited to the above six works.
John Paul II speaks of ``inventiveness of which Christian charity is capable.''
(DD 72) To increase solidarity among lapsed Catholics, Christian charity demands
that active Christians spend a few hours each week finding, contacting and
visiting non-practicing Catholics in their homes, introducing them to Christian
fellowship in Small Ecclesial Faith Communities in their parish, and experience
in them love and caring of Christians in true solidarity with one another. Most
importantly they will experience within SEFCs Christian solidarity when the tide
of God's grace begins to flood their hearts.
Ultimate Christian solidarity takes place when they re-unite with Christ in the
Sunday Eucharist and receive him in Holy Communion. This work of mercy has no
equal. It is a merciful act of self-giving to a person in need of
salvation. What better way to love another than to help a disconnected fellow
Christian return to solidarity with Christ and to solidarity with his brothers
and sisters in the ``Church of the Eucharist,'' the Body of Christ!

10. Millions of non-church-going Catholics are drowning in the floodwaters of
the secular and de-Christianized culture. They have cut themselves adrift from
the safe moorings of the Catholic Church to which they were once joined. Not a
few secretly worry about their religious state. They are hoping for a rescuer to
come along. Christians should make saving lapsed Catholics their chief work of
mercy. This brings about solidarity, the re-uniting and bonding of drifting
Catholics into the body of the Catholic Church. Jesus said to his Apostles. ``I
will make you fishers of men.'' (Mk 4:19) In an appearance to seven disciples
after his Resurrection who were fishing, Jesus shouted to them from the shore,
``Cast your net over the right side of the boat and you will find something,''
``Duc in altum.'' They took in a huge catch of fish, one hundred and
fifty-three. (Jn 21:6,11) Christians and SEFC Home Visitors (cf. Stage 4),
should be ``fishers of men!'' Bring lapsed, lost and indifferent fellow
Catholics to the haven of the Catholic Church. Without doubt it is the most
urgent work of mercy and charity in the CLCC apostolate.
Yet another way to look at the Day of Solidarity is to hear this word spoken by
Jesus on the Cross, ``I thirst.'' (Jn 19:28) He thirsts for the salvation of all
because he died for all. As members of his Body, the Church, we are obliged to
satisfy his intense thirst. We do this by retrieving stray Catholics and
restoring them into the community of believers, koinonia, into solidarity with
him. In the words of St. Paul, achieving solidarity means ``to equip the holy
ones for the work of ministry, for building up the body of Christ, until we all
attain to the unity of faith and knowledge of the Son of God, to mature manhood,
to the extent of the full stature of Christ.'' (Eph 4:12-13) Paul's words are in
part the inspiration of this process of in-depth faith formation and outreach to
the lapsed called Celebrating Life as a Catholic Christian.
Summary
Stage 5 contains three parts. All parts are linked and the title of each part
immediately suggests its content. The Introduction establishes the fact that
Catholic believers are ``Stewards of the Eucharist.'' ``Let everyone see us as
servants of Christ and stewards of the mysteries of God.'' (1 Cor 4:1)

A. Eucharist, God's Greatest Gift to Man
This part establishes the great truth of faith that the Eucharist is Christ
himself, his body and blood, soul and divinity. St. Thomas Aquinas states ``That
in this sacrament are the true Body of Christ and his true Blood is something
that `cannot be apprehended by the senses, but only by faith, which relies on
divine authority.''' (CCC 1381) The Gift of the Eucharist is food, a spiritual
banquet. To receive Holy Communion is to take in and be nourished by Christ, the
Lord God himself.

B. Stewardship of the Eucharist
This part identifies ten areas, where Catholics as Stewards of the Eucharist,
have opportunities to demonstrate their firm belief in this great mystery of
faith. As stewards they also have responsibility to defend the Eucharist from
modern evils that assault it either directly or indirectly especially by the
grave sin of sacrilege.

C. and D. Eucharist, Catholic Worship and the Celebration of Sunday
In these two parts sixteen aspects of worship of the Eucharist are identified
and explained. The first seven are: the need and usefulness of the Sacrament of
Penance and Reconciliation; the First Commandment of God, the basis of worship;
Worship, as an act of religion; Transubstantiation; Signs of Adoration;
Celebration of believers and the Real Presence. The next seven aspects of
worship focus on Sunday, the Day of the Lord, the Day of Christ, and the Day of
the Church and the Day of Man. The last emphasizes Sunday as a Day of Rest, Joy
and Solidarity with fellow believers, with lapsed Catholics who returned to the
faith, and rejoined the Christian community.

% ------------------------------------------------------------------------------

\section{Reflection and Discussion}
\lxRDFa{property=stage-rd-content,resource={manual}}

Newcomers should read: Putting the CLCC Process to Work found in the front of
this manual, and then Order for Conducting Formation Sessions. These
explanations and instructions will help you to gain the greatest good from the
manual's contents.

Note: Stage 5 has 39 numbered segments, they form the main text. In this
Reflection and Discussion section there are corresponding numbered segments
designed to stimulate prayerful reflection, discussion and sharing. When you
have chosen a paragraph from the main text as the subject matter for your
formation session, please refer to the like numbered segment in this section as
they should be used together.

Introduction
``Let everyone see us as servants of Christ and stewards of the
  mysteries of God.'' (1 Cor 4:1)
1. This text from St. Paul's First Letter to the Corinthians is addressed first
to bishops. It includes clergy and the lay faithful. It's a service
text. ``Stewards'' are service people to God's people in need of the sacraments,
``the mysteries of God'' and other works for growing in faith and
holiness. Imagine the millions of baptized Catholics who do not receive God's
life-giving mysteries, some for many years. They are spiritually starved. Young
people and even children are hungry for the truth but do not know they hunger
for the ``bread'' which only God can give, not what their eyes see at the
shopping mall. As a ``steward'' of God's mysteries, say to yourself: I can feed
them with loving concern for their salvation with truth about Jesus, and help
them come to him in his Church. Share the insights the Holy Spirit gives you..

2. Do you have exposition of the Blessed Sacrament in your parish? How often,
once a month, daily, all night? Is public exposition just a beautifully
decorated altar with candles, flowers and lights? Does your priest invite people
to come daily to make an hour of adoration of Jesus in the Blessed Sacrament?
Are there persons in your parish who have expressed interest in this public
devotion? As we have learned, isn't the Holy Eucharist ``the font and apex of
the whole Christian life?'' (LG 11) Ask, how can we celebrate as Catholic
Christians if we don't have a deep love of Jesus in the Blessed Sacrament? What
can I/we do to get eucharistic adoration started if it isn't started, or if
started to grow among believers? Share.

3. The chief mystery of which St. Paul speaks (cf. above, par 1.), and we are
stewards of this mystery,is the Real Presence of Jesus among us. The place he
has chosen to be among us apart from the Mass is in the tabernacle of our
churches. If we are persons of deep faith shouldn't his presence prompt us to
visit him, to want to ``see'' him, to adore him, thank and praise him, and at
the same time express sorrow for our sins and ask him to help us love him more
and more? Reflect, how strong is my faith in his real presence? How deep is my
love of Christ in the Blessed Sacrament?

4. A conflict exists between those who accept humbly the great truth that Jesus'
Body and Blood are food and drink that nourish the souls of believers, and those
who reject this truth such as non-believers, skeptics, and even some dissenters
who choose only truths they like. We who believe in the great eucharistic
mystery feel an urge at times to speak to them, telling them what they are
depriving themselves of, and praying they will open their hearts to the gift of
faith in the Eucharist. Or, do we tend to ``write them off'' as hopeless
cases?. As good stewards of the Eucharist, reflect and ask the Lord what he
wants you to do. Listen and share.

5. The aim of Stage 5 is to awaken in the faithful a new and deeper
understanding of the mystery of the Holy Eucharist. Having gone through four
stages of the CLCC process, reflect: do I have a much improved appreciation of
the Eucharistic Sacrifice of the Mass, that Jesus is indeed the ``source and
summit of the whole Christian life,'' and that he is the ``source and summit''
of my life? Is this my aim? Be open to the Holy Spirit. Share.



A. Eucharist, God's Greatest Gift to Man
1. and 2. Vatican Council II has given the Church 16 precious documents of
teachings, biblical texts and quotations from distinguished teachers of the
faith through its 2000 year history. In this week's formation session, open your
Catechism of the Catholic Church to article 3 on The Sacrament of the Eucharist,
pages 334-356. Browse through the sections to see how the sacred doctrine on
Eucharist is presented. Go through all pages. Select the section that catches
your eye, read it carefully. Reflect on it. Digest it as best you can, and share
your insight with your SEFC.

The Eucharist is God the Father's supreme gift of his beloved Son to us
3. What is most important to remember about the Eucharist is that it is a
memorial, a sacred action which serves to perpetuate the memory of the
God-Christ re-enacting his Sacrifice on the Cross and becoming spiritual food
and drink to those who believe in him. A memorial can be a lifeless ``thing''
like the beautiful memorial tomb of the Unknown Introduction.

4. A conflict exists between those accepting humbly the great truth of Jesus
that his body and blood are food and drink which nourish the souls of believers,
and non-believers, skeptics and dissenters who question this truth or reject
it. Fortunately believing in this great Eucharistic mystery as we do, what does
this gift who is Christ impel us to do toward non-believers? Feel moved to tell
them about the great sacrament they are depriving Soldier in Arlington Cemetery
in Washington, D.C. But the Eucharist is a living memorial which helps us
remember Jesus' death on the Cross and his resurrection from
death. Simultaneously the Eucharist is a remembering of the Last Supper when
Jesus changed bread and wine into his Body and Blood. The Mass is, therefore, a
memorial, the ``remembering'' of what Jesus did 2000 years ago. What he does
here and now, today and every day in Eucharistic celebrations is a
``remembering.'' Reflect on the Eucharist as a memorial. Pray and
believe. Share.
We can study and reflect on the mystery of the Holy Eucharist as a memorial, as
sacrifice, as sacrament, and make progress in appreciating it, but we will not
fully understand it until we reach heaven and see Christ face to
face. ``Remember'' that Mass is a 'live' re-presentation of Jesus' suffering,
death on the Cross and resurrection from the dead enacted by Jesus himself
through the actions of the priest. Christ making himself really present in the
Mass is the incomparable gift he gives us. That he makes himself present to us
as sacrifice and sacrament at all times and places is the great Mystery of
Faith. This mystery requires your deepest and most loving reflection. Share.

Absolutely central to the Holy Eucharist is its sacrificial meaning
5. The Mass is not only a 'live' memorial, a remembering of Jesus' Last Supper,
his Passion and Death and Resurrection, but the main event of Mass is his
sacrifice, a living re-enactment of his Suffering and Death on the Cross. He
makes present on the altar really and truly His broken and bruised body and his
blood that flowed from his wounds at the scourging and from his hands and feet
and side on the Cross. To be intimately united with him in his sacrifice, he
makes his body eatable and his blood drinkable. Reverently reflect on the
sacrificial aspect of the Eucharist. Pray for an increase of faith and
love. Share.

What about the ``real presence?''
6. Having reflected on the Eucharist as memorial, as the Mystery of Faith, and
sacrifice, these sacred truths are expressed by a single phrase, the ``real
presence.'' This is the term that the Church uses for stating definitively that
Christ is really and truly present in the Eucharist. He is fully present, that
is substantially. What continues to appear as bread and wine is in reality his
body and blood, his soul and divine nature in the Eucharist. The truth of the
``real presence'' is awesome. Again it demands our deepest reflection, and
prayer for faith. Share.

Transubstantiation! What is it?
7. Transubstantiation, the conversion of bread and wine into the body and blood
of Christ, takes place through Jesus' spoken words, ``This is my Body'' over
bread, and ``This is my Blood'' over a cup of wine. These are words that Jesus
spoke as God. They had the power to make change at the Last Supper. They
continue to have the power to make change now. The priest at Mass vocalizes
Jesus' words. The ultimate Mystery of Faith is to accept the unfathomable
mystery of Transubstantiation. Reflect and pray: Lord, ``I do believe, help my
unbelief.'' (cf. Mk 9:24)
The next Mystery of Faith, Communion
8. The last Mystery of Faith is that the Eucharist is a banquet, a festive meal
of believers coming together in celebration. Recall that Jesus had no problem of
eating and drinking with sinners, multiplying loaves and fishes for thousands
and changing water into wine at the marriage feast in Cana. The ``banquet''
words of Jesus at the Last Supper were ``take and eat...'' handing his Body
appearing as bread to his apostles. He then said ``take this and drink...''
handing his Blood appearing to be a cup of wine to his apostles. The banquet is
communion, the many sharing one and same food and drink. Eucharistic communion
is more than nourishment that enhances union with Christ only the
individual. The Eucharistic banquet ``binds us to the Lord, and in that way,
binds us to one another.'' (Cardinal Ratzinger addressing John Paul II and the
College of Cardinals on the Holy Father's 25th anniversary of his pontificate,
October, 2003) The eucharistic banquet forgives venial sin, helps us avoid sin
and grow in holiness. Indeed the Eucharist is God's greatest gift to us. Reflect
and share.

B. Stewardship of the Eucharist
Responsibility to oneself
1. To be a good steward means to be a responsible person. Privileged as we are
to be given the Holy Eucharist, certain responsibilities go with
privilege. Among the responsibilities given in par. 1, choose one, perhaps two,
which you recognize in yourself as needing improvement. Example. If you receive
Holy Communion often, perhaps daily, ask yourself, has it become routine, do I
make a conscious effort to realize what I am doing, am I reverent and filled
with faith, do I make a heartfelt thanksgiving? Reflect.

Responsibility to others
2. Recall from an earlier formation session (A.8.) that receiving the Eucharist
is not an individualistic action of a person but a communal action of many
persons in celebration of their faith. The very term ``Holy Communion''
indicates that the community of believers comes together at the table of the
Lord to partake of his Body and Blood as food and drink. Sadly not all Catholics
receive Holy Communion, some for serious reasons. Millions of Catholics,
especially the lapsed, do not know what they are missing, that they are
depriving themselves deliberately of union with the Lord. Responsible stewards
of the Eucharist are obliged to help them return to the Lord. Reflect on this
duty. Share.
3. Jesus' ``washing the feet'' of the apostles was not only a humble action of
charitable service (stewardship) but is also a powerful symbol to us for doing
good to others. Hundreds of Catholics are out of the Church in your parish, your
family, neighborhood and place of work. They are in need of symbolic
``washing,'' a cleansing from sin, a change of lifestyle, formation in Catholic
values and return to the Eucharist. Resolve to bring at least one of these
people in need of ``washing'' into your SEFC. Help them to get back on the road
to Christ. Reflect. Pray for God's grace to do this soon. Share.

4. The causes for low Mass attendance are many. More could be added to the list
given here, for example the sex explosion now more explicit than ever imagined
brings untold moral and spiritual degradation to souls. But deficient
catechetical formation in one's youth is the top culprit. This is why weekly
CLCC adult formation sessions provided in SEFCs bring about lifestyle changes
not believed to be achievable. In liturgy numerous instructions, directives,
admonitions and need for correction of liturgical abuses have been issued by the
Vatican in recent years in order to cultivate reverence, a sense of the
transcendent and safeguard the mystery of worship of the Triune God especially
the Eucharist. The CLCC process can and does produce change for the
better. Reflect on what you are expected to do in your SEFC to bring about this
change. Share.

5. Ask yourself, to trivialize the Mass, the Eucharist, by injecting comedy into
it; will this bring a sense of the sacred into the celebration? Aren't children
awed by mystery, the unseen but real, rather than by jokes and showmanship? In
short, Mass must become a holy experience as it is meant to be by Christ and his
Church. Pray that Mass in your parish and all parishes is a prayerful event. If
it is truly prayerful it will be joyous, be assured. All of us have attended a
Mass or been to a church where something was said or done which made us say,
``This isn't right. It's out of line. I've never seen or heard anything like
that before.'' Without getting specific, Pope John Paul II has called these
novelties ``shadows'' which blur the beauty and ``lights'' of the
liturgy. Cardinal Arinze calls this ``horizontalism'' where the celebrant and
people stay on the level of themselves, when in fact the Mass is and should be
``perpendicular,'' a celebration pointing up from earth to the glory of the
Father with the Son through the Holy Spirit. Reverence in the Mass cannot be
stressed enough. Reflect on how liturgical abuses can be corrected. Share. In
the Spirit keep discussion charitable.

6. How do you react to the notion of freedom as it is understand by millions
today? ``Any thing goes,'' ``No one is going to tell me what to do,'' ``nobody
is going to go to hell anyway'' ``I'll live my life the way I want.'' These
expressions describe mildly the attitudes millions of people have about
freedom. What can be done? Certainly don't give up. With grace from God and the
gift of fortitude from the Holy Spirit, our responsible stewardship toward them
is this: we must pray, discreetly advise, point out misconduct, teach Christian
behavior [the Ten Commandments], insist that solid truths be preached in church,
not get discouraged, be patient and live in strong hope that the light of God
will overcome their darkness. Believe that Christ wants everyone to be
saved. Trust the Holy Spirit to do the rest for his Church. Reflect and share.

7. The two cultural heresies pointed out here, denial of sin and tolerance, can
be disastrous to a person for his salvation on the one hand, and on the other
hand they compromise one's own religious convictions or show indulgence to
another's error thus giving scandal. In either case it is an assault on the
truth of Christ's words at the consecration when he said that his blood is shed
``so that sins may be forgiven.'' Resolve not to be influenced by passing
fallacies or be taken in by political correctness. Tolerance of sin contradicts
true charity. Let us be firm in our belief in the reality of sin, mortal and
venial. Reflect and share.

8. The sacred institution of marriage was initiated by God when at the beginning
he created man and woman. He made man first (Gen 2:7). Feeling alone, God
created for the first man a ``suitable partner'' (cf. Gen 2:20) and called her
``woman,'' and ``the two of them became one.'' (cf. Gen 2:23-24) ``God blessed
them, saying 'Be fertile and multiply.''' (Gen 1:28) Down through the centuries
every society has understood that the purpose of marriage was the ``begetting
and education of children, and secondly mutual help.'' Do you see why the vast
majority of people are opposed to same-sex unions and angry at federal courts
for legalizing them in certain states? This is a frontal attack on marriage, the
foundation of society, on the family, the fundamental unit of society, and the
two-parent home for having and raising children. No nation or society can
survive that approves as normal same-sex partnerships. As responsible stewards,
reflect on ways you can help persons with this disordered orientation to live
chaste and holy lives. Share.

9. When groups of people gather at dinner, at work-breaks, at parties or church
socials, talk is usually about the weather, sports and children. When
conversation drifts into politics or religion often strong opinions are
expressed. Some statements might be objectively true. Others might be ``off the
wall.'' ``Off the wall'' opinions are given often when religion is discussed in
a mixed group of solid, orthodox Catholics and liberal, dissenting Catholics,
also among Catholics and non-Catholics. When discussion gets animated, usually
someone will say in an appeasing voice, ``But all religions are the same; we all
believe in the same God, lead a good life and we'll all go to heaven.''
Not so! This is what Dominus Iesus, the Declaration of the Congregation on the
Doctrine of the Faith, 2000 stated: salvation is for all but only in Christ and
in his Church. St. Peter said it concisely and perfectly: ``There is no
salvation through anyone else, nor is there any other name under heaven given to
the human race by which we are to be saved.''(cf. Acts 4:12) Reflect on
St. Peter's statement. Religions are not relative nor all the same .For all its
implications and distinctions, you may enjoy reading the Declaration. Stick to
truth. Share.

10. All of us live in the midst of the culture wars. We're fighting big and
small battles almost every day, and we're scarcely aware of them. Ask yourself,
where am I in these struggles? Am I on the right or wrong side of these hot
buttoned issues? As a Catholic striving for holiness and doing outreach to
Catholics outside of the Church, do I firmly accept the Church's and the Popes'
teachings on so-called ``controversial issues?'' Do I question, doubt or reject
the Church's teaching authority on issues I don't like? Am I willing to change
my mind on ``controversial issues,'' and humbly embrace the truth which the
Church teaches on these matters? In quiet time, reflect and ask the Holy Spirit
for light and grace to follow the truth. Share.


C. Eucharist and Catholic Worship
1. As a way to prepare for and appreciate receiving Holy Communion as worthily
and fruitfully as possible, it is necessary that we be conscious of no serious
sin and at the same time feel confident we are in the state of grace. To make
sure we are in the state of grace at all times, Christ gave us the sacrament of
Penance and Reconciliation. This is his loving gift to us, the powerful means
whereby through confession and sorrow (contrition) sins are forgiven, grace is
restored in our soul, the wounds of ``ecclesial communion'' healed, and the road
to holiness is opened. What frequent confession does for us for advancing in the
holiness is inestimable.
The good example of going to the Sacrament of Penance frequently which members
of SEFCs give to returnees in the group is also inestimable. This example shows
returnees that practicing Catholics really ``practice'' their faith. Observing
this practice impacts favorably on their desire to return to the
Church. Reflect: do I receive this Sacrament frequently; am I a good example?

2. Worship of God according to the clear statement of the First Commandment is
neither an option nor a personal preference. The true God is to be adored as the
one and only God. To dare to choose another way is to worship a ``strange''
god. Reflect deeply on how the devil, the Evil One, used ``choice'' on Jesus on
the mountain top, on the pinnacle of the temple and in Gethsemani to worship him
rather than the Father. Imagine a fallen angel who wanted to be God still
pursues that end, tempting even Christ to worship him and the world he
rules. None of us are immune from encountering this kind of temptation. Jesus'
answer to the devil must be our answer, ``Be gone, Satan.'' Share.

Worship is an act of religion (cf. CCC 2096)
3. Note the clear and concise description of adoration in relation to worship as
given in the Catechism of the Catholic Church, 2000. The central words are ``to
acknowledge'' God as Lord and ``to thank'' him for his love. In adoration note
also the importance of ``respect and submission'' that we should give to God,
acknowledging our ``nothingness'' without him, our need, therefore, to praise
and exalt him, and reject temptations to self-importance and
vainglory. Reflecting on these aspects, would you say, in brief, that adoration
is humble prayer to our Almighty and loving God? Share.

4. The highest form of worship that we are to give to God is of course to the
Holy Eucharist, where Christ, according to the will of the Father, and the
intervention of the Holy Spirit, makes himself present to his people under the
``signs'' of bread and wine. By means of the Eucharist Jesus Christ himself
becomes the ``source and summit of the Christian life'' (LG 11). This process,
Celebrating Life as a Catholic Christian, is a preparation for and leads up to a
faith-filled action of extolling and honoring the God-Man, Christ, for his
monumental sacrifice on the Cross re-enacted through ritual of the Mass. The
Eucharist is the ``sublime cause'' that unites God with the worldwide community
of faith-persons called the Church. Reflect how we are a part of it requiring
our grateful worship of praise and thanks. Share.

5. Outward signs of adoration of Christ in the Blessed Sacrament within Mass and
outside of Mass are not only appropriate but required. On entering and leaving
church for example, the genuflection or profound bow are customary reverential
gestures due to the Lord who is present in the sacrament. In some parishes
genuflections are almost a thing of the past. Also signing oneself with holy
water on entering and leaving church. Lost in great part is silence in church
even during Mass. Loud talking in church after Mass is becoming commonplace even
in parishes where the tabernacle is visible and occupies a prominent place in
the sanctuary. Loud talking distracts parishioners who try to pray after Mass in
quiet. Reflect: while in church do I give appropriate outward signs of adoration
to the Blessed Sacrament?

6. The following sentence near the bottom of par. 6 demands our closest
attention because it contains a fundamental truth: Without the shadow of a doubt
Christ is the principal agent and center of Eucharistic worship. The next
sentence taken from the Constitution on the Sacred Liturgy (SC 7) affirms this
truth. Why this emphasis? So that it is not forgotten that the central and
primary focus of the Eucharistic liturgy is Jesus, not on the celebrant and not
on the assembly. It is on Jesus Christ, the sacrificial victim, in unity with
the Holy Spirit who re-enacts his passion, death and resurrection to the glory
of the Father. It is God who is worshiped, no one else. Do you consciously try
to give your fullest attention to Jesus, the Redeemer, when you participate in
the Eucharistic worship? In quiet reflect. Share.

D. Eucharist and the Celebration of Sunday
Sunday, the Day of the Lord
1. and 2. The third commandment is stated simply: Remember to keep holy the
Lord's Day. For us Christians the Lord's Day is a day of rest. After six days of
creating the universe, God ``gazed'' on his creation and saw that it was ``very
good.'' (Gen 1:31) It was the seventh day and God blessed it making it ``holy.''
How is it made holy? Not because God set it aside as a ``day of rest,'' but it
became a day for ``remembering'' creation (``gazing''), a day for
``remembering'' deliverance exemplified in the Exodus, and for ``remembering''
the day when Jesus rose from the dead.
These monumental events in the mystery of the God's relationship with man demand
``remembering.'' They make the day ``holy.'' On the Lord's Day, our ``gaze''
should be on these events. What does ``gaze'' mean to you? On what do you
``gaze'' at Mass? Reflect. Share.

Sunday, The Day of Christ, Dies Christi
3. Sunday is the Lord's Day because it is the day after the Sabbath, the first
day of the week, the great day of Jesus' resurrection. When Christ rose from the
tomb early Sunday morning he became Christ our Light in darkness. On Pentecost
Sunday Christ showed himself brilliantly as ``the light of the world'' when he
sent his Holy Spirit on Mary and the Apostles. Thus Sunday is a day of light and
faith. Reflect, is it a day of light and faith for me, my family, my SEFC? With
the Lord's loving grace, make it so. Share.

Sunday, Day of the Church, Dies Ecclesiae
4. a. Nothing gives stronger witness to the Church than when ``the Church''
assembles on Sunday, the Lord's Day, for Eucharistic worship. Worship is a
community celebration when the baptized faithful gather at the table of the
word, the readings, and at the table of the Bread of Life, where Christ makes
his Body and Blood spiritual food and drink for his body, the Church. The
Eucharistic prayer is a prayer of thanksgiving. At the ``Great Amen'' the
assembly affirms aloud that Christ's in unity with the Holy Spirit gives his
Almighty Father ``all glory and honor.'' At the same time the Eucharistic
sacrifice of Christ unites all the praises, prayers, sufferings and works of the
members of his Church in total offering to the Father. Do you see why Sunday is
the Day of the Church, the celebration of ``the living presence of the risen
Lord in the midst of his own people?'' Give the Lord humble and profound thanks
and praise. As ``Church''' reflect on Sunday, and share your inmost thoughts
with your SEFC and your family.

b. Ask yourself, how strong is my belief in the Holy Eucharist, God's
``unsurpassable gift of himself'' to us? How intense is my commitment to going
to Sunday Mass every week? Millions of martyrs are glorified in heaven for their
unwavering defense of this gift. But millions of Catholics ignore the gift and
excuse themselves from accepting it. Do you see why the Church legislates that
to casually skip Sunday Mass is a mortal sin against God's first commandment?
And that to miss Sunday Mass without sufficient reason is a great ingratitude to
Christ who re-enacts in Mass the ``memorial'' of his sacrifice on the Cross for
our salvation? In prayer to the Holy Spirit reflect quietly. Share.

Sunday, the Day of Man, Dies Hominis
5. Besides being the Day of the Lord, and the Day of the Church, Sunday is the
Day of Man as Pope John Paul II has stated. Being a very human person he affirms
Sunday in terms of joy, rest and solidarity. His observation is not entirely his
own. God ``rested from all the work he had undertaken.'' (Gen 2:2) Rest is
refreshing. It restores a feeling of joy to be with family and friends. Reflect
that in early Christianity Sunday was not a civic day free from
work. Christians, nevertheless, strove to be joyful. Joy is one of the fruits of
the Holy Spirit, a gift that helps a person to bear suffering and burdens and
not be overcome or depressed by them. Reflect, am I a joyful person? Holiness
and joy go together. Share your insights.

The Day of Man is a Day of Rest and Joy
6. Joy is indeed a gift of the Holy Spirit. Joy is in fact one of the
distinguishing features of the CLCC process. Look carefully at the joy-filled
words in the title: Celebrating Life as a Catholic Christian. A key dynamic of
the process is to generate solid spiritual joy in the people who belong to Small
Ecclesial Faith Communities, and to all in the parish, acting as a ``leaven'' to
them. Prayerfully reflect, do you feel you are growing in the joy of the Spirit?
Recall especially some spiritual joy you have personally experienced since you
have joined your SEFC. Share it.

7. The Day of Man, according to Sacred Scripture and the Popes' teachings, ought
to be a day of rest and joy. You are probably one of countless workers who have
said on the fifth day of the work week, ``Thank God it's Friday'' as a broad
smile crosses your face and already the good feeling of being able to do ``your
thing'' surges in your heart. But it was not always that way. Nor is the good
feeling of a two-day weekend for everyone. Some people must work on weekends by
virtue of the kind of job they have such as firemen and policemen, nurses, cooks
and waitresses and others.
That we have a five-day work week and a two-day weekend plus a day or two more
for observing certain holidays, we can thank greatly Pope Leo XIII. Early in the
20th century he announced to leaders in the work world and in state government
to give their workers a day of rest and joy, and from the Church's perspective
this would allow workers to give worship and thanksgiving to the Lord by coming
to and participating in Sunday Mass. They listened but not without
struggle. This calls for us to reflect on the blessing we have of a free
weekend. Thank the Lord and the Church for obtaining this ``right'' to rest and
joy and to worship. Share.

The Day of Man is also a Day of Solidarity
8. Sunday as a Day of Solidarity is characterized by Christians sharing, caring,
loving and serving others by performing various kind of corporal and spiritual
works of mercy. These are works of charity. They show the heart of the
doer. Read par. 14 carefully. Center your reflection on the last two sentences
of paragraph two. They capsulize the Day of Man and the CLCC process especially
Stages 4 and 5. Reflect: when faith people do good works for lapsed people, they
draw them to religious faith, to God and to truth. Is this not a goal of the
CLCC process? Share your inmost thoughts.

The Day of Solidarity and the Commandment to Love One Another
9. Moreover, Sunday as a Day of Solidarity means it is a day to renew and carry
out our commitment to the commandment to love one another not merely in desire
or thought but in action. God has given us six actions that have priority. But
solidarity is not limited these actions. Solidarity includes making connection
with lapsed Catholics, finding the lost, reaching out to the wayward and
bringing hope to the despairing and rejected. Solidarity means not resting until
every lapsed Catholic living in the parish boundaries is brought into the loving
center of a SEFC where he begins his return to Christ and the Church by grace of
the Holy Spirit. Doing this is the acid test of the commandment: ``love one
another.'' Ask yourself, am I loving in this way? Do I need help from the
brothers and sisters in my SEFC? Am I humble enough to ask for help? Pray and
share.

10. When Christ on the lakeshore shouted to the apostles in their boats, ``Cast
into the deep,'' Duc in altum. They made a huge catch. He was saying in effect,
``I have come for the salvation of all.'' I will make you ``fishers of men.''
(Mt 4:19; Mk 1:17) Indeed, members of SEFCs, committed to bring into the Church
all lapsed, lost and indifferent Catholics, are indeed ``fishers of men'' of
whom Christ spoke. They cast the nets of the CLCC conversion process not on the
surface but into the deep where fishing is not easy. Apply Jesus' analogy to
yourself. I too am a fisher of men. I am also called to satisfy Jesus' immense
``thirst'' for souls that he expressed on the Cross. I/we satisfy his thirst by
seeking out non-practicing Catholics, and with the Lord's grace we bring them
into the community of believers in order to build up the Body of Christ to
``full stature.'' Pray and say to the Lord, I want to be a holy disciple and an
effective fisher of men, thus, Celebrating Life as a Catholic Christian.

% ------------------------------------------------------------------------------

\addtocontents{toc}{\cftpagenumbersoff{part}}
\part{Authoritative Sources}
\addtocontents{toc}{\cftpagenumberson{part}}

% ------------------------------------------------------------------------------

\chapter{Introduction}

Please read: Putting the CLCC Process to Work found in the front of this manual,
and then read Order for Conducting Formation Sessions. These explanations and
instructions will help you to gain the greatest good from the manual's contents.

Part II, Authoritative Sources, consists of an extensive compilation of choice
excerpts from major Church documents beginning with Vatican Council II
(1962-1965). Four leading pontiffs, Bl. John XXIII, Paul VI, John Paul II and
Benedict XVI have issued documents to light the way for God's People to travel
the Third Millennium. They constitute a body of superb ecclesial teachings
urgently needed by Catholics who recognize their need of deeper formation in
Catholic truth in order to live Christian values in a culture becoming
increasingly anti-Catholic. Hungering to grow in truth and holiness, they find
strength and support in Small Ecclesial Faith Communities (SEFC) established
throughout the parish. In weekly formation sessions within SEFCs they grow in
knowledge of the faith and strive to reach higher levels of personal holiness.

In SEFCs members also become concerned about the thousands of non-church going
Catholics who live within the parish boundaries. As ``lost sheep'' and wounded
lambs they need to be found and healed. Part of formation is to prepare them to
reach out to and help lapsed Catholics return to Christ and his Church. Outreach
is best done, as Pope Paul VI states inEvangelii Nuntiandi, through ``person to
person contact.'' Excerpts from official Church teachings given here are
eye-openers to most Catholics. They have never heard of them, or have forgotten
or chosen to ignore them. Conversion of mind and heart to live one's life in
line with God's truths is a condition for salvation and mission. Part II of this
manual is a veritable goldmine of truths presented by the Church's highest
teaching authority, the Magisterium.

More Authentic Catholic Life
Having gone through Part I, parish leaders of the CLCC Process and members of
Small Ecclesial Faith Communities are now invited, ready and urged to take up
Part II for advanced formation in the faith. This section of the manual will
introduce participants to teachings of the Church necessary for living an
exemplary, vibrant Catholic life in private and public without compromise in
this secular age. The teachings come from the Popes. The propositions of
bishops' synods have also been the inspiration of encyclicals, apostolic
exhortations and pastoral letters. They are enhanced with texts from Vatican
Council II documents, the Catechism of the Catholic Church and related biblical
texts.

Part II does not have corresponding paragraphs for stimulating reflection and
discussion, as given in the stages of Part I. Rather, under each paragraph in
Part II you will see the letters R-D-S.

R stands for Reflection on a word, phrase or sentence that you feel, after
praying in quiet time, the Holy Spirit has inspired you to look at in the
paragraph. D stands for Development. This means you develop in your mind the
insights the Holy Spirit has given you and formulate them for effective
sharing. S stands for Sharing. You relate to your SEFC faith group the insights
you have been given on the word, phrase or sentence in the text. You contribute
thereby to everyone's growth in the faith. R-D-S is second level catechesis. It
is called maturation. This means each person strives to increase his or her
knowledge of the faith by ``interiorizing it,'' that is, rooting it not only in
one's mind but also in the heart. This element of formation equips members to
live and act by faith both privately and in exterior actions in the secular
environment. (cf. John Paul II, Catechesi tradendae, Catechesis in our Time, CT,
20)

% ------------------------------------------------------------------------------

\chapter{On Evangelization in the Modern World}

Some document stuff.

% ------------------------------------------------------------------------------

\chapter{The Redeemer of Man}

Some document stuff.

% ------------------------------------------------------------------------------

\chapter{Mission of the Redeemer}

Some document stuff.

% ------------------------------------------------------------------------------

\chapter{The Lay Faithful of Christ}

Some document stuff.

% ------------------------------------------------------------------------------

\chapter{The Gospel of Life}

Some document stuff.

% ------------------------------------------------------------------------------

\chapter{Letter to Families}

Some document stuff.

% ------------------------------------------------------------------------------

\chapter{On the Regulation of Births}

Some document stuff.

% ------------------------------------------------------------------------------

\chapter{God is Love}

Some document stuff.

% ------------------------------------------------------------------------------

\chapter{Saved in Hope}

Some document stuff.

% ------------------------------------------------------------------------------

\chapter{I Will Give You Shepherds}

Some document stuff.

% ------------------------------------------------------------------------------

\chapter{Charity in Truth}

Some document stuff.

% ------------------------------------------------------------------------------

\chapter{Observing the Day of the Lord}

Some document stuff.

% ------------------------------------------------------------------------------

\chapter{The Splendor of Truth}

Some document stuff.

% ------------------------------------------------------------------------------

\chapter{The Holy Spirit in the Life of the Church and the World}

Some document stuff.

% ------------------------------------------------------------------------------

\chapter{The Lord Jesus, One and Only Savior}

Some document stuff.

% ------------------------------------------------------------------------------

\chapter{The Christian Meaning of Human Suffering}

Some document stuff.

% ------------------------------------------------------------------------------

\chapter{The Church of the Eucharist}

Some document stuff.

% ------------------------------------------------------------------------------

\chapter{The Authentic Liturgy}

Some document stuff.

% ------------------------------------------------------------------------------

\chapter{That All May Be One}

Some document stuff.

% ------------------------------------------------------------------------------

\chapter{Mother of the Redeemer}

Some document stuff.

% ------------------------------------------------------------------------------

\appendix \setcounter{secnumdepth}{-2}
{
\Hide  
\addtocontents{toc}{\cftpagenumbersoff{part}}
\part{Appendices}
\addtocontents{toc}{\cftpagenumberson{part}}
}

% ------------------------------------------------------------------------------

\chapter{Questions and Answers}

Q. Why do you call Celebrating Life as a Catholic Christian (CLCC) a process and
not a program? What's the difference?

A. A process is developmental. It moves progressively from one stage to
another. It has a beginning but no end. It's ongoing. A program, on the other
hand, is a succession of actions carried out within a structured order that has
a beginning and a predefined end.

Q. Is one year long enough for carrying out this process?

A. No. In Galilee Jesus took three years to form his disciples in his new way of
life. Compare this process, for example, to capital funds campaigns that are
conducted in dioceses and parishes. They usually take 3 years or more. Formation
in Christ's new way of life is also a ``capital campaign'' requiring a lifetime
for becoming a saint and serving our neighbor.

Q. Is the pastor involved in this process? If so, how, when, where?

A. YES! He together with key people (example, the pastoral council) find
qualified parishioners who are willing to serve as lay leaders of the process,
and other lay persons to begin the establishment of Small Ecclesial Faith
Communities (SEFCs) throughout the parish. He need not get involved in
details. He gives it leadership, and above all shows interest and support.
As often as possible and despite many demands, he should visit the different
SEFCs once in awhile when they hold their spiritual formation sessions,
encourage them, pray with them, answer their questions, and help them keep their
focus.
He is to preach on the process with enthusiasm and conviction, as often as he
deems it helpful to the people. In his remarks he is encouraged to choose some
points from the stage his parishioners are working through, and which he feels
will most benefit them. This also applies to the material in Part II,
Authoritative Sources. On Sunday the people in the pews who are not yet involved
have a right to be informed about the process and invited, in fact urged, to
join and participate in a SEFC.

Q. How many sessions for spiritual formation can be, or should be, drawn from
the material presented in each of the 5 stages and in Authoritative Sources?

A. As many as there are paragraphs. The material in both parts is so abundant
that it is, relatively speaking, inexhaustible, and of course repeatable. What
the Matrix 12 and members of SEFCs seek is in-depth not superficial formation in
faith and holiness. Each of the 5 stages provides material for multiple
formation sessions, one Stage has 20, another 43 sessions, and so on.
Materials for formation sessions in Part II, Authoritative Sources are taken
entirely from magisterial documents. Under each paragraph or segment you will
find these letters: R-D-S in bold. They mean Reflection, Development and
Sharing. Members of SEFCs should follow these three steps as they did in Part I.

Q. Does the CLCC process have criteria to measure success or failure as it
carries out each of the stages?

A. Each of the 5 stages has built-in goals, or targets to reach. Criteria for
the general goal are: how well or poorly was the stage carried out; did it
attain what it was supposed to attain? For example, in Stage 1, did twelve
gifted people respond to the ``call'' to be members of the Matrix, the lay group
which leads the parish through the CLCC process? Did enough parishioners respond
to the invitation to form one, two, three or more Small Ecclesial Faith
Communities for a start of the process? Not to reach these goals the first time
around is not failure, but a signal to continue to recruit.
Each stage also contains intermediate objectives. These are specific actions
that are carried out as exactly as possible within the formation session so
everyone in the SEFC gets the full benefit. For example, does the moderator
start and end the session on time; is the reflection issue clear to all; is
enough time given to quiet time; do all or most participants get a chance to
share their insights; does the moderator politely cut off long-winded sharing;
are the hymns, prayers and intercessions appropriate? More importantly, do the
participants feel they are growing spiritually and gaining better knowledge of
the faith?
Recommended is that each SEFC chooses a recorder to keep track of how well or
poorly goals and objectives are attained. This record is helpful for review and
can induce the group to strive for excellence in their formation. It tells
participants how they are conducting the formation sessions, what needs
improvement, and how to make the sessions run smoothly to the best advantage of
all.

Q. What is the connection between the National Catholic Conference for Total
Stewardship and Celebrating Life as a Catholic Christian?

A. In 1980 the National Catholic Conference for Total Stewardship (NCCTS) was
founded by Francis A. Novak, a Redemptorist priest, with approval of the NCCB,
now called the United States Conference of Catholic Bishops (USCCB). Its purpose
is to promote the biblical concept of stewardship in its totality. Stewardship
has been and still is widely used by dioceses and parishes for fund raising,
increased offertory appeals and capital funds campaigns, under the banner of 3
Ts: Time, Talent and Treasure.
Fr. Novak and the NCCTS felt compelled to explore the deeper implications of
stewardship in Sacred Scripture. Under the guidance of expert biblical scholars,
the NCCTS learned that the Old and New Testaments are a fascinating account of
how God ``practices'' holistic stewardship in creation through his providential
care of the universe and the planet earth. The three components of stewardship
found in both testaments are the good news (Evangelization), holiness of life
(Discipleship) and service to the Lord and his people (Stewardship). The Bible's
three components have become the template of the NCCTS and its CLCC
process. They are living terms of a trilogy of transcendent realities that are
inextricably linked. In 1986 the language chosen to express the three realities
was Total Stewardship.
Expressive as the NCCTS thought the new title was, response to it was limited
but it did stir curiosity in many quarters. In 1990 a new language was developed
to convey that Total Stewardship has an uplifting quality: joy in celebrating
Christ as the Way, the Truth and the Life for his pilgrim people on earth. The
new title was The Christian Celebration of Life (CCL). The new title did arouse
a wide reception of the process. But some times it was mistaken for something
else, for a pro-life group and even an offshoot of Alcoholics Anonymous. Thus
another name change was needed.
The new name for this total stewardship process came in 1998: Celebrating Life
as a Catholic Christian (CLCC). The NCCTS now feels this newly named process
expresses not only the biblical triad of Evangelization, Discipleship and
Stewardship, but also ``New Evangelization'', ``Call to Holiness'' and
``Communion,'' goals which the Second Vatican Council, Pope John Paul II, the
Congregation for the Doctrine of the Faith and Benedict XVI have called for
repeatedly and passionately for God's people in today's world of conflicting
cultures.

Q. With the name change were any adjustments made to the process: its focus,
goals and manner of carrying out the Church's mission of New Evangelization?

A. The name change seems to excite the human spirit. It opens the imagination to
new possibilities and challenges like ``I'm going to grow in holiness, me?'' and
``we'll reach out to help lapsed Catholics return to the faith, terrific!''
Indeed, the process has been given a substantial revision and its content has
been expanded. Each stage makes formation in holiness, in doctrine and in
pastoral outreach clearer. Readers of this manual will be helped immensely in
Celebrating their Lives as Catholic Christians.

Q. What is the cost for implementing this process in a parish?

A. There is no set fee. However, cost is in the purchase of the CLCC books for
the pastor, the lay leaders and all members of Small Ecclesial Faith
Communities. Based on experience the best way to go is for the parish to
purchase the needed number of manuals at a bulk rate, and then request each
participant to purchase a copy so as reimburse the parish. Without a book in
hand participants will find it impossible to carry out their formation. The idea
to buy one or two books and run photocopies each week for the 12 lay leaders and
members of SEFCs, as an economy measure, has been tried before with disastrous
results. The end result is not an economy measure - it costs more in time and
paper. Handout sheets get lost. People need a complete, bound book.
Having a personal copy allows each person to study from his own book, get an
idea of the structure of the process and the order of how formation sessions are
carried out, and allows participants to use their free time to quick-read the
contents before going to the next week's session.
Special cost arrangements can be negotiated with the National Catholic
Conference for Total Stewardship on an as needed basis.

Q. The CLCC process, it appears, puts a lot of importance on the lay leaders
called the Matrix 12 and on Small Ecclesial Faith Communities. Where can I go to
find reliable and official Church teaching on these kinds of communities?

A. Go to Stage 2, titled Formation of Disciples and Apostles in the CLCC
manual. Read section C. Small Ecclesial Faith Communities, numbered paragraphs 1
to 10. Here the teaching on Small Ecclesial Faith Communities is drawn directly
from John Paul II's encyclical Redemptoris Missio and from Pope Paul VI's
Evangelii Nuntiandi. You can find additional information on SEFCs in Part II of
this manual, Authoritative Sources. See the first four documents treated there.

Q. Could you tell me in one sentence what the CLCC process is?

A. The CLCC process is pure, unadulterated biblical stewardship framed in the
Second Vatican Council's ecclesiology and magisterial teachings, which give
solid theological formation and mission motivation to faithful believers,
enabling them to Celebrate their Lives as Catholic Christians in a
dechristianized culture.

% ------------------------------------------------------------------------------

\chapter{Glossary}

Glossary entries.

% ------------------------------------------------------------------------------

\backmatter
{
\Hide  
\addtocontents{toc}{\cftpagenumbersoff{part}}
\part{Indexes}
\addtocontents{toc}{\cftpagenumberson{part}}
}

% ------------------------------------------------------------------------------

\chapter{Alphabetical Index}

Some index entries.

% ------------------------------------------------------------------------------

\chapter{Topical Index}

Some index entries.

% ------------------------------------------------------------------------------

{
\Hide  
\addtocontents{toc}{\cftpagenumbersoff{part}}
\part{}
\addtocontents{toc}{\cftpagenumberson{part}}
}

% ------------------------------------------------------------------------------

\chapter{Creative Commons License}

Attribution-NonCommercial 3.0 Unported

URI:  http://creativecommons.org/licenses/by-nc/3.0/legalcode

License
THE WORK (AS DEFINED BELOW) IS PROVIDED UNDER THE TERMS OF THIS CREATIVE COMMONS
PUBLIC LICENSE (``CCPL'' OR ``LICENSE''). THE WORK IS PROTECTED BY COPYRIGHT
AND/OR OTHER APPLICABLE LAW. ANY USE OF THE WORK OTHER THAN AS AUTHORIZED UNDER
THIS LICENSE OR COPYRIGHT LAW IS PROHIBITED.

BY EXERCISING ANY RIGHTS TO THE WORK PROVIDED HERE, YOU ACCEPT AND AGREE TO BE
BOUND BY THE TERMS OF THIS LICENSE. TO THE EXTENT THIS LICENSE MAY BE CONSIDERED
TO BE A CONTRACT, THE LICENSOR GRANTS YOU THE RIGHTS CONTAINED HERE IN
CONSIDERATION OF YOUR ACCEPTANCE OF SUCH TERMS AND CONDITIONS.
1. Definitions
``Adaptation'' means a work based upon the Work, or upon the Work and other
pre-existing works, such as a translation, adaptation, derivative work,
arrangement of music or other alterations of a literary or artistic work, or
phonogram or performance and includes cinematographic adaptations or any other
form in which the Work may be recast, transformed, or adapted including in any
form recognizably derived from the original, except that a work that constitutes
a Collection will not be considered an Adaptation for the purpose of this
License. For the avoidance of doubt, where the Work is a musical work,
performance or phonogram, the synchronization of the Work in timed-relation with
a moving image (``synching'') will be considered an Adaptation for the purpose
of this License.
``Collection'' means a collection of literary or artistic works, such as
encyclopedias and anthologies, or performances, phonograms or broadcasts, or
other works or subject matter other than works listed in Section 1(f) below,
which, by reason of the selection and arrangement of their contents, constitute
intellectual creations, in which the Work is included in its entirety in
unmodified form along with one or more other contributions, each constituting
separate and independent works in themselves, which together are assembled into
a collective whole. A work that constitutes a Collection will not be considered
an Adaptation (as defined above) for the purposes of this License.
``Distribute'' means to make available to the public the original and copies of
the Work or Adaptation, as appropriate, through sale or other transfer of
ownership.
``Licensor'' means the individual, individuals, entity or entities that offer(s)
the Work under the terms of this License.
``Original Author'' means, in the case of a literary or artistic work, the
individual, individuals, entity or entities who created the Work or if no
individual or entity can be identified, the publisher; and in addition (i) in
the case of a performance the actors, singers, musicians, dancers, and other
persons who act, sing, deliver, declaim, play in, interpret or otherwise perform
literary or artistic works or expressions of folklore; (ii) in the case of a
phonogram the producer being the person or legal entity who first fixes the
sounds of a performance or other sounds; and, (iii) in the case of broadcasts,
the organization that transmits the broadcast.
``Work'' means the literary and/or artistic work offered under the terms of this
License including without limitation any production in the literary, scientific
and artistic domain, whatever may be the mode or form of its expression
including digital form, such as a book, pamphlet and other writing; a lecture,
address, sermon or other work of the same nature; a dramatic or
dramatico-musical work; a choreographic work or entertainment in dumb show; a
musical composition with or without words; a cinematographic work to which are
assimilated works expressed by a process analogous to cinematography; a work of
drawing, painting, architecture, sculpture, engraving or lithography; a
photographic work to which are assimilated works expressed by a process
analogous to photography; a work of applied art; an illustration, map, plan,
sketch or three-dimensional work relative to geography, topography, architecture
or science; a performance; a broadcast; a phonogram; a compilation of data to
the extent it is protected as a copyrightable work; or a work performed by a
variety or circus performer to the extent it is not otherwise considered a
literary or artistic work.
``You'' means an individual or entity exercising rights under this License who
has not previously violated the terms of this License with respect to the Work,
or who has received express permission from the Licensor to exercise rights
under this License despite a previous violation.
``Publicly Perform'' means to perform public recitations of the Work and to
communicate to the public those public recitations, by any means or process,
including by wire or wireless means or public digital performances; to make
available to the public Works in such a way that members of the public may
access these Works from a place and at a place individually chosen by them; to
perform the Work to the public by any means or process and the communication to
the public of the performances of the Work, including by public digital
performance; to broadcast and rebroadcast the Work by any means including signs,
sounds or images.

``Reproduce'' means to make copies of the Work by any means including without
limitation by sound or visual recordings and the right of fixation and
reproducing fixations of the Work, including storage of a protected performance
or phonogram in digital form or other electronic medium.
2. Fair Dealing Rights.
Nothing in this License is intended to reduce, limit, or restrict any uses free
from copyright or rights arising from limitations or exceptions that are
provided for in connection with the copyright protection under copyright law or
other applicable laws.
3. License Grant.
Subject to the terms and conditions of this License, Licensor hereby grants You
a worldwide, royalty-free, non-exclusive, perpetual (for the duration of the
applicable copyright) license to exercise the rights in the Work as stated below:

to Reproduce the Work, to incorporate the Work into one or more Collections, and
to Reproduce the Work as incorporated in the Collections;
to create and Reproduce Adaptations provided that any such Adaptation, including
any translation in any medium, takes reasonable steps to clearly label,
demarcate or otherwise identify that changes were made to the original Work. For
example, a translation could be marked ``The original work was translated from
English to Spanish,'' or a modification could indicate ``The original work has
been modified.'';
to Distribute and Publicly Perform the Work including as incorporated in
Collections; and,
to Distribute and Publicly Perform Adaptations.

The above rights may be exercised in all media and formats whether now known or
hereafter devised. The above rights include the right to make such modifications
as are technically necessary to exercise the rights in other media and
formats. Subject to Section 8(f), all rights not expressly granted by Licensor
are hereby reserved, including but not limited to the rights set forth in
Section 4(d).
4. Restrictions.
The license granted in Section 3 above is expressly made subject to and limited
by the following restrictions:

You may Distribute or Publicly Perform the Work only under the terms of this
License. You must include a copy of, or the Uniform Resource Identifier (URI)
for, this License with every copy of the Work You Distribute or Publicly
Perform. You may not offer or impose any terms on the Work that restrict the
terms of this License or the ability of the recipient of the Work to exercise
the rights granted to that recipient under the terms of the License. You may not
sublicense the Work. You must keep intact all notices that refer to this License
and to the disclaimer of warranties with every copy of the Work You Distribute
or Publicly Perform. When You Distribute or Publicly Perform the Work, You may
not impose any effective technological measures on the Work that restrict the
ability of a recipient of the Work from You to exercise the rights granted to
that recipient under the terms of the License. This Section 4(a) applies to the
Work as incorporated in a Collection, but this does not require the Collection
apart from the Work itself to be made subject to the terms of this License. If
You create a Collection, upon notice from any Licensor You must, to the extent
practicable, remove from the Collection any credit as required by Section 4(c),
as requested. If You create an Adaptation, upon notice from any Licensor You
must, to the extent practicable, remove from the Adaptation any credit as
required by Section 4(c), as requested.

You may not exercise any of the rights granted to You in Section 3 above in any
manner that is primarily intended for or directed toward commercial advantage or
private monetary compensation. The exchange of the Work for other copyrighted
works by means of digital file-sharing or otherwise shall not be considered to
be intended for or directed toward commercial advantage or private monetary
compensation, provided there is no payment of any monetary compensation in
connection with the exchange of copyrighted works.

If You Distribute, or Publicly Perform the Work or any Adaptations or
Collections, You must, unless a request has been made pursuant to Section 4(a),
keep intact all copyright notices for the Work and provide, reasonable to the
medium or means You are utilizing: (i) the name of the Original Author (or
pseudonym, if applicable) if supplied, and/or if the Original Author and/or
Licensor designate another party or parties (e.g., a sponsor institute,
publishing entity, journal) for attribution (``Attribution Parties'') in
Licensor's copyright notice, terms of service or by other reasonable means, the
name of such party or parties; (ii) the title of the Work if supplied; (iii) to
the extent reasonably practicable, the URI, if any, that Licensor specifies to
be associated with the Work, unless such URI does not refer to the copyright
notice or licensing information for the Work; and, (iv) consistent with Section
3(b), in the case of an Adaptation, a credit identifying the use of the Work in
the Adaptation (e.g., ``French translation of the Work by Original Author,'' or
``Screenplay based on original Work by Original Author''). The credit required
by this Section 4(c) may be implemented in any reasonable manner; provided,
however, that in the case of a Adaptation or Collection, at a minimum such
credit will appear, if a credit for all contributing authors of the Adaptation
or Collection appears, then as part of these credits and in a manner at least as
prominent as the credits for the other contributing authors. For the avoidance
of doubt, You may only use the credit required by this Section for the purpose
of attribution in the manner set out above and, by exercising Your rights under
this License, You may not implicitly or explicitly assert or imply any
connection with, sponsorship or endorsement by the Original Author, Licensor
and/or Attribution Parties, as appropriate, of You or Your use of the Work,
without the separate, express prior written permission of the Original Author,
Licensor and/or Attribution Parties.

For the avoidance of doubt:
i.  Non-waivable Compulsory License Schemes. In those jurisdictions in which the
right to collect royalties through any statutory or compulsory licensing scheme
cannot be waived, the Licensor reserves the exclusive right to collect such
royalties for any exercise by You of the rights granted under this License;

ii.  Waivable Compulsory License Schemes. In those jurisdictions in which the
right to collect royalties through any statutory or compulsory licensing scheme
can be waived, the Licensor reserves the exclusive right to collect such
royalties for any exercise by You of the rights granted under this License if
Your exercise of such rights is for a purpose or use which is otherwise than
noncommercial as permitted under Section 4(b) and otherwise waives the right to
collect royalties through any statutory or compulsory licensing scheme; and,

iii.  Voluntary License Schemes. The Licensor reserves the right to collect
royalties, whether individually or, in the event that the Licensor is a member
of a collecting society that administers voluntary licensing schemes, via that
society, from any exercise by You of the rights granted under this License that
is for a purpose or use which is otherwise than noncommercial as permitted under
Section 4(c).

Except as otherwise agreed in writing by the Licensor or as may be otherwise
permitted by applicable law, if You Reproduce, Distribute or Publicly Perform
the Work either by itself or as part of any Adaptations or Collections, You must
not distort, mutilate, modify or take other derogatory action in relation to the
Work which would be prejudicial to the Original Author's honor or
reputation. Licensor agrees that in those jurisdictions (e.g. Japan), in which
any exercise of the right granted in Section 3(b) of this License (the right to
make Adaptations) would be deemed to be a distortion, mutilation, modification
or other derogatory action prejudicial to the Original Author's honor and
reputation, the Licensor will waive or not assert, as appropriate, this Section,
to the fullest extent permitted by the applicable national law, to enable You to
reasonably exercise Your right under Section 3(b) of this License (right to make
Adaptations) but not otherwise.


5. Representations, Warranties and Disclaimer
UNLESS OTHERWISE MUTUALLY AGREED TO BY THE PARTIES IN WRITING, LICENSOR OFFERS
THE WORK AS-IS AND MAKES NO REPRESENTATIONS OR WARRANTIES OF ANY KIND CONCERNING
THE WORK, EXPRESS, IMPLIED, STATUTORY OR OTHERWISE, INCLUDING, WITHOUT
LIMITATION, WARRANTIES OF TITLE, MERCHANTIBILITY, FITNESS FOR A PARTICULAR
PURPOSE, NONINFRINGEMENT, OR THE ABSENCE OF LATENT OR OTHER DEFECTS, ACCURACY,
OR THE PRESENCE OF ABSENCE OF ERRORS, WHETHER OR NOT DISCOVERABLE. SOME
JURISDICTIONS DO NOT ALLOW THE EXCLUSION OF IMPLIED WARRANTIES, SO SUCH
EXCLUSION MAY NOT APPLY TO YOU.
6. Limitation on Liability.
EXCEPT TO THE EXTENT REQUIRED BY APPLICABLE LAW, IN NO EVENT WILL LICENSOR BE
LIABLE TO YOU ON ANY LEGAL THEORY FOR ANY SPECIAL, INCIDENTAL, CONSEQUENTIAL,
PUNITIVE OR EXEMPLARY DAMAGES ARISING OUT OF THIS LICENSE OR THE USE OF THE
WORK, EVEN IF LICENSOR HAS BEEN ADVISED OF THE POSSIBILITY OF SUCH DAMAGES.

7. Termination
This License and the rights granted hereunder will terminate automatically upon
any breach by You of the terms of this License. Individuals or entities who have
received Adaptations or Collections from You under this License, however, will
not have their licenses terminated provided such individuals or entities remain
in full compliance with those licenses. Sections 1, 2, 5, 6, 7, and 8 will
survive any termination of this License.
Subject to the above terms and conditions, the license granted here is perpetual
(for the duration of the applicable copyright in the Work). Notwithstanding the
above, Licensor reserves the right to release the Work under different license
terms or to stop distributing the Work at any time; provided, however that any
such election will not serve to withdraw this License (or any other license that
has been, or is required to be, granted under the terms of this License), and
this License will continue in full force and effect unless terminated as stated
above.

8. Miscellaneous
Each time You Distribute or Publicly Perform the Work or a Collection, the
Licensor offers to the recipient a license to the Work on the same terms and
conditions as the license granted to You under this License.
Each time You Distribute or Publicly Perform an Adaptation, Licensor offers to
the recipient a license to the original Work on the same terms and conditions as
the license granted to You under this License.
If any provision of this License is invalid or unenforceable under applicable
law, it shall not affect the validity or enforceability of the remainder of the
terms of this License, and without further action by the parties to this
agreement, such provision shall be reformed to the minimum extent necessary to
make such provision valid and enforceable.
No term or provision of this License shall be deemed waived and no breach
consented to unless such waiver or consent shall be in writing and signed by the
party to be charged with such waiver or consent.
This License constitutes the entire agreement between the parties with respect
to the Work licensed here. There are no understandings, agreements or
representations with respect to the Work not specified here. Licensor shall not
be bound by any additional provisions that may appear in any communication from
You. This License may not be modified without the mutual written agreement of
the Licensor and You.
The rights granted under, and the subject matter referenced, in this License
were drafted utilizing the terminology of the Berne Convention for the
Protection of Literary and Artistic Works (as amended on September 28, 1979),
the Rome Convention of 1961, the WIPO Copyright Treaty of 1996, the WIPO
Performances and Phonograms Treaty of 1996 and the Universal Copyright
Convention (as revised on July 24, 1971). These rights and subject matter take
effect in the relevant jurisdiction in which the License terms are sought to be
enforced according to the corresponding provisions of the implementation of
those treaty provisions in the applicable national law. If the standard suite of
rights granted under applicable copyright law includes additional rights not
granted under this License, such additional rights are deemed to be included in
the License; this License is not intended to restrict the license of any rights
under applicable law.

Creative Commons Notice
Creative Commons is not a party to this License, and makes no warranty
whatsoever in connection with the Work. Creative Commons will not be liable to
You or any party on any legal theory for any damages whatsoever, including
without limitation any general, special, incidental or consequential damages
arising in connection to this license. Notwithstanding the foregoing two (2)
sentences, if Creative Commons has expressly identified itself as the Licensor
hereunder, it shall have all rights and obligations of Licensor.

Except for the limited purpose of indicating to the public that the Work is
licensed under the CCPL, Creative Commons does not authorize the use by either
party of the trademark ``Creative Commons'' or any related trademark or logo of
Creative Commons without the prior written consent of Creative Commons. Any
permitted use will be in compliance with Creative Commons' then-current
trademark usage guidelines, as may be published on its website or otherwise made
available upon request from time to time. For the avoidance of doubt, this
trademark restriction does not form part of the License.

Creative Commons may be contacted at
  http://creativecommons.org/

% ------------------------------------------------------------------------------

\chapter{CLCC Prayer}

Most loving Lord,
You said, ``I am the way, the truth and the life.''
Source of every good, fill us with your Holy Spirit.
Inspire us to celebrate our lives as Catholic Christians.
Show us the way to bring Celebrating Life as a
Catholic Christian, a faith formation process, to
your people in parishes. Thank you for entrusting
this apostolate to us, and choosing us to live it,
spread it, and teach it in your Church.

Through the power of your Spirit enlighten our minds.
Open our hearts to value the CLCC process as your
special gift to your Church. For this mission, form us
in holiness. Set us on fire to promote the CLCC process
not only to those in your Church but to those outside of it.

Help us to form Small Ecclesial Faith Communities and
bond with each other in sincere ``communion.''
Through your grace move us to repent of our faults
and weaknesses that may be lingering in our hearts.

Awaken in us a hunger ``to be evangelized [in order to]
evangelize.'' Give us courage to reach out to family
and friends who have left the Church, and to millions
of ``unchurched'' Catholics gravely in need of salvation.

Mary, Mother of Jesus, our Lady of Good Counsel, Seat
of Wisdom, and Mother of Perpetual Help, supreme
model of Celebrating Life as a Catholic Christian, lead us
to deep conversion, growth in holiness and self-giving in
service to your Son, Jesus, for building up ``the Church
as Communion'' like the Blessed Trinity. We ask this of
you Father, Son and Holy Spirit. Amen.

% ------------------------------------------------------------------------------

\chapter{About the Author}

Francis A. Novak, a Redemptorist priest, is a nationally recognized authority
and promoter of Total Stewardship. Earlier he had been a missionary, retreat
director, manager of promotion for Liguori Publications on the East Coast,
pastor and director of development and pastoral councils in the Diocese of Grand
Rapids. In 1974 he was made the first full-time executive director of the
National Catholic Stewardship Council in Washington, DC, and currently is
president of the National Catholic Conference for Total Stewardship. Over the
past 30 years he has developed Celebrating Life as a Catholic Christian, a
pastoral program of Total Stewardship for dioceses and parishes. This integrated
process provides comprehensive and systematic formation of the lay faithful in
Church teaching, spirituality and the apostolate of reaching out to lapsed
Catholics, thus furthering the Church's goal of New Evangelization. Author of
several publications treating Total Stewardship, he holds a Masters degree in
Liturgical Studies and a Doctorate in Ministry. He currently resides at
St. Clement in Liguori, Missouri, 63057.

% ------------------------------------------------------------------------------

\end{document}
