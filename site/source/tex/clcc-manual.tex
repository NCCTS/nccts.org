% ------------------------------ sample ----------------------------------------

% WARNING!  Do not type any of the following 10 characters except as directed:
%                &   $   #   %   _   {   }   ^   ~   \

% \section{Simple Text}          % This command makes a section title.

% Words are separated by one or more spaces.  Paragraphs are separated by
% one or more blank lines.  The output is not affected by adding extra
% spaces or extra blank lines to the input file.

% Double quotes are typed like this: ``quoted text''.
% Single quotes are typed like this: `single-quoted text'.

% Long dashes are typed as three dash characters---like this.

% Emphasized text is typed like this: \emph{this is emphasized}.
% Bold       text is typed like this: \textbf{this is bold}.

% \subsection{A Warning or Two}  % This command makes a subsection title.

% If you get too much space after a mid-sentence period---abbreviations
% like etc.\ are the common culprits)---then type a backslash followed by
% a space after the period, as in this sentence.

% Remember, don't type the 10 special characters (such as dollar sign and
% backslash) except as directed!  The following seven are printed by
% typing a backslash in front of them:  \$  \&  \#  \%  \_  \{  and  \}.
% The manual tells how to make other symbols.

% ------------------------------------------------------------------------------

\documentclass[oneside]{book}
\usepackage{graphicx}
\usepackage{hyperref}
\usepackage[utf8]{inputenc}
\usepackage{lxRDFa}
\usepackage[explicit]{titlesec}
\usepackage{tocloft}

\title{\textbf{CELEBRATING LIFE} \\ AS A \\ \textbf{CATHOLIC CHRISTIAN}}
\author{Fr. Francis A. Novak, C.Ss.R.}  \date{June 2013}

\newcommand*\Hide{%
\titleformat{\part}
  {}{}{0pt}{}
}

\begin{document}
\pagestyle{plain}

% ------------------------------------------------------------------------------

\frontmatter

\setcounter{secnumdepth}{-1}
\section*{} \lxRDFa{property=manual-top-title,resource={manual}}

CELEBRATING LIFE AS A CATHOLIC CHRISTIAN

% ------------------------------------------------------------------------------

Nihil Obstat: + Kelvin E. Felix, D.D.  Archbishop of Castries Castries,
St. Lucia, West Indies

Imprimatur: + John F. Donoghue, D.D.  Archbishop of Atlanta October 15, 2004

Celebrating Life as a Catholic Christian Copyright 2010 by Fr. Francis A. Novak,
C.Ss.R.  St. Clement, Liguori, Missouri, 63057, USA www.clcc.info

ISBN: 978-0-615-31588-1

The contents of this book are distributed under the terms of the Creative
Commons license, Attribution-NonCommercial 3.0 Unported. The full text of the
license may be found at the back of the book, beginning on page 309.

A summary of the license terms You are free: to share to copy, distribute and
transmit the work; to remix to adapt the work. Under the following conditions:
attribution you must attribute the work in the manner specified by the author or
licensor (but not in any way that suggests that they endorse you or your use of
the work); noncommercial you may not use this work for commercial purposes. With
the understanding that: any of the above conditions can be waived if you get
permission from the copyright holder; public domain where the work or any of its
elements is in the public domain under applicable law, that status is in no way
affected by the license; other rights in no way are any of the following rights
affected by the license: your fair dealing or fair use rights, or other
applicable copyright exceptions and limitations; the authors moral rights;
rights other persons may have either in the work itself or in how the work is
used, such as publicity or privacy rights; notice for any reuse or distribution,
you must make clear to others the license terms of this work.

Scripture selections, unless otherwise noted, are taken from the New American
Bible, Copyright 1991, 1986, 1970 by the Confraternity of Christian Doctrine,
Inc., Washington, DC. Used with permission. All rights reserved. No portion of
the New American Bible may be reprinted without permission in writings from the
copyright holder.

Selections from official documents of the Catholic Church are taken from the
website of The Holy See, wwww.vatican.va. Copyright Libreria Editrice Vaticana.

Cover design by Wendy Barnes, Liguori, Missouri. Artwork and design of the CLCC
Centerfold found on pages 144-145 by Chris Schwartz Studios, Saint Louis,
Missouri.

For more information and additional electronic resources, please visit the CLCC
website.  http://www.clcc.info e-mail: questions@clcc.info

National Catholic Conference for Total Stewardship 2010

\section*{} \lxRDFa{property=cc-lic-section,resource={cc-lic}}

\begin{center}

This work is distributed under the terms of the Creative Commons license,
\emph{\href{https://creativecommons.org/licenses/by-nc/3.0/us/legalcode}{
Attribution-NonCommercial 3.0 Unported}}.

\href{https://creativecommons.org/licenses/by-nc/3.0/us/}{
  \includegraphics[scale=0.4]{by-nc-nd}
\lxRDFa{property=cc-lic-graphic,resource={cc-lic}}}

\end{center}

% ------------------------------------------------------------------------------

To teach in order to lead others to the faith is the task of every preacher and
of every believer.

--- St. Thomas Aquinas

% ------------------------------------------------------------------------------

The world aflame with Pentecosts fire, Evangelization, the Good News spreading,
Gods people true conversion seeking; Enlightened by the Spirits fire, Enlivened
by the Saviors Body and Blood, Protected under Virgin Mothers care, Entrusted
with the Churchs teaching faithfully to live and share.

% ------------------------------------------------------------------------------

To my Father and Mother, and siblings, Sister Elizabeth, Marie, Fr. Henry,
Philip and Bishop Alfred and to my Redemptorist family and friends

% ------------------------------------------------------------------------------

\maketitle

% ------------------------------------------------------------------------------

Celebrating Life as a Catholic Christian is a proven and efficacious tool for
the New Evangelization, that is, the teaching and celebrating and living of our
Catholic faith with the enthusiasm and engagement of the first Christians and
the first missionaries to every part of the world.  In small communities formed
by the parish, Catholics are helped to deepen their own knowledge and love of
Christ, so that they can give an account of their faith to others, especially
those who do not know Christ or have drifted from His friendship.  The small
communities are particularly effective in carrying out the catechesis of those
who are new members of the Church.  The deepening of faith is secured by the
study of the authoritative sources of the Church's Magisterium, Her official
teachings, which are presented in a most accessible manner.  Through the
discussion of the texts of recent Papal Magisterium, members of the small
communities, under the pastoral guidance of the Vicar of Christ on earth, are
prepared in the best manner possible to be effective heralds and agents of the
New Evangelization.

Raymond Leo Cardinal Burke
Archbishop Emeritus of Saint Louis
Prefect, Supreme Tribunal of the Apostolic Signatura

% ------------------------------------------------------------------------------

\chapter{Foreword}

In 1980 Father Francis A. Novak, C.Ss.R., founded the National Catholic
Conference for Total Stewardship (NCCTS) with approval of the National
Conference of Catholic Bishops (NCCB) now renamed the United States Conference
of Catholic Bishops (USCCB).  His intention was to carry the concept of
stewardship far beyond the appeals for monetary giving to which it is so often
restricted in many dioceses and parishes.  To this end the NCCTS in 1985
developed a comprehensive process called the Christian Celebration of Life (CCL)
for the formation of laity and clergy in Vatican Council II documents and
magisterial teachings. The aim of the process was to deepen spirituality and
generate wider participation among the laity for doing more meaningful ministry
in the Church and in the world.  In many parishes in the U.S., in the Diocese of
Grand Falls, Newfoundland, and in the Archdiocese of St. Lucia in the Caribbean,
the CCL process, where correctly implemented, was resoundingly successful.
Observing the process unfold in dioceses and parishes, it became clear that some
changes in the process would be useful. Also gracious and positive feedback from
clergy and laity prompted some changes. A few users of the CCL felt that the
process ought to be made more contemporary to counter the aggressive
secularization of culture. In 1998 changes were made, the most obvious change
was in the title, Celebrating Life as a Catholic Christian (CLCC). The new title
was chosen to suggest the tensions Catholics have and will increasingly have in
their struggle to live authentic Christian lives in a culture hell-bent on its
de-Christianization.  Unfortunately in 1995, Fr. Novak contracted a staph
infection during a knee replacement operation. This event greatly curtailed and
slowed down the promotion of the newly named and improved CLCC process. Six
years and nineteen surgeries later, the continuing infection necessitated
removal of his right leg above the knee on Tuesday of Holy Week, April 10, 2001.
Functioning on a prosthetic leg, Fr. Novak is back in action, though with some
limitations. This book, Celebrating Life as a Catholic Christian, is written for
priests and lay leaders in parishes strongly drawn to bring Gods people to a
deeper knowledge of the Catholic faith, growth in holiness and commitment to
evangelize ones family, the parish community and world, with special focus on
helping lapsed Catholics return to the Church.

G. William Sefton Chairman of NCCTS Board of Directors

% ------------------------------------------------------------------------------

\chapter{Acknowledgments}

Some thirty years ago a seminal idea about a faith formation process was planted
in my psyche. I prayed it fell on good soil. Half way through the stretch of
nurturing, a title for the process came to me: Celebrating Life as a Catholic
Christian (CLCC). By no means was the laborious move from idea to reality a
one-man project. The Lord provided a partnership of talented people to help me
advance the project, namely Mr. Bill Sefton, first chairman of the National
Catholic Conference for Total Stewardship (NCCTS). He was a man of exceptional
commitment to the CLCC apostolate, and a friend of nearly five decades. Our two
episcopal moderators, Archbishop John F. Donoghue, now emeritus, of Atlanta and
Archbishop Kelvin E. Felix, formerly of St. Lucia in the Caribbean and now in
Grand Bay, Roseau, Dominica, provided their friendly presence faithfully and
their input has always been positive. I thank also all other original donors of
their time and gifts to help define the nucleus of a much-needed renewal
apostolate in the Church, the CLCC process.  The CLCC faith formation process
has incalculable potential and requires strong participant commitment. Inspiring
and exemplary commitment has been given by a small group eager to grow
personally in the faith, and eager to spread it by reaching out to
non-practicing, lapsed Catholics. According to the latest survey, the number of
non-church-going Catholics in the U.S. stands at a staggering 45
million. Helping these millions return to Christ and his Church is the CLCCs
apostolic outreach. Groups formed in parishes that carry out the CLCC process
are known as Small Ecclesial Faith Communities (SEFC). Members of one such group
have met faithfully on Tuesday evenings without interruption for 4 years in over
200 consecutive ninety-minute Faith Formation Sessions. Their commitment to the
CLCC process is not only transparent, but their growth in knowledge of the faith
and in their spiritual life has been a climb of the mountain. Lay leaders in
this group, noted for their extraordinary dedication and perseverance, are Rob
Winkler, Adelaide Herrell and Kimberly Dowd.

The CLCC process, we feel confident, is the work of the Holy Spirit. How else
explain the following occurrence? Out of the masses emerged mysteriously a
Michael Bradley, the right man at the right time, a computer professional. He
came to help us put the CLCC faith formation Manual into final form for
publication. To Michael Bradley very special and sincere thanks for the amount
of selfless time he has generously donated to the CLCC process, even taking six
weeks off from work. Thanks also to Fr. Brian Van Hove, S.J., Board member of
the NCCTS, for introducing Michael Bradley to us to become part of this
mission. Finally, warm thanks to family and many dear friends, who by their
daily prayers, were in large and small ways instrumental in making the CLCC
faith formation process a successful journey from idea to reality.

% ------------------------------------------------------------------------------

\pagebreak
\tableofcontents

% ------------------------------------------------------------------------------

\chapter{Primary Sources}

Citations in this manual are selected from the following list of official Church
documents: Vatican II documents, encyclicals, apostolic exhortations and papal
pronouncements. They are listed here in the order in which they appear in this
manual, not alphabetically. Also listed is nomenclature proper to the CLCC
process.

CL Christifideles Laici, The Lay Faithful of Christ Apostolic Letter, John Paul
II, 1988 CCC Catechism of the Catholic Church Libreria Editrice Vaticana, Second
Edition, 1997 PDV Pastores Dabo Vobis, I Will Give You Shepherds Apostolic
Exhortation, John Paul II, 1991 NMI Novo Millennio Ineunte, Beginning of the New
Millennium Apostolic Letter, John Paul II, 2001 LG Lumen Gentium, Light of the
Nations, Dogmatic Constitution of the Church Second Vatican Council, 1964 EN
Evangelii Nuntiandi, On Evangelization in the Modern World Apostolic
Exhortation, Paul VI, 1975 RMiss Redemptoris Missio, Mission of the Redeemer
Encyclical Letter, John Paul II, 1990 PO Presbyterorum Ordinis, Decree on the
Ministry and Life of Priests Second Vatican Council, 1965 CD Christus Dominus,
Decree on the Bishops Pastoral Office in the Church Second Vatican Council, 1965
AA Apostolicam Auctuositatem, Decree on the Apostolate of the Laity Second
Vatican Council, 1965

DV Dei Verbum, Dogmatic Constitution on Divine Revelation Second Vatican
Council, 1965 CT Catechesi Tradendae, Catechesis in Our Time Apostolic
Exhortation, John Paul II, 1979 GS Gaudium et Spes, Pastoral Constitution of the
Church in the Modern World Second Vatican Council, 1965 EE Ecclesia de
Eucharistia, Church of the Eucharist Encyclical Letter, John Paul II, 2003 SC
Sacrosanctum Concilium, Constitution on the Sacred Liturgy Second Vatican
Council, 1963 DI Dominus Iesus, On the Unicity and Unity of Christs Church
Declaration, Congregation for the Doctrine of the Faith, 2000 EV Evangelium
vitae, The Gospel of Life Encyclical Letter, John Paul II, 1995 DD Dies Domini,
Day of the Lord Apostolic Letter, John Paul II, 1998 DeV Dominum et
vivificantem, The Holy Spirit in the Life of the Church and the World Encyclical
Letter, John Paul II, 1986 Did Didascalia, Teaching of the 12 Apostles and
Disciples of Our Savior Catechetical text of the 3rd century DC Deus Caritas
Est, God is Love Encyclical Letter, Benedict XVI, 2005 SC Sacramentum Caritatis,
The sacrament of charity Apostolic Exhortation, Benedict XVI, 2007 SD Salvifici
Doloris, The Christian Meaning of Human Suffering Apostolic Letter, John Paul
II, 1984 CV Caritas in veritate, Charity in Truth Encyclical Letter, Benedict
XVI, 2009

CLCC Celebrating Life as a Catholic Christian SEFC Small Ecclesial Faith
Communities TS Total Stewardship

% ------------------------------------------------------------------------------

\chapter{Introduction}

This is introduction stuff.1. Celebrating Life as a Catholic Christian (CLCC) is
a process by which lay Catholics are formed in spirituality within Small
Ecclesial Faith Communities (SEFC) which are established in and by the
parish. In addition to becoming better formed in the faith, the CLCC process
prepares SEFC members to carry out the critically needed apostolate of reaching
out to lapsed, non-practicing Catholics with the aim of helping them return to
the Church. For those who are willing to join a SEFC for re-introduction to the
faith, the faith community then acts as the staging ground for their re-entry
into the Church. Fundamental to this process is Saint James astounding and
unequivocal teaching faith of itself, if it does not have works, is dead
... [F]aith without works is useless. (cf. Jas 2:17,20; cf. CCC 1815) Obviously
the goal of this process is to develop faith more deeply in the People of God,
and its apostolic outreach is conversion of fallen-away Catholics for building
up the kingdom of God.  Recently a saddened father confided to me that five of
his six adult children do not go to Mass on Sunday. All were educated in the
best of Catholic schools from the elementary level to college. All were trained
in the faith. They dont deny the faith, he said, they just ignore it, or try to
justify their troubled conscience by saying they dont buy it all.

2. This book will show why Pope John Paul II has stressed the need of formation
in holiness and the need of New Evangelization that he describes in several
apostolic letters and exhortations issued during his pontificate. For example in
the apostolic exhortation, Christifideles Laici, The Lay Faithful of Christ, he
states that between God who offers his gifts, and the person who receives them,
that person is called to exercise responsibility, indeed the necessity, of a
total and ongoing formation. Quoting the Synod Fathers (1988), the Holy Father
writes, Christian formation (is) a continual process [emphasis added] of
maturation in faith and a likening to Christ in the individual, according to the
will of the Father, under the guidance of the Holy Spirit. The Holy Father
emphasizes that formation of the lay faithful must be placed among the
priorities of a diocese. It ought to be so placed within the plan of pastoral
action that the efforts of the whole community (clergy, lay faithful and
religious) converge on this goal. (cf. CL 57)

3. The plan of action which the Pope has in mind is embodied in this
process. Because Catholics for two or more generations have been minimally
catechized and therefore minimally formed in the faith, the need of sound,
orthodox catechesis is not only imperative but may not be delayed. The RCIA
boasts of newcomers and returnees to the faith in the tens of thousands each
year. But in not a few cases because they feel unanchored to the Church many
drift way.

4. Unique to Small Ecclesial Faith Communities in the CLCC process is that not
only do fallen-away Catholics have a welcome place to come to, but also new
Catholics, graduates of the RCIA who have the glow of their Easter Vigil
sacramental experience, have a haven to come to for bonding, support, prayer
life, ongoing formation in doctrine and maturation in the faith.

5. Many conscientious parents express concern when they see in their adolescent
and college age children a faith crisis. Their concern is selection, to which
Catholic school should we send them, since some Catholic schools are Catholic in
name only. They see the faith crisis in their own parish: low Mass attendance,
often irreverent liturgies, and certain sex education programs in their schools
which instead of promoting and preserving chastity destroy the childrens
innocence.

6. The growing priest shortage in both diocesan and religious orders is finally
hitting home. Where will they go for Mass? Hard to believe but true, the priest
shortage began some forty years ago when an admission policy to seminaries was
adopted which has been tabbed as artificial and contrived. This means that
screening of applicants favored those who were liberal, open and modern. Men
found to be traditional, obedient and prayerful were believed to be rigid and
therefore not accepted. The fruits of this policy are being reaped today. We
ought never to feel that God is not calling men to the priesthood. He is. For
reasons too numerous to mention here, young men are not responding, but in time
they will. Moreover, the new priests ordained in a given year are far too few to
replace old and ill priests who retire from ministry or die, increasing the
number of priestless parishes.

7. Hope should be our trademark. The SEFC is an ideal environment for a young
man to experience Church and discern his vocation to serve it. Members will help
him follow his vocation quest, pray for him, support him and see him to
ordination. The duty of fostering vocations falls on the whole Christian
community, stated Pope John Paul II, and they should discharge it principally by
living full Christian lives. (cf. PD 41)

8. Do something is the outcry today. A Spirit-filled laity is emerging to do
something about the crisis of faith they see around them. No longer do they feel
comfortable shifting problems to Father or the Bishop. The Holy Spirit is
working in their lives and in families awakening them to the powers that
Baptism, Confirmation and Eucharist impart, especially awareness of their right
and duty to participate in the saving mission of the Church. They learn that to
share in the mission of the Church does not give them license to act
independently of the local pastor or of the bishop or pope. Rather they strive
to share responsibility in the Churchs apostolates in friendly communion with
the pastor, the designated head and shepherd of the flock.  Part I of this book
presents the CLCC process in its 5 stages. Part II, titled Authoritative Sources
consists of passages taken from 20 key teachings of the Church. Both Part I and
II present almost inexhaustible material for faith formation, preaching and the
apostolate to the lapsed.

% ------------------------------------------------------------------------------

\chapter{Putting the CLCC Process to Work}

The Process 1. Celebrating Life as a Catholic Christian (CLCC) is a process
which systematically evangelizes Gods people in spirituality. At the same time
it stimulates them to carry out their missionary duties, the major one, helping
lapsed, non-church going Catholics return to the Church. In addition to citing
many official Church teachings, this process incorporates a number of the new
initiatives that Pope John Paul II calls for in his apostolic letter, Novo
Millennio Ineunte, On Entering the New Millennium, for spreading the Gospel in
the new century.  This process confronts the faith and cultural crisis that
currently plagues millions of Catholics, many of whom seem blissfully unaware of
the gravity of their situation.  By means of New Evangelization and a catechesis
for holiness, the CLCC process aims to give Gods people a bold Catholic
identity, a way to live and a way to help other people live genuinely Christian
lives not succumbing to the pressures of secular society nor to the corrosive
influences of anti-Christian and atheistic forces that are corrupting the
culture. In opposition to these forces is this thoroughly Christ-centered CLCC
faith formation process. It presents Jesus public ministry in Galilee in five
stages, while at the parish level the CLCC process enacts Jesus Galilean
ministry in five parallel stages.

Key People 2. Key people who endorse the process are the priest, pastoral
council and heads of parish organizations. They make a united decision to
undertake the CLCC process in the parish and commit to support it.  Their first
job is to select 12 men and women, after the 12 apostles, called the
Matrix. These 12 become the core group of lay persons who lead the process. They
agree to lead the process through its 5 Stages. Their first responsibility is to
learn the process in a general way. As the process unfolds they learn the
specific responsibilities relative to each stage.

Small Ecclesial Faith Communities 3. The Matrix 12, the core lay leadership
group, is the first of many Small Ecclesial Faith Communities (SEFC) formed in
the parish. Many more Small Ecclesial Faith Communities will follow. Indeed
small communities of different kinds already exist in parishes. However, it is
advised and highly useful for parish community building (communion) that
existing groups be willing to be part of the CLCC process. The following and all
others are invited and urged to join: existing bible study groups, prayer
groups, religious education instructors, parish school teachers, the liturgy
committee, choir members, the youth group, young adults, music ministers, the
married, single, widowed, the pro-life group, St. Vincent de Paul Society,
ushers, greeters, collection counters, the finance committee, sports directors,
maintenance people, all. There is no one in the parish not in need of further
spiritual growth and updating in his faith. After Jesus called the 12 Apostles
(cf. Mt 4:18,21; 10:1-4), and appointed 72 disciples (cf. Lk 10:1; Mk 3:16),
they followed Jesus to learn his way of life. They grew in the blessedness Jesus
proclaimed and exemplified. Jesus formed the first small faith
community. Through the centuries thousands of small faith communities have grown
into communions, called dioceses and parishes, which constitute the
infrastructure of the Church, the Body of Christ.

CLCC is not a Quick-Fix 4. The CLCC is decidedly a process, and not a program! A
program has a beginning and an end, whereas a process is developmental. Through
a series of small steps taken within several stages it advances toward certain
desired goals. Unlike programs, the CLCC process is ongoing. Its
open-ended. Being a process of formation in holiness and in knowledge of the
faith, usually a radical change in lifestyle begins. The CLCC process is,
therefore, not a quick-fix. Saints are not made overnight. Time is needed for
lay leaders and parishioners to get the process underway. People need time to
understand the process and act on it, especially the astounding prospect of
becoming holy. True spiritual warriors who do battle with Satan to save lapsed
Catholics need time for solid spiritual formation. Thus, the CLCC process may
not be hurried or short-circuited. Every temptation to get it over must be
banished. Time is the soul of the CLCC process. Time will show that the process
achieves hoped for and measurable results. Personal holiness and zeal for
recovering lost and lapsed Catholics cannot be bought online by credit card.

How to Start the CLCC Process 5. Parishes have a choice of three ways to start
the CLCC process.

A five day mission, Sunday to Thursday, with preaching centered on the meaning
and need of the process and an explanation of the 5 stages, one stage explained
each evening.  A three, four or five day traditional mission on the eternal
truths: need of salvation, the reality of death, judgment, Gods loving mercy,
the need of making a good, integral confession, heaven or hell, and the Blessed
Mother.  The parish priest announces the process over several weekends,
explaining what it is, and the absolute usefulness of everyone entering in and
belonging to a Small Ecclesial Faith Community. He emphasizes its twofold focus:
catechetical formation in the faith and in holiness, and outreach to recover
lapsed Catholics who have strayed from the faith. He points out the glory that
will be given to the Lord when all Catholics in the parish, active and
alienated, will begin Celebrating Life as Catholic Christians.

Holding a mission prior to initiating the process has its advantages, but a
mission is optional. A mission can bring the people to a spiritual high. But to
implant a faith formation process in the parish like the CLCC as a follow-up to
a mission is a perfect way to build on the missions high and its momentum. In
other words, strike while the iron is hot! In any case the parish priests
preaching on the process has value that ought not be underestimated. Hearing him
parishioners will know immediately that he and the lay leaders alike are making
the CLCC a top parish priority.

Like advertising on TV, parish leaders have a sell job that is non-stop. Those
slow to buy into the process will eventually come around. However parishioners
who for reasons of health, age, handicap or other good reasons cannot join a
SEFC are encouraged to take up the spiritual formation process privately. They
are also urged to pray for the success of the CLCC process, for example, praying
the rosary daily or doing some good work, invoking Gods blessing on all those
entering the process and for a great spiritual awakening in the parish.

Importance and Necessity of Prayer 6. Jesus said solemnly and emphatically,
Whoever remains in me and I in him will bear much fruit. (cf. Jn 15:5) Prayer
unites us to the Lord. Prayer creates a bond between Christ and the person who
prays. For where two or three are gathered together in my name, there am I in
the midst of them. (Mt 18:20) In prayer we come to him, and he comes to
us. Through prayer good things happen, and good things are given us as Jesus
promised. Thus, to begin the CLCC process is one thing. To have the CLCC process
bear much fruit is another. The fruit is a new level of holiness in all parish
members who participate in the formation process. Holiness is brought about
through the experience of planting, pruning, cultivation and reaping of New
Evangelization, orthodox catechetical formation, holiness, conversion of the
lapsed and the return of many to Christ.  Why the emphasis on remaining united
to the Lord through prayer? Jesus gives this straight reply, because without me
you can do nothing.Jesus analogy of the vine and branches applies to the
importance and necessity of prayer. He said, I am the vine, you are the
branches. He exhorts, Remain in me, as I remain in you. He warns, Just as a
branch cannot bear fruit on its own unless it remains on the vine, so neither
can you unless you remain in me. (Jn 15:4-5) In sum, the CLCC process cannot
bear fruit in the parish unless all persons involved are united to the Lord
through the grace-filled power of prayer.

Desired Results of the Process 7. What are some of them? When parishioners show
greater faith and fervor in their worship at Mass; when growing numbers of
parishioners, by Gods grace, show real earnestness about living their lives as
Catholic Christians; when after a period of time a detectable air of humility
and holiness wafts through the parish; when a new and fresh spirit of
involvement and cooperation become evident among the People of God.  Evidence of
progress in holiness will also be seen when Catholics who have been away from
the sacraments for many years seek reconciliation with the Church through the
sacrament of penance; when new faces begin to appear at Sunday Mass alongside
the regulars, thus increasing Mass attendance; when liturgical celebration
becomes more transcendent and less banal; when people ask for public adoration
of the Blessed Sacrament and come faithfully to make a personal hour of
adoration; when marriage becomes esteemed as a sacred life-long covenant; when
spouses become more forgiving and compatible; when family life becomes more
peaceful and stable and children are parent-directed, rather than parents being
directed by children.  As holiness grows among the actives in SEFCs, so interest
in missionary activities will grow. Awakened will be a sense of responsibility
to reach out to Catholics who are religiously comatose and critically in need of
New Evangelization. The agent of this responsibility is Gods grace. The Holy
Spirit impels them to seek out and help lapsed brothers and sisters to join a
SEFC for return to the Church. The grace-filled actives recover the lost. They
are builders of communion in the Church. They help bring the Body of Christ to
full stature.

Lay Leaders and Small Ecclesial Faith Communities Organization 8. Lay leaders of
the CLCC process are chosen for having shown some organizational and management
skills. More importantly they possess a strong desire to grow in personal
holiness and in knowledge of the faith and want to help fellow parishioners do
likewise. Recognizing their own hunger for truth and holiness, they feel certain
others experience the same hungers but have not diagnosed their condition as
primarily spiritual rather than psychological, social, marital or material. Thus
leaders are willing to learn and lead a spiritual conversion process, the CLCC,
which is of utmost importance for soul, mind, heart and body.  Lay leaders first
organize themselves. They form the first Small Ecclesial Faith Community
(SEFC). Their main function is to hold weekly spiritual formation sessions for
themselves. In them they learn how to conduct a spiritual formation session, how
to use the material in the 5 Stages starting with Stage 1, and later going into
Part II, titled Authoritative Sources. Formation consists of internalizing the
content of each paragraph in each stage following the order for proper conduct
of a session.

9. Lay leaders in the first SEFC, also called the Matrix 12 after the 12
Apostles, choose from among themselves a Moderator and an Associate
Moderator. They are advised not to hire an outside facilitator. Educated,
qualified and motivated people that they are, they can and will become expert
facilitators themselves. Should questions arise, you are welcome to e-mail the
NCCTS at anytime: questions@clcc.info.  Lay leaders are also responsible for
initiating in the parish other Small Ecclesial Faith Communities (SEFC). Like
the Matrix 12, SEFCs should ideally have 12 members. They may have 10 but ought
not to go below 7. There is no limit to the number of SEFCs that can be formed
in a parish, the more the better. They should dot every area within the parish
boundaries. Like the Matrix 12, members of SEFCs moderate themselves, choosing
persons within the group whom they feel are best qualified to lead.

Functions 10. Moderators call and conduct the sessions. They follow strictly the
Order for Conducting Spiritual Formation Sessions (see the next chapter). All
members should be willing and given a chance to take turns moderating a
session. Sessions may be held in peoples homes or in parish
facilities. Experience shows that preference tilts to peoples homes. A private
home is less institutional and less threatening to returnees to the faith than a
church meeting hall.  Most important is that the moderator keeps the session on
track without deviation, that is, he stays strictly to the content of the
paragraph under consideration. Allowing participants to wander off track,
letting some members dominate the discussion, to air personal gripes and
distract the group from its spiritual focus -- these are sure ways to kill the
group and destroy the process.  Moderators are to begin the session at the
agreed upon time, giving each member a chance to briefly share (without
coercion) their spiritual insight or experience with the group. The session must
conclude exactly at the agreed upon time. To go overtime, no matter how lively
the discussion, is also a sure way to kill the formation process and the groups
commitment to it.  If some members choose to continue the sessions discussion,
they are free to do so after it concludes. The moderator must conclude the
formation session in 90 minutes in consideration of those who have other
obligations.

The weekly session is 90 minutes, 45 minutes for the first half, Interior
Formation, then 45 minutes for the second half, Exterior Action and Conclusion.

% ------------------------------------------------------------------------------

\chapter{Order for Conducting Formation Sessions}

Hymn Select an appropriate religious song familiar to all missalette, hymn book
or CD.

Time Frame 90 minutes Start on time; end on time; do not go overtime.

Opening Prayer Suggestion, use a liturgical prayer in accord with Churchs
season.

Source Part I, material as given in 5 stages.  Part II, material in
Authoritative Sources.

Subject Matter Each stage contains many short and long paragraphs. For example,
Stage 1 has about 45 paragraphs, enough material for 45 formation sessions. Read
aloud only one paragraph per session, and from it choose a sentence, phrase or
short excerpt that strikes you. That choice is your subject matter; every person
in the SEFC makes a similar choice from the same paragraph.

Interior Spiritual Formation 45 minutes All spend a short time in quiet,
prayerful reflection, each member opening himself to the Holy Spirit to receive
insights for spiritual development. After a short time members of the SEFC share
their insights (gifts) with the full group. Stay with the subject matter of the
paragraph; do not wander off to other paragraphs, as this will distract you and
confuse the group. Strive for togetherness, bonding, and the all-important
communion in spirit and truth that links the People of God in SEFCs.

Exterior Action 45 minutes, includes Action Resolution and Conclusion In
personal, quiet time all reflect briefly on an action the Holy Spirit
suggests. The action centers always on how you or the group can advance the CLCC
process. Invoke the Holy Spirit for light and Mary, Mother of the Church, for
heavenly assistance. Share with your group the action you choose to do either
alone or with the group; try to reach consensus on the best action to take.

Action Resolution Deciding on a personal or group action has unlimited
possibilities. Example: help make the formation sessions run smoothly so that
all members learn and get used to the Order given here; make interior formation
prayerful and holy; respect everyones right to speak; start thinking of
parishioners to invite into your SEFC to bring the number up to ten or twelve;
begin making a list of lapsed, non-church going Catholics who need to be visited
and brought into your SEFC. Exterior action includes, if possible, going to
weekday Mass, making a visit to the adoration chapel and praying for the
conversion of lapsed Catholics.

Conclusion Prayers of intercession come next led by the moderator or a group
member. Such prayers may be printed, pre-prepared or spontaneous. Always include
a petition that ALL parishioners enter into the CLCC formation process.

Closing Prayer Selected from the Churchs liturgical prayers for the previous
Sunday, the current feast or memorial, etc.

Hymn From the missalette, hymn book or CD.  Total session time should not exceed
90 minutes.

% ------------------------------------------------------------------------------

\mainmatter
\addtocontents{toc}{\cftpagenumbersoff{part}}
\part{The Process}
\addtocontents{toc}{\cftpagenumberson{part}}

% ------------------------------------------------------------------------------

\chapter{Stage 1.\ Call of Apostles and Disciples}

\section*{} \lxRDFa{property=stage-main-content,resource={manual}}

Important note:  Before beginning Stage One, you are asked, please, not to
quick-read this book as you may read a novel or the newspaper. This is a study
manual of deep content that aims to help Catholics rise to higher levels of
formation in the faith for growth in personal holiness. The Small Ecclesial
Faith Community (SEFC) is the best suited environment for implanting in
participants' minds, hearts and souls some of the great truths taught by Christ
and his Church. So that everyone is on the same page, please go to the front
pages of this manual and reread Putting the CLCC Process to Work. Like a three
dimensional image you will ``see'' clearly the two specifics of the CLCC
process: formation of God's people in personal holiness, and motivating them for
mission:  to reach out to and help lapsed and alienated Catholics return to
Christ and his Church.

The first action in a CLCC formation session is the leader's reading aloud a
segment, a paragraph or group of paragraphs, from the manual to the SEFC. At the
bottom of the segments stand the letters RDS. R stands for Reflection: quiet
time given to understand the content of the section. D stands for Discussion:
personal prayer time directed to the Holy Spirit asking him for light,
dialoguing with him within oneself and ordering the received insights. S stands
for Sharing: each participant contributes to the group the insights he or she
has received and wants to share. Alongside RDS is given a page number to turn to
for aiding the group's Reflection, Discussion and Sharing.


Introduction
1. Jesus' three-year public ministry in Galilee was a divinely planned process
of salvation for the whole human race. It is divided into five stages, each
stage is distinguishable from the other, yet each stage is connected to the
other. Celebrating Life as a Catholic Christian (CLCC) is a five-stage faith
formation process modeled on Jesus' public ministry in Galilee. It is designed
for implementation in parishes of the local Church.
RDS, page \#

2. Jesus' ministry has two goals: giving his disciples a new way of life in
faith, the other is missionary outreach. The two goals of the CLCC go together:
interior spiritual formation conjoined to an exterior action.

The first goal focuses on personal and communal spiritual growth that gives the
believer a deeper knowledge of Christ and his Church, and on improving his or
her prayer life. To do this believers enter a small faith group to experience
bonding with like-minded people for community building. These goals help God's
people answer Christ's call, ``Come, follow me,'' (Lk 18:22) and his command,
``Love one another.'' (Jn 13:34)

The second goal focuses on a particular exterior missionary action, reaching out
to lapsed, non-church going Catholics, estimated at 45 million in the US, with a
view to helping them return to Christ for reconciliation and salvation, thus
building up the kingdom of God on earth, the Church.
RDS, page \#

For better understanding and results-producing pastoral use of the CLCC process,
each of the five stages has essentially two parts: A. What Jesus did in Galilee
and B. Enacting in the Parish What Jesus did in Galilee.

A. What Jesus Did in Galilee
1. The first thing Jesus did in his public ministry in Galilee was to gather
followers and from among them he called and selected 12 apostles. They were
select witnesses of his teachings, his miracles, his way of life and his
communion with the Father in prayer. Through his Passion, Death, Resurrection
and Pentecost, Christ founded his Church, the kingdom of God's people on earth
as designed by the Father, a kingdom that will last till the end of time through
the sustaining presence and power of the Holy Spirit. In founding his Church on
earth Jesus guaranteed its continuity and triumph despite the fierce powers of
hell raging against it. (cf. Mt 16:18)
RDS, page \#

2. The terms ``Christian'' and ``Catholic'' were unknown in Jesus'
day. (cf. Acts 11:26) Yet what Jesus was teaching and sharing with his Apostles
and disciples was Celebrating Life as Catholic Christians (CLCC). With their own
eyes and ears they saw and heard the primacy Jesus gave to Commandments of love
of God and love of neighbor. Daily he showed this love by his self-giving to
everyone to the point of exhaustion. They saw the compassion he showed toward
repentant sinners and his rebuke to those who exalted themselves; how he
detested hypocrisy, religiosity, pseudo-piety and false worship. In the many
different ways that Jesus dealt with people his followers were to imitate him.
RDS, page \#

3. Catholics and other people are immersed in the present-day culture of
secularism and neo-paganism. They need a defense against the barrage of
anti-Christian influences that surround them. They must be armed to deal with
the growing, overt hostility against Christianity, the Catholic Church in
particular, at themselves and the values they live by. Thus, an empowerment
process like Celebrating Life as Catholic Christians becomes imperative, and
should be classed among the leading parish apostolates. God's people under siege
need guidance, group support and grace to live authentic Catholic lives, gifts
God gives through the CLCC formation process.
RDS, page \#

4. Call of the 12 Apostles and 72 Disciples
Biblical texts

Mt 4:18-22
``As Jesus was walking by the Sea of Galilee, he saw two brothers, Simon whom he
called Peter, and his brother Andrew casting a net into the sea; they were
fishermen. He said to them, 'Come after me, and I will make you fishers of men.'
At once they left their nets and followed him. Walking a little farther he saw
other brothers, James and John, Zebedee's sons, and he called them and
immediately they left their boat and their father and followed him.''

Mt 9:9-10
``He also called Levi, named Matthew, who was sitting in his customs post
collecting taxes. He said to him, 'Follow me.' And he got up and followed
him. While he was at table in his house, many tax collectors and sinners came
and sat with Jesus and his disciples.''

Lk 6:12-13
``In those days he departed to the mountain to pray, and he spent the night in
prayer to God. When day came, he called his disciples to himself, and from them
he chose Twelve, whom he named apostles.''

Mt 10:1-4
``The names of the twelve apostles are: first, Simon called Peter, and his
brother Andrew, James, son of Zebedee and his brother John; Philip and
Bartholomew, Thomas and Matthew, the tax collector; James, son of Alpheus, and
Thaddeus, Simon the Cananean and Judas Iscariot who betrayed him.''
RDS, page \#

5. Mission of the Apostles and Disciples
Biblical Texts

Lk 9:1-2,6
``He summoned the Twelve and gave them power and authority over all demons and
to cure diseases, and he sent them to proclaim the kingdom of God and to heal
[the sick]. Then they set out and went from village to village proclaiming the
good news and curing diseases everywhere.''

Lk 10:1-3
``After this the Lord appointed seventy [two] others whom he sent ahead of him
in pairs to every town and village he intended to visit. He said to them, 'The
harvest is abundant but the laborers are few; so ask the master of the harvest
to send laborers for his harvest. Go on your way; behold, I am sending you like
lambs among wolves.'''
RDS, page \#

6. Catechism of the Catholic Church
The following quotations give a concise summary of the Church's teaching on the
Mission of the Apostles.

CCC 75
``Christ the Lord, in whom the entire Revelation of the most high God is summed
up, commanded the apostles to preach the Gospel, which had been promised
beforehand by the prophets, and which he fulfilled in his own person and
promulgated with his own lips. In preaching the Gospel, they were to communicate
the gifts of God to all men. This Gospel was to be the source of all saving
truth and moral discipline.''

CCC 1575
``Christ himself chose the apostles and gave them a share in his mission and
authority. Raised to the Father's right hand, he has not forsaken his flock but
keeps it under his constant protection through the apostles, and guides it still
through these same pastors who continue his work today. Thus, it is Christ whose
gift it is that some be apostles and others pastors. He continues to act through
the bishops.''
RDS, page \#

B. Enacting What Jesus Did in Galilee in the Parish
1. Leadership of the Priest
Jesus chose 12 apostles to assist him in his mission and provided for their
continuing his mission. Likewise the parish priest must find lay persons to
assist him in his mission, an important one, of launching the CLCC process. Pope
John Paul II states that the parish priest is ``configured to Christ, the good
shepherd.'' (cf. PD 22) It is the pastor who goes out among the sheep and lambs
of his parish. He scans the flock, selects and calls 12 men and women of all age
groups, young, middle aged and older, who can help him carry out the CLCC
process in the parish. They become the lay leaders of the CLCC process. They are
called the Matrix 12.

The priest informs the people in the pews about the CLCC process, explains what
it is, and asks for their participation in it. The mission of the lay leadership
group, the Matrix 12, is to learn the process and guide it through its 5 stages,
receiving throughout support and encouragement from the priest and cooperation
from fellow parishioners. For its success the priest's initiative and active
involvement in the process cannot be emphasized enough. Strongly recommended is
that this process becomes a parish priority, taking precedence over less urgent
ministries and parish activities.

In order to select the right people for carrying out the process, the priest
should pray, as Christ did, and greatly depend on the light of the Holy Spirit
for the gift of discernment to ensure he chooses the right parishioners to
become the Matrix 12. He should also pray to be open to the Holy Spirit's
bidding, that he chooses other gifted parishioners to form SEFCs, and resist the
temptation to select ``never say no'' workers, favorites or super-involved
people often already over-burdened with parish projects.

The priest must always keep in mind the fundamental purposes of the CLCC
process.The first purpose of the CLCC process is to help people attain new
levels of personal and communal holiness in preparation for doing effective
evangelization. The second purpose is missionary, to find and help save the
wayward sheep, lapsed Catholics. Recall the gospel parable of the lost
sheep. ``What man among you having a hundred sheep and losing one of them would
not leave the ninety nine in the desert and go after the lost one until he finds
it? When he does find it  he calls together his friends and neighbors  Rejoice
with me because I have found my lost sheep  in just the same way there will be
more joy in heaven over one sinner who repents than over ninety-nine righteous
people who have no need of repentence.'' (cf. Lk 15:4-7)
RDS, page \#

2. Teachings of Vatican Council II
``Just as the Son was sent by the Father, so too he sent apostles.'' (cf. Jn
20:21) The Church has received from the apostles the task to be Christ's
``witnesses'' and proclaim his saving truth ``even to the ends of the earth.''
(cf. Acts 1:8) ``Hence she makes the words of the Apostle her own: 'Woe to me,
if I do not preach the gospel' (cf. 1 Cor 9:16), and unceasingly sends heralds
of the gospel [to]  carry on the work of evangelizing. The Church is compelled
by the Holy Spirit to do her part towards the full realization of the will of
God, who has established Christ as the source of salvation for the whole
world. By proclamation of the gospel, she prepares her hearers to receive and
profess the faith, disposes them for baptism, snatches them from the slavery of
error, and incorporates them into Christ so that through charity they may grow
up into full maturity in Christ  The obligation of spreading the faith is
imposed on every disciple of Christ according to his ability.'' [emphasis added]
(cf. LG 17)
RDS, page \#

3. Apostles and Disciples, the First Small Faith Community!
During his three-year public ministry Jesus not only chose 12 apostles and 72
disciples, but also formed ``the 12'' into a small, close-knit community, the
first of its kind. They were not men who merely followed Jesus passively. Jesus
``called'' them, formed them and ``appointed'' them by divine design to become
active co-workers with him. They were the first to come to know the purpose of
his mission: ``to be evangelized [in order to] evangelize,'' words of Blessed
John Paul II.

``He appointed 72 others whom he sent ahead of him in pairs to every town and
place he intended to visit.'' (Lk 10:1). They were witnesses to Christ. He
empowered them to ``cure the sick'' in whatever town they entered and welcomed
them, and to say, ``The kingdom of God is at hand for you.'' (cf. Lk 1:5,8-9)

The community of apostles and disciples in Galilee has become a model for the
establishment of Small Ecclesial Faith Communities (SEFCs) in parishes and
dioceses. The importance of SEFCs cannot be measured, for they implant in God's
people a good that is at once spiritual, personal and communal. They strive to
achieve an in depth sense of community, which is most needed in today's secular
and individualized culture. Fostering community among God's people is imperative
for families, for parishes and dioceses, and above all in the secular work
places of the laity.
RDS, page \#

4. Models of Leadership in the Old Testament
Words like ``call,'' ``chosen,'' ``mandated,'' ``sent,'' and ``mission'' were
central in Christ's ministry. Consider some of the leading figures in the Old
Testament who were also specially called by God and formed to carry out a
mission. They were models for Christ's disciples and are models for our
formation and mission.

Examples
Samuel, when only a boy, was called by God (1 Sam 3:3b-19); became a great seer
(prophet) and judge (ruler) of the 12 tribes of Israel (1 Sam 7:12,15-17).

Jonah, a prophet, was sent on mission to the city of Nineveh to predict
destruction for its evil ways. In sackcloth the king and people turned away from
their sins and God preserved them and their city (cf. Jon 3:1-5,10).

What about Abraham, his ``call'' (cf. Gen 12:1-3), the spiritual seed (Gen 15),
and the test of his faith? (Gen 22)

Joseph, son of Jacob, born of Rachel (Gen 30:24), a shepherd boy (Gen 37:2-3),
sold to slavery in Egypt only to rise to be steward of Pharaoh's vast country
(Gen 41:41).

Moses who was ``called'' (Ex 3:2-10), led Israel out of Egypt across the Red Sea
(Ex 13-14), and to whom God gave the 10 Commandments on Mt. Sinai (Ex 19-20).

Who can forget the courageous Esther, a Jewess, who by divine plotting became
queen to King Ahasuerus of Persia where Jews were targeted for extermination by
order of the king, but who by her brave intervention with him and admission she
was Jewish saved her people from massacre? (Esth 1-9; cf. also Dictionary of the
Bible, John L. McKenzie, S.J)

To appreciate how God in his grand plan of salvation called, chose and
dispatched certain people to carry out his will, it is useful to look up the
above texts in your bible. Many more models could be added.
RDS, page \#

5. Small Ecclesial Faith Communities, Why Are They Needed?
Most parishes, large, medium-sized and small have a sizable number of persons
who call themselves Catholic, but are in fact lapsed, alienated, non-practicing
Catholics. Many are weak in their knowledge of the faith, and in numerous cases
almost totally ignorant of even the fundamentals of the Catholic Faith.

Some who dissent from the faith nurse a bias against certain truths of the faith
they don't like, for example, artificial contraception, cohabitation, same sex
marriage, etc. Some reject outright various teachings of the Church, and justify
their rejection by appealing to the principle of ``religious freedom.'' They
claim their conscience is at peace. Not a few ``in name only Catholics'' are
graduates of Catholic colleges. They enrolled in the school with shaky faith and
not infrequently left the school with ``freedom'' not to practice the faith.

Catholics such as these need to be reached, befriended and won back to the
Church. Some may be looking for a way back into the fold. Others may be
closed-minded and resist any invitation to return to the Church. The power of
God's grace can never be under-estimated. For some, the RCIA program fills their
faith gap. Others, really the majority, need to be brought to a more lively
practice of the Faith through the slower-moving, more in-depth dynamics of Small
Ecclesial Faith Communities where welcome is warm and camaraderie
long-term. Within them results are more predictably measurable.Two outstanding
Pontiffs have explicitly endorsed and promoted small faith communities in their
teachings (cf. Evangelii Nuntiandi, EN, Paul VI, 1975, 58, 60; Redemptoris
Missio, RM, John Paul II, 1990, 51). In faith communities, Church-going
Catholics advance admirably to higher levels of knowledge of their Church's
teachings and in holiness. Alienated, lapsed and ``Christmas and Easter''
Catholics who accept invitations to come into such a community will first
experience bonding with the members; then, by association with them in study,
prayer and sharing, and by the power of God's grace, they will become disposed
to return to the Church.
RDS, page \#

6. Blessed Trinity, Archetype of Small Ecclesial Faith Communities
Under the heading THE PERSON AND SOCIETY, the Catechism of the Catholic Church
includes this sub-heading, The Communal Character of the Human Vocation, and
makes this statement, ``All men are called to the same end: God himself. There
is a certain resemblance between the unity of the three divine persons and the
fraternity that men are to establish among themselves in truth and love. Love of
neighbor is inseparable from love of God.'' (CCC 1878) It goes on to say that
each community is defined not only ``by its purpose  and consequently obeys
specific rules; but the human person  is and ought to be the principle, the
subject and the end of all social institutions.'' (CCC 1881)

Singled out as first is the communal family, followed by ``the state,''
``voluntary associations and institutions  on both national and international
levels  various professions, economic, cultural and recreational, including
sports.'' (cf. Mater et Magistra, John XXIII, 1961, 60; cf. also CCC 1882)

In his encyclical, Peace on Earth, Blessed John XXIII wrote: ``Human society
must primarily be considered something pertaining to the spiritual. Through it,
in the bright light of truth, men should share their knowledge, be able to
exercise their rights and fulfill their obligations, be inspired to seek
spiritual values  and eagerly strive to make their own the spiritual
achievements of others.'' (Pacem in Terris, PT, 1963, 36; cf. also CCC 1886)

Addressing the issue of authority, Blessed John XXIII stated: ``Human society
can be neither well-ordered nor prosperous unless it has some people invested
with legitimate authority to preserve its institutions and to devote themselves
as far as is necessary to the work and care for the good of all.'' (PT 46;
cf. also CCC 1897) ``Authority is exercised legitimately only when it seeks the
common good of the group concerned, and if it employs morally licit means to
attain it, [otherwise it is] not binding in conscience.'' (PT 51; cf. also CCC
1903)
RDS, page \#

7. SEFCs, Front Doors for Lapsed Catholics to Come Home to the Church
Small Ecclesial Faith Communities (SEFCs) are front doors with welcome mats out
where Catholics away from the Church can enter the house of the Lord and begin
to feel accepted and and be enkindled with love of the Catholic faith.

Every Catholic, including the lapsed, on entering a Small Ecclesial Faith
Community, begins a process of change. SEFCs are established for the specific
purpose of change. Within them members become notably better formed in the
faith, strive to develop a deep love for Christ from the truths they learn about
him and his Church, and grow spiritually. Invigorated by the power of truth,
many then are inspired to tell others about their conversion experience
(evangelization). They want to evangelize others.

The second purpose of an SEFC is for its members to reach out to and invite
lapsed Catholics to join their community. Having entered the group, the lapsed,
by association with practicing Catholics and exposed to the community's
formation and prayer life, are re-introduced to God's saving grace. They set
foot on the road that leads to a great change of heart. They leave their former
life of separation from the Church. They begin Celebrating Life as Catholic
Christians.

As the group bonds together, practicing Catholics must be on their toes to give
good example in Christian living. They ``celebrate'' the Christian life not for
show, but from conviction that they are Christ's disciples, genuinely committed
to Jesus' way of life in spite of the influences of a secular, anti-Christian
world. On the other hand, newcomers to the group become witnesses of this
counter-cultural way of life. Through gradual formation in the faith and by the
outpouring of grace received from the Holy Spirit, they move toward re-entry
into the larger community of Christ  the One, Holy, Catholic and Apostolic
Church.
RDS, page \#

8. Priest's Mission is ``To Beget''
Vatican Council II's Decree on the Ministry and Life of Priests, states
``Acknowledging Christ's desire and inspired by the Holy Spirit, the apostles
considered it their duty to select ministers 'who shall be competent in turn to
teach others.' (2 Tim 2:2) This duty then 'to beget' is part of the priestly
mission by which every priest is made a partaker in the care of the whole
Church, so that workers may never be lacking for the People of God on earth.''
(PO 11). This statement applies to bishops, successors of the apostles, and to
priests who aid their bishops in the offices of teaching, sanctifying and
governing. Stated here are the basic principles of shared responsibility.

In their mission priests also share responsibility with the laity, particularly
in the role of teaching: priests teach truth from the pulpit, teachers of
religion form children in the faith in Catholic schools, in the religious
education programs, in the Rite of Christian Initiation of Adults, in
preparation for the Sacraments of Baptism and Confirmation, and in the
consultative body called the parish pastoral council. Vatican Council II's
statement about church leaders, bishops, having the ``duty to select ministers
'who shall be competent to teach others,''' points first to priests. Part of the
priests' mission is to select competent laypersons to help teach eternal truths
in pastoral work such as in the CLCC process. Theirs is an educational,
formational and spiritual responsibility shared with the bishop.The Decree sums
up shared responsibility in these words, ``This duty then is a part of the
priestly mission by which every priest is made a partaker in the care of the
whole Church, so that workers may never be lacking for the People of God on
earth.''(cf. also LG 34-37; CCC 783; RH 19)

Thus, the meaning ``to beget'': Jesus, the God-Man came to earth to save all
mankind. He is the one and only Savior. There is no other. (cf. DI 13) To Peter
and the 12 Apostles he entrusted the great responsibility to carry out his plan
of salvation in the whole world. (cf. Mt 28:18-20; also 16:18) Through the
apostles' successors, bishops, their helpers, priests, and lay persons, Christ
continues his work, by their preaching the good news, administering the
sacraments to his people, and sharing responsibility with the laity who can
contribute much to the development and spread of the faith of Christ's Church
throughout the world.
RDS, page \#

9. Formation in and Functions of Small Ecclesial Faith Communities
The primary goal of every person who belongs to an SEFC is to strive to advance
in personal holiness through the CLCC faith formation process as given in this
study manual. Formation in holiness of life is based on Jesus' teachings during
his public ministry, as recorded in the Bible. Members of SEFCs are also formed
in official Church teachings, many of which may relatively unknown to them
(cf. Part II, Authoritative Sources). Jesus' teachings recorded in the gospels
and the Church's official teachings interlink. Each participant in an SEFC
should be open ``to be evangelized [in order to] evangelize'' in his or her
parish; that is, willing, if possible, to do mission outreach to lapsed,
alienated, non-regular church goers, befriend them and invite them to join a
Small Ecclesial Faith Community for conversion to Christ and his Church.
RDS, page \#

10. Catechetical Instruction Holds Primacy of Place
Jesus assures us, ``Where two or three are gathered together in my name, there
am I in the midst of them.'' (Mt 18:20) Moreover he dispenses his graces
generously to those who ask. It is in the close-knit Small Ecclesial Faith
Community where good will and fraternity exist, and where training in the faith
and motivation to reach out to non-regular Church-going Catholics takes place.

In the lapsed who join one of the SEFCs a change of heart begins. Enkindled by
Christ's love they begin to have a positive ``feeling'' for return to the
Church. Through God's grace Small Ecclesial Faith Communities are the ``rich
soil'' that ``produce(s) a hundred or sixty or thirtyfold.'' (cf. Mt 13:8)
Lapsed Catholics are like ``the harvest [that] is abundant but the laborers are
few; so [as Jesus urged] ask the master of the harvest to send out laborers for
the harvest.'' (cf. Mt 9:37-38) To gather in the ``abundant harvest,'' laborers
for the harvest are needed. Thus, SEFCs are a ``must'' where laborers are
schooled both in personal holiness and for gathering the harvest.
RDS, page \#

Summary
Stage One of the CLCC process is summed up by this statement from Vatican II's
Decree on the Bishops' Pastoral Office in the Church, ``Catechetical training is
intended to make men's faith living, conscious and active through the light of
instruction. This instruction is based on Sacred Scripture, tradition, the
liturgy, the teaching authority and life of the Church.'' (CD 14) This statement
is directed to the bishops. However, various forms of ``catechetical training,''
as stated above, are the primary ministry of most priests with assistance from
laity.

Good preaching by clergy and formation of laity through catechesis given in
Small Ecclesial Faith Communities by, through and with competent laypersons
holds primacy of place in the process called Celebrating Life as a Catholic
Christian.

The first thing Christ did was call 12 Apostles and 72 disciples. After being
with him awhile he declared them ``fishers of men.'' He then sent them into the
villages and towns around Lake Genesareth  to announce the Good News about his
coming as Messiah and to cure the sick. People who accepted their words become
followers of Jesus.

Enacting in a parish what Jesus did in Galilee consists of establishing Small
Ecclesial Faith Communities (SEFCs) modeled on Jesus' small faith community of
the apostles and disciples. SEFCs are composed of regular church-going
Catholics. The focus of SEFCs is to provide, through the CLCC process, formation
of the members in Catholic doctrine and Church teachings for growth in personal
holiness. The CLCC faith formation process can be summed up in the famous words
of Blessed John Paul II, ``be evangelized'' in order ``to evangelize.''

``To evangelize'' includes seeking out lapsed and alienated Catholics,
befriending them and winning them over to join one of the SEFCs in the
parish. Moved by the Holy Spirit and by association with exemplary practicing
Catholics, they begin their journey back to full union with Christ and his
Church.
RDS, page \#

% ------------------------------------------------------------------------------

\section{Reflection and Discussion}
\lxRDFa{property=stage-rd-content,resource={manual}}

Note:  for obtaining the most value from the CLCC faith formation process,
please read again Putting the CLCC Process to Work found in the introductory
pages of this manual; also look again at Order for Conducting Formation
Sessions.  The aids and insights given in this Reflection and Discussion section
correspond to paragraphs in the main text of Stage One.


Introduction
1. The first paragraphs contain two issues for reflection and formation. Choose
one issue, one that appeals to you most. Reflect on it prayerfully under the
guidance of the Holy Spirit. Be open to his insights. He is helping you make
your first tiny step in the CLCC faith formation process.

Also open your manual to the centerfold and study the graphic. Note the five
stages of Jesus' ministry in Galilee (left side) and the parallel five-stage
structure of the CLCC process for SEFCs (right side). Reflect how you and your
fellow parishioners can become more Christlike by carrying out this process.

2. In his public ministry Jesus had two goals: to give his people a new way of
life, and do missionary outreach to the lapsed and ``lost.'' For us, his new way
of life means first answering the ``call to holiness.'' In silence, reflect on
his ``two greatest commandments'' and his ``call'' to us to become holy. Share
your insights with your SEFC group.

Missionary outreach can be done in different ways. In the CLCC process, outreach
means contacting and helping an estimated 45 million lapsed and fallen-away
Catholics return to Christ and his Church. On a scale of 1 to 10 how important
and urgent do you rate this outreach in your family and parish. Reflect: how do
we help the lapsed return to Christ and his Church?

A. What Jesus Did in Galilee
1. Choosing apostles and disciples was a major step in Jesus' great divine plan
of founding his Church on earth. Reflect prayerfully on the two reasons Jesus
chose these men as stated in the main text. Reflect how you fit into Jesus'
divine plan and how you can benefit spiritually from it. Be open to the insights
the Holy Spirit gives you.

2. To Celebrate Life as a Catholic Christian means to imitate Christ in the ways
he has shown us. Grade yourself. Choose one of the ways given here and reflect
how you live it: well, sort of, need improvement, news to me.

3. What you know of the CLCC process to this point, do you feel it can
``empower'' you and other Catholics to live more authentic Christian lives in
our current anti-Christian culture? Reflect on the area of empowerment you feel
will bring about the most change in your life. Reread the paragraph. Share.

4. Call of the 12 Apostles and 72 Disciples
The two texts taken from St. Matthew's gospel have action words such as
``walking,'' ``casting,'' ``followed'' and so forth. Find a few more. Reflect
how these actions can apply to your relationship with Jesus. St. Luke's gospel
states, Jesus ``prayed'' a whole night before choosing the 12. This action shows
us the importance and necessity of praying prior to making major choices or
decisions.

Before Jesus' time and even in his day people were filled with great expectation
of the promised Messiah. When Jesus came and called certain men, they were drawn
to his extraordinary persona. They may also have been dissatisfied with the work
they were doing, e.g. fishing, tax collecting. God respects the free will with
which he endowed every human being. What teaching does Jesus' ``call'' suggest
to you about vocations that people respond to for different states of life:
single, married, religious, priestly? Each individual is free to accept or
reject God's invitation. He forces no one. Choose up to three ``action words.''
Prayerfully reflect on them. Call on the Holy Spirit for light. What meaning do
they have for you? Share.

5. Mission of the Apostles and Disciples
The two texts from St. Luke's gospel are Jesus' mission statements for the 12
apostles and 72 disciples. Choose an action that Jesus gave them to do and
reflect on it prayerfully. Ask yourself, how can it apply to me? Listen to the
Holy Spirit. Try to think of some comparable action you can do. Share.

6. Catechism of the Catholic Church
The first of the two paragraphs taken from the Catechism of the Catholic Church
(CCC 75), centers on Jesus' command to his apostles to preach the gospel, the
source of truth and moral discipline. The second paragraph (CCC 1575) states how
Jesus protects his flock through the apostles and their successors, bishops, who
are pastors. Reflect on how these statements complement each other. Tell Jesus
how safe you feel believing it is he who cares for us, his Church, through these
responsible shepherds. Share your insights.

B. Enacting in the Parish What Jesus Did in Galilee

Note: Under ``Leadership of the Priest'' are four paragraphs. They describe the
priest's all-essential role in getting the CLCC process underway in the
parish. The priest, whether resident and full-time, part-time, itinerant or a
once a month priest to his people, should take on the process with great
enthusiasm especially since the laity in SEFCs share the greater responsibility
for carrying it out. Like Christ, the priest should take the lead in choosing
and calling ``apostles and disciples,'' lay leaders from among the
parishioners. He helps them launch the process. To assure its progress and
growth he oversees it as best he can and preaches about it as often as he deems
it necessary to make the ``fire burn brightly.''

1. Leadership of the Priest
The priest's initiative for successful launching of the CLCC process is
imperative. What about parishes with only one priest who is overwhelmed with
work? Do you feel lay leaders should enact it counting on only limited oversight
from the priest? What about getting help from a deacon? Recall the two purposes
of the CLCC process! Neither are ``take it or leave it'' options.  The call to
holiness is mandated by Jesus: ``Be perfect as your heavenly Father is perfect''
(cf. Mt 5:48; also CCC 2013) and the Dogmatic Constitution of the Church, Light
of the Nations (Lumen gentium, LG, 39-42). The second purpose is missionary, to
reach out to and help save the wayward sheep (Lk 15:4-6). The two purposes are
complimentary to each other. Choose a question above. Call on the Holy Spirit
for light. Reflect in quiet. Share your inmost thoughts.

2. Teachings of Vatican Council II
The two teachings are power-packed mandates deriving from Christ and the
apostles. Ask yourself: how do these mandates pertain to me? How obligated am I
to do what they say?  Humbly be open to the Holy Spirit. Be inspired. Share your
insights with your SEFC.

3. Small Ecclesial Faith Communities (SEFCs)
This paragraph contains phrases and words that show how Jesus formed his first
community. They are ``community,'' ``faith,'' ``small,'' ``by design,''
``co-workers,'' and ``model.'' Asking the Holy Spirit for light, what meaning do
these words have for you, for your growth in holiness and mission outreach to
lapsed fellow Catholics? Reflect quietly in prayer. Share.

4. Models of Leadership in the Old Testament
The models given here are ordinary people whom God called to do extraordinary
things in his great plan of salvation. Select one model. Open your Bible to the
texts that describe the model you chose. In prayer to the Holy Spirit reflect on
this person specially chosen to work with God on behalf of his people. Reflect
on how you, also special to God, are to carry out his plan of salvation within
your family, parish and work place through the CLCC process.

5. Small Ecclesial Faith Communities, Why Are They Needed?
In the three paragraphs given here, identify the reason for establishing SEFCs,
the faith problems that many Catholics have, and by what means they can be
brought back to the faith. Reflect.

If the cited documents are accessible, you may want to read what Popes Paul VI
and Blessed John Paul II have said about small faith communities. Both Pontiffs
gave them strong endorsement. Marginal Catholics has reached astronomic
numbers. An estimated 45 million Catholics do not go to Church regularly or not
at all. They need to be reached and won back to the Church. Try to recall some
of the ways you know or have heard to win wayward Catholics back into the
Church. What you know of the CLCC and SEFCs, do you feel this non-intimidating,
sensitive formation process will in the long run be the most effective
conversion method? Prayerfully reflect. Share.

6. Blessed Trinity, Archetype of Small Ecclesial Faith Communities
The following sentence taken from the Catechism (CCC 1878) identifies the
source, character and the links that connect Small Ecclesial Faith Communities:
``There is a certain resemblance between the unity of the three divine persons
[source, God] and the fraternity (character of SEFCs) that men are to establish
among themselves in truth and love [connecting links].''

What should this sentence mean to you in the CLCC formation process? (a) The
Blessed Trinity, the source, is certainly a ``community'' of Father, Son and
Holy Spirit, the source of life, redemption and agent of salvation; (b) the
character of SEFCs consists of Christian brotherhood, neighborly fellowship and
solidarity among its members; (c) the connecting links between the Trinity and
SEFCs are truth and love. Christ said, ``I am the truth'' (Jn 14:6) and God is
defined as ``love'' (1 Jn 4:8). Clearly SEFCs have their foundation in and flow
from the Triune God.

Central to this reflection is the human person, also the close-knit society of
persons, the communal family, each having a spiritual life. Their rights and
obligations, their spiritual values and spiritual achievements must be
respected, as taught by Blessed John XXIII (cf. MM 60; PT 36) Having supreme
authority over all creation and man, God also set up civil authority which must
be ``morally licit'' to further the common good and maintain good order.

Prayerfully reflect on these fundamental truths. Choose one, preferably the
first one. Make the connection between the Blessed Trinity and the CLCC
process. See how the CLCC process flows radically from the Blessed
Trinity. Invoke the Holy Spirit. Share your insights.

7. SEFCs, Front Doors for Lapsed Catholics to Come Home to the Church
The four paragraphs here give the reasons for the existence (raison d'etre) of
SEFCs, what they do, and why they are valuable in pastoral ministry. Lapsed
Catholics have reached astronomic numbers. An estimated 45 million Catholics do
not go to Church regularly in the United States. They need to be reached and won
back to the Church. Try to recall some of the ways you know or have heard to win
wayward Catholics back into the Church. What you know of the CLCC to this point,
do you feel this non-intimidating, formative process is an effective apostolate
for helping people grow in holiness and reaching out to help save the lapsed?
In quiet time, pray and reflect on at least one or two of these key words and
phrases: ``welcome mat,'' ``re-introduce,'' ``change of heart,''
``association,'' ``good example,'' ``witness,'' ``grace of the Holy Spirit.''
What impact do these key words or phrases have on you? Pray for light. Share
your God-given thoughts.

8. Priest's Mission in the CLCC Process is ``To Beget''
``To beget'' means the ``duty'' to propagate the kingdom of God for one's own
salvation and the salvation of others. Jesus' mission was ``to beget.'' He
called apostles first, who chose successors (bishops), who then selected priests
to share their care of the Church, and priests in turn seek out lay persons
(workers), so that the people of God are adequately cared for. The pastoral term
for ``duties'' is shared responsibility. The laity's duty is to share
responsibility, helping the priest by exercising primarily their prophetic role
of teaching and formation of God's people in the faith. The CLCC is in fact a
process of ``begetting.'' Participants train in personal holiness, a quality
that helps them deliver lapsed Catholics into the arms of Mother Church.  Pray
to the Holy Spirit to get the impact of these truths. Reflect in quiet. Share
with your SEFC how you see the CLCC as a ``begetter'' of higher levels of
holiness and of conversion in your parish.

9. Formation in SEFCs Consists of Two Functions
They are:  (a) growth in personal holiness of the group's members, and (b)
reaching out to help lapsed Catholics to return to the Church.  The two
functions are inter-linked to ensure good results. Reflect why and how formation
in personal holiness is a vital prerequisite for doing successful outreach
evangelization to those in need of conversion. Pray to the Holy Spirit for his
gifts. For productive evangelization, all seven gifts of the Holy Spirit need to
be active within you: wisdom, council, knowledge, understanding, fortitude,
piety and fear of the Lord (CCC 1831). Invoke the Lord to energize the Holy
Spirit's ``gifts'' in you. Share the light you receive.

10. Catechetical Instruction Holds Primacy of Place in SEFCs
Small Ecclesial Faith Communites are invaluable for parishes. Parishioners in
great part are in need of a booster shot of truth. Too often, they are strangers
to each other, having time only for a quick ``Hi'' as they rush to the parking
lot after Sunday Mass. SEFCs have two purposes. First to invite, draw and
convince ``Hi Catholics'' that SEFCs are also for them. Systematic formation in
the faith is necessary for them, and for their own growth in personal holiness
perhaps a stunning surprise. The second purpose is that members of SEFCs do not
stop with ``interior formation.'' They are expected to follow up with an
``exterior action,'' to reach out to and help save lapsed Catholics return to
the Church. This was Jesus' mode of operation in Galilee. He called apostles,
formed them in his truths by his preaching (interior formation), then sent them
out two by two (exterior action) to towns and villages to teach, preach, convert
and drive out demons. The CLCC process is modeled on Jesus' method of
evangelizing. See the centerfold, page \#. Invoke the Holy Spirit for
light. Reflect in quiet on the remarkable parallel between Jesus' form of
evangelizing in Galilee and the CLCC's corresponding form of evangelizing in the
parish. Share the insights the Holy Spirit gives you.

Summary
In Stage One, the main text of the CLCC process is summed up by a single
authoritative statement from Vatican Council II's Decree on the Bishops'
Pastoral Office in the Church (Christus Dominus, CD, 14). Please reread it. In
this one sentence do you see the CLCC in outline? Do you feel this sentence
gives a glimpse of the CLCC process, highlighting its main points? Prayerfully
reflect. Share.

% ------------------------------------------------------------------------------

\chapter{Stage 2.\ Formation of Disciples and Apostles}

\section*{} \lxRDFa{property=stage-main-content,resource={manual}}

This is stage 2 stuff.

% ------------------------------------------------------------------------------

\section{Reflection and Discussion}
\lxRDFa{property=stage-rd-content,resource={manual}}

This is stage 2 RD stuff.

% ------------------------------------------------------------------------------

\chapter{Stage 3.\ Stewardship of Our Gifts}

\section*{} \lxRDFa{property=stage-main-content,resource={manual}}

This is stage 3 stuff.

% ------------------------------------------------------------------------------

\section{Reflection and Discussion}
\lxRDFa{property=stage-rd-content,resource={manual}}

This is stage 3 RD stuff.

% ------------------------------------------------------------------------------

\chapter{Stage 4.\ Home Visitation and Evangelization}

\section*{} \lxRDFa{property=stage-main-content,resource={manual}}

This is stage 4 stuff.

% ------------------------------------------------------------------------------

\section{Reflection and Discussion}
\lxRDFa{property=stage-rd-content,resource={manual}}

This is stage 4 RD stuff.

% ------------------------------------------------------------------------------

\chapter{Stage 5.\ Stewards of the Eucharist}

\section*{} \lxRDFa{property=stage-main-content,resource={manual}}

This is stage 5 stuff.

% ------------------------------------------------------------------------------

\section{Reflection and Discussion}
\lxRDFa{property=stage-rd-content,resource={manual}}

This is stage 5 RD stuff.

% ------------------------------------------------------------------------------

\addtocontents{toc}{\cftpagenumbersoff{part}}
\part{Authoritative Sources}
\addtocontents{toc}{\cftpagenumberson{part}}

% ------------------------------------------------------------------------------

\chapter{Introduction}

Some introductory stuff.

% ------------------------------------------------------------------------------

\chapter{On Evangelization in the Modern World}

Some document stuff.

% ------------------------------------------------------------------------------

\chapter{The Redeemer of Man}

Some document stuff.

% ------------------------------------------------------------------------------

\chapter{Mission of the Redeemer}

Some document stuff.

% ------------------------------------------------------------------------------

\chapter{The Lay Faithful of Christ}

Some document stuff.

% ------------------------------------------------------------------------------

\chapter{The Gospel of Life}

Some document stuff.

% ------------------------------------------------------------------------------

\chapter{Letter to Families}

Some document stuff.

% ------------------------------------------------------------------------------

\chapter{On the Regulation of Births}

Some document stuff.

% ------------------------------------------------------------------------------

\chapter{God is Love}

Some document stuff.

% ------------------------------------------------------------------------------

\chapter{Saved in Hope}

Some document stuff.

% ------------------------------------------------------------------------------

\chapter{I Will Give You Shepherds}

Some document stuff.

% ------------------------------------------------------------------------------

\chapter{Charity in Truth}

Some document stuff.

% ------------------------------------------------------------------------------

\chapter{Observing the Day of the Lord}

Some document stuff.

% ------------------------------------------------------------------------------

\chapter{The Splendor of Truth}

Some document stuff.

% ------------------------------------------------------------------------------

\chapter{The Holy Spirit in the Life of the Church and the World}

Some document stuff.

% ------------------------------------------------------------------------------

\chapter{The Lord Jesus, One and Only Savior}

Some document stuff.

% ------------------------------------------------------------------------------

\chapter{The Christian Meaning of Human Suffering}

Some document stuff.

% ------------------------------------------------------------------------------

\chapter{The Church of the Eucharist}

Some document stuff.

% ------------------------------------------------------------------------------

\chapter{The Authentic Liturgy}

Some document stuff.

% ------------------------------------------------------------------------------

\chapter{That All May Be One}

Some document stuff.

% ------------------------------------------------------------------------------

\chapter{Mother of the Redeemer}

Some document stuff.

% ------------------------------------------------------------------------------

\appendix \setcounter{secnumdepth}{-2}
{
\Hide  
\addtocontents{toc}{\cftpagenumbersoff{part}}
\part{Appendices}
\addtocontents{toc}{\cftpagenumberson{part}}
}

% ------------------------------------------------------------------------------

\chapter{Questions and Answers}

Q. Why do you call Celebrating Life as a Catholic Christian (CLCC) a process and
not a program? Whats the difference?

A. A process is developmental. It moves progressively from one stage to
another. It has a beginning but no end. Its ongoing. A program, on the other
hand, is a succession of actions carried out within a structured order that has
a beginning and a predefined end.


Q. Is one year long enough for carrying out this process?

A. No. In Galilee Jesus took three years to form his disciples in his new way of
life. Compare this process, for example, to capital funds campaigns that are
conducted in dioceses and parishes. They usually take 3 years or more. Formation
in Christs new way of life is also a capital campaign requiring a lifetime for
becoming a saint and serving our neighbor.


Q. Is the pastor involved in this process? If so, how, when, where?

A. YES! He together with key people (example, the pastoral council) find
qualified parishioners who are willing to serve as lay leaders of the process,
and other lay persons to begin the establishment of Small Ecclesial Faith
Communities (SEFCs) throughout the parish. He need not get involved in
details. He gives it leadership, and above all shows interest and support.  As
often as possible and despite many demands, he should visit the different SEFCs
once in awhile when they hold their spiritual formation sessions, encourage
them, pray with them, answer their questions, and help them keep their focus.
He is to preach on the process with enthusiasm and conviction, as often as he
deems it helpful to the people. In his remarks he is encouraged to choose some
points from the stage his parishioners are working through, and which he feels
will most benefit them. This also applies to the material in Part II,
Authoritative Sources. On Sunday the people in the pews who are not yet involved
have a right to be informed about the process and invited, in fact urged, to
join and participate in a SEFC.


Q. How many sessions for spiritual formation can be, or should be, drawn from
the material presented in each of the 5 stages and in Authoritative Sources?

A. As many as there are paragraphs. The material in both parts is so abundant
that it is, relatively speaking, inexhaustible, and of course repeatable. What
the Matrix 12 and members of SEFCs seek is in-depth not superficial formation in
faith and holiness. Each of the 5 stages provides material for multiple
formation sessions, one Stage has 20, another 43 sessions, and so on.  Materials
for formation sessions in Part II, Authoritative Sources are taken entirely from
magisterial documents. Under each paragraph or segment you will find these
letters: R-D-S in bold. They mean Reflection, Development and Sharing. Members
of SEFCs should follow these three steps as they did in Part I.


Q. Does the CLCC process have criteria to measure success or failure as it
carries out each of the stages?

A. Each of the 5 stages has built-in goals, or targets to reach. Criteria for
the general goal are: how well or poorly was the stage carried out; did it
attain what it was supposed to attain? For example, in Stage 1, did twelve
gifted people respond to the call to be members of the Matrix, the lay group
which leads the parish through the CLCC process? Did enough parishioners respond
to the invitation to form one, two, three or more Small Ecclesial Faith
Communities for a start of the process? Not to reach these goals the first time
around is not failure, but a signal to continue to recruit.  Each stage also
contains intermediate objectives. These are specific actions that are carried
out as exactly as possible within the formation session so everyone in the SEFC
gets the full benefit. For example, does the moderator start and end the session
on time; is the reflection issue clear to all; is enough time given to quiet
time; do all or most participants get a chance to share their insights; does the
moderator politely cut off long-winded sharing; are the hymns, prayers and
intercessions appropriate? More importantly, do the participants feel they are
growing spiritually and gaining better knowledge of the faith?  Recommended is
that each SEFC chooses a recorder to keep track of how well or poorly goals and
objectives are attained. This record is helpful for review and can induce the
group to strive for excellence in their formation. It tells participants how
they are conducting the formation sessions, what needs improvement, and how to
make the sessions run smoothly to the best advantage of all.


Q. What is the connection between the National Catholic Conference for Total
Stewardship and Celebrating Life as a Catholic Christian?

A. In 1980 the National Catholic Conference for Total Stewardship (NCCTS) was
founded by Francis A. Novak, a Redemptorist priest, with approval of the NCCB,
now called the United States Conference of Catholic Bishops (USCCB). Its purpose
is to promote the biblical concept of stewardship in its totality. Stewardship
has been and still is widely used by dioceses and parishes for fund raising,
increased offertory appeals and capital funds campaigns, under the banner of 3
Ts: Time, Talent and Treasure.  Fr. Novak and the NCCTS felt compelled to
explore the deeper implications of stewardship in Sacred Scripture. Under the
guidance of expert biblical scholars, the NCCTS learned that the Old and New
Testaments are a fascinating account of how God practices holistic stewardship
in creation through his providential care of the universe and the planet
earth. The three components of stewardship found in both testaments are the good
news (Evangelization), holiness of life (Discipleship) and service to the Lord
and his people (Stewardship). The Bibles three components have become the
template of the NCCTS and its CLCC process. They are living terms of a trilogy
of transcendent realities that are inextricably linked. In 1986 the language
chosen to express the three realities was Total Stewardship.  Expressive as the
NCCTS thought the new title was, response to it was limited but it did stir
curiosity in many quarters. In 1990 a new language was developed to convey that
Total Stewardship has an uplifting quality: joy in celebrating Christ as the
Way, the Truth and the Life for his pilgrim people on earth. The new title was
The Christian Celebration of Life (CCL). The new title did arouse a wide
reception of the process. But some times it was mistaken for something else, for
a pro-life group and even an offshoot of Alcoholics Anonymous. Thus another name
change was needed.  The new name for this total stewardship process came in
1998: Celebrating Life as a Catholic Christian (CLCC). The NCCTS now feels this
newly named process expresses not only the biblical triad of Evangelization,
Discipleship and Stewardship, but also New Evangelization, Call to Holiness and
Communion, goals which the Second Vatican Council, Pope John Paul II, the
Congregation for the Doctrine of the Faith and Benedict XVI have called for
repeatedly and passionately for Gods people in todays world of conflicting
cultures.


Q. With the name change were any adjustments made to the process: its focus,
goals and manner of carrying out the Churchs mission of New Evangelization?

A. The name change seems to excite the human spirit. It opens the imagination to
new possibilities and challenges like Im going to grow in holiness, me? and well
reach out to help lapsed Catholics return to the faith, terrific! Indeed, the
process has been given a substantial revision and its content has been
expanded. Each stage makes formation in holiness, in doctrine and in pastoral
outreach clearer. Readers of this manual will be helped immensely in Celebrating
their Lives as Catholic Christians.



Q. What is the cost for implementing this process in a parish?

A. There is no set fee. However, cost is in the purchase of the CLCC books for
the pastor, the lay leaders and all members of Small Ecclesial Faith
Communities. Based on experience the best way to go is for the parish to
purchase the needed number of manuals at a bulk rate, and then request each
participant to purchase a copy so as reimburse the parish. Without a book in
hand participants will find it impossible to carry out their formation. The idea
to buy one or two books and run photocopies each week for the 12 lay leaders and
members of SEFCs, as an economy measure, has been tried before with disastrous
results. The end result is not an economy measure - it costs more in time and
paper. Handout sheets get lost. People need a complete, bound book.  Having a
personal copy allows each person to study from his own book, get an idea of the
structure of the process and the order of how formation sessions are carried
out, and allows participants to use their free time to quick-read the contents
before going to the next weeks session.  Special cost arrangements can be
negotiated with the National Catholic Conference for Total Stewardship on an as
needed basis.


Q. The CLCC process, it appears, puts a lot of importance on the lay leaders
called the Matrix 12 and on Small Ecclesial Faith Communities. Where can I go to
find reliable and official Church teaching on these kinds of communities?

A. Go to Stage 2, titled Formation of Disciples and Apostles in the CLCC
manual. Read section C. Small Ecclesial Faith Communities, numbered paragraphs 1
to 10. Here the teaching on Small Ecclesial Faith Communities is drawn directly
from John Paul IIs encyclical Redemptoris Missio and from Pope Paul VIs
Evangelii Nuntiandi. You can find additional information on SEFCs in Part II of
this manual, Authoritative Sources. See the first four documents treated there.




Q. Could you tell me in one sentence what the CLCC process is?

A. The CLCC process is pure, unadulterated biblical stewardship framed in the
Second Vatican Councils ecclesiology and magisterial teachings, which give solid
theological formation and mission motivation to faithful believers, enabling
them to Celebrate their Lives as Catholic Christians in a dechristianized
culture.

% ------------------------------------------------------------------------------

\chapter{Glossary}

Glossary entries.

% ------------------------------------------------------------------------------

\backmatter
{
\Hide  
\addtocontents{toc}{\cftpagenumbersoff{part}}
\part{Indexes}
\addtocontents{toc}{\cftpagenumberson{part}}
}

% ------------------------------------------------------------------------------

\chapter{Alphabetical Index}

Some index entries.

% ------------------------------------------------------------------------------

\chapter{Topical Index}

Some index entries.

% ------------------------------------------------------------------------------

{
\Hide  
\addtocontents{toc}{\cftpagenumbersoff{part}}
\part{}
\addtocontents{toc}{\cftpagenumberson{part}}
}

% ------------------------------------------------------------------------------

\chapter{Creative Commons License}

Attribution-NonCommercial 3.0 Unported

URI: http://creativecommons.org/licenses/by-nc/3.0/legalcode


License THE WORK (AS DEFINED BELOW) IS PROVIDED UNDER THE TERMS OF THIS CREATIVE
COMMONS PUBLIC LICENSE ("CCPL" OR "LICENSE"). THE WORK IS PROTECTED BY COPYRIGHT
AND/OR OTHER APPLICABLE LAW. ANY USE OF THE WORK OTHER THAN AS AUTHORIZED UNDER
THIS LICENSE OR COPYRIGHT LAW IS PROHIBITED.

BY EXERCISING ANY RIGHTS TO THE WORK PROVIDED HERE, YOU ACCEPT AND AGREE TO BE
BOUND BY THE TERMS OF THIS LICENSE. TO THE EXTENT THIS LICENSE MAY BE CONSIDERED
TO BE A CONTRACT, THE LICENSOR GRANTS YOU THE RIGHTS CONTAINED HERE IN
CONSIDERATION OF YOUR ACCEPTANCE OF SUCH TERMS AND CONDITIONS.  1. Definitions
"Adaptation" means a work based upon the Work, or upon the Work and other
pre-existing works, such as a translation, adaptation, derivative work,
arrangement of music or other alterations of a literary or artistic work, or
phonogram or performance and includes cinematographic adaptations or any other
form in which the Work may be recast, transformed, or adapted including in any
form recognizably derived from the original, except that a work that constitutes
a Collection will not be considered an Adaptation for the purpose of this
License. For the avoidance of doubt, where the Work is a musical work,
performance or phonogram, the synchronization of the Work in timed-relation with
a moving image ("synching") will be considered an Adaptation for the purpose of
this License.  "Collection" means a collection of literary or artistic works,
such as encyclopedias and anthologies, or performances, phonograms or
broadcasts, or other works or subject matter other than works listed in Section
1(f) below, which, by reason of the selection and arrangement of their contents,
constitute intellectual creations, in which the Work is included in its entirety
in unmodified form along with one or more other contributions, each constituting
separate and independent works in themselves, which together are assembled into
a collective whole. A work that constitutes a Collection will not be considered
an Adaptation (as defined above) for the purposes of this License.  "Distribute"
means to make available to the public the original and copies of the Work or
Adaptation, as appropriate, through sale or other transfer of ownership.
"Licensor" means the individual, individuals, entity or entities that offer(s)
the Work under the terms of this License.  "Original Author" means, in the case
of a literary or artistic work, the individual, individuals, entity or entities
who created the Work or if no individual or entity can be identified, the
publisher; and in addition (i) in the case of a performance the actors, singers,
musicians, dancers, and other persons who act, sing, deliver, declaim, play in,
interpret or otherwise perform literary or artistic works or expressions of
folklore; (ii) in the case of a phonogram the producer being the person or legal
entity who first fixes the sounds of a performance or other sounds; and, (iii)
in the case of broadcasts, the organization that transmits the broadcast.
"Work" means the literary and/or artistic work offered under the terms of this
License including without limitation any production in the literary, scientific
and artistic domain, whatever may be the mode or form of its expression
including digital form, such as a book, pamphlet and other writing; a lecture,
address, sermon or other work of the same nature; a dramatic or
dramatico-musical work; a choreographic work or entertainment in dumb show; a
musical composition with or without words; a cinematographic work to which are
assimilated works expressed by a process analogous to cinematography; a work of
drawing, painting, architecture, sculpture, engraving or lithography; a
photographic work to which are assimilated works expressed by a process
analogous to photography; a work of applied art; an illustration, map, plan,
sketch or three-dimensional work relative to geography, topography, architecture
or science; a performance; a broadcast; a phonogram; a compilation of data to
the extent it is protected as a copyrightable work; or a work performed by a
variety or circus performer to the extent it is not otherwise considered a
literary or artistic work.  "You" means an individual or entity exercising
rights under this License who has not previously violated the terms of this
License with respect to the Work, or who has received express permission from
the Licensor to exercise rights under this License despite a previous violation.
"Publicly Perform" means to perform public recitations of the Work and to
communicate to the public those public recitations, by any means or process,
including by wire or wireless means or public digital performances; to make
available to the public Works in such a way that members of the public may
access these Works from a place and at a place individually chosen by them; to
perform the Work to the public by any means or process and the communication to
the public of the performances of the Work, including by public digital
performance; to broadcast and rebroadcast the Work by any means including signs,
sounds or images.  "Reproduce" means to make copies of the Work by any means
including without limitation by sound or visual recordings and the right of
fixation and reproducing fixations of the Work, including storage of a protected
performance or phonogram in digital form or other electronic medium.  2. Fair
Dealing Rights.  Nothing in this License is intended to reduce, limit, or
restrict any uses free from copyright or rights arising from limitations or
exceptions that are provided for in connection with the copyright protection
under copyright law or other applicable laws.  3. License Grant.  Subject to the
terms and conditions of this License, Licensor hereby grants You a worldwide,
royalty-free, non-exclusive, perpetual (for the duration of the applicable
copyright) license to exercise the rights in the Work as stated below:

to Reproduce the Work, to incorporate the Work into one or more Collections, and
to Reproduce the Work as incorporated in the Collections; to create and
Reproduce Adaptations provided that any such Adaptation, including any
translation in any medium, takes reasonable steps to clearly label, demarcate or
otherwise identify that changes were made to the original Work. For example, a
translation could be marked "The original work was translated from English to
Spanish," or a modification could indicate "The original work has been
modified."; to Distribute and Publicly Perform the Work including as
incorporated in Collections; and, to Distribute and Publicly Perform
Adaptations.

The above rights may be exercised in all media and formats whether now known or
hereafter devised. The above rights include the right to make such modifications
as are technically necessary to exercise the rights in other media and
formats. Subject to Section 8(f), all rights not expressly granted by Licensor
are hereby reserved, including but not limited to the rights set forth in
Section 4(d).  4. Restrictions.  The license granted in Section 3 above is
expressly made subject to and limited by the following restrictions:

You may Distribute or Publicly Perform the Work only under the terms of this
License. You must include a copy of, or the Uniform Resource Identifier (URI)
for, this License with every copy of the Work You Distribute or Publicly
Perform. You may not offer or impose any terms on the Work that restrict the
terms of this License or the ability of the recipient of the Work to exercise
the rights granted to that recipient under the terms of the License. You may not
sublicense the Work. You must keep intact all notices that refer to this License
and to the disclaimer of warranties with every copy of the Work You Distribute
or Publicly Perform. When You Distribute or Publicly Perform the Work, You may
not impose any effective technological measures on the Work that restrict the
ability of a recipient of the Work from You to exercise the rights granted to
that recipient under the terms of the License. This Section 4(a) applies to the
Work as incorporated in a Collection, but this does not require the Collection
apart from the Work itself to be made subject to the terms of this License. If
You create a Collection, upon notice from any Licensor You must, to the extent
practicable, remove from the Collection any credit as required by Section 4(c),
as requested. If You create an Adaptation, upon notice from any Licensor You
must, to the extent practicable, remove from the Adaptation any credit as
required by Section 4(c), as requested.  You may not exercise any of the rights
granted to You in Section 3 above in any manner that is primarily intended for
or directed toward commercial advantage or private monetary compensation. The
exchange of the Work for other copyrighted works by means of digital
file-sharing or otherwise shall not be considered to be intended for or directed
toward commercial advantage or private monetary compensation, provided there is
no payment of any monetary compensation in connection with the exchange of
copyrighted works.  If You Distribute, or Publicly Perform the Work or any
Adaptations or Collections, You must, unless a request has been made pursuant to
Section 4(a), keep intact all copyright notices for the Work and provide,
reasonable to the medium or means You are utilizing: (i) the name of the
Original Author (or pseudonym, if applicable) if supplied, and/or if the
Original Author and/or Licensor designate another party or parties (e.g., a
sponsor institute, publishing entity, journal) for attribution ("Attribution
Parties") in Licensor's copyright notice, terms of service or by other
reasonable means, the name of such party or parties; (ii) the title of the Work
if supplied; (iii) to the extent reasonably practicable, the URI, if any, that
Licensor specifies to be associated with the Work, unless such URI does not
refer to the copyright notice or licensing information for the Work; and, (iv)
consistent with Section 3(b), in the case of an Adaptation, a credit identifying
the use of the Work in the Adaptation (e.g., "French translation of the Work by
Original Author," or "Screenplay based on original Work by Original
Author"). The credit required by this Section 4(c) may be implemented in any
reasonable manner; provided, however, that in the case of a Adaptation or
Collection, at a minimum such credit will appear, if a credit for all
contributing authors of the Adaptation or Collection appears, then as part of
these credits and in a manner at least as prominent as the credits for the other
contributing authors. For the avoidance of doubt, You may only use the credit
required by this Section for the purpose of attribution in the manner set out
above and, by exercising Your rights under this License, You may not implicitly
or explicitly assert or imply any connection with, sponsorship or endorsement by
the Original Author, Licensor and/or Attribution Parties, as appropriate, of You
or Your use of the Work, without the separate, express prior written permission
of the Original Author, Licensor and/or Attribution Parties.  For the avoidance
of doubt: i.  Non-waivable Compulsory License Schemes. In those jurisdictions in
which the right to collect royalties through any statutory or compulsory
licensing scheme cannot be waived, the Licensor reserves the exclusive right to
collect such royalties for any exercise by You of the rights granted under this
License;

ii.  Waivable Compulsory License Schemes. In those jurisdictions in which the
right to collect royalties through any statutory or compulsory licensing scheme
can be waived, the Licensor reserves the exclusive right to collect such
royalties for any exercise by You of the rights granted under this License if
Your exercise of such rights is for a purpose or use which is otherwise than
noncommercial as permitted under Section 4(b) and otherwise waives the right to
collect royalties through any statutory or compulsory licensing scheme; and,

iii.  Voluntary License Schemes. The Licensor reserves the right to collect
royalties, whether individually or, in the event that the Licensor is a member
of a collecting society that administers voluntary licensing schemes, via that
society, from any exercise by You of the rights granted under this License that
is for a purpose or use which is otherwise than noncommercial as permitted under
Section 4(c).

Except as otherwise agreed in writing by the Licensor or as may be otherwise
permitted by applicable law, if You Reproduce, Distribute or Publicly Perform
the Work either by itself or as part of any Adaptations or Collections, You must
not distort, mutilate, modify or take other derogatory action in relation to the
Work which would be prejudicial to the Original Author's honor or
reputation. Licensor agrees that in those jurisdictions (e.g. Japan), in which
any exercise of the right granted in Section 3(b) of this License (the right to
make Adaptations) would be deemed to be a distortion, mutilation, modification
or other derogatory action prejudicial to the Original Author's honor and
reputation, the Licensor will waive or not assert, as appropriate, this Section,
to the fullest extent permitted by the applicable national law, to enable You to
reasonably exercise Your right under Section 3(b) of this License (right to make
Adaptations) but not otherwise.


5. Representations, Warranties and Disclaimer UNLESS OTHERWISE MUTUALLY AGREED
TO BY THE PARTIES IN WRITING, LICENSOR OFFERS THE WORK AS-IS AND MAKES NO
REPRESENTATIONS OR WARRANTIES OF ANY KIND CONCERNING THE WORK, EXPRESS, IMPLIED,
STATUTORY OR OTHERWISE, INCLUDING, WITHOUT LIMITATION, WARRANTIES OF TITLE,
MERCHANTIBILITY, FITNESS FOR A PARTICULAR PURPOSE, NONINFRINGEMENT, OR THE
ABSENCE OF LATENT OR OTHER DEFECTS, ACCURACY, OR THE PRESENCE OF ABSENCE OF
ERRORS, WHETHER OR NOT DISCOVERABLE. SOME JURISDICTIONS DO NOT ALLOW THE
EXCLUSION OF IMPLIED WARRANTIES, SO SUCH EXCLUSION MAY NOT APPLY TO YOU.
6. Limitation on Liability.  EXCEPT TO THE EXTENT REQUIRED BY APPLICABLE LAW, IN
NO EVENT WILL LICENSOR BE LIABLE TO YOU ON ANY LEGAL THEORY FOR ANY SPECIAL,
INCIDENTAL, CONSEQUENTIAL, PUNITIVE OR EXEMPLARY DAMAGES ARISING OUT OF THIS
LICENSE OR THE USE OF THE WORK, EVEN IF LICENSOR HAS BEEN ADVISED OF THE
POSSIBILITY OF SUCH DAMAGES.  7. Termination This License and the rights granted
hereunder will terminate automatically upon any breach by You of the terms of
this License. Individuals or entities who have received Adaptations or
Collections from You under this License, however, will not have their licenses
terminated provided such individuals or entities remain in full compliance with
those licenses. Sections 1, 2, 5, 6, 7, and 8 will survive any termination of
this License.  Subject to the above terms and conditions, the license granted
here is perpetual (for the duration of the applicable copyright in the
Work). Notwithstanding the above, Licensor reserves the right to release the
Work under different license terms or to stop distributing the Work at any time;
provided, however that any such election will not serve to withdraw this License
(or any other license that has been, or is required to be, granted under the
terms of this License), and this License will continue in full force and effect
unless terminated as stated above.  8. Miscellaneous Each time You Distribute or
Publicly Perform the Work or a Collection, the Licensor offers to the recipient
a license to the Work on the same terms and conditions as the license granted to
You under this License.  Each time You Distribute or Publicly Perform an
Adaptation, Licensor offers to the recipient a license to the original Work on
the same terms and conditions as the license granted to You under this License.
If any provision of this License is invalid or unenforceable under applicable
law, it shall not affect the validity or enforceability of the remainder of the
terms of this License, and without further action by the parties to this
agreement, such provision shall be reformed to the minimum extent necessary to
make such provision valid and enforceable.  No term or provision of this License
shall be deemed waived and no breach consented to unless such waiver or consent
shall be in writing and signed by the party to be charged with such waiver or
consent.  This License constitutes the entire agreement between the parties with
respect to the Work licensed here. There are no understandings, agreements or
representations with respect to the Work not specified here. Licensor shall not
be bound by any additional provisions that may appear in any communication from
You. This License may not be modified without the mutual written agreement of
the Licensor and You.  The rights granted under, and the subject matter
referenced, in this License were drafted utilizing the terminology of the Berne
Convention for the Protection of Literary and Artistic Works (as amended on
September 28, 1979), the Rome Convention of 1961, the WIPO Copyright Treaty of
1996, the WIPO Performances and Phonograms Treaty of 1996 and the Universal
Copyright Convention (as revised on July 24, 1971). These rights and subject
matter take effect in the relevant jurisdiction in which the License terms are
sought to be enforced according to the corresponding provisions of the
implementation of those treaty provisions in the applicable national law. If the
standard suite of rights granted under applicable copyright law includes
additional rights not granted under this License, such additional rights are
deemed to be included in the License; this License is not intended to restrict
the license of any rights under applicable law.  Creative Commons Notice
Creative Commons is not a party to this License, and makes no warranty
whatsoever in connection with the Work. Creative Commons will not be liable to
You or any party on any legal theory for any damages whatsoever, including
without limitation any general, special, incidental or consequential damages
arising in connection to this license. Notwithstanding the foregoing two (2)
sentences, if Creative Commons has expressly identified itself as the Licensor
hereunder, it shall have all rights and obligations of Licensor.

Except for the limited purpose of indicating to the public that the Work is
licensed under the CCPL, Creative Commons does not authorize the use by either
party of the trademark "Creative Commons" or any related trademark or logo of
Creative Commons without the prior written consent of Creative Commons. Any
permitted use will be in compliance with Creative Commons' then-current
trademark usage guidelines, as may be published on its website or otherwise made
available upon request from time to time. For the avoidance of doubt, this
trademark restriction does not form part of the License.

Creative Commons may be contacted at http://creativecommons.org/

% ------------------------------------------------------------------------------

\chapter{CLCC Prayer}

Most loving Lord, You said, ``I am the way, the truth and the life.''  Source of
every good, fill us with your Holy Spirit.  Inspire us to celebrate our lives as
Catholic Christians.  Show us the way to bring Celebrating Life as a Catholic
Christian, a faith formation process, to your people in parishes. Thank you for
entrusting this apostolate to us, and choosing us to live it, spread it, and
teach it in your Church.

Through the power of your Spirit enlighten our minds.  Open our hearts to value
the CLCC process as your special gift to your Church. For this mission, form us
in holiness. Set us on fire to promote the CLCC process not only to those in
your Church but to those outside of it.

Help us to form Small Ecclesial Faith Communities and bond with each other in
sincere ``communion.''  Through your grace move us to repent of our faults and
weaknesses that may be lingering in our hearts.

Awaken in us a hunger ``to be evangelized [in order to] evangelize.'' Give us
courage to reach out to family and friends who have left the Church, and to
millions of ``unchurched'' Catholics gravely in need of salvation.

Mary, Mother of Jesus, our Lady of Good Counsel, Seat of Wisdom, and Mother of
Perpetual Help, supreme model of Celebrating Life as a Catholic Christian, lead
us to deep conversion, growth in holiness and self-giving in service to your
Son, Jesus, for building up ``the Church as Communion'' like the Blessed
Trinity. We ask this of you Father, Son and Holy Spirit. Amen.

% ------------------------------------------------------------------------------

\chapter{About the Author}

Francis A. Novak, a Redemptorist priest, is a nationally recognized authority
and promoter of Total Stewardship. Earlier he had been a missionary, retreat
director, manager of promotion for Liguori Publications on the East Coast,
pastor and director of development and pastoral councils in the Diocese of Grand
Rapids. In 1974 he was made the first full-time executive director of the
National Catholic Stewardship Council in Washington, DC, and currently is
president of the National Catholic Conference for Total Stewardship. Over the
past 30 years he has developed Celebrating Life as a Catholic Christian, a
pastoral program of Total Stewardship for dioceses and parishes. This integrated
process provides comprehensive and systematic formation of the lay faithful in
Church teaching, spirituality and the apostolate of reaching out to lapsed
Catholics, thus furthering the Church's goal of New Evangelization. Author of
several publications treating Total Stewardship, he holds a Masters degree in
Liturgical Studies and a Doctorate in Ministry. He currently resides at
St. Clement in Liguori, Missouri, 63057.

% ------------------------------------------------------------------------------

\end{document}
