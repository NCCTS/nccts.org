% ------------------------------ sample ----------------------------------------

% WARNING!  Do not type any of the following 10 characters except as directed:
%                &   $   #   %   _   {   }   ^   ~   \

% \section{Simple Text}          % This command makes a section title.

% Words are separated by one or more spaces.  Paragraphs are separated by
% one or more blank lines.  The output is not affected by adding extra
% spaces or extra blank lines to the input file.

% Double quotes are typed like this: ``quoted text''.
% Single quotes are typed like this: `single-quoted text'.

% Long dashes are typed as three dash characters---like this.

% Emphasized text is typed like this: \emph{this is emphasized}.
% Bold       text is typed like this: \textbf{this is bold}.

% \subsection{A Warning or Two}  % This command makes a subsection title.

% If you get too much space after a mid-sentence period---abbreviations
% like etc.\ are the common culprits)---then type a backslash followed by
% a space after the period, as in this sentence.

% Remember, don't type the 10 special characters (such as dollar sign and
% backslash) except as directed!  The following seven are printed by
% typing a backslash in front of them:  \$  \&  \#  \%  \_  \{  and  \}.
% The manual tells how to make other symbols.

% ------------------------------------------------------------------------------

\documentclass[oneside]{book}
\usepackage{graphicx}
\setcounter{secnumdepth}{-1}

\begin{document}
\pagestyle{plain}

% ------------------------------------------------------------------------------

\frontmatter

CELEBRATING LIFE AS A CATHOLIC CHRISTIAN

% ------------------------------------------------------------------------------

Nihil Obstat: + Kelvin E. Felix, D.D. Archbishop of Castries Castries,
St. Lucia, West Indies Imprimatur: + John F. Donoghue, D.D. Archbishop of
Atlanta October 15, 2004 Celebrating Life as a Catholic Christian Copyright ©
2010 by Fr. Francis A. Novak, C.Ss.R. St. Clement, Liguori, Missouri, 63057, USA
www.clcc.info ISBN: 978-0-615-31588-1 The contents of this book are distributed
under the terms of the Creative Commons license, Attribution- NonCommercial 3.0
Unported. The full text of the license may be found at the back of the book,
beginning on page 309.  A summary of the license terms You are free: to share –
to copy, distribute and transmit the work; to remix – to adapt the work. Under
the following conditions: attribution – you must attribute the work in the
manner specified by the author or licensor (but not in any way that suggests
that they endorse you or your use of the work); noncommercial you may not use
this work for commercial purposes. With the understanding that: any of the above
conditions can be waived if you get permission from the copyright holder; public
domain – where the work or any of its elements is in the public domain under
applicable law, that status is in no way affected by the license; other rights –
in no way are any of the following rights affected by the license: your fair
dealing or fair use rights, or other applicable copyright exceptions and
limitations; the author’s moral rights; rights other persons may have either in
the work itself or in how the work is used, such as publicity or privacy rights;
notice – for any reuse or distribution, you must make clear to others the
license terms of this work.  Scripture selections, unless otherwise noted, are
taken from the New American Bible, Copyright © 1991, 1986, 1970 by the
Confraternity of Christian Doctrine, Inc., Washington, DC. Used with
permission. All rights reserved. No portion of the New American Bible may be
reprinted without permission in writings from the copyright holder.  Selections
from official documents of the Catholic Church are taken from the website of The
Holy See, wwww.vatican.va. Copyright © Libreria Editrice Vaticana.  Cover design
by Wendy Barnes, Liguori, Missouri. Artwork and design of the CLCC Centerfold
found on pages 144-145 by Chris Schwartz Studios, Saint Louis, Missouri.  For
more information and additional electronic resources, please visit the CLCC
website.  http://www.clcc.info e-mail: questions@clcc.info National Catholic
Conference for Total Stewardship 2010

\begin{center}
\includegraphics[scale=0.6]{by-nc-nd}
\end{center}

% ------------------------------------------------------------------------------

To teach in order to lead others to the faith is the task of every preacher and
of every believer.– St. Thomas Aquinas

% ------------------------------------------------------------------------------

The world aflame with Pentecost’s fire, Evangelization, the Good News
spreading,God’s people true conversion seeking;Enlightened by the Spirit’s
fire,Enlivened by the Savior’s Body and Blood,Protected under Virgin Mother’s
care,Entrusted with the Church’s teaching faithfully to live and share.

% ------------------------------------------------------------------------------

To my Father and Mother, and siblings, Sister Elizabeth, Marie, Fr. Henry,
Philip and Bishop Alfred and to my Redemptorist family and friends

% ------------------------------------------------------------------------------

\title{\textbf{CELEBRATING LIFE} \\ AS A \\ \textbf{CATHOLIC CHRISTIAN}}
\author{Fr. Francis A. Novak, C.Ss.R.}
\date{June 2013}

\maketitle

% ------------------------------------------------------------------------------

\chapter{Foreword}

\chapter{Acknowledgments}

\chapter{Primary Sources}

\chapter{Introduction}

\chapter{Putting the CLCC Process to Work}

\chapter{Order for Conducting Formation Sessions in Small Ecclesial Faith Communities}

\tableofcontents

\mainmatter

\part{The Process}

\chapter{Stage One}

This is stage 1 stuff.

\chapter{Stage Two}

This is stage 2 stuff.

\chapter{Stage Three}

This is stage 3 stuff.

\chapter{Stage Four}

This is stage 4 stuff.

\chapter{Stage Five}

This is stage 5 stuff.

\part{Authoritative Sources}

\chapter{Document A}

This is dcoument A stuff.

\chapter{Document B}

This is document B stuff.

\backmatter

\setcounter{secnumdepth}{-2}
\part{Appendices}

\chapter{Questions and Answers on the CLCC Process}

Q. Why do you call Celebrating Life as a Catholic Christian (CLCC) a process and
not a program? What's the difference?  A. A process is developmental. It moves
progressively from one stage to another. It has a beginning but no end. It's
ongoing. A program, on the other hand, is a succession of actions carried out
within a structured order that has a beginning and a predefined end.  Q. Is one
year long enough for carrying out this process?  A. No. In Galilee Jesus took
three years to form his disciples in his new way of life. Compare this process,
for example, to capital funds campaigns that are conducted in dioceses and
parishes. They usually take 3 years or more. Formation in Christ's new way of
life is also a "capital campaign" requiring a lifetime for becoming a saint and
serving our neighbor.  Q. Is the pastor involved in this process? If so, how,
when, where?  A. YES! He together with key people (example, the pastoral
council) find qualified parishioners who are willing to serve as lay leaders of
the pro- cess, and other lay persons to begin the establishment of Small
Ecclesial Faith Communities (SEFCs) throughout the parish. He need not get in-
volved in details. He gives it leadership, and above all shows interest and
support.  As often as possible and despite many demands, he should visit the
different SEFCs once in awhile when they hold their spiritual formation
301CELEBRATING LIFE AS A CATHOLIC CHRISTIAN sessions, encourage them, pray with
them, answer their questions, and help them keep their focus.  He is to preach
on the process with enthusiasm and conviction, as of- ten as he deems it helpful
to the people. In his remarks he is encouraged to choose some points from the
stage his parishioners are working through, and which he feels will most benefit
them. This also applies to the material in Part II, Authoritative Sources. On
Sunday the people in the pews who are not yet involved have a right to be
informed about the process and in- vited, in fact urged, to join and participate
in a SEFC.  Q. How many sessions for spiritual formation can be, or should be,
drawn from the material presented in each of the 5 stages and in Authoritative
Sources?  A. As many as there are paragraphs. The material in both parts is so
abun- dant that it is, relatively speaking, inexhaustible, and of course
repeatable. What the Matrix 12 and members of SEFCs seek is in-depth not
superfi- cial formation in faith and holiness. Each of the 5 stages provides
material for multiple formation sessions, one Stage has 20, another 43 sessions,
and so on.  Materials for formation sessions in Part II, Authoritative Sources
are taken entirely from magisterial documents. Under each paragraph or seg- ment
you will find these letters: R-D-S in bold. They mean Reflection, Development
and Sharing. Members of SEFCs should follow these three steps as they did in
Part I.  Q. Does the CLCC process have criteria to measure success or failure as
it carries out each of the stages?  A. Each of the 5 stages has built-in goals,
or targets to reach. Criteria for the general goal are: how well or poorly was
the stage carried out; did it attain what it was supposed to attain? For
example, in Stage 1, did twelve gifted people respond to the "call" to be
members of the Matrix, the lay group which leads the parish through the CLCC
process? Did enough pa- rishioners respond to the invitation to form one, two,
three or more Small 302 APPENDIX QUESTIONS AND ANSWERS ON THE CLCC PROCESS
Ecclesial Faith Communities for a start of the process? Not to reach these goals
the first time around is not failure, but a signal to continue to recruit. Each
stage also contains intermediate objectives. These are specific actions that are
carried out as exactly as possible within the formation session so everyone in
the SEFC gets the full benefit. For example, does the modera- tor start and end
the session on time; is the reflection issue clear to all; is enough time given
to quiet time; do all or most participants get a chance to share their insights;
does the moderator politely cut off long-winded shar- ing; are the hymns,
prayers and intercessions appropriate? More important- ly, do the participants
feel they are growing spiritually and gaining better knowledge of the faith?
Recommended is that each SEFC chooses a recorder to keep track of how well or
poorly goals and objectives are attained. This record is helpful for review and
can induce the group to strive for excellence in their forma- tion. It tells
participants how they are conducting the formation sessions, what needs
improvement, and how to make the sessions run smoothly to the best advantage of
all.  Q. What is the connection between the National Catholic Conference for
Total Stewardship and Celebrating Life as a Catholic Christian?  A. In 1980 the
National Catholic Conference for Total Stewardship (NCCTS) was founded by
Francis A. Novak, a Redemptorist priest, with approval of the NCCB, now called
the United States Conference of Catho- lic Bishops (USCCB). Its purpose is to
promote the biblical concept of stewardship in its totality. Stewardship has
been and still is widely used by dioceses and parishes for fund raising,
increased offertory appeals and cap- ital funds campaigns, under the banner of 3
Ts: Time, Talent and Treasure.  Fr. Novak and the NCCTS felt compelled to
explore the deeper impli- cations of stewardship in Sacred Scripture. Under the
guidance of expert biblical scholars, the NCCTS learned that the Old and New
Testaments are a fascinating account of how God "practices" holistic stewardship
in creation through his providential care of the universe and the planet
earth. The three components of stewardship found in both testaments are the good
news (Evangelization), holiness of life (Discipleship) and service to the Lord
and his people (Stewardship). The Bible's three components have 303 CELEBRATING
LIFE AS A CATHOLIC CHRISTIAN become the template of the NCCTS and its CLCC
process. They are living terms of a trilogy of transcendent realities that are
inextricably linked. In 1986 the language chosen to express the three realities
was Total Steward- ship.  Expressive as the NCCTS thought the new title was,
response to it was limited but it did stir curiosity in many quarters. In 1990 a
new language was developed to convey that Total Stewardship has an uplifting
quality: joy in celebrating Christ as the Way, the Truth and the Life for his
pil- grim people on earth. The new title was The Christian Celebration of Life
(CCL). The new title did arouse a wide reception of the process. But some times
it was mistaken for something else, for a pro-life group and even an offshoot of
Alcoholics Anonymous. Thus another name change was needed.  The new name for
this total stewardship process came in 1998: Cel- ebrating Life as a Catholic
Christian (CLCC). The NCCTS now feels this newly named process expresses not
only the biblical triad of Evangeliza- tion, Discipleship and Stewardship, but
also "New Evangelization", "Call to Holiness" and "Communion," goals which the
Second Vatican Council, Pope John Paul II, the Congregation for the Doctrine of
the Faith and Benedict XVI have called for repeatedly and passionately for God's
people in today's world of conflicting cultures.  Q. With the name change were
any adjustments made to the process: its fo- cus, goals and manner of carrying
out the Church's mission of New Evan- gelization?  A. The name change seems to
excite the human spirit. It opens the imagi- nation to new possibilities and
challenges like "I'm going to grow in holi- ness, me?" and "we'll reach out to
help lapsed Catholics return to the faith, terrific!" Indeed, the process has
been given a substantial revision and its content has been expanded. Each stage
makes formation in holi- ness, in doctrine and in pastoral outreach
clearer. Readers of this manual will be helped immensely in Celebrating their
Lives as Catholic Chris- tians.  304 APPENDIX QUESTIONS AND ANSWERS ON THE CLCC
PROCESS Q. What is the cost for implementing this process in a parish?  A. There
is no set fee. However, cost is in the purchase of the CLCC books for the
pastor, the lay leaders and all members of Small Ecclesial Faith
Communities. Based on experience the best way to go is for the parish to
purchase the needed number of manuals at a bulk rate, and then request each
participant to purchase a copy so as reimburse the parish. Without a book in
hand participants will find it impossible to carry out their forma- tion. The
idea to buy one or two books and run photocopies each week for the 12 lay
leaders and members of SEFCs, as an economy measure, has been tried before with
disastrous results. The end result is not an economy measure - it costs more in
time and paper. Handout sheets get lost. People need a complete, bound book.
Having a personal copy allows each person to study from his own book, get an
idea of the structure of the process and the order of how for- mation sessions
are carried out, and allows participants to use their free time to quick-read
the contents before going to the next week's session. Special cost arrangements
can be negotiated with the National Catholic Conference for Total Stewardship on
an as needed basis.  Q. The CLCC process, it appears, puts a lot of importance
on the lay lead- ers called the Matrix 12 and on Small Ecclesial Faith
Communities. Where can I go to find reliable and official Church teaching on
these kinds of communities?  A. Go to Stage 2, titled Formation of Disciples and
Apostles in the CLCC manual. Read section C. Small Ecclesial Faith Communities,
numbered paragraphs 1 to 10. Here the teaching on Small Ecclesial Faith Commu-
nities is drawn directly from John Paul II's encyclical Redemptoris Mis- sio and
from Pope Paul VI's Evangelii Nuntiandi. You can find additional information on
SEFCs in Part II of this manual, Authoritative Sources. See the first four
documents treated there.  305 CELEBRATING LIFE AS A CATHOLIC CHRISTIAN Q. Could
you tell me in one sentence what the CLCC process is?  A. The CLCC process is
pure, unadulterated biblical stewardship framed in the Second Vatican Council's
ecclesiology and magisterial teachings, which give solid theological formation
and mission motivation to faithful believers, enabling them to Celebrate their
Lives as Catholic Christians in a dechristianized culture.  306

\part{Indexes}

\chapter{Alphabetical}

\chapter{Topical}

\part{License}

CREATIVE COMMONS LICENSE Attribution-NonCommercial 3.0 Unported URI:
http://creativecommons.org/licenses/by-nc/3.0/legalcode License THE WORK (AS
DEFINED BELOW) IS PROVIDED UNDER THE TERMS OF THIS CREATIVE COMMONS PUBLIC
LICENSE ("CCPL" OR "LICENSE"). THE WORK IS PROTECTED BY COPYRIGHT AND/OR OTHER
APPLICABLE LAW. ANY USE OF THE WORK OTHER THAN AS AUTHORIZED UNDER THIS LICENSE
OR COPYRIGHT LAW IS PROHIBITED.  BY EXERCISING ANY RIGHTS TO THE WORK PROVIDED
HERE, YOU ACCEPT AND AGREE TO BE BOUND BY THE TERMS OF THIS LICENSE. TO THE
EXTENT THIS LICENSE MAY BE CONSIDERED TO BE A CONTRACT, THE LICENSOR GRANTS YOU
THE RIGHTS CONTAINED HERE IN CONSIDERATION OF YOUR ACCEPTANCE OF SUCH TERMS AND
CONDITIONS.  1. Definitions a. "Adaptation" means a work based upon the Work, or
upon the Work and other pre-existing works, such as a translation, adaptation,
derivative work, arrangement of music or other alterations of a literary or ar-
tistic work, or phonogram or performance and includes cinematographic
adaptations or any other form in which the Work may be recast, transformed, or
adapted including in any form recognizably derived from the original, except
that a work that constitutes a Collection will not be considered an Adaptation
for the purpose of this License. For the avoidance of doubt, where the Work is a
musical work, performance or phonogram, the synchronization of the Work in
timed-relation with a moving image ("synching") will be considered an Adaptation
for the purpose of this License.  b. "Collection" means a collection of literary
or artistic works, such as encyclopedias and anthologies, or performances,
phonograms or broadcasts, or other works or subject matter other than works
listed in Section 1(f) below, which, by reason of the selection and arrangement
of their contents, constitute intel- lectual creations, in which the Work is
included in its entirety in unmodified form along with one or more other
contributions, each constituting separate and independent works in themselves,
which together are assembled into a collective whole. A work that constitutes a
Collection will not be considered an Adapta- tion (as defined above) for the
purposes of this License.  c. "Distribute" means to make available to the public
the original and copies of the Work or Adaptation, as appropriate, through sale
or other transfer of ownership.  d. "Licensor" means the individual,
individuals, entity or entities that offer(s) the Work under the terms of this
License.  e. "Original Author" means, in the case of a literary or artistic
work, the individual, individuals, entity or entities who created the Work or if
no individual or entity can be identified, the publisher; and in addition (i) in
the case of a performance the actors, singers, musicians, dancers, and other
persons who act, sing, deliver, declaim, play in, interpret or otherwise perform
literary or artistic works or expressions of folklore; (ii) in the case of a
phonogram the producer being the person or legal entity who first fixes the
sounds of a performance or other sounds; and, (iii) in the case of broadcasts,
the organization that transmits the broadcast.  f. "Work" means the literary
and/or artistic work offered under the terms of this License including without
limitation any production in the literary, scientific and artistic domain,
whatever may be the mode or form of its expression including digital form, such
as a book, pamphlet and other writing; a lecture, address, sermon or other work
of the same nature; a dramatic or dramatico-musical work; a choreographic work
or entertainment in dumb show; a musical composition with or without words; a
cinematographic work to which are assimilated works expressed by a process
analogous to cinematography; a work of drawing, painting, architecture,
sculpture, engraving or lithography; a photographic work to which are
assimilated works expressed by a process analogous to photography; a work of
applied art; an illustration, map, plan, sketch or three-dimensional work
relative to geography, topography, architecture or science; a perfor- mance; a
broadcast; a phonogram; a compilation of data to the extent it is protected as a
copyrightable work; or a work performed by a variety or circus performer to the
extent it is not otherwise considered a literary or artistic work.
309CELEBRATING LIFE AS A CATHOLIC CHRISTIAN g. "You" means an individual or
entity exercising rights under this License who has not previously violated the
terms of this License with respect to the Work, or who has received express
permission from the Licensor to exercise rights under this License despite a
previous violation.  h. "Publicly Perform" means to perform public recitations
of the Work and to communicate to the public those public recitations, by any
means or process, including by wire or wireless means or public digital
performances; to make available to the public Works in such a way that members
of the public may access these Works from a place and at a place individually
chosen by them; to perform the Work to the public by any means or process and
the communication to the public of the performances of the Work, including by
public digital performance; to broadcast and rebroadcast the Work by any means
including signs, sounds or images.  i. "Reproduce" means to make copies of the
Work by any means including without limitation by sound or visual recordings and
the right of fixation and reproducing fixations of the Work, including storage
of a protected performance or phonogram in digital form or other electronic
medium.  2. Fair Dealing Rights.  Nothing in this License is intended to reduce,
limit, or restrict any uses free from copyright or rights arising from
limitations or exceptions that are provided for in connection with the copyright
protection under copyright law or other applicable laws.  3. License Grant.
Subject to the terms and conditions of this License, Licensor hereby grants You
a worldwide, royalty-free, non-exclu- sive, perpetual (for the duration of the
applicable copyright) license to exercise the rights in the Work as stated
below: a. to Reproduce the Work, to incorporate the Work into one or more
Collections, and to Reproduce the Work as incorporated in the Collections; b. to
create and Reproduce Adaptations provided that any such Adaptation, including
any translation in any medium, takes reasonable steps to clearly label,
demarcate or otherwise identify that changes were made to the original Work. For
example, a translation could be marked "The original work was translated from
English to Spanish," or a modification could indicate "The original work has
been modified."; c. to Distribute and Publicly Perform the Work including as
incorporated in Collections; and, d. to Distribute and Publicly Perform
Adaptations.  The above rights may be exercised in all media and formats whether
now known or hereafter devised. The above rights include the right to make such
modifications as are technically necessary to exercise the rights in other media
and formats. Subject to Section 8(f), all rights not expressly granted by
Licensor are hereby reserved, including but not limited to the rights set forth
in Section 4(d).  4. Restrictions.  The license granted in Section 3 above is
expressly made subject to and limited by the following restrictions: a. You may
Distribute or Publicly Perform the Work only under the terms of this
License. You must include a copy of, or the Uniform Resource Identifier (URI)
for, this License with every copy of the Work You Distribute or Publicly
Perform. You may not offer or impose any terms on the Work that restrict the
terms of this License or the ability of the recipient of the Work to exercise
the rights granted to that recipient under the terms of the License. You may not
sublicense the Work. You must keep intact all notices that refer to this License
and to the disclaimer of warranties with every copy of the Work You Distribute
or Publicly Perform. When You Distribute or Publicly Perform the Work, You may
not impose any effective technological measures on the Work that restrict the
ability of a recipient of the Work from You to exer- cise the rights granted to
that recipient under the terms of the License. This Section 4(a) applies to the
Work as incorporated in a Collection, but this does not require the Collection
apart from the Work itself to be made subject to the terms of this License. If
You create a Collection, upon notice from any Licensor You must, to the extent
practicable, remove from the Collection any credit as required by Section 4(c),
as requested. If You create an Adaptation, upon notice from any Licensor You
must, to the extent practi- cable, remove from the Adaptation any credit as
required by Section 4(c), as requested.  b. You may not exercise any of the
rights granted to You in Section 3 above in any manner that is primarily
intended for or directed toward commercial advantage or private monetary
compensation. The exchange 310 LICENSE of the Work for other copyrighted works
by means of digital file-sharing or otherwise shall not be considered to be
intended for or directed toward commercial advantage or private monetary
compensa- tion, provided there is no payment of any monetary compensation in
connection with the exchange of copyrighted works.  c. If You Distribute, or
Publicly Perform the Work or any Adaptations or Collections, You must, unless a
request has been made pursuant to Section 4(a), keep intact all copyright
notices for the Work and provide, reasonable to the medium or means You are
utilizing: (i) the name of the Original Author (or pseudonym, if applicable) if
supplied, and/or if the Original Author and/or Licensor designate another party
or parties (e.g., a sponsor institute, publishing entity, journal) for
attribution ("Attribution Parties") in Licensor's copyright notice, terms of
service or by other reasonable means, the name of such party or parties; (ii)
the title of the Work if supplied; (iii) to the extent reasonably practicable,
the URI, if any, that Licensor specifies to be associated with the Work, unless
such URI does not refer to the copyright notice or licensing information for the
Work; and, (iv) consistent with Section 3(b), in the case of an Adapta- tion, a
credit identifying the use of the Work in the Adaptation (e.g., "French
translation of the Work by Original Author," or "Screenplay based on original
Work by Original Author"). The credit required by this Section 4(c) may be
implemented in any reasonable manner; provided, however, that in the case of a
Adaptation or Collection, at a minimum such credit will appear, if a credit for
all contributing authors of the Adaptation or Collection appears, then as part
of these credits and in a manner at least as prominent as the credits for the
other contributing authors. For the avoidance of doubt, You may only use the
credit required by this Section for the purpose of attribution in the manner set
out above and, by exercising Your rights under this License, You may not
implicitly or explicitly assert or imply any connection with, sponsorship or
endorsement by the Original Author, Licensor and/or Attribution Parties, as
appropriate, of You or Your use of the Work, without the separate, express prior
written permission of the Original Author, Licensor and/or Attribution Parties.
d. For the avoidance of doubt: i. Non-waivable Compulsory License Schemes. In
those jurisdictions in which the right to collect royalties through any
statutory or compulsory licensing scheme cannot be waived, the Licensor reserves
the exclusive right to collect such royalties for any exercise by You of the
rights granted under this License; ii. Waivable Compulsory License Schemes. In
those jurisdictions in which the right to collect royalties through any
statutory or compulsory licensing scheme can be waived, the Licensor reserves
the exclusive right to collect such royalties for any exercise by You of the
rights granted under this License if Your exercise of such rights is for a
purpose or use which is otherwise than noncommercial as permitted under Section
4(b) and otherwise waives the right to collect royalties through any statutory
or compulsory licensing scheme; and, iii. Voluntary License Schemes. The
Licensor reserves the right to collect royalties, whether individually or, in
the event that the Licensor is a member of a collecting society that administers
voluntary licensing schemes, via that society, from any exercise by You of the
rights granted under this License that is for a purpose or use which is
otherwise than noncommercial as permitted under Section 4(c).  e. Except as
otherwise agreed in writing by the Licensor or as may be otherwise permitted by
applicable law, if You Reproduce, Distribute or Publicly Perform the Work either
by itself or as part of any Adapta- tions or Collections, You must not distort,
mutilate, modify or take other derogatory action in relation to the Work which
would be prejudicial to the Original Author's honor or reputation. Licensor
agrees that in those jurisdictions (e.g. Japan), in which any exercise of the
right granted in Section 3(b) of this License (the right to make Adaptations)
would be deemed to be a distortion, mutilation, modification or other de-
rogatory action prejudicial to the Original Author's honor and reputation, the
Licensor will waive or not assert, as appropriate, this Section, to the fullest
extent permitted by the applicable national law, to enable You to reasonably
exercise Your right under Section 3(b) of this License (right to make
Adaptations) but not otherwise.  311 CELEBRATING LIFE AS A CATHOLIC CHRISTIAN
5. Representations, Warranties and Disclaimer UNLESS OTHERWISE MUTUALLY AGREED
TO BY THE PARTIES IN WRITING, LICENSOR OFFERS THE WORK AS-IS AND MAKES NO
REPRESENTATIONS OR WARRANTIES OF ANY KIND CONCERNING THE WORK, EXPRESS, IMPLIED,
STATUTORY OR OTHERWISE, INCLUDING, WITHOUT LIMITATION, WAR- RANTIES OF TITLE,
MERCHANTIBILITY, FITNESS FOR A PARTICULAR PURPOSE, NONINFRINGEMENT, OR THE
ABSENCE OF LATENT OR OTHER DEFECTS, ACCURACY, OR THE PRESENCE OF ABSENCE OF
ERRORS, WHETHER OR NOT DISCOVERABLE. SOME JURISDICTIONS DO NOT ALLOW THE EXCLU-
SION OF IMPLIED WARRANTIES, SO SUCH EXCLUSION MAY NOT APPLY TO YOU.
6. Limitation on Liability.  EXCEPT TO THE EXTENT REQUIRED BY APPLICABLE LAW, IN
NO EVENT WILL LICENSOR BE LIABLE TO YOU ON ANY LEGAL THEORY FOR ANY SPECIAL,
INCIDENTAL, CONSEQUENTIAL, PUNITIVE OR EXEMPLARY DAMAGES ARISING OUT OF THIS
LICENSE OR THE USE OF THE WORK, EVEN IF LICEN- SOR HAS BEEN ADVISED OF THE
POSSIBILITY OF SUCH DAMAGES.  7. Termination a. This License and the rights
granted hereunder will terminate automatically upon any breach by You of the
terms of this License. Individuals or entities who have received Adaptations or
Collections from You under this License, however, will not have their licenses
terminated provided such individuals or entities remain in full compliance with
those licenses. Sections 1, 2, 5, 6, 7, and 8 will survive any termination of
this License.  b. Subject to the above terms and conditions, the license granted
here is perpetual (for the duration of the applicable copyright in the
Work). Notwithstanding the above, Licensor reserves the right to release the
Work under different license terms or to stop distributing the Work at any time;
provided, however that any such election will not serve to withdraw this License
(or any other license that has been, or is required to be, granted under the
terms of this License), and this License will continue in full force and effect
unless terminated as stated above.  8. Miscellaneous a. Each time You Distribute
or Publicly Perform the Work or a Collection, the Licensor offers to the
recipient a license to the Work on the same terms and conditions as the license
granted to You under this License.  b. Each time You Distribute or Publicly
Perform an Adaptation, Licensor offers to the recipient a license to the
original Work on the same terms and conditions as the license granted to You
under this License.  c. If any provision of this License is invalid or
unenforceable under applicable law, it shall not affect the validity or
enforceability of the remainder of the terms of this License, and without
further action by the parties to this agreement, such provision shall be
reformed to the minimum extent necessary to make such provision valid and
enforceable.  d. No term or provision of this License shall be deemed waived and
no breach consented to unless such waiver or consent shall be in writing and
signed by the party to be charged with such waiver or consent.  e. This License
constitutes the entire agreement between the parties with respect to the Work
licensed here. There are no understandings, agreements or representations with
respect to the Work not specified here. Licensor shall not be bound by any
additional provisions that may appear in any communication from You. This
License may not be modified without the mutual written agreement of the Licensor
and You.  f. The rights granted under, and the subject matter referenced, in
this License were drafted utilizing the terminology of the Berne Convention for
the Protection of Literary and Artistic Works (as amended on September 28,
1979), the Rome Convention of 1961, the WIPO Copyright Treaty of 1996, the WIPO
Performances and Phonograms Treaty of 1996 and the Universal Copyright
Convention (as revised on July 24, 1971). These rights and subject matter take
effect in the relevant jurisdiction in which the License terms are sought to be
enforced according to the corresponding provisions of the implementa- tion of
those treaty provisions in the applicable national law. If the standard suite of
rights granted under applicable copyright law includes additional rights not
granted under this License, such additional rights are deemed to be included in
the License; this License is not intended to restrict the license of any rights
under applicable law.  312 LICENSE Creative Commons Notice Creative Commons is
not a party to this License, and makes no warranty whatsoever in connection with
the Work. Creative Commons will not be liable to You or any party on any legal
theory for any damages whatsoever, including without limitation any general,
special, incidental or consequential damages arising in connection to this
license. Notwithstanding the foregoing two (2) sentences, if Creative Commons
has expressly identified itself as the Licensor hereunder, it shall have all
rights and obligations of Licensor.  Except for the limited purpose of
indicating to the public that the Work is licensed under the CCPL, Creative Com-
mons does not authorize the use by either party of the trademark "Creative
Commons" or any related trademark or logo of Creative Commons without the prior
written consent of Creative Commons. Any permitted use will be in compliance
with Creative Commons' then-current trademark usage guidelines, as may be
published on its website or otherwise made available upon request from time to
time. For the avoidance of doubt, this trademark restriction does not form part
of the License.  Creative Commons may be contacted at
http://creativecommons.org/

\part{CLCC Prayer}

Most loving Lord, You said, "I am the way, the truth and the life." Source of
every good, fill us with your Holy Spirit. Inspire us to celebrate our lives as
Catholic Christians. Show us the way to bring Celebrating Life as a Catholic
Christian, a faith formation process, to your people in parishes. Thank you for
entrusting this apostolate to us, and choosing us to live it, spread it, and
teach it in your Church.  Through the power of your Spirit enlighten our
minds. Open our hearts to value the CLCC process as your special gift to your
Church. For this mission, form us in holiness. Set us on fire to promote the
CLCC process not only to those in your Church but to those outside of it.  Help
us to form Small Ecclesial Faith Communities and bond with each other in sincere
"communion." Through your grace move us to repent of our faults and weaknesses
that may be lingering in our hearts.  Awaken in us a hunger "to be evangelized
[in order to] evangelize." Give us courage to reach out to family and friends
who have left the Church, and to millions of "unchurched" Catholics gravely in
need of salvation.  Mary, Mother of Jesus, our Lady of Good Counsel, Seat of
Wisdom, and Mother of Perpetual Help, supreme model of Celebrating Life as a
Catholic Christian, lead us to deep conversion, growth in holiness and
self-giving in service to your Son, Jesus, for building up "the Church as
Communion" like the Blessed Trinity. We ask this of you Father, Son and Holy
Spirit. Amen.

% ------------------------------------------------------------------------------

\end{document}
