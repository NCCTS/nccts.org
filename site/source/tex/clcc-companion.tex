% ------------------------------------------------------------------------------
% Copyright (c) 2014 by the NCCTS
% This work is distributed under the terms of the Creative Commons license,
% Attribution-NonCommercial 3.0 Unported.
% https://creativecommons.org/licenses/by-nc/3.0/us/
% https://creativecommons.org/licenses/by-nc/3.0/us/legalcode

% ------------------------------ sample ----------------------------------------

% WARNING!  Do not type any of the following 10 characters except as directed:
%                &   $   #   %   _   {   }   ^   ~   \

% \section{Simple Text}          % This command makes a section title.

% Words are separated by one or more spaces.  Paragraphs are separated by
% one or more blank lines.  The output is not affected by adding extra
% spaces or extra blank lines to the input file.

% Double quotes are typed like this: ``quoted text''.
% Single quotes are typed like this: `single-quoted text'.

% Long dashes are typed as three dash characters---like this.

% Emphasized text is typed like this: \emph{this is emphasized}.
% Bold       text is typed like this: \textbf{this is bold}.

% \subsection{A Warning or Two}  % This command makes a subsection title.

% If you get too much space after a mid-sentence period---abbreviations
% like etc.\ are the common culprits)---then type a backslash followed by
% a space after the period, as in this sentence.

% Remember, don't type the 10 special characters (such as dollar sign and
% backslash) except as directed!  The following seven are printed by
% typing a backslash in front of them:  \$  \&  \#  \%  \_  \{  and  \}.
% The manual tells how to make other symbols.

% ------------------------------------------------------------------------------

\documentclass{article}
\usepackage{graphicx}
\usepackage{hyperref}
\usepackage{lxRDFa}

\title{\textbf{Companion to the Manual} \\
               Celebrating Life as a Catholic Christian}
\author{Fr. Francis A. Novak, C.Ss.R.}
\date{2006}
\begin{document}

% ------------------------------------------------------------------------------

\maketitle

\renewcommand{\abstractname}{\emph{\textmd{Veni, Sancte Spiritus!}}}
\begin{abstract}

This booklet is a companion to a larger book, a manual titled, \emph{Celebrating
Life as a Catholic Christian} (CLCC). In a few pages it tells you what the CLCC
is: a process that forms Catholics in the faith through official Church
teachings and helps them respond to their \emph{call to holiness}. It will also
explain CLCC's special apostolate, reaching out to and helping an estimated 45
million lapsed, non-practicing Catholics return to Christ and his Church, a
mammoth challenge for the Church.

\end{abstract}

% ------------------------------------------------------------------------------

\setcounter{secnumdepth}{0}
\section{Small Ecclesial Faith Communities}

The process of formation takes place in \emph{Small Ecclesial Faith Communities}
(SEFCs). \textbf{Small}, a group of eight to twelve people who meet in someone's
home. \textbf{Ecclesial}, the group is composed of parish members and is
approved by the pastor. \textbf{Faith}, the group is eager to know the faith
better, grow in love of God and neighbor, is earnest about spreading the Good
News, and building up the Church to ``full stature.'' \textbf{Community}, an
assortment of fellow Catholics, who previously knew each other only by face now
come together in loving ``communion'' and bond through weekly study, sharing,
prayer and mutual support.

Jesus' primary reason for coming to earth was to establish his Church, at once
an earthly and heavenly kingdom. He started by forming his own small faith
community of twelve Apostles. Since his day, the apostles and their successors
through the centuries have established and depended on small faith communities
for building the kingdom. Historically small faith communities have a missionary
purpose. They are a catalyst for parishes, from which dioceses are
erected. Today they dot every corner of the earth making the Church undeniably
``catholic'' or universal. They are the starting point for giving growth and
renewal to the Church.

Pope John Paul II during his twenty-seven year pontificate preached and promoted
what he called ``small ecclesial faith communities.'' He added the word
``ecclesial'' to make sure they do not function independently of the
Church. These he saw as an efficient way for doing New Evangelization, the Good
News that he said must be \emph{``new in ardor, new in expression and new in
method.''}  (Haiti, 1983) The CLCC process is indeed ``new'' in all three
categories.

% ------------------------------------------------------------------------------

\section{Catholic Identity}

A currently critical issue in the Church is loss of \emph{Catholic
identity}. The CLCC process, \emph{Celebrating Life as a Catholic Christian}, is
a way to recover it. Loss occurs because so many Catholics know so little about
the ``mystery of Church.'' Since the Second Vatican Council, the Popes, in
particular John Paul II and Paul VI, have been urging the People of God to learn
more about the Church through systematic formation drawn from her magisterial
and authoritative teachings. Better understanding of the Church and deeper
spirituality result.

Being Catholic means more than being baptized and attending Sunday Mass in some
nearby Catholic church. It means more than pumping a little religion into
gradeschool children, and after the Sacrament of Confirmation is conferred
seeing them no longer. Formation in Church teachings and holiness must be
ongoing especially for adults, young adults and youth. In the past forty years
most Catholic unfortunately have been given a disastrously inferior education in
the faith. Religious illiteracy among great number of Catholic is endemic, Asked
a question about the faith, the answer is either slow in coming or
embarrassingly inadequate.

Catholic identity means that a Catholic's lifestyle, grounded in solid
theological and moral principles, sets him apart from the crowd. Good character
qualities that are observable also make him or her different. Identity is not a
religious medal dangling on a chain from one's neck. It's attitude, values,
demeanor and conviction, all are centered in a derived from supernatural faith,
hope and charity, humility and wisdom that emanate outward. Both Catholic
\emph{identity} and Catholic \emph{``in name only''} are transparent.

To be ``different'' means becoming engaged in the Spirit-filled promise that
Christ made without any reservations, ``Where two or three are gathered together
in my name, there am I in the midst of them'' (\emph{Mt} 18:20). Members of
SEFCs gradually get to the point that they sense Christ's Spirit is present to
them. As they acknowledge that the Holy Spirit is pouring out an abundance of
graces on them, so their sense of Catholic identity grows stronger. Gathering
together week after week for formation in holiness and truth marks each
individual of the group as ``different,'' a complement not a criticism.

Grace is the ``living water'' that satisfies the soul's thirst for knowledge and
holiness as Jesus stated to the Samaritan woman at the well (cf.\ \emph{Jn}
4:10). The CLCC formation process brings about conversion from a run of the mill
Catholic to one who identifies genuinely as Catholic. What the CLCC formation
process does is raise consciousness among God's People to the Catholic identity
crisis that exists in the Church and in our culture. It invites them not to
ignore it but be willing to do something about it: \emph{begin celebrating life
as a Catholic Christian}.

% ------------------------------------------------------------------------------

\section{St.\ Peter's Words}

The Apostle St.\ Peter in his First Letter identifies newly established small
Christian communities in the early church in this way. You are ``a chosen race,
a community of priest-kings, a consecrated nation, a people God has made his own
to proclaim his wonders'' (\emph{1 Pet} 2:9). By these words St.\ Peter is
stating how essential small Christian faith communities are for celebrating life
as a Catholic Christian.

In the same letter St.\ Peter uses another analogy likening small faith
communities to living stones. ``Like living stones,'' he writes, ``let
yourselves be built into a spiritual house to be a holy priesthood to offer
spiritual sacrifices acceptable to God through Jesus Christ'' (\emph{1 Pet} 2:5)
St.\ Peter envisioned the Church as a ``spiritual house,'' the House of God,
built with ``living stones,'' small faith communities, growing into bigger
spiritual houses, and ultimately becoming the universal Church. In the two texts
one can see Catholic identity, ``living stones'' starting with small faith
communities and growing into the Body of Christ. The Church's beginning on
Pentecost Sunday, its increase and renewal over the centuries, starts now as
then through small faith communities. Both of St.\ Peter's statements hold a
wealth of meaning. To comprehend the depth of their meaning is food for a
lifetime of reflection.

% ------------------------------------------------------------------------------

\section{First Considerations of the CLCC Process}

Jesus' parting words to his Apostles before ascending into heaven were: ``Go,
therefore, make disciples of all nations \ldots teaching the to observe all that
I have commanded you. And Behold, I am with you always, until the end of the
age'' (\emph{Mt} 2:19-20). This command is Christ's blueprint for building up
his Church. The Second Vatican Council's \emph{Decree on the Apostolate of the
Laity} states that the laity have a ``right and duty to exercise the apostolate
\ldots [having] their own proper roles in building up the Church.''
(\emph{Apostolicam actositatem}, 25) Practically this means the laity can and
should share the function of building up the Church. One way to achieve this is
for the laity to become deeply involved in the CLCC process, a process that
forms them systematically in \emph{evangelization, discipleship and service}.

It is the laity who usually conduct the formation sessions. The pastor, however,
must take the initiative. He takes time to become familiar with the process,
gives it his blessing, and promises to promote it by preaching and including it
in the Prayers of the Faithful each Sunday. He allows little progress reports to
be placed in the parish bulletin. These are reminders that help keep the CLCC
alive in the parishioners' minds. They help people decide to join an existing
SEFC or form new ones in the parish.

% ------------------------------------------------------------------------------

\section{How and Where to Begin}

Ideally, action begins when the pastor and parish council select a group of top
men and women called the \emph{Matrix 12} (after the Twelve Apostles) who are
qualified to organize and lead the CLCC process. The Matrix 12 becomes the first
Small Ecclesial Faith Community in the parish. It models itself on Mary, Mother
of God and of the Church. As she conceived and delivered Jesus to the world in
Bethlehem, so the Matrix 12 delivers the CLCC process into the parish. Being the
first Small Ecclesial Faith Community in the parish, they begin their formation
in the faith, a spiritual journey of growing in grace by study, prayer and
bonding. Simultaneously they prepare for their mission, to reach out to lapsed,
unchurched Catholics with the aim of helping them return to the faith.

After some nine to twelve months, the Matrix 12 is formed well enough to
establish more SEFCs --- a second, third, fourth, tenth, twentieth and so
on. Hundreds of Catholics will experience religious formation they've never had
before. All the while they are mindful of their mission to reach out to the
lapsed. Filled with the light of truth, they sense the power of Jesus' words in
their lives, ``I am the way, the truth and the life, no one comes to the Father
\emph{except through me}'' (\emph{Jn} 14:4). They begin truly \emph{celebrating
their lives as Catholic Christians}.

% ------------------------------------------------------------------------------

\section{The CLCC Process of Formation}

The CLCC formation process is an imitation, or as nearly as possible a
duplication, in the parish of Jesus' mission in Galilee. He carried it out in
five stages. So too the CLCC process is carried out in five corresponding
stages.

% ------------------------------------------------------------------------------

\subsection{Stage 1. Call of Apostles and Disciples}

Jesus first called disciples and later chose the Twelve Apostles. Most notably
they were ordinary men: fishermen, tax collectors and others who were looking
for the \emph{Messiah}. They were attracted to his personality. Some perhaps
were looking for a career change. He said simply, ``Come after me, and I will
make you fishers of men'' (\emph{Mt} 4:19).

% ------------------------------------------------------------------------------

\subsection{Stage 2. Formation of the Apostles and Disciples}

Jesus' school of evangelization was not a classroom but the great outdoors: the
Sea of Galilee, a mountain, a vineyard, a field, a well-traveled road. He also
evangelized at a wedding feast, a banquet hall, a tax collector's house, the
home of Mary and Martha, the Temple and finally the Cross on Calvary and the
empty tomb. ``They were spellbound at his teaching because he spoke with
authority'' (\emph{Lk} 4:32).

% ------------------------------------------------------------------------------

\subsection{Stage 3. Stewardship of our Gifts}

The ``many and varied'' gifts we freely receive from God are in such abundance
and continuous flow that it is impossible for us to count them or even be aware
of them all. But as persons of faith we must ``count our blessings'' as best we
can. This means we must acknowledge God as the \emph{Giver} of all gifts. It
means cultivating the habit of daily prayer, in justice, to praise God for his
gifts, thanking him humbly, rejoicing like the leper who was healed returned to
Christ ``glorifying God in a loud voice'' (\emph{Lk} 17:15).

% ------------------------------------------------------------------------------

\subsection{Stage 4. Home Visitation and Evangelization}

Jesus ``went around to the villages in the vicinity teaching. He summoned the
Twelve and began to send them out two by two and gave them authority over
unclean spirits'' (cf.\ \emph{Mk} 6:6-7). Jesus and the Twelve tried first to
evangelize the Chosen People. In Acts of the Apostles we read, ``the word of God
be spoken \ldots first [to the Jews] \ldots we now turn to the Gentiles''
(cf. \emph{Acts} 13:46,47-48). Stage 4 continues to form practicing Catholics
more deeply in faith and holiness, but now prepares them to turn to the
thousands of lapsed, non-church going, nominal Catholics residing within the
parish boundaries who are in need of salvation.

% ------------------------------------------------------------------------------

\subsection{Stage 5. Stewards of the Eucharist}

Most important is Stage 5, the climax of the process the CLCC formation
process. To be a Catholic Christian one cannot get by with only a meager
understanding of the mystery of the Holy Eucharist, the ``greatest of all
gifts.'' The Eucharist is Jesus Christ himself, the Incarnate Word, the God-Man,
who is really and truly present to us in the Sacrament of his body and
blood. God the Father gave man the supreme gift of himself, his only-begotten
Son, Jesus, out of his infinite, inestimable love of man. In turn, Jesus gave
himself to us.

Before his ascension into heaven, he chose to remain present to man out of love
in a manner never known before. Under the appearance of common bread and wine,
eatable and potable substances, he gave his Body and Blood, Soul and Divinity, a
reality known and believed only through the sight of faith. By this
unsurpassable gift, Catholics are given access to union with Jesus Christ, a
union unlike any other union on earth. Thus, we must regard ourselves as
extraordinarily privileged ``servants of Christ and stewards of the mysteries of
God,'' as the Apostle St.\ Paul says. (\emph{1 Cor} 4:1) Note that first
encyclical letter of Pope Benedict XVI is titled \emph{``God is Love''}.

% ------------------------------------------------------------------------------

\section{Formation is Weekly}

Formation sessions in the CLCC process are held once a week, every week
throughout the year. The success of this process is there are no interruptions
in the sequence. There are no lengthy intervals between get-togethers, for
example the months between Lent and autumn; no vacation breaks in the summer or
during the Christmas and Easter seasons or during the World Series. Continuity
keeps a steady flow of grace to all members. Their love of Christ and his Church
grows steadily, and bonding within the group is a sure sign of the presence and
influence of the Holy Spirit. In view of experience with the process, it is most
strongly urged that adherence to the weekly session be held ``sacred,'' i.e. not
abandoned or yielded to personal feelings or easy excuses. Realistically,
however, when a member of the SEFC must be absent for a good reason, the weekly
formation nonetheless goes on. Celebrating Life as a Catholic Christian is
perennial. It is not like a seasonal sport.

% ------------------------------------------------------------------------------

\section{Interior Spiritual Formation}

The weekly formation session has two parts. The first part is called
\textbf{Interior Spiritual Formation}, the second part is called
\textbf{Exterior Action}. The first part is a teaching, a presentation of key
quotations from official Church documents such as the Second Vatican Council,
papal encyclicals, apostolic exhortations and letters, the \emph{Catechism of
the Catholic Church}, and related texts of Sacred Scripture. The quotations are
chosen in a way that their meaning and relevance to the Stage under
consideration is clear to all in the SEFC. The group is asked to look carefully
at the phrases and important words in the paragraph, and center on one or two
such phrases or words invoking the Holy Spirit for light to get a hold on the
main point. To be avoided is to try to absorb the whole content, thereby getting
lost in a multiplicity of thoughts and words that distract a person from the
teaching. Imperative also is that only \textbf{one paragraph is read, studied,
prayed over and shared in the session}. Reflection on one paragraph only at each
session is unique to this formation process. The insights the members obtain
from the paragraph and share with the group are always surprising and inspiring.

Most important to add here is that this process ought \textbf{\emph{not}} be
rushed through, streamlined, shortened or gotten through with quickly. Rather,
each paragraph must be read carefully, reflected on, savored, digested and
shared in order to draw from the knowledge and graces that only the Holy Spirit
offers and only God can count. Remember, the majority of paragraphs are
summaries of the official teachings of the Church. Therefore they have the power
of truth derived from the Holy Spirit. Cherish what you read, share what you
have been given, and receive what you hear form the others as gifts of the Holy
Spirit.

To help understand the \textbf{one paragraph per week procedure} and to gain
maximum spiritual formation from each session, at the end of each stage is a set
of \emph{Reflection and Discussion Issues} (RDI) that correspond to the
paragraph under consideration. It is most helpful to refer to and use the
RDIs. They are offered to stimulate reflection. At the same time, pray to the
Holy Spirit for his gifts of understanding and knowledge.

% ------------------------------------------------------------------------------

\section{Exterior Action, Part Two of CLCC Formation}

The second part of the weekly session is called Exterior Action. The paragraph
under Interior Spiritual Formation is reread and prayerfully reflected on again
in quiet time, but from the perspective of action: \emph{how can we act on what
we find in the paragraph? What worthwhile action can be chosen that translates
the mental process of reflection into a concrete missionary action that will
help us prepare for our outreach apostolate to fallen-away Catholics?} Ideas
often fly freely. But, without fail, someone in the SEFC comes up with an action
that is acceptable to all. Not infrequently the agreeable action is fine-tuned
with someone offering a related idea. The Exterior Action agreed upon becomes
the action of the week for all. Obviously the Holy Spirit is present and
working. He is helping each person in the SEFC to move forward on his or her
journey of celebrating life as a Catholic Christian.

% ------------------------------------------------------------------------------

\section{90 Minute Syndrome}

What makes the weekly CLCC formation session successful? The strict 90-minute
time frame within which the weekly session is conducted! Packed into ninety
minutes are the two parts: 45 minutes for \emph{Interior Spiritual Formation},
and 45 minutes for \emph{Exterior Action}. Each 45-minute period follows a
prescribed order.

% ------------------------------------------------------------------------------

\section{Order for Conducting Formation Sessions}

This \emph{Order} is found at the beginning of the \emph{CLCC Manual}. It is the
solemn duty of the Moderator, usually the leader of the SEFC, to follow it
faithfully. As members grow familiar with the procedure, other members are
invited, in fact urged, to take a turn at moderating a session. Moderating a
session gives each member a sense of belonging, and heightens his own
participation within the group. The experience gives everyone a feeling of
responsibility for the process and to the group.

The moderator must always stick to the Order and not deviate, nor let others get
off track. He or she must be alert, wise, kind and strong. This means the
moderator may not allow people to wander off to other subjects not being
considered. Everyone should be encouraged to speak up and share. Should someone
get long-winded, the moderator (with charity and prudence) must cut off the
person in order to stick to the subject and stay within the time frame.

With respect to each person's circumstances it is most important that the
moderator starts the session on time and ends on time. A sure way to kill the
process is to start late or go overtime. After the session is over, if people
wish to stay to socialize, that is their choice.

Where are the weekly meetings held? A room in the parish facility \emph{seems}
like a good place because the SEFC then identifies with the parish. But it's
also \emph{not a good place}. Other groups in the parish have a right to use the
room too. The better place is someone's private home. Someone in the SEFC is
usually willing to have the group of up to twelve persons. This is how the
early, apostolic Church started --- in peoples' homes.

% ------------------------------------------------------------------------------

\section{``By their fruits will you know them''}

To help the Matrix 12 (the lay leadership group) establish as many Small
Ecclesial Faith Communities as possible in the parish, it is useful for the
twelve and those deciding whether to enter an SEFC to ponder the parable of the
``good tree'' that bears ``good fruit'' (cf.\ \emph{Lk} 6:43-45) The application
is easy to see. People who are thinking about entering the process, that is,
people desirous of growing in holiness and deeper knowledge of the faith, can be
likened to a ``good tree'' that bears ``good fruit.'' Already the blossoms of
faith begin to show. The buds of Christ-centered holiness break forth. The tiny
fruit, \emph{Celebrating Life as a Catholic Christian}, is ripening to become
good fruit.

It is logical that a \emph{good tree} bear \emph{good fruit}. In the Gospel of
St.\ Luke we read, ``every tree is known by its fruit'' (cf.\ \emph{Lk}
6:44). People who enter the CLCC formation process are like the good tree that
produces good fruit. The ``good tree'' is that large company of baptized but
minimally catechized ``Sunday Catholics'' who, hungering for truth, enter the CLCC
process. Doing so, they feel they will ripen into formed Catholic
Christians. Their ``fruit'' is twofold: holiness and zeal to spread the good news,
and a committment to building up the House of God by brining lapsed Catholics
back into the fold. Jesus said this, ``for every tree is known by its own
fruit.''

St.\ Matthew narrates Our Lord's fruit parable in this way, ``A good tree cannot
bear bad fruit, nor can a rotten tree bear good fruit. Every tree that does not
bear good fruit will be cut down and thrown into the fire. So \emph{by their
fruits you will know them}'' (\emph{Mt} 7:18-20). Simply stated, the fruit of
the CLCC process is that people in parishes begin celebrating their lives as
Catholic Christians.

St.\ John in his Gospel records Jesus' parable of the vine, the vine grower,
branches and the fruit --- an analogy that can be applied to the CLCC formation
process. It brings a powerful message to those in the process as well as to
those not yet in the process.

Jesus said, ``I am the vine, and my Father is the vine grower. Remain in me, as
I remain in you. Just as a branch cannot bear fruit on its own unless it remains
on the vine, so neither can you unless you remain in me. I am the vine, you are
the branches. Whoever remains in me and I in him will bear much fruit, because
without me you can do nothing. It was not you who chose me, but I who chose you
and appointed you to go and bear fruit that will remain. So whatever you ask the
Father in my name he may give you'' (cf.\ \emph{Jn} 15:1,4,5,16).

% ------------------------------------------------------------------------------

\section{Conclusion}

\textbf{CLCC}. This four-letter acronym has been used may times in this
booklet. It describes a formation process that every member of the Church will
find life-changing, as well as spiritually and intellectually rewarding, if he
chooses to enter it. \newline

The letters mean: \newline

\textbf{C Celebrating:} God's grace together with the individual's strong faith
make life a joy --- a jubilation despite tribulation. \newline

\textbf{L Life:} Because God created us to his image, and in baptism Christ
recreated us to his likeness, our lives are infinitely precious in his
sight. \newline

\textbf{C Catholic:} A unique identity has been given us. We are members of
Christ's Body. We belong to the one, holy, catholic and apostolic Church that
the Lord Jesus founded. \newline

\textbf{C Christian:} As true followers of Christ we are committed totally to
his way of life (cf.\ \emph{Acts} 11:26).

% ------------------------------------------------------------------------------

\section{} \lxRDFa{property=cc-lic-section,resource={cc-lic}}

\begin{center}

This work is distributed under the terms of the Creative Commons license,
\emph{\href{https://creativecommons.org/licenses/by-nc/3.0/legalcode}{
            Attribution-NonCommercial 3.0 Unported}}.

\href{https://creativecommons.org/licenses/by-nc/3.0/}{
  \includegraphics[scale=0.4]{by-nc-nd}
  \lxRDFa{property=cc-lic-graphic,resource={cc-lic}}}

The \href{http://www.latex-project.org/}{\LaTeX} source code for this book can
be freely accessed on GitHub.

\texttt{\href{https://github.com/NCCTS/nccts.org/tree/latest/site/source/tex/}
             {https://github.com/NCCTS/nccts.org/tree/latest/site/source/tex/}}

\end{center}

% ------------------------------------------------------------------------------

\end{document}
